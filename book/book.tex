% LaTeX source for ``Think Python: How to Think Like a Computer Scientist''
% Copyright (c)  2015  Allen B. Downey.

% License: Creative Commons Attribution-NonCommercial 3.0 Unported License.
% http://creativecommons.org/licenses/by-nc/3.0/
%

% Traducción al español por Jorge Espinoza.

%\documentclass[10pt,b5paper]{book}
\documentclass[10pt]{book}
\usepackage[width=5.5in,height=8.5in,hmarginratio=3:2,vmarginratio=1:1]{geometry}

% for some of these packages, you might have to install
% texlive-latex-extra (in Ubuntu)

\usepackage[T1]{fontenc}
\usepackage{textcomp}
\usepackage{mathpazo}
\usepackage{url}
\usepackage{fancyhdr}
\usepackage{graphicx}
\usepackage{amsmath}
\usepackage{amsthm}
%\usepackage{amssymb}
\usepackage{exercise}                        % texlive-latex-extra
\usepackage{makeidx}
\usepackage{setspace}
\usepackage{hevea}
\usepackage{upquote}
\usepackage{appendix}
\usepackage[spanish,es-noquoting]{babel}
\usepackage[utf8]{inputenc}
\usepackage[bookmarks]{hyperref}

\hyphenation{Python}

\title{Think Python}
\author{Allen B. Downey}
\newcommand{\thetitle}{Think Python: Aprende a pensar como un informático}
\newcommand{\theversion}{2da Edición, Versión 2.4.0}
\newcommand{\thedate}{}

% these styles get translated in CSS for the HTML version
\newstyle{a:link}{color:black;}
\newstyle{p+p}{margin-top:1em;margin-bottom:1em}
\newstyle{img}{border:0px}

% change the arrows
\setlinkstext
  {\imgsrc[ALT="Previous"]{back.png}}
  {\imgsrc[ALT="Up"]{up.png}}
  {\imgsrc[ALT="Next"]{next.png}}

\makeindex

\newif\ifplastex
\plastexfalse

\begin{document}

\frontmatter

% PLASTEX ONLY
\ifplastex
    \usepackage{localdef}
    \maketitle

\newcount\anchorcnt
\newcommand*{\Anchor}[1]{%
  \@bsphack%
    \Hy@GlobalStepCount\anchorcnt%
    \edef\@currentHref{anchor.\the\anchorcnt}%
    \Hy@raisedlink{\hyper@anchorstart{\@currentHref}\hyper@anchorend}%
    \M@gettitle{}\label{#1}%
    \@esphack%
}


\else
% skip the following for plastex

\newtheorem{exercise}{Ejercicio}[chapter]

% LATEXONLY

\input{latexonly}

\begin{latexonly}

\renewcommand{\blankpage}{\thispagestyle{empty} \quad \newpage}

%\blankpage
%\blankpage

% TITLE PAGES FOR LATEX VERSION

%-half title--------------------------------------------------
\thispagestyle{empty}

\begin{flushright}
\vspace*{2.0in}

\begin{spacing}{3}
{\huge Think Python}\\
{\Large Aprende a pensar como un informático}
\end{spacing}

\vspace{0.25in}

\theversion

\thedate

\vfill

\end{flushright}

%--verso------------------------------------------------------

\blankpage
\blankpage
%\clearemptydoublepage
%\pagebreak
%\thispagestyle{empty}
%\vspace*{6in}

%--title page--------------------------------------------------
\pagebreak
\thispagestyle{empty}

\begin{flushright}
\vspace*{2.0in}

\begin{spacing}{3}
{\huge Think Python}\\
{\Large Aprende a pensar como un informático}
\end{spacing}

\vspace{0.25in}

\theversion

\thedate

\vspace{1in}


{\Large
Allen Downey\\
}


\vspace{0.5in}

{\Large Green Tea Press}

{\small Needham, Massachusetts}

%\includegraphics[width=1in]{figs/logo1.pdf}
\vfill

\end{flushright}


%--copyright--------------------------------------------------
\pagebreak
\thispagestyle{empty}

{\small
Copyright \copyright ~2015 Allen Downey.


\vspace{0.2in}

\begin{flushleft}
Green Tea Press       \\
9 Washburn Ave        \\
Needham MA 02492
\end{flushleft}

Se concede permiso para copiar, distribuir y/o modificar este documento
bajo los términos de la Creative Commons Attribution-NonCommercial
3.0 Unported License, la cual está disponible en \url{http://creativecommons.org/licenses/by-nc/3.0/}.

La forma original de este libro está en código fuente de \LaTeX.  La compilación
de esta fuente de \LaTeX\ tiene el efecto de generar una representación de un libro
de texto que es independiente del dispositivo, la cual se puede convertir a otros formatos
e imprimir.

La fuente de \LaTeX\ para este libro está disponible en
\url{http://www.thinkpython.com}

Título original: {\em Think Python: How to Think Like a Computer Scientist}

Traducción de Jorge Espinoza.

\vspace{0.2in}

} % end small

\end{latexonly}


% HTMLONLY

\begin{htmlonly}

% TITLE PAGE FOR HTML VERSION

{\Large \thetitle}

{\large Allen B. Downey}

\theversion

\thedate

\setcounter{chapter}{-1}

\end{htmlonly}

\fi
% END OF THE PART WE SKIP FOR PLASTEX


\chapter{Prefacio}

\section*{La extraña historia de este libro}

En enero de 1999, me estaba preparando para enseñar un curso introductorio
de programación en Java.  Lo había enseñado tres veces y me estaba frustrando.
La tasa de fracaso en el curso era muy alta y, aún para estudiantes que
tenían éxito, el nivel general de logros era muy bajo.

Uno de los problemas que vi tenía relación con los libros.
Eran muy grandes, con demasiados detalles innecesarios sobre Java, y
no tenían suficiente orientación de alto nivel acerca de cómo programar.  Y todos
sufrían el efecto trampilla: comenzaban con facilidad,
avanzaban de manera gradual y luego, en algún lugar alrededor del Capítulo 5,
fallaban.  Los estudiantes conseguían demasiado material nuevo,
muy rápido, y yo ocupaba el resto del semestre recogiendo los pedazos.

Dos semanas antes del primer día de clases, decidí escribir mi propio libro.
Mis objetivos eran:

\begin{itemize}

\item Que sea corto: es mejor para los estudiantes leer 10 páginas que
no leer 50 páginas.

\item Tener cuidado con el vocabulario: intenté minimizar la jerga y
definir cada término en su primer uso.

\item Construir de manera gradual: para evitar trampillas, tomé los temas más difíciles
y los dividí en series de pasos pequeños.

\item Concentrarse en la programación, no en el lenguaje de programación: incluí
el mínimo subconjunto útil de Java y excluí el resto.

\end{itemize}

Necesitaba un título, así que por capricho escogí {\em Aprende a pensar como
un informático} ({\em How to Think Like a Computer Scientist}).

Mi primera versión fue áspera, pero funcionó.  Los estudiantes hicieron la
lectura y entendieron lo suficiente como para que yo pudiera ocupar el tiempo
de la clase en los temas difíciles, los temas interesantes y (más importante)
dejar a los estudiantes practicar.

Publiqué el libro bajo la Licencia de documentación libre de GNU,
permitiendo a los usuarios copiar, modificar y distribuir el libro.
\index{GNU Free Documentation License}
\index{Free Documentation License, GNU}

Lo que ocurrió después es la parte genial.  Jeff Elkner, un profesor
de escuela secundaria en Virginia, adoptó mi libro y lo tradujo a Python.
Me envió una copia de su traducción y tuve la extraña experiencia de
aprender Python leyendo mi propio libro. Como Green Tea Press,
publiqué la primera versión en Python en 2001.
\index{Elkner, Jeff}

En 2003 comencé a enseñar en el Olin College y tuve que enseñar Python
por primera vez.  El contraste con Java fue notable.
Los estudiantes se esforzaban menos, aprendían más, trabajaban en más
proyectos interesantes y generalmente se divertían mucho.
\index{Olin College}

Desde entonces he contiunado desarrollando el libro,
corrigiendo errores, mejorando algunos de los ejemplos y agregando material,
especialmente ejercicios.

El resultado es este libro, ahora con el título menos grandioso
{\em Think Python}.  Algunos de los cambios son:

\begin{itemize}

\item Agregué una sección sobre depuración al final de cada capítulo.
  Estas secciones presentan técnicas generales para encontrar y evitar
  errores de programación y advertencias sobre trampas de Python.

\item Agregué más ejercicios, que van desde pruebas cortas de comprensión
  hasta algunos proyectos sustanciales.  La mayoria de los ejercicios
  incluyen un enlace a mi solución.

\item Agregué una serie de estudios de caso: ejemplos más largos con
  ejercicios, soluciones y discusión.

\item Expandí la discusión de planes de desarrollo de programa y
  pautas de diseño básicas.

\item Agregué apéndices sobre depuración y análisis de algoritmos.

\end{itemize}

La segunda edición de {\em Think Python} tiene nuevas características:

\begin{itemize}

\item El libro y todo el código de apoyo han sido actualizados a Python 3.

\item Agregué unas pocas secciones, y más detalles en la web, para ayudar a
los principiantes a empezar a ejecutar Python en un navegador, así que no
tienes que lidiar con la instalación de Python hasta que quieras hacerlo.

\item Para el Capítulo~\ref{turtle} cambié mi propio paquete de gráfica
  tortuga, llamado Swampy, por un módulo de Python más estándar, {\tt
    turtle}, que es más fácil de instalar y más poderoso.

\item Agregué un nuevo capítulo llamado ``Trucos extra'', que introduce
algunas características adicionales de Python que no son estrictamente
necesarias, pero a veces son prácticas.

\end{itemize}

Espero que disfrutes trabajando con este libro y que te ayude a aprender
a programar y a pensar como un informático, al menos un poco.


Allen B. Downey \\

Olin College \\


\section*{Agradecimientos}

Muchas gracias a Jeff Elkner, quien
tradujo mi libro de Java a Python, lo cual comenzó este
proyecto y me presentó lo que ha resultado ser mi
lenguaje favorito.
\index{Elkner, Jeff}

Gracias también a Chris Meyers, quien contribuyó a varias secciones
de {\em How to Think Like a Computer Scientist}.
\index{Meyers, Chris}

Gracias a la Free Software Foundation por desarrollar
la Licencia de documentación libre de GNU, la cual me ayudó
a hacer posible mi colaboración con Jeff y Chris, y a Creative
Commons por la licencia que uso ahora.
\index{Licencia de documentación libre de GNU}
\index{GNU!Licencia de documentación libre}
\index{Creative Commons}

Gracias a los editores de Lulu que trabajaron en
{\em How to Think Like a Computer Scientist}.

Gracias a los editores de O'Reilly Media que trabajaron en
{\em Think Python}.

Gracias a todos los estudiantes que trabajaron con las primeras
versiones de este libro y a todos los colaboradores (nombrados
a continuación) que enviaron correcciones y sugerencias.


\section*{Lista de colaboradores}

\index{colaboradores}
Más de 100 lectores perspicaces y atentos han enviado
sugerencias y correcciones en los últimos años.  Sus
contribuciones, y su entusiasmo por este proyecto, han sido una
ayuda enorme.

Si tienes una sugerencia o corrección, por favor envía un email a
{\tt feedback@thinkpython.com}.  Si hago un cambio basado en tu
retroalimentación, te agregaré a la lista de colaboradores
(a menos que pidas que te omita).

Si incluyes al menos una parte de la oración en donde
aparece el error, eso me facilita la búsqueda.  Los números de página
y de sección también me ayudan, pero no es tan fácil trabajar con estos.
¡Gracias!

\begin{itemize}

\small
\item Lloyd Hugh Allen envió una corrección a la Sección 8.4.

\item Yvon Boulianne envió una corrección a un error semántico en
el Capítulo 5.

\item Fred Bremmer envió una corrección en la Sección 2.1.

\item Jonah Cohen escribió los scripts de Perl que convierten la
fuente de LaTeX de este libro en un hermoso HTML.

\item Michael Conlon envió una corrección gramatical en el Capítulo 2
y una mejora de estilo en el Capítulo 1, e inició la discusión
sobre los aspectos técnicos de los intérpretes.

\item Beno\^{i}t Girard envió una
corrección a un error chistoso en la Sección 5.6.

\item Courtney Gleason y Katherine Smith escribieron {\tt horsebet.py}, que
fue usado como un estudio de caso en una versión anterior del libro.  Su
programa puede encontrarse ahora en el sitio web.

\item Lee Harr envió más correcciones de las que cabrían acá en una lista
y, de hecho, debería aparecer como uno de los principales editores
del texto.

\item James Kaylin es un estudiante que usa el texto. Ha enviado
numerosas correcciones.

\item David Kershaw arregló la función incorrecta {\tt catTwice} en la Sección
3.10.

\item Eddie Lam ha enviado numerosas correcciones a los Capítulos
1, 2 y 3.
También arregló el Makefile para que cree un índice la primera vez que
se ejecute y nos ayudó a configurar un esquema de versionamiento.

\item Man-Yong Lee envió una corrección al código de ejemplo en la
Sección 2.4.

\item David Mayo advirtió que la palabra ``inconsciente''
en el Capítulo 1 necesitaba
ser cambiada a ``subconsciente''.

\item Chris McAloon envió muchas correcciones a las Secciones 3.9 y
3.10.

\item Matthew J. Moelter ha sido un colaborador por mucho tiempo que envió
numerosas correcciones y sugerencias al libro.

\item Simon Dicon Montford informó sobre una definición de función faltante y
muchos errores tipográficos en el Capítulo 3.  Además, encontró errores en la función {\tt increment}
en el Capítulo 13.

\item John Ouzts corrigió la definición de ``valor de retorno''
en el Capítulo 3.

\item Kevin Parks envió valiosos comentarios y sugerencias en cuanto a cómo
mejorar la distribución del libro.

\item David Pool envió un error tipográfico en el glosario del Capítulo 1, así como
amables palabras de aliento.

\item Michael Schmitt envió una corrección al capítulo de archivos
y excepciones.

\item Robin Shaw señaló un error en la Sección 13.1, donde la
función printTime se usó en un ejemplo sin estar definida.

\item Paul Sleigh encontró un error en el Capítulo 7 y un error en el script de Perl
de Jonah Cohen que genera HTML a partir de LaTeX.

\item Craig T. Snydal está probando el texto en un curso en la Drew
University.  Ha aportado muchas sugerencias y correcciones valiosas.

\item Ian Thomas y sus alumnos están usando el texto en un curso de
programación.  Ellos son los primeros en probar los capítulos de la segunda mitad
del libro, y han hecho numerosas correcciones y sugerencicas.

\item Keith Verheyden envió una correción en el Capítulo 3.

\item Peter Winstanley nos hizo saber sobre un error en
nuestro Latin que estuvo por mucho tiempo en el Capítulo 3.

\item Chris Wrobel hizo correcciones al código en el capítulo de
entrada/salida de archivo y excepciones.

\item Moshe Zadka ha hecho contribuciones invaluables a este proyecto.
Además de escribir el primer borrador del capítulo de Diccionarios,
proporcionó constante orientación en las primeras etapas del libro.

\item Christoph Zwerschke envió muchas correcciones y
sugerencias pedagógicas, y explicó la diferencia entre {\em gleich}
y {\em selbe}.

\item James Mayer nos envió una gran cantidad de errores ortográficos
y tipográficos, incluyendo dos en la lista de colaboradores.

\item Hayden McAfee encontró una inconsistencia potencialmente confusa
entre dos ejemplos.

\item Angel Arnal es parte de un equipo internacional de traductores
trabajando en la versión en español del texto.  Además, encontró muchos
errores en la versión en inglés.

\item Tauhidul Hoque y Lex Berezhny crearon las ilustraciones
en el Capítulo 1 y mejoraron muchas de las otras ilustraciones.

\item Dr. Michele Alzetta encontró un error en el Capítulo 8 y envió
algunos comentarios pedagógicos interesantes y sugerencias sobre Fibonacci
y Old Maid.

\item Andy Mitchell encontró un error tipográfico en el Capítulo 1 y un ejemplo
incorrecto en el Capítulo 2.

\item Kalin Harvey sugirió una aclaración en el Capítulo 7 y
captó algunos errores tipográficos.

\item Christopher P. Smith encontró muchos errores tipográficos y nos ayudó a
actualizar el libro a Python 2.2.

\item David Hutchins encontró un error tipográfico en el Prólogo.

\item Gregor Lingl está enseñando Python en una escuela secundaria en Vienna,
Austria.  Está trabajando en una traducción del libro al alemán
y encontró un par de errores malos en el Capítulo 5.

\item Julie Peters encontró un error tipográfico en el Prefacio.

\item Florin Oprina envió una mejora a {\tt makeTime},
una corrección a {\tt printTime} y un buen error tipográfico.

\item D.~J.~Webre sugirió una aclaración en el Capítulo 3.

\item Ken encontró un puñado de errores en los Capítulos 8, 9 y 11.

\item Ivo Wever encontró un error tipográfico en el Capítulo 5 y sugirió una aclaración
en el Capítulo 3.

\item Curtis Yanko sugirió una aclaración en el Capítulo 2.

\item Ben Logan envió una serie de errores tipográficos y problemas con la traducción
del libro a HTML.

\item Jason Armstrong vio la palabra que faltaba en el Capítulo 2.

\item Louis Cordier notó un lugar en el Capítulo 16 donde el código
no coincidía con el texto.

\item Brian Cain sugirió varias aclaraciones en los Capítulos 2 y 3.

\item Rob Black envió un montón de correcciones, incluyendo algunos
cambios para Python 2.2.

\item Jean-Philippe Rey de la \'{E}cole Centrale
Paris envió una serie de parches, incluyendo algunas actualizaciones para Python 2.2
y otras mejoras para pensar.

\item Jason Mader en la George Washington University hizo una serie 
de sugerencias y correcciones útiles.

\item Jan Gundtofte-Bruun nos recordó que ``a error'' es un error.

\item Abel David y Alexis Dinno nos recordaron que el plural de
``matrix'' es ``matrices'', no ``matrixes''.  Este error estuvo en el
libro por años, pero dos lectores con las mismas iniciales lo informaron en
el mismo día. Extraño.

\item Charles Thayer nos animó a deshacernos de los punto y coma que
habíamos puesto al final de algunas sentencias y a limpiar nuestro
uso de ``argumento'' y ``parámetro''.

\item Roger Sperberg advirtió sobre una lógica retorcida en el Capítulo 3.

\item Sam Bull advirtió sobre un párrafo confuso en el Capítulo 2.

\item Andrew Cheung señaló dos instancias de ``use before def''.

\item C. Corey Capel vio la palabra que faltaba en el Tercer Teorema
de la Depuración y un error tipográfico en el Capítulo 4.

\item Alessandra ayudó a aclarar alguna confusión con Turtle.

\item Wim Champagne encontró un error en un ejemplo de diccionario.

\item Douglas Wright señaló un problema con la división entera en
{\tt arco}.

\item Jared Spindor encontró algo de basura al final de una oración.

\item Lin Peiheng envió una serie de sugerencias muy útiles.

\item Ray Hagtvedt envió dos errores y un no tan error.

\item Torsten H\"{u}bsch señaló una inconsistencia en Swampy.

\item Inga Petuhhov corrigió un ejemplo en el Capítulo 14.

\item Arne Babenhauserheide envió muchas correcciones útiles.

\item Mark E. Casida es es bueno mirando palabras repetidas.

\item Scott Tyler rellenó una que faltaba.  Y envió
un montón de correcciones.

\item Gordon Shephard envió varias correcciones, todas en correos
separados.

\item Andrew Turner encontró un error en el Capítulo 8.

\item Adam Hobart arregló un problema con la división entera en {\tt arco}.

\item Daryl Hammond y Sarah Zimmerman advirtieron que mostré a
{\tt math.pi} demasiado pronto.  Y Zim vio un error tipográfico.

\item George Sass encontró un error en una sección de Depuración.

\item Brian Bingham sugirió el Ejercicio~\ref{exrotatepairs}.

\item Leah Engelbert-Fenton advirtió que usé {\tt tuple}
como un nombre de variable, contrario a mi propio consejo.  Y luego encontró
un montón de errores tipográficos y un ``use before def''.

\item Joe Funke vio un error tipográfico.

\item Chao-chao Chen encontró una inconsistencia en el ejemplo de Fibonacci.

\item Jeff Paine sabe la diferencia entre space y spam.

\item Lubos Pintes envió un error tipográfico.

\item Gregg Lind y Abigail Heithoff sugirieron el Ejercicio~\ref{checksum}.

\item Max Hailperin ha enviado una serie de correcciones y
  sugerencias.  Max es uno de los autores del extraordinario {\em
    Concrete Abstractions}, que tal vez quieras leer cuando termines
  con este libro.

\item Chotipat Pornavalai encontró un error en un mensaje de error.

\item Stanislaw Antol envió una lista de sugerencias muy útiles.

\item Eric Pashman envió una serie de correcciones para los Capítulos 4--11.

\item Miguel Azevedo encontró algunos errores tipográficos.

\item Jianhua Liu envió una larga lista de correcciones.

\item Nick King encontró una palabra que faltaba.

\item Martin Zuther envió una larga lista de sugerencias.

\item Adam Zimmerman encontró una inconsistencia en mi ejemplo
de una ``instancia'' y muchos otros errores.

\item Ratnakar Tiwari sugirió una nota al pie explicando los triángulos
degenerados.

\item Anurag Goel sugirió otra solución para \verb"es_abecedario"
y envió algunas correcciones adicionales.  Y sabe cómo
deletrear Jane Austen.

\item Kelli Kratzer vio uno de los errores tipográficos.

\item Mark Griffiths señaló un ejemplo confuso en el Capítulo 3.

\item Roydan Ongie encontró un error en mi método de Newton.

\item Patryk Wolowiec me ayudó con un problema en la versión HTML.

\item Mark Chonofsky me habló de una nueva palabra clave en Python 3.

\item Russell Coleman me ayudó con mi geometría.

\item Nam Nguyen encontró un error tipográfico y advirtió que usé el patrón Decorator
pero sin mencionar el nombre.

\item St\'{e}phane Morin envió varias correcciones y sugerencias.

\item Paul Stoop corrigió un error tipográfico en \verb+usa_solo+.

\item Eric Bronner advirtió sobre una confusión en la discusión del
orden de operaciones.

\item Alexandros Gezerlis estableció un nuevo estándar para el número y
la calidad de sugerencias que envió.  ¡Estamos profundamente agradecidos!

\item Gray Thomas distingue su derecha de su izquierda.

\item Giovanni Escobar Sosa envió una larga lista de correcciones y
sugerencias.

\item Daniel Neilson corrigió un error sobre el orden de operaciones.

\item Will McGinnis advirtió que {\tt polilinea} fue definida
de manera diferente en dos lugares.

\item Frank Hecker advirtió sobre un ejercicio que estaba subespecificado y
algunos enlaces rotos.

\item Animesh B me ayudó a limpiar un ejemplo confuso.

\item Martin Caspersen encontró dos errores de redondeo.

\item Gregor Ulm envió varias correcciones y sugerencias.

\item Dimitrios Tsirigkas me sugirió que aclarara un ejercicio.

\item Carlos Tafur envió una página de correcciones y sugerencias.

\item Martin Nordsletten encontró un error en una solución de un ejercicio.

\item Sven Hoexter advirtió que una variable con nombre {\tt input}
le hace sombra a una función incorporada.

\item Stephen Gregory advirtió el problema con {\tt cmp}
en Python 3.

\item Ishwar Bhat corrigió mi enunciado del último teorema de Fermat.

\item Andrea Zanella tradujo el libro al italiano y envió una
serie de correcciones en el camino.

\item Muchas, muchas gracias a Melissa Lewis y Luciano Ramalho por
  los excelentes comentarios y sugerencias sobre la segunda edición.

\item Gracias a Harry Percival de PythonAnywhere por su ayuda
al hacer que la gente comience ejecutando Python en un navegador.

\item Xavier Van Aubel hizo muchas correcciones útiles en la segunda
edición.

\item William Murray corrigió mi definición de división entera.

\item Per Starb{\"a}ck me puso al día sobre nuevas líneas universales en Python 3.

\item Laurent Rosenfeld y Mihaela Rotaru tradujeron este libro al francés.  En el camino, me enviaron muchas correcciones y sugerencias.

% ENDCONTRIB

Adicionalmente, las personas que vieron errores tipográficos o hicieron correcciones incluyen a
Czeslaw Czapla, Dale Wilson, Francesco Carlo Cimini,
Richard Fursa, Brian McGhie, Lokesh Kumar Makani, Matthew Shultz, Viet
Le, Victor Simeone, Lars O.D. Christensen, Swarup Sahoo, Alix Etienne,
Kuang He, Wei Huang, Karen Barber y Eric Ransom.




\end{itemize}

\normalsize
\clearemptydoublepage

% TABLE OF CONTENTS
\begin{latexonly}

\tableofcontents

\clearemptydoublepage

\end{latexonly}

% START THE BOOK
\mainmatter

\chapter{El camino del programa}

El objetivo de este libro es enseñarte a pensar como un informático. Esta forma
de pensar combina algunas de las mejores características de las matemáticas,
la ingeniería y las ciencias naturales.  Al igual que los matemáticos, los
informáticos utilizan lenguajes formales para denotar ideas (específicamente,
computaciones). Al igual que los ingenieros, diseñan cosas, ensamblando
componentes en sistemas y evaluando compensasiones entre alternativas.
Al igual que los científicos, observan el comportamiento de sistemas complejos,
a partir de hipótesis, y prueban predicciones.  \index{resolución de problemas}

La habilidad más importante de un informático es la {\bf resolución
 de problemas}.  La resolución de problemas supone la capacidad de
 formular problemas, pensar creativamente en soluciones y emitir una solución
de manera clara y precisa.  Como resultado, el proceso de aprender a
programar es una excelente oportunidad para practicar habilidades de
resolución de problemas.  Es por eso que este capítulo se llama ``El camino del
programa''.

En un nivel, aprenderás a programar, una habilidad útil por
sí misma.  En otro nivel, usarás la programación como un medio para un
fin.  Mientras avancemos, ese fin se volverá más claro.


\section{¿Qué es un programa?}

Un {\bf programa} es una secuencia de instrucciones que especifica cómo
realizar una computación.  La computación puede ser algo matemático,
tal como resolver un sistema de ecuaciones o encontrar las raíces de un
polinomio, pero también puede ser una computación simbólica, tal como
buscar y reemplazar texto en un documento, o algo gráfico, como procesar
una imagen o reproducir un video.
\index{programa}

Los detalles se ven diferentes en lenguajes diferentes, pero unas pocas
instrucciones básicas aparecen en casi todos los lenguajes:

\begin{description}

\item[entrada ({input}):] Obtener datos desde el teclado, un archivo, la red u
otro dispositivo.

\item[salida ({output}):] Mostrar datos en la pantalla, guardarlos en un
archivo, enviarlos a través de la red, etc.

\item[matemáticas:] Realizar operaciones matemáticas básicas como la suma y
la multiplicación.

\item[ejecución condicional:] Verificar ciertas condiciones y
ejecutar el código apropiado.

\item[repetición:] Realizar alguna acción repetidas veces, generalmente con
alguna variación.

\end{description}

Lo creas o no, eso es prácticamente todo lo que hay.  Cada
programa que has utilizado, no importa cuán complicado, está compuesto de
instrucciones que se parecen mucho a estas.  Así que puedes pensar en
la programación como el proceso de separar una tarea grande y compleja en
subtareas cada vez más pequeñas, hasta que cada subtarea sea lo suficientemente
simple para hacerla con una de estas instrucciones básicas.


\section{Ejecutar Python}

Uno de los desafíos de comenzar con Python es que quizás debas
instalar Python y software relacionado en tu computador.
Si conoces bien tu sistema operativo, y especialmente
si manejas bien la interfaz de línea de comandos, no tendrás
problemas instalando Python.  Sin embargo, puede ser doloroso para
los principiantes aprender sobre administración del sistema y programación al
mismo tiempo.
\index{ejecutar Python}
\index{Python, ejecutar}

Para evitar ese problema, recomiendo que comiences ejecutando Python
en un navegador.  Después, cuando te acostumbres a Python, haré
sugerencias para instalar Python en tu computador.
\index{Python en un navegador}

Hay una serie de páginas web que puedes utilizar para ejecutar Python.  Si ya
tienes una favorita, ve y úsala.  De otra manera, recomiendo
PythonAnywhere.  En \url{http://tinyurl.com/thinkpython2e}
proporciono instrucciones detalladas para comenzar.
\index{PythonAnywhere}

Hay dos versiones de Python, llamadas Python 2 y Python 3.
Son muy similares, así que si aprendes una, es fácil cambiar
a la otra.  De hecho, hay solo unas pocas diferencias que encontrarás
como principiante.
Este libro está escrito para Python 3, pero incluyo algunas notas
sobre Python 2.
\index{Python 2}

El {\bf intérprete} de Python es un programa que lee y ejecuta
código de Python.  Dependiendo de tu entorno, puedes iniciar el intérprete
haciendo clic en un ícono o escribiendo {\tt python} en una
línea de comandos.
Cuando se inicia, deberías ver una salida como esta:
\index{interprete@intérprete}

\begin{verbatim}
Python 3.4.0 (default, Jun 19 2015, 14:20:21)
[GCC 4.8.2] on linux
Type "help", "copyright", "credits" or "license" for more information.
>>>
\end{verbatim}
%
Las primeras tres líneas contienen información acerca del intérprete
y el sistema operativo en el cual se está ejecutando, así que puede ser
diferente para ti.  Sin emgargo, deberías verificar que el número de versión, que
en este ejemplo es {\tt 3.4.0}, comience con 3, lo cual indica que estás
ejecutando Python 3.  Si comienza con 2, estás ejecutando
(lo adivinaste) Python 2.

La última línea es un {\bf prompt} que indica que el intérprete está listo
para que introduzcas código.
Si escribes una línea de código y presionas la tecla Enter, el intérprete
muestra el resultado:
\index{prompt}

\begin{verbatim}
>>> 1 + 1
    2
\end{verbatim}
%
Ahora estás listo para comenzar.
A partir de aquí, doy por sentado que sabes cómo iniciar el intérprete de Python
y ejecutar código.


\section{El primer programa}
\label{hello}
\index{Hola, mundo}

Tradicionalmente, el primer programa que escribes en un nuevo lenguaje
de programación se llama ``Hola, mundo'' porque todo lo que hace es mostrar
las palabras ``Hola, mundo''.  En Python, se ve así:

\begin{verbatim}
>>> print('Hola, mundo')
\end{verbatim}
%
Este es un ejemplo de una {\bf sentencia print}, aunque
en realidad no imprime nada en papel.  Esta sentencia muestra un resultado en la
pantalla.  En este caso, el resultado es la frase

\begin{verbatim}
Hola, mundo
\end{verbatim}
%
Las comillas en el programa marcan el principio y el final
del texto a visualizar: estas no aparecen en el resultado.
\index{comillas}
\index{sentencia!print}
\index{print, sentencia}

Los paréntesis indican que {\tt print} es una función.  Llegaremos
a las funciones en el Capítulo~\ref{funcchap}.
\index{función} \index{función!print}

En Python 2, la sentencia print es un poco diferente; no es
una función, así que no utiliza paréntesis.
\index{Python 2}

\begin{verbatim}
>>> print 'Hola, mundo'
\end{verbatim}
%
Esta distinción pronto tendrá más sentido, pero eso es suficiente para
comenzar.


\section{Operadores aritméticos}
\index{aritmético, operador}
\index{operador!aritmético}

Después de ``Hola, mundo'', el siguiente paso es la aritmética.  Python
proporciona {\bf operadores}, los cuales son símbolos especiales que
representan computaciones como la suma y la multiplicación.

Los operadores {\tt +}, {\tt -} y {\tt *} realizan sumas,
restas y multiplicaciones, como en los siguientes ejemplos:

\begin{verbatim}
>>> 40 + 2
    42
>>> 43 - 1
    42
>>> 6 * 7
    42
\end{verbatim}
%
El operador {\tt /} realiza divisiones:

\begin{verbatim}
>>> 84 / 2
    42.0
\end{verbatim}
%
Puedes preguntarte por qué el resultado es {\tt 42.0} en lugar de {\tt 42}.
Lo explicaré en la sección siguiente.

Finalmente, el operador {\tt **} realiza potenciaciones; es decir,
eleva un número a una potencia:

\begin{verbatim}
>>> 6**2 + 6
    42
\end{verbatim}
%
En algunos otros lenguajes, \verb"^" se utiliza para potenciación, pero
en Python es un operador bit a bit llamado XOR.  Si no conoces
los operadores bit a bit, el resultado te sorprenderá:

\begin{verbatim}
>>> 6 ^ 2
    4
\end{verbatim}
%
No cubriré
operadores bit a bit en este libro, pero puedes leer sobre
estos en \url{http://wiki.python.org/moin/BitwiseOperators}.
\index{operador!bit a bit}
\index{bit a bit, operador}


\section{Valores y tipos}
\index{valor}
\index{tipo}
\index{cadena}\index{string}

Un {\bf valor} es una de las cosas básicas con las que funciona un programa,
como una letra o un número.  Algunos valores que hemos visto hasta ahora son
{\tt 2}, {\tt 42.0} y \verb"'Hola, mundo'".

Estos valores pertenecen a diferentes {\bf tipos}: {\tt 2} es un {\bf entero}
(en inglés, {\em integer}), {\tt 42.0} es un {\bf número de coma flotante} (en
inglés, {\em floating-point number}), y \verb"'Hola, mundo'" es una {\bf
cadena} (en inglés, {\em string}), llamada así porque las letras que contiene están unidas.
\index{entero}
\index{coma flotante}

Si no sabes bien qué tipo tiene algún valor, el intérprete te lo
puede decir:

\begin{verbatim}
>>> type(2)
    <class 'int'>
>>> type(42.0)
    <class 'float'>
>>> type('Hola, mundo')
    <class 'str'>
\end{verbatim}
%
En estos resultados, la palabra ``class'' se utiliza en el sentido de
una categoría: un tipo es una categoría de valores.
\index{class}

Evidentemente, los enteros pertenecen al tipo {\tt int},
las cadenas pertenecen al tipo {\tt str} y los números de
coma flotante pertenecen al tipo {\tt float}.
\index{tipo}
\index{tipo!string}
\index{tipo!cadena}
\index{str!tipo}
\index{tipo!int}
\index{int!tipo}
\index{tipo!float}
\index{float, tipo}

¿Qué pasa con los valores como \verb"'2'" y \verb"'42.0'"?
Se ven como números, pero están en comillas como
cadenas.
\index{comillas}

\begin{verbatim}
>>> type('2')
    <class 'str'>
>>> type('42.0')
    <class 'str'>
\end{verbatim}
%
Son cadenas.

Cuando escribas un entero grande, tal vez sientas la tentación de utilizar comas
entre grupos de dígitos, como en {\tt 1,000,000}.  No es un
{\em entero} legal en Python, pero es legal:

\begin{verbatim}
>>> 1,000,000
    (1, 0, 0)
\end{verbatim}
%
¡Eso no es lo que esperábamos en absoluto!  Python interpreta {\tt
  1,000,000} como una secuencia de enteros separados por comas.  Aprenderemos
más sobre este tipo de secuencia más adelante.
\index{secuencia}

%This is the first example we have seen of a semantic error: the code
%runs without producing an error message, but it doesn't do the
%``right'' thing.
%\index{semantic error}
%\index{error!semantic}
%\index{error message}
% TODO: use this as an example of a semantic error later



\section{Lenguajes formales y lenguajes naturales}
\index{lenguaje!formal}
\index{lenguaje!natural}
\index{formal, lenguaje}
\index{natural, lenguaje}

Los {\bf lenguajes naturales} son los lenguajes que hablan las personas,
tales como el inglés, el español y el francés.  No fueron diseñados
por las personas (aunque las personas traten de imponerles algo de orden):
evolucionan de manera natural.

Los {\bf lenguajes formales} son lenguajes que están diseñados por personas
para aplicaciones específicas.  Por ejemplo, la notación que utilizan los
matemáticos es un lenguaje formal que es particularmente bueno al denotar
relaciones entre números y símbolos.  Los químicos utilizan un lenguaje formal
para representar la estructura química de las moléculas.  Y
más importante:

\begin{quote}
{\bf Los lenguajes de programación son lenguajes formales que han sido
diseñados para expresar computaciones.}
\end{quote}

Los lenguajes formales tienden a tener reglas de {\bf sintaxis} que
gobiernan la estructura de las sentencias.
Por ejemplo, en matemáticas la sentencia
$3 + 3 = 6$ tiene sintaxis correcta, pero
$3 + = 3 \$ 6$ no la tiene.  En química,
$H_2O$ es una fórmula sintácticamente correcta, pero $_2Zz$ no.
\index{sintaxis}

Hay dos tipos de reglas de sintaxis, relacionadas con los {\bf tokens} y la
estructura.  Los tokens son los elementos básicos del lenguaje, tales como
palabras, números y elementos químicos.  Uno de los problemas con
$3 += 3 \$ 6$ es que \( \$ \) no es un token legal en matemáticas
(al menos por lo que sé).  Del mismo modo, $_2Zz$ no es legal porque
no hay ningún elemento con la abreviatura $Zz$.
\index{token}
\index{estructura}

El segundo tipo de regla de sintaxis tiene relación con la manera en que los
tokens están combinados.  La ecuación $3 +/ 3$ es ilegal porque aunque $+$
y $/$ son tokens legales, no puedes tener uno justo después del otro.
Del mismo modo, en una fórmula química el subíndice viene después del nombre
del elemento, no antes.

Esta es un@ oración en e\$paño$/$ bien estructurada
con t*kens inválidos.  Esta oración todos los tokens válidos
tiene, pero estructura inválida presenta.

Cuando lees una oración en español o una sentencia en un lenguaje
formal, tienes que descifrar la estructura
(aunque en un lenguaje natural lo haces de manera subconsciente).  Este
proceso se llama {\bf análisis sintáctico} (en inglés, {\em parsing}).
\index{analisis sintactico@análisis sintáctico}\index{parse}

Aunque los lenguajes formales y los lenguajes naturales tienen muchas características en
común ---tokens, estructura y sintaxis--- existen algunas
diferencias:
\index{ambigüedad}
\index{redundancia}
\index{literalidad}

\begin{description}

\item[ambigüedad:] Los lenguajes naturales están llenos de ambigüedad, con la cual
las personas lidian mediante el uso de pistas contextuales y otra información.
Los lenguajes formales están diseñados para ser casi o completamente inequívocos,
lo cual significa que cualquier sentencia tiene exactamente un solo significado,
sin importar el contexto.

\item[redundancia:] Para compensar la ambigüedad y reducir
los malentendidos, los lenguajes naturales emplean mucha
redundancia.  Como consecuencia, a menudo son verbosos.  Los lenguajes formales
son menos redundantes y más concisos.

\item[literalidad:] Los lenguajes naturales están llenos de modismo y metáfora.
Si yo digo ``Cayó en la cuenta'', probablemente no hay ninguna cuenta ni
nadie cayendo (este modismo significa que alguien entendió algo
después de un periodo de confusión).  Los lenguajes formales
expresan exactamente lo que dicen.

\end{description}

Debido a que todos crecemos hablando lenguajes naturales, a veces es
difícil adaptarse a los lenguajes formales.  La diferencia entre lenguaje formal
y lenguaje natural es como la diferencia entre poesía y prosa, pero
más aún: \index{poesía} \index{prosa}

\begin{description}

\item[Poesía:] Las palabras se utilizan por su sonido tanto como por
su significado y todo el poema junto crea un efecto o
respuesta emocional.  La ambigüedad no solo es común sino a menudo
deliberada.

\item[Prosa:] El significado literal de las palabras es más importante
y la estructura aporta más significado.  La prosa es más susceptible de
análisis que la poesía pero todavía a menudo ambigua.

\item[Programas:] El significado de un programa de computador es inequívoco
y literal, y puede entenderse enteramente analizando los
tokens y su estructura.

\end{description}

Los lenguajes formales son más densos
que los lenguajes naturales, así que leerlos requiere más tiempo.  Además, la
estructura es importante, por lo que no siempre es mejor leer
de arriba a abajo y de izquierda a derecha.  En su lugar, hay que aprender a analizar
sintácticamente el programa en tu cabeza, identificar los tokens e interpretar
la estructura.  Finalmente, los detalles importan.  Pequeños errores en
la ortografía y puntuación, que puedes cometer
con los lenguajes naturales, pueden hacer una gran diferencia en un lenguaje
formal.


\section{Depuración}
\index{depuración}

Los programadores cometen errores.  Por razones caprichosas, los {\bf errores
de programación} se llaman {\em bugs} y el proceso de localizarlos se llama
{\bf depuración} (en inglés, {\em debugging}).
\index{depuración}\index{debugging}
\index{bug}\index{error!de programación}

La programación, y especialmente la depuración, a veces provoca emociones
fuertes.  Si estás luchando con un error de programación difícil, podrías
sentir ira, desánimo o vergüenza.

Hay evidencia de que las personas responden naturalmente a los computadores
como si estos fueran personas.  Cuando funcionan bien, pensamos en ellos como
compañeros de equipo, y cuando son obstinados o rudos, respondemos a ellos
de la misma manera que respondemos a las personas rudas
y obstinadas (Reeves and Nass, {\it The Media
    Equation: How People Treat Computers, Television, and New Media
    Like Real People and Places}).
\index{depuración!respuesta emocional}
\index{depuración emocional}

Prepararse para estas reacciones podría ayudarte a lidiar con ellas.
Un enfoque es pensar en el computador como un empleado con
ciertas fortalezas, como la velocidad y la precisión, y
debilidades particulares, como la falta de empatía y la incapacidad
para comprender el panorama general.

Tu trabajo es ser un buen jefe: encontrar maneras de aprovechar las
fortalezas y mitigar las debilidades.  Y encontrar maneras
de utilizar tus emociones para abordar el problema,
sin dejar que tus reacciones interfieran en tu capacidad
de trabajar eficazmente.

Aprender a depurar puede ser frustrante, pero es una habilidad valiosa
que es útil para muchas actividades más allá de la programación.  Al final
de cada capítulo hay una sección, como esta,
con mis sugerencias para la depuración.  ¡Espero que ayuden!


\section{Glosario}

\begin{description}

\item[resolución de problemas:]  El proceso de formular un problema, encontrar
una solución y expresarla.
\index{resolución de problemas}

\item[lenguaje de alto nivel:]  Un lenguaje de programación como Python que
está diseñado para que los humanos puedan leer y escribir fácilmente.
\index{lenguaje!de alto nivel}

\item[lenguaje de bajo nivel:]  Un lenguaje de programación que está diseñado
para que sea fácil de ejecutar por un computador; también llamado ``lenguaje de máquina'' o
``lenguaje ensamblador''.
\index{lenguaje!de bajo nivel}

\item[portabilidad:]  Una propiedad de un programa que puede ejecutarse en más
de un tipo de computador.
\index{portabilidad}

\item[intérprete:]  Un programa que lee otro programa y lo
ejecuta.
\index{interprete@intérprete}

\item[prompt:] Caracteres mostrados por el intérprete que indican
que está listo para recibir la entrada del usuario.
\index{prompt}

\item[programa:] Un conjunto de instrucciones que especifica una computación.
\index{programa}

\item[sentencia print:]  Una instrucción que hace que el intérprete de Python
muestre un valor en la pantalla.
\index{print, sentencia}
\index{sentencia!print}

\item[operador:]  Un símbolo especial que representa una computación simple
como suma, multiplicación o concatenación de cadenas.
\index{operador}

\item[valor:]  Una de las unidades básicas de datos, como un número o una cadena,
que manipula un programa.
\index{valor}

\item[tipo:] Una categoría de valores.  Los tipos que hemos visto hasta ahora
son los enteros (tipo {\tt int}), los números de coma flotante (tipo {\tt
float}) y cadenas (tipo {\tt str}).
\index{tipo}

\item[entero:] Un tipo que representa números enteros.
\index{entero}

\item[coma flotante:] Un tipo que representa números con partes
fraccionarias.
\index{coma flotante}

\item[cadena:] Un tipo que representa secuencias de caracteres.
\index{cadena}

\item[lenguaje natural:]  Cualquiera de los lenguajes que hablan las personas y
que evolucionaron de manera natural.
\index{lenguaje!natural}

\item[lenguaje formal:]  Cualquiera de los lenguajes que las personas han diseñado
para propósitos específicos, tales como representar ideas matemáticas o
programas de computador; todos los lenguajes de programación son lenguajes formales.
\index{lenguaje!formal}

\item[token:]  Uno de los elementos básicos de la estructura sintáctica de
un programa, análogo a una palabra en un lenguaje natural.
\index{token}

\item[sintaxis:] Las reglas que rigen la estructura de un programa.
\index{sintaxis}

\item[análisis sintáctico ({\em parse}):] Examinar un programa y analizar la
estructura sintáctica.\index{parse}\index{analisis sintactico@análisis sintáctico}

\item[error de programación ({\em bug}):] Un error en un programa.
\index{bug}\index{error!de programación}

\item[depuración:] El proceso de encontrar y corregir
errores de programación.\index{depuración}

\end{description}


\section{Ejercicios}

\begin{exercise}

Es una buena idea leer este libro en frente de un computador para que puedas
probar los ejemplos mientras avanzas.

Cada vez que experimentes con una nueva característica, deberías intentar
cometer errores.  Por ejemplo, en el programa ``Hola, mundo'',
¿qué ocurre si omites una de las comillas?  ¿Y
si omites ambas?  ¿Qué ocurre si escribes {\tt print} de manera incorrecta?
\index{mensaje de error}

Este tipo de experimento te ayuda a recordar lo que leíste; también
te ayuda cuando estás programando porque logras saber lo que significan los
mensajes de error.  Es mejor cometer errores ahora y a propósito que
después y de manera accidental.

\begin{enumerate}

\item En una sentencia print, ¿qué ocurre si omites uno
de los paréntesis, o ambos?

\item Si estás intentando imprimir una cadena con print, ¿qué ocurre si
omites una de las comillas, o ambas?

\item Puedes utilizar un signo menos para hacer un número negativo como
{\tt -2}.  ¿Qué ocurre si pones un signo más antes de un número?
¿Qué pasa con {\tt 2++2}?

\item En notación matemática, los ceros a la izquierda están bien, como en {\tt 09}.
¿Qué ocurre si intentas esto en Python?  ¿Qué pasa con {\tt 011}?

\item ¿Qué ocurre si tienes dos valores sin operador
entre ellos?

\end{enumerate}

\end{exercise}



\begin{exercise}

Inicia el intérprete de Python y úsalo como una calculadora.

\begin{enumerate}

\item ¿Cuántos segundos hay en 42 minutos con 42 segundos?

\item ¿Cuántas millas hay en 10 kilómetros?  Pista: hay 1.61
  kilómetros en una milla.

\item Si corres una carrera de 10 kilómetros en 42 minutos con 42 segundos,
  ¿Cuál es tu ritmo promedio (tiempo por cada milla en minutos y segundos)?  ¿Cuál
  es tu rapidez promedio en millas por hora?

\index{calculadora}
\index{ritmo de carrera}

\end{enumerate}

\end{exercise}




\chapter{Variables, expresiones y sentencias}

Una de las características más poderosas de un lenguaje de programación es la
posibilidad de manipular {\bf variables}.  Una variable es un nombre que
se refiere a un valor.
\index{variable}


\section{Sentencias de asignación}
\label{variables}
\index{sentencia!de asignación}
\index{asignación!sentencia de}

Una {\bf sentencia de asignación} crea una nueva variable y le da
un valor:

\begin{verbatim}
>>> mensaje = 'Y ahora algo completamente diferente'
>>> n = 17
>>> pi = 3.1415926535897932
\end{verbatim}
%
Este ejemplo hace tres asignaciones.  La primera asigna una cadena
a una nueva variable llamada {\tt mensaje};
la segunda pone al entero {\tt 17} en {\tt n}; la tercera
asigna el valor (aproximado) de $\pi$ a {\tt pi}.
\index{diagrama!de estado}
\index{estado, diagrama de}

Una forma común de representar en papel las variables es escribir el nombre con
una flecha apuntando a su valor.  Este tipo de figura se
llama {\bf diagrama de estado} porque muestra en qué estado está cada una
de las variables (piénsalo como el estado mental de la variable).
La Figura~\ref{fig.state2} muestra el resultado del ejemplo anterior.

\begin{figure}
\centerline
{\includegraphics[scale=0.8]{figs/state2.pdf}}
\caption{Diagrama de estado.}
\label{fig.state2}
\end{figure}



\section{Nombres de variable}
\index{variable}

Los programadores generalmente escogen nombres para sus variables que
sean significativos: documentan para qué se utiliza la variable.

Los nombres de variable pueden ser tan largos como quieras.  Pueden contener
tanto letras como números, pero no pueden comenzar con un número.
Es legal utilizar letras mayúsculas, pero es convencional
utilizar solo minúsculas para los nombres de variables.

El guión bajo, \verb"_", puede aparecer en un nombre.
A menudo se utiliza en nombres con varias palabras, tales como
\verb"tu_nombre" o \verb"velocidad_de_golondrina_sin_carga".
\index{guión bajo}

Si le das un nombre ilegal a una variable, obtendrás un error de sintaxis:

\begin{verbatim}
>>> 76trombones = 'gran desfile'
    SyntaxError: invalid syntax
>>> mas@ = 1000000
    SyntaxError: invalid syntax
>>> class = 'Cimología teórica avanzada'
    SyntaxError: invalid syntax
\end{verbatim}
%
{\tt 76trombones} es ilegal porque comienza con un número.
{\tt mas@} es ilegal porque contiene un carácter ilegal, {\tt
@}.  Sin embargo, ¿qué tiene de malo {\tt class}?

Resulta que {\tt class} es una de las {\bf palabras clave} de Python.  El
intérprete utiliza las palabras clave para reconocer la estructura del programa
y no se pueden utilizar como nombres de variable.
\index{keyword}

Python 3 tiene estas palabras clave:

\begin{verbatim}
False      class      finally    is         return
None       continue   for        lambda     try
True       def        from       nonlocal   while
and        del        global     not        with
as         elif       if         or         yield
assert     else       import     pass
break      except     in         raise
\end{verbatim}
%
No tienes que memorizar esta lista.  En la mayoría de los entornos de desarrollo,
las palabras clave se muestran con un color diferente; si intentas utilizar una
como nombre de variable, lo sabrás.


\section{Expresiones y sentencias}

Una {\bf expresión} es una combinación de valores, variables y operadores.
Un valor por sí mismo es considerado una expresión, y por consigiuente lo es
una variable, así que las siguientes son todas expresiones legales:
\index{expresión}

\begin{verbatim}
>>> 42
    42
>>> n
    17
>>> n + 25
    42
\end{verbatim}
%
Cuando escribes una expresión en el prompt, el intérprete
la {\bf evalúa}, lo cual significa que encuentra el valor de
la expresión.
En este ejemplo, {\tt n} tiene el valor 17 y
{\tt n + 25} tiene el valor 42.
\index{evaluar}

Una {\bf sentencia} es una unidad de código que tiene un efecto, como
crear una variable o mostrar un valor.
\index{sentencia}

\begin{verbatim}
>>> n = 17
>>> print(n)
\end{verbatim}
%
La primera línea es una sentencia de asignación que le da un valor a
{\tt n}.  La segunda línea es una sentencia print que muestra el
valor de {\tt n}.

Cuando escribes una sentencia, el intérprete la {\bf ejecuta},
lo cual significa que hace lo que dice la sentencia.  En general,
las sentencias no tienen valores.
\index{ejecutar}


\section{Modo Script}

Hasta ahora hemos ejecutado Python en {\bf modo interactivo}, lo cual
significa que interactúas directamente con el intérprete.
El modo interactivo es una buena manera de comenzar,
pero si estás trabajando con más que unas pocas líneas de código, puede ser
algo torpe.
\index{modo interactivo}

La alternativa es guardar código en un archivo llamado {\bf script} y
entonces utilizar el intérprete en {\bf modo script} para ejecutar el script.  Por
convención, los scripts de Python tienen nombres que terminan con {\tt .py}.
\index{script}
\index{modo script}

Si sabes cómo crear y ejecutar un script en tu computador, estás
listo para seguir.  De lo contrario, recomiendo de nuevo utilizar PythonAnywhere.
He publicado instrucciones para utilizarlo en modo script en
\url{http://tinyurl.com/thinkpython2e}.

Debido a que Python proporciona ambos modos,
puedes probar pedazos de código en modo interactivo antes de ponerlos
en un script.  Sin embargo, hay diferencias entre el modo interactivo
y el modo script que pueden confundir.
\index{modo interactivo}
\index{modo script}

Por ejemplo, si utilizas Python como una calculadora, puedes escribir

\begin{verbatim}
>>> millas = 26.2
>>> millas * 1.61
    42.182
\end{verbatim}

La primera línea asigna un valor a {\tt millas}, pero no tiene un efecto
visible.  La segunda línea es una expresión, por lo cual el
intérprete la evalúa y muestra el resultado.  Resulta que una
maratón es de unos 42 kilómetros.

Sin embargo, si escribes el mismo código dentro de un script y lo ejecutas, no obtienes
ninguna salida.
En modo script, una expresión por sí misma no tiene
efecto visible.  Python evalúa la expresión, pero no
muestra el resultado.
Para mostrar el resultado, necesitas una sentencia {\tt print} como esta:

\begin{verbatim}
millas = 26.2
print(millas * 1.61)
\end{verbatim}

Este comportamiento puede confundir al principio.
Para comprobar tu comprensión, escribe las siguientes sentencias en el
intérprete de Python y mira lo que hacen:

\begin{verbatim}
5
x = 5
x + 1
\end{verbatim}

Ahora pon las mismas sentencias en un script y ejecútalo.  ¿Cuál
es la salida?  Modifica el script transformando cada
expresión en una sentencia print y luego ejecútalo de nuevo.

\section{Orden de operaciones}
\index{orden de operaciones}
\index{PEMDAS}

Cuando una expresión contiene más de un operador, el orden de
evaluación depende del {\bf orden de operaciones}.  Para
operadores matemáticos, Python sigue la convención matemática.
El acrónimo {\bf PEMDAS} es una manera útil de
recordar las reglas:

\begin{itemize}

\item Los {\bf P}aréntesis tienen la mayor prioridad y se pueden utilizar 
para forzar una expresión a evaluar en el orden que tú quieras. Dado que
las expresiones en paréntesis se evalúan primero, {\tt 2 * (3-1)} es 4,
y {\tt (1+1)**(5-2)} es 8. También puedes utilizar paréntesis para hacer una
expresión más fácil de leer, como en {\tt (minuto * 100) / 60}, incluso
si no cambia el resultado.

\item Los {\bf E}xponentes de potencias tienen la siguiente proridad, así que
{\tt 1 + 2**3} es 9, no 27, y {\tt 2 * 3**2} es 18, no 36.

\item La {\bf M}ultiplicación y la {\bf D}ivisión tienen mayor prioridad
	que la {\bf A}dición (suma) y la {\bf S}ustracción (resta).  Así que {\tt 2*3-1} es 5, no
  4, y {\tt 6+4/2} es 8, no 5.

\item Los operadores con la misma prioridad se evalúan de izquierda a
  derecha (excepto la potenciación).  Así que en la expresión {\tt grados /
    2 * pi}, la división ocurre primero y el resultado se multiplica
  por {\tt pi}.  Para dividir por $2 \pi$, puedes utilizar paréntesis o escribir
  {\tt grados / 2 / pi}.

\end{itemize}

Yo no me esfuerzo mucho en recordar la prioridad de los
operadores. Si no puedo saber mirando la expresión, utilizo
paréntesis para hacerlo obvio.


\section{Operaciones con cadenas}
\index{operación con cadena}
\index{cadena!operación con}

En general, no puedes realizar operaciones matemáticas con cadenas, incluso
si las cadenas parecen números, así que las siguientes expresiones son ilegales:

\begin{verbatim}
'comida'-'china'    'huevos'/'fácil'    'la tercera'*'la vencida'
\end{verbatim}
%
Sin embargo, hay dos excepciones: {\tt +} y {\tt *}.

El operador {\tt +} realiza una {\bf concatenación}, lo cual significa que
une las cadenas enlazándolas de extremo a extremo.  Por ejemplo:
\index{concatenación}

\begin{verbatim}
>>> primero = 'curruca'
>>> segundo = 'garganta'
>>> primero + segundo
    currucagarganta
\end{verbatim}
%
El operador {\tt *} también funciona en cadenas: hace repetición.
Por ejemplo, \verb"'Spam'*3" es \verb"'SpamSpamSpam'".  Si uno de los
valores es una cadena, el otro tiene que ser un entero.

Este uso de {\tt +} y {\tt *} tiene sentido por
analogía con la suma y la multiplicación.  Tal como {\tt 4*3} es
equivalente a {\tt 4+4+4}, esperamos que \verb"'Spam'*3" sea lo mismo que
\verb"'Spam'+'Spam'+'Spam'", y lo es.  Por otro lado, hay una
manera significativa en la que la concatenación y la repetición son
diferentes de la suma y multiplicación de enteros.
¿Puedes pensar en una propiedad que tiene la suma
que la concatenación no tiene?
\index{conmutatividad}


\section{Comentarios}
\index{comentario}

A medida que los programas se hacen más grandes y complicados, se vuelven más difíciles
de leer.  Los lenguajes formales son densos y a menudo es difícil
mirar un pedazo de código y descifrar lo que hace, o por qué lo hace.

Por esta razón, es una buena idea añadir notas a tus programas para explicar
en lenguaje natural lo que hace el programa.  Estas notas se llaman
{\bf comentarios}, y comienzan con el símbolo \verb"#":

\begin{verbatim}
# calcular el porcentaje de la hora que ha transcurrido
porcentaje = (minuto * 100) / 60
\end{verbatim}
%
En este caso, el comentario aparece en una línea por sí sola.  Puedes también poner
comentarios al final de una línea:

\begin{verbatim}
porcentaje = (minuto * 100) / 60     # porcentaje de una hora
\end{verbatim}
%
Todo desde el {\tt \#} hasta el final de la línea es ignorado: no
tiene efecto en la ejecución del programa.

Los comentarios son más útiles cuando documentan características del
código que no son obvias.  Es razonable suponer que el lector puede descifrar
{\em qué} hace el código; es más útil explicar {\em por qué}.

Este comentario es redundante con el código e inútil:

\begin{verbatim}
v = 5     # asigna 5 a v
\end{verbatim}
%
Este comentario contiene información útil que no está en el código:

\begin{verbatim}
v = 5     # velocidad en metros/segundos.
\end{verbatim}
%
Los buenos nombres de variable pueden reducir la necesidad de comentarios, pero
los nombres largos pueden hacer que las expresiones complejas sean difíciles de leer,
así que hay una compensación.


\section{Depuración}
\index{depuración}
\index{bug}\index{error}

En un programa pueden ocurrir tres tipos de errores: errores de sintaxis, errores
de tiempo de ejecución y errores semánticos.  Es útil
distinguir entre ellos para rastrearlos de manera más rápida.

\begin{description}

\item[Error de sintaxis:] La ``sintaxis'' se refiere a la estructura de un programa
  y las reglas sobre esa estructura.  Por ejemplo, los paréntesis tienen
  que venir en pares que coincidan, por lo que {\tt (1 + 2)} es legal, pero {\tt 8)}
  es un {\bf error de sintaxis}.  \index{error!de sintaxis} \index{sintaxis!error de}
  \index{mensaje de error}
\index{sintaxis}

Si hay un error de sintaxis
en cualquier lugar de tu programa, Python muestra un mensaje de error y se detiene,
y no podrás ejecutar el programa.  Durante las primeras
semanas de tu carrera de programación, podrías pasar mucho
tiempo rastreando errores de sintaxis.  A medida que ganes experiencia,
cometerás menos errores y los encontrarás más rápido.


\item[Error de tiempo de ejecución:] El segundo tipo de error es el error de tiempo de ejecución,
  llamado así porque el error no aparece hasta después de que el programa ha
  comenzado a ejecutarse.  Estos errores también se llaman {\bf excepciones}
  porque usualmente indican que algo excepcional (y malo)
  ha ocurrido.  \index{error!de tiempo de ejecución} \index{tiempo de ejecución, error de}
  \index{excepción} \index{lenguaje!seguro} \index{seguro, lenguaje}

Los errores de tiempo de ejecución son poco comunes en programas simples que verás en los
primeros capítulos, así que puede pasar un tiempo antes de que encuentres uno.


\item[Error semántico:] El tercer tipo de error es ``semántico'', lo cual
  significa que se relaciona con el significado.  Si hay un error semántico en tu
  programa, se ejecutará sin generar mensajes de error, pero
  no hará lo correcto.  Hará otra cosa.  Específicamente,
  hará lo que le dijiste que hiciera.  \index{error!semántico}
  \index{semántico, error} \index{mensaje de error}

Identificar errores semánticos puede ser complicado porque requiere que trabajes
hacia atrás mirando la salida del programa e intentando averiguar
lo que hace.

\end{description}


\section{Glosario}

\begin{description}

\item[variable:]  Un nombre que se refiere a un valor.
\index{variable}

\item[asignación:]  Una sentencia que asigna un valor a una variable.
\index{asignación}

\item[diagrama de estado:]  Una representación gráfica de un conjunto de variables y los
valores a los cuales se refieren.
\index{diagrama!de estado}

\item[palabra clave:]  Una palabra reservada que se utiliza como parte de la sintaxis
de un programa; no puedes utilizar palabras claves tales como {\tt if}, {\tt  def} y {\tt while} como
nombres de variables.
\index{palabra clave}

\item[operando:]  Uno de los valores en los cuales opera un operador.
\index{operando}

\item[expresión:]  Una combinación de variables, operadores y valores que
representa un resultado único.
\index{expresión}

\item[evaluar:]  Simplificar una expresión realizando las operaciones
para obtener un valor único.

\item[sentencia:]  Una sección de código que representa un comando o acción.  Hasta
aquí, las sentencias que hemos visto son asignaciones y sentencias print.
\index{sentencia}

\item[ejecutar:]  Llevar a efecto una sentencia y hacer lo que dice.
\index{ejecutar}

\item[modo interactivo:] Una manera de utilizar el intérprete de Python
escribiendo código en el prompt.
\index{modo interactivo}

\item[modo script:] Una manera de utilizar el intérprete de Python para leer
código de un script y ejecutarlo.
\index{modo script}

\item[script:] Un programa almacenado en un archivo.
\index{script}

\item[orden de operaciones:]  Reglas que gobiernan el orden en el cual
se evalúan las expresiones que involucran múltiples operadores y operandos.
\index{orden de operaciones}

\item[concatenar:]  Unir dos operandos de extremo a extremo.
\index{concatenación}

\item[comentario:]  Información en un programa que está destinada a otros
programadores (o cualquiera que lea el código fuente) y no tiene efecto en la
ejecución del programa.
\index{comentario}

\item[error de sintaxis:]  Un error en un programa que hace imposible reconocer
la estructura sintáctica (y por lo tanto imposible de interpretar).
\index{error!de sintaxis}

\item[excepción:]  Un error que es detectado mientras el programa se ejecuta.
\index{excepción}

\item[semántica:]  El significado de un programa.
\index{semántica}

\item[error semántico:]   Un error en un programa que supone hacer algo
distinto a lo que el programador pretendía.
\index{error!semántico}

\end{description}


\section{Ejercicios}

\begin{exercise}

Repitiendo mi consejo del capítulo anterior, cuando aprendas
una nueva característica, deberías intentar probarla en modo interactivo y cometer
errores a propósito para ver qué sale mal.

\begin{itemize}

\item Hemos visto que {\tt n = 42} es legal.  ¿Qué pasa con {\tt 42 = n}?

\item ¿Qué ocurre con {\tt x = y = 1}?

\item En algunos lenguajes cada sentencia termina con un punto y coma, {\tt ;}.
¿Qué ocurre si pones un punto y coma al final de una sentencia de Python?

\item ¿Qué ocurre si pones un punto al final de una sentencia?

\item En notación matemática puedes multiplicar $x$ e $y$ así: $x y$.
¿Qué ocurre si intentas eso en Python?

\end{itemize}

\end{exercise}


\begin{exercise}

Practica utilizando el intérprete de Python como una calculadora:
\index{calculadora}

\begin{enumerate}

\item El volumen de una esfera con radio $r$ es $\frac{4}{3} \pi r^3$.
  ¿Cuál es el volumen de una esfera con radio 5?

\item Supongamos que el precio original de un libro es \$24.95, pero las librerías obtienen
  un 40\% de descuento.  El envío cuesta \$3 para la primera copia y 75 centavos
  por cada copia adicional.  ¿Cuál es el costo al por mayor para
  60 copias?

\item Si dejo mi casa a las 6:52 a.m. y corro 1 milla a un ritmo fácil
  (8 minutos y 15 segundos por milla), luego 3 millas a tempo (7 minutos y 12 segundos por milla) y 1 milla
  a ritmo fácil de nuevo, ¿a qué hora llego a casa para el desayuno?
\index{ritmo de carrera}

\end{enumerate}
\end{exercise}


\chapter{Funciones}
\label{funcchap}

En el contexto de la programación, una {\bf función} es una secuencia de sentencias
que realiza una computación y posee un nombre.  Cuando defines una función,
especificas el nombre y la secuencia de sentencias.  Después, puedes
``llamar'' a la función por su nombre.
\index{función}

\section{Llamadas a funciones}
\label{functionchap}
\index{llamada a función}

Ya hemos visto un ejemplo de una {\bf llamada a función}:

\begin{verbatim}
>>> type(42)
    <class 'int'>
\end{verbatim}
%
El nombre de la función es {\tt type}.  La expresión en paréntesis
se llama {\bf argumento} de la función.  El resultado, para esta
función, es el tipo del argumento.
\index{paréntesis!argumento en}

Es común decir que una función ``toma'' un argumento y ``devuelve''
un resultado.  El resultado también se llama {\bf valor de retorno} (en inglés, {\em return value}).
\index{argumento}
\index{valor de retorno}\index{return value}

Python proporciona funciones que convierten valores
de un tipo a otro.  La función {\tt int} toma cualquier valor y
lo convierte en un entero, si puede, o de lo contrario reclama:
\index{tipo!conversión de}
\index{conversión de tipo}
\index{función!int}
\index{int!función}

\begin{verbatim}
>>> int('32')
    32
>>> int('Hola')
    ValueError: invalid literal for int(): Hola
\end{verbatim}
%
{\tt int} puede convertir valores de coma flotante en enteros, pero
no redondea; corta la parte de fracción:

\begin{verbatim}
>>> int(3.99999)
    3
>>> int(-2.3)
    -2
\end{verbatim}
%
{\tt float} convierte enteros y cadenas en números coma
flotante:
\index{función!float}
\index{float, función}

\begin{verbatim}
>>> float(32)
    32.0
>>> float('3.14159')
    3.14159
\end{verbatim}
%
Por último, {\tt str} convierte su argumento en una cadena:
\index{str!función}
\index{función!str}

\begin{verbatim}
>>> str(32)
    '32'
>>> str(3.14159)
    '3.14159'
\end{verbatim}
%

\section{Funciones matemáticas}
\index{función!matemática}
\index{matemática, función}

Python tiene un módulo matemático que proporciona la mayor parte de las
funciones matemáticas conocidas.  Un {\bf módulo} es un archivo que contiene una
colección de funciones relacionadas entre sí.
\index{modulo@módulo}
\index{objeto!de módulo}

Antes de que podamos utilizar las funciones de un módulo, tenemos que importarlo con
una {\bf sentencia import}:

\begin{verbatim}
>>> import math
\end{verbatim}
%
Esta sentencia crea un {\bf objeto de módulo} llamado math.  Si
muestras el objeto de módulo en pantalla, obtienes información sobre este:

\begin{verbatim}
>>> math
    <module 'math' (built-in)>
\end{verbatim}
%
El objeto de módulo contiene las funciones y variables definidas en el
módulo. Para tener acceso a una de las funciones, tienes que especificar el nombre
del módulo y el nombre de la función, separados por un punto.
Este formato se llama {\bf notación de punto}.
\index{notación de punto}

\begin{verbatim}
>>> relacion = potencia_senal / potencia_ruido
>>> decibeles = 10 * math.log10(relacion)

>>> radianes = 0.7
>>> altura = math.sin(radianes)
\end{verbatim}
%
El primer ejemplo utiliza \verb"math.log10" para calcular
una relación señal/ruido en decibeles (suponiendo que \verb"potencia_senal" y
\verb"potencia_ruido" están definidas).  El módulo math también proporciona a {\tt log},
que calcula logaritmos en base {\tt e}.
\index{función!log}
\index{log, función}
\index{función!seno}
\index{radian@radián}
\index{función!trigonométrica}
\index{trigonométrica, función}

El segundo ejemplo encuentra el seno de {\tt radianes}.  El nombre de la variable {\tt radianes} es un indicio de que {\tt sin} y las otras funciones
trigonométricas ({\tt cos}, {\tt tan}, etc.)  toman argumentos en radianes. Para
convertir de grados a radianes, divide por 180 y multiplica por
$\pi$:

\begin{verbatim}
>>> grados = 45
>>> radianes = grados / 180.0 * math.pi
>>> math.sin(radianes)
    0.707106781187
\end{verbatim}
%
La expresión {\tt math.pi} obtiene la variable {\tt pi} del módulo
math.  Su valor es una aproximación en coma flotante
de $\pi$, con precisión de alrededor de 15 digitos.
\index{pi}

Si sabes
trigonometría, puedes verificar los resultados anteriores comparándolos con
la raíz cuadrada de dos, dividida por dos:
\index{función!sqrt}
\index{sqrt, función}

\begin{verbatim}
>>> math.sqrt(2) / 2.0
    0.707106781187
\end{verbatim}
%

\section{Composición}
\index{composición}

Hasta aquí, hemos visto los elementos de un programa ---variables,
expresiones y sentencias--- de forma aislada, sin hablar sobre cómo
combinarlos.

Una de las características más útiles de los lenguajes de programación es su
posibilidad de tomar pequeños bloques de construcción y {\bf componerlos}.  Por
ejemplo, el argumento de una función puede ser cualquier tipo de expresión,
incluyendo operadores aritméticos:

\begin{verbatim}
x = math.sin(grados / 360.0 * 2 * math.pi)
\end{verbatim}
%
También llamadas a funciones:

\begin{verbatim}
x = math.exp(math.log(x+1))
\end{verbatim}
%
Casi en cualquier lugar que puedes poner un valor, puedes poner una expresión
arbitraria, con una excepción: el lado izquierdo de una sentencia
de asignación tiene que ser un nombre de variable. Cualquier otra expresión en el lado
izquierdo es un error de sintaxis (veremos excepciones a esta regla
más tarde).

\begin{verbatim}
>>> minutos = horas * 60                 # correcto
>>> horas * 60 = minutos                 # ¡incorrecto!
    SyntaxError: can't assign to operator
\end{verbatim}
%
\index{SyntaxError}\index{error!de sintaxis}\index{sintaxis!error de}
\index{excepción!SyntaxError}


\section{Agregar nuevas funciones}

Hasta aquí, solo hemos estado utilizando las funciones que vienen con Python,
pero también es posible agregar nuevas funciones.
Una {\bf definición de función} especifica el nombre de una nueva función y
la secuencia de sentencias que se ejecutan cuando la función es llamada.
\index{función}
\index{definición de función}
\index{función!definición de}

Aquí hay un ejemplo:

\begin{verbatim}
def imprimir_letra():
    print("I'm a lumberjack, and I'm okay.")
    print("I sleep all night and I work all day.")
\end{verbatim}
%
{\tt def} es una palabra clave que indica que esta es una definición
de función. El nombre de la función es \verb"imprimir_letra".  Las
reglas para los nombres de funciones son las mismas que para los nombres de variables: las letras,
los números y el guión bajo son legales, pero el primer carácter
no puede ser un número.  No puedes utilizar una palabra clave como nombre de una función,
y deberías evitar tener una variable y una función con el mismo
nombre.
\index{palabra clave!def}
\index{def, palabra clave}
\index{argumento}

Los paréntesis vacíos después del nombre indican que esta función
no toma ningún argumento.
\index{paréntesis!vacíos}
\index{encabezado}\index{header}
\index{cuerpo}\index{body}
\index{sangría}
\index{dos puntos}

La primera línea de la definición de función se llama {\bf encabezado} (en inglés, {\em header});
el resto se llama {\bf cuerpo} (en inglés, {\em body}).  El encabezado debe terminar con el signo de dos puntos
y el cuerpo debe tener sangría.  Por convención, la sangría
siempre se hace con cuatro espacios.  El cuerpo puede contener
cualquier número de sentencias.

Las cadenas en las sentencias print están encerradas en comillas
dobles.  Las comillas simples y las comillas dobles hacen lo mismo;
la mayoría de la gente utiliza comillas simples excepto en casos como este donde
una comilla simple (que también es un apóstrofe) aparece en la cadena.

Todas las comillas (simples y dobles)
deben ser ``comillas rectas'', usualmente
ubicadas cerca de Enter en el teclado.  Las ``comillas tipográficas'', como
las de esta oración, no son legales en Python.

Si escribes una definición de función en modo interactivo, el intérprete
imprime puntos ({\tt ...}) que te hacen saber que la definición
no está completa:
\index{puntos suspensivos}

\begin{verbatim}
>>> def imprimir_letra():
...     print("I'm a lumberjack, and I'm okay.")
...     print("I sleep all night and I work all day.")
...
\end{verbatim}
%
Para terminar una función, tienes que insertar una línea vacía.

Al definir una función, se crea un {\bf objeto de función} que tiene
tipo \verb"function":
\index{tipo!function}
\index{function, tipo}

\begin{verbatim}
>>> print(imprimir_letra)
    <function imprimir_letra at 0xb7e99e9c>
>>> type(imprimir_letra)
    <class 'function'>
\end{verbatim}
%
La sintaxis para llamar a la nueva función es la misma que
para las funciones incorporadas:

\begin{verbatim}
>>> imprimir_letra()
    I'm a lumberjack, and I'm okay.
    I sleep all night and I work all day.
\end{verbatim}
%
Una vez que hayas definido una función, puedes utilizarla dentro de otra
función.  Por ejemplo, para repetir el estribillo anterior, podríamos escribir
una función llamada \verb"repetir_letra":

\begin{verbatim}
def repetir_letra():
    imprimir_letra()
    imprimir_letra()
\end{verbatim}
%
Y luego llamar a \verb"repetir_letra":

\begin{verbatim}
>>> repetir_letra()
    I'm a lumberjack, and I'm okay.
    I sleep all night and I work all day.
    I'm a lumberjack, and I'm okay.
    I sleep all night and I work all day.
\end{verbatim}
%
Pero así no es como sigue la canción realmente.


\section{Definiciones y usos}
\index{definición de función}

Reuniendo los fragmentos de código de la sección anterior, el
programa completo se ve así:

\begin{verbatim}
def imprimir_letra():
    print("I'm a lumberjack, and I'm okay.")
    print("I sleep all night and I work all day.")

def repetir_letra():
    imprimir_letra()
    imprimir_letra()

repetir_letra()
\end{verbatim}
%
Este programa contiene dos definiciones de función: \verb"imprimir_letra" y
\verb"repetir_letra".  Las definiciones de funciones se ejecutan al igual que otras
sentencias, pero el efecto es crear objetos de función.  Las sentencias
dentro de la función no se ejecutan hasta que la función es llamada, y
la definición de función no genera salida.
\index{use before def}

Como podrías esperar, tienes que crear la función antes de que puedas
ejecutarla.  En otras palabras, la definición de función tiene que efectuarse
antes de que la función sea llamada.

Como ejercicio, mueve la última línea de este programa
hasta el principio, así la llamada a la función aparece antes que las definiciones. Ejecuta
el programa y mira qué mensaje
de error obtienes.

Ahora regresa la llamada de función al final
y mueve la definición de \verb"imprimir_letra" a después de la definición de
\verb"repetir_letra".  ¿Qué ocurre cuando ejecutas este programa?


\section{Flujo de ejecución}
\index{flujo de ejecución}

Para asegurarte de que una función está definida antes de su primer uso,
tienes que conocer el orden en que se ejecutan las sentencias, lo cual se
llama {\bf flujo de ejecución}.

La ejecución siempre comienza con la primera sentencia del programa.
Las sentencias se ejecutan una a la vez, en orden desde arriba hacia abajo.

Las definiciones de función no alteran el flujo de ejecución del
programa, pero recuerda que las sentencias dentro de la función no
se ejecutan hasta que la función es llamada.

Una llamada a función es como un desvío en el flujo de ejecución. En lugar de ir
a la siguiente sentencia, el flujo salta al cuerpo de
la función, ejecuta las sentencias que están allí y luego regresa
para retomar donde lo había dejado.

Eso suena bastante simple, hasta que recuerdas que una función puede
llamar a otra.  Mientras está en el medio de una función, el programa quizás
tenga que ejecutar las sentencias en otra función.  Luego, mientras
se ejecuta esa nueva función, ¡el programa quizás tenga que ejecutar
otra función más!

Afortunadamente, Python es bueno haciendo seguimiento de dónde está, así que cada
vez que se completa una función, el programa retoma donde lo había dejado en
la función que la llamó.  Cuando llega al final del programa,
termina.

En resumen, cuando lees un programa,
no siempre quieres leer desde arriba hacia abajo.  A veces tiene
más sentido si sigues el flujo de ejecución.


\section{Parámetros y argumentos}
\label{parameters}
\index{parámetro}
\index{función!parámetro de}
\index{argumento}
\index{función!argumento de}

Algunas de las funciones que hemos visto requieren argumentos. Por
ejemplo, cuando llamas a {\tt math.sin} pasas un número
como argumento.  Algunas funciones toman más de un argumento;
{\tt math.pow} toma dos: la base y el exponente.

Dentro de la función, los argumentos son asignados a
variables llamadas {\bf parámetros}.  Aquí hay una definición para
una función que toma un argumento:
\index{paréntesis!parámetros en}

\begin{verbatim}
def impr_2veces(bruce):
    print(bruce)
    print(bruce)
\end{verbatim}
%
Esta función asigna el argumento a un parámetro
con nombre {\tt bruce}.  Cuando la función es llamada, esta imprime el valor del
parámetro (sea lo que sea) dos veces.

Esta función puede utilizarse con cualquier valor que se pueda imprimir.

\begin{verbatim}
>>> impr_2veces('Spam')
    Spam
    Spam
>>> impr_2veces(42)
    42
    42
>>> impr_2veces(math.pi)
    3.14159265359
    3.14159265359
\end{verbatim}
%
Las mismas reglas de composición que se aplican a las funciones incorporadas también
se aplican a las funciones definidas por el programador, así que podemos utilizar cualquier tipo de expresión
como un argumento para \verb"impr_2veces":
\index{composición}
\index{función!definida por el programador}
\index{programador, función definida por el}

\begin{verbatim}
>>> impr_2veces('Spam '*4)
    Spam Spam Spam Spam
    Spam Spam Spam Spam
>>> impr_2veces(math.cos(math.pi))
    -1.0
    -1.0
\end{verbatim}
%
El argumento es evaluado antes de que se llame a la función, por lo que
en los ejemplos las expresiones \verb"'Spam '*4" y
{\tt math.cos(math.pi)} son evaluadas una sola vez.
\index{argumento}

También puedes utilizar una variable como un argumento:

\begin{verbatim}
>>> michael = 'Eric, the half a bee.'
>>> impr_2veces(michael)
Eric, the half a bee.
Eric, the half a bee.
\end{verbatim}
%
El nombre de la variable que pasamos como argumento ({\tt michael}) no tiene
nada que ver con el nombre del parámetro ({\tt bruce}).  No
importa cómo se le llame al valor en su casa (en la sentencia llamadora);
aquí en \verb"impr_2veces", a todos les llamamos {\tt bruce}.


\section{Las variables y los parámetros son locales}
\index{variable local}
\index{local, variable}

Cuando creas una variable dentro de una función, esta es {\bf local},
lo cual significa que existe
solamente dentro de la función.  Por ejemplo:
\index{paréntesis!parámetros en}

\begin{verbatim}
def cat_2veces(parte1, parte2):
    cat = parte1 + parte2
    impr_2veces(cat)
\end{verbatim}
%
Esta función toma dos argumentos, los concatena e imprime
el resultado dos veces.  Aquí hay un ejemplo que la utiliza:
\index{concatenación}

\begin{verbatim}
>>> linea1 = 'Bing tiddle '
>>> linea2 = 'tiddle bang.'
>>> cat_2veces(linea1, linea2)
    Bing tiddle tiddle bang.
    Bing tiddle tiddle bang.
\end{verbatim}
%
Cuando \verb"cat_2veces" termina, la variable {\tt cat}
se destruye.  Si intentamos imprimirla, obtenemos una excepción:
\index{NameError}
\index{excepción!NameError}

\begin{verbatim}
>>> print(cat)
    NameError: name 'cat' is not defined
\end{verbatim}
%
Los parámetros también son locales.
Por ejemplo, afuera de \verb"impr_2veces", no hay
tal cosa como {\tt bruce}.
\index{parámetro}


\section{Diagramas de pila}
\label{stackdiagram}
\index{stack diagram}\index{diagrama!de pila}
\index{marco de una función}
\index{frame}\index{marco}

Para hacer un seguimiento de qué variables se pueden utilizar en qué lugar, a veces es
útil dibujar un {\bf diagrama de pila} (en inglés, {\em stack diagram}).  Al igual que los diagramas de estado,
los diagramas de pila muestran el valor de cada variable, pero también muestran la
función a la cual pertenece cada variable.
\index{diagrama!de pila}
\index{pila, diagrama de}

Cada función se representa por un {\bf marco} (en inglés, {\em frame}).  Un marco es un recuadro que tiene
el nombre de una función al lado y los parámetros y variables de
la función adentro.  El diagrama de pila para el ejemplo anterior se
muestra en la Figura~\ref{fig.stack}.

\begin{figure}
\centerline
{\includegraphics[scale=0.8]{figs/stack.pdf}}
\caption{Diagrama de pila.}
\label{fig.stack}
\end{figure}


Los marcos se organizan en una pila que indica cuál función
llama a cuál, y así sucesivamente.  En este ejemplo, \verb"impr_2veces"
fue llamado por \verb"cat_2veces", y \verb"cat_2veces" fue llamado por
\verb"__main__", el cual es un nombre especial para el marco más alto.  Cuando
creas una variable afuera de todas las funciones, pertenece a
\verb"__main__".

\index{main}

Cada parámetro se refiere al mismo valor que su argumento
correspondiente.  Así que, {\tt parte1} tiene el mismo valor que
{\tt linea1}, {\tt parte2} tiene el mismo valor que {\tt linea2}
y {\tt bruce} tiene el mismo valor que {\tt cat}.

Si ocurre un error durante una llamada a función, Python imprime el
nombre de la función, el nombre de la función que la llamó
y el nombre de la función que llamó a {\em esa}, todo el
camino de vuelta a \verb"__main__".

Por ejemplo, si intentas acceder a {\tt cat} desde adentro de
\verb"impr_2veces", obtienes un {\tt NameError}:

\begin{verbatim}
Traceback (innermost last):
  File "test.py", line 13, in __main__
    cat_2veces(linea1, linea2)
  File "test.py", line 5, in cat_2veces
    impr_2veces(cat)
  File "test.py", line 9, in impr_2veces
    print(cat)
NameError: name 'cat' is not defined
\end{verbatim}
%
Esta lista de funciones se llama {\bf rastreo} (en inglés, {\em traceback}).  Te dice en qué
archivo de programa ocurrió el error, y en qué línea, y qué funciones
se estaban ejecutando en ese momento.  Además, te muestra la línea de código que
causó el error.
\index{traceback}\index{rastreo}

El orden de las funciones en el rastreo es el mismo que el
orden de los marcos en el diagrama de pila.  La función que se está
ejecutando actualmente está al final.


\section{Funciones productivas y funciones nulas}
\index{función!productiva}
\index{función!nula}
\index{productiva, función}
\index{nula, función}

Algunas de las funciones que hemos utilizado, tales como las funciones matemáticas, devuelven
resultados; por falta de un mejor nombre, las llamo {\bf funciones
  productivas}.  Otras funciones, como \verb"impr_2veces", realizan una
acción pero no devuelven un valor.  Son llamadas {\bf funciones
  nulas} (en inglés, {\em void functions}).

Cuando llamas a una función productiva, casi siempre
quieres hacer algo con el resultado; por ejemplo, podrías
asignarlo a una variable o utilizarlo como parte de una expresión:

\begin{verbatim}
x = math.cos(radianes)
dorado = (math.sqrt(5) + 1) / 2
\end{verbatim}
%
Cuando llamas a una función en modo interactivo, Python muestra
el resultado:

\begin{verbatim}
>>> math.sqrt(5)
    2.2360679774997898
\end{verbatim}
%
Pero en un script, si llamas a una función productiva por sí sola,
¡el valor de retorno se pierde para siempre!

\begin{verbatim}
math.sqrt(5)
\end{verbatim}
%
Este script calcula la raíz cuadrada de 5, pero ya que no almacena
ni muestra el resultado, no es muy útil.
\index{modo interactivo}
\index{modo script}

Las funciones nulas podrían mostrar algo en la pantalla o tener algún
otro efecto, pero no tienen un valor de retorno.  Si
asignas el resultado a una variable, obtienes un valor especial llamado
{\tt None}.
\index{valor especial!None}
\index{None, valor especial}

\begin{verbatim}
>>> resultado = impr_2veces('Bing')
    Bing
    Bing
>>> print(resultado)
    None
\end{verbatim}
%
El valor {\tt None} no es lo mismo que la cadena \verb"'None'".
Es un valor especial que tiene su propio tipo:

\begin{verbatim}
>>> type(None)
    <class 'NoneType'>
\end{verbatim}
%
Las funciones que hemos escrito hasta ahora son todas nulas.  Comenzaremos
a escribir funciones productivas en unos capítulos más adelante.
\index{tipo!NoneType}
\index{NoneType, tipo}


\section{¿Por qué funciones?}
\index{función!razones para utilizar una}

Puede que no esté claro por qué vale la pena el problema de dividir
un programa en funciones.  Hay muchas razones:

\begin{itemize}

\item Crear una nueva función te da la oportunidad de ponerle nombre a un grupo
de sentencias, lo cual hace que tu programa sea más fácil de leer y depurar.

\item Las funciones pueden hacer que un programa sea más pequeño al eliminar código
repetitivo.  Después, si quieres hacer un cambio, solo tienes
que hacerlo en un lugar.

\item Dividir un programa largo en funciones te permite depurar las
partes una a la vez y luego reunirlas en un todo funcional.

\item Las funciones bien diseñadas son a menudo útiles para muchos programas.
Una vez que escribes y depuras una, la puedes reutilizar.

\end{itemize}


\section{Depuración}

Una de las habilidades más importantes que adquirirás es la depuración.
Aunque puede ser frustrante, la depuración es una de las partes más
intelectualmente ricas, desafiantes e interesantes de
la programación.
\index{depuración experimental}
\index{experimental, depuración}

En algunas formas la depuración es como un trabajo de detective.  Te enfrentas
a pistas y tienes que inferir los procesos y eventos que te guían
a los resultados que ves.

La depuración es también como una ciencia experimental.  Una vez que tienes una idea
sobre qué va mal, modificas tu programa e intentas de nuevo.  Si
tu hipótesis era correcta, puedes predecir el resultado de la
modificación y das un paso más cerca hacia un programa que funcione.  Si
tu hipótesis era incorrecta, tienes que inventar una nueva.  Como
señaló Sherlock Holmes, ``Una vez descartado lo
imposible, lo que queda, por improbable que parezca, debe ser la verdad.''
(A. Conan Doyle, {\em El signo de los cuatro})
\index{Holmes, Sherlock}
\index{Doyle, Arthur Conan}

Para algunas personas, programar y depurar son lo mismo.  Es
decir, programar es el proceso de depurar gradualmente un programa hasta que
haga lo que tú quieres.  La idea es que deberías comenzar con un
programa que funcione y hacer pequeñas modificaciones,
depurándolas a medida que avanzas.

Por ejemplo, Linux es un sistema operativo que contiene millones de
líneas de código, pero comenzó como un programa simple que Linus Torvalds
utilizaba para explorar el chip Intel 80386.  Según Larry Greenfield,
``Uno de los proyectos anteriores de Linus era un programa que cambiaría
entre imprimir AAAA y BBBB.  Esto evolucionó más tarde a Linux.''
({\em The Linux Users' Guide} Beta Version 1).
\index{Linux}


\section{Glosario}

\begin{description}

\item[función:] Una secuencia de sentencias que tiene nombre y realiza alguna
operación útil.  Las funciones pueden o no tomar argumentos y pueden o no
producir un resultado.
\index{función}

\item[definición de función:]  Una sentencia que crea una nueva función,
especificando su nombre, parámetros y las sentencias que contiene.
\index{definición de función}

\item[objeto de función:]  Un valor creado por una definición de función.
El nombre de la función es una variable que se refiere a un objeto
de función.
\index{objeto!de función}

\item[encabezado:] La primera línea de una definición de función.
\index{encabezado}

\item[cuerpo:] La secuencia de sentencias dentro de una definición de función.
\index{cuerpo}

\item[parámetro:] Un nombre utilizado dentro de una función para referirse al valor
pasado como argumento.
\index{parámetro}

\item[llamada a función:] Una sentencia que ejecuta una función.
Consiste en el nombre de la función seguido de una lista de argumentos en
paréntesis.
\index{llamada a función}

\item[argumento:]  Un valor proporcionado a la función cuando la función es llamada.
Este valor es asignado al parámetro correspondiente en la función.
\index{argumento}

\item[variable local:]  Una variable definida dentro de una función. Una variable
local solo puede utilizarse dentro de su función.
\index{variable local}

\item[valor de retorno:]  El resultado de una función.  Si una llamada a función
se utiliza como expresión, el valor de retorno es el valor de
la expresión.
\index{valor de retorno}

\item[función productiva:] Una función que devuelve un valor.
\index{función!productiva}

\item[función nula:] Una función que siempre devuelve {\tt None}.
\index{función!nula}

\item[{\tt None}:]  Un valor especial devuelto por funciones nulas.
\index{valor especial!None}
\index{None, valor especial}

\item[módulo:] Un archivo que contiene una
colección de funciones relacionadas entre sí y otras definiciones.
\index{modulo@módulo}

\item[sentencia import:] Una sentencia que lee un archivo de módulo y crea
un objeto de módulo.
\index{sentencia!import}
\index{import, sentencia}

\item[objeto de módulo:] Un valor creado por una sentencia {\tt import}
que proporciona acceso a los valores definidos en el módulo.
\index{objeto!de módulo}

\item[notación de punto:]  La sintaxis para llamar a una función de otro
módulo especificando el nombre del módulo, seguido de un punto y
el nombre de la función.
\index{notación de punto}

\item[composición:] Usar una expresión como parte de una expresión más grande,
o una sentencia como parte de una sentencia más grande.
\index{composición}

\item[flujo de ejecución:]  El orden en que las sentencias se ejecutan.
\index{flujo de ejecución}

\item[diagrama de pila:]  Una representación de una pila de funciones,
sus variables y los valores a los cuales se refieren.
\index{stack diagram}

\item[marco:]  Un recuadro en un diagrama de pila que representa una llamada a función.
Contiene las variables locales y los parámetros de la función.
\index{función!marco de}
\index{marco}

\item[rastreo:]  Una lista de las funciones que se están ejecutando,
impresas cuando ocurre una excepción.
\index{rastreo}


\end{description}


\section{Ejercicios}

\begin{exercise}
\index{función!len}
\index{len, función}

Escribe una función con nombre \verb"justificar_derecha" que tome una cadena
con nombre {\tt s} como parámetro e imprima la cadena con suficientes
espacios al inicio de tal manera que la última letra de la cadena esté en la columna 70
de la pantalla.

\begin{verbatim}
>>> justificar_derecha('monty')
                                                                 monty
\end{verbatim}

Pista: utiliza la repetición de cadenas y la concatenación.  Además,
Python proporciona una función incorporada llamada {\tt len} que
devuelve la longitud de una cadena, por lo que el valor de \verb"len('monty')" es 5.

\end{exercise}


\begin{exercise}
\index{objeto!de función}
\index{función!objeto de}

Un objeto de función es un valor que puedes asignar a una variable o
pasarlo como argumento.  Por ejemplo, \verb"hacer_2veces" es una función
que toma un objeto de función como argumento y lo llama dos veces:

\begin{verbatim}
def hacer_2veces(f):
    f()
    f()
\end{verbatim}

Aquí hay un ejemplo que utiliza \verb"hacer_2veces" para llamar a una función
con nombre \verb"imprimir_spam" dos veces.

\begin{verbatim}
def imprimir_spam():
    print('spam')

hacer_2veces(imprimir_spam)
\end{verbatim}

\begin{enumerate}

\item Escribe este ejemplo en un script y pruébalo.

\item Modifica \verb"hacer_2veces" para que tome dos argumentos, un
objeto de función y un valor, y llame a la función dos veces,
pasando al valor como argumento.

\item Copia la definición de
\verb"impr_2veces", presentada previamente en este capítulo, a tu script.

\item Usa la versión modificada de \verb"hacer_2veces" para llamar a
\verb"impr_2veces" dos veces, pasando a \verb"'spam'" como argumento.

\item Define una nueva función llamada
\verb"hacer_4veces" que tome un objeto de función y un valor
y llame a la función cuatro veces, pasando al valor
como argumento.  Debería haber solo
dos sentencias en el cuerpo de esta función, no cuatro.

\end{enumerate}

Solution: \url{http://thinkpython.com/code/do_four.py}.

\end{exercise}



\begin{exercise}

Nota: Este ejercicio debería hacerse
utilizando solo las sentencias y otras características que hemos aprendido hasta
ahora.

\begin{enumerate}

\item Escribe una función que dibuje una cuadrícula como la siguiente:
\index{cuadrícula}
\newpage
\begin{verbatim}
+ - - - - + - - - - +
|         |         |
|         |         |
|         |         |
|         |         |
+ - - - - + - - - - +
|         |         |
|         |         |
|         |         |
|         |         |
+ - - - - + - - - - +
\end{verbatim}
%
Pista: para imprimir más de un valor en una línea, puedes imprimir
una secuencia de valores separada por comas:

\begin{verbatim}
print('+', '-')
\end{verbatim}
%
Por defecto, {\tt print} avanza a la siguiente línea, pero
puedes anular ese comportamiento y poner un espacio al final, como esto:

\begin{verbatim}
print('+', end=' ')
print('-')
\end{verbatim}
%
La salida de estas sentencias es \verb"'+ -'" en la misma línea.
La salida desde la siguiente sentencia print debería comenzar en la siguiente línea.

\item Escribe una función que dibuje una cuadrícula similar
con cuatro filas y cuatro columnas.

\end{enumerate}

Solución: \url{http://thinkpython.com/code/grid.py}.
Crédito: este ejercicio está basado en un ejercicio de Oualline, {\em
    Practical C Programming, Third Edition}, O'Reilly Media, 1997.

\end{exercise}





\chapter{Estudio de caso: diseño de interfaz}
\label{turtlechap}

Este capítulo presenta un estudio de caso que demuestra un proceso para
diseñar funciones que interactúen entre sí.

Se presenta el módulo {\tt turtle}, el cual te permite
crear imágenes utilizando gráficas tortuga.  El módulo {\tt turtle}
está incluido en la mayoría de las instalaciones de Python, pero si estás ejecutando Python
utilizando PythonAnywhere, no podrás ejecutar los ejemplos de tortuga (al
menos no podías cuando escribí esto).

Si ya has instalado Python en tu computador, deberías
poder ejecutar los ejemplos.  Si no, ahora es un buen momento
para instalarlo.  He publicado instrucciones en
\url{http://tinyurl.com/thinkpython2e}.

Los códigos de ejemplo de este capítulo están disponibles en
\url{http://thinkpython.com/code/polygon.py}.


\section{El módulo turtle}
\label{turtle}

Para verificar si tienes el módulo {\tt turtle}, abre el intérprete de Python
y escribe

\begin{verbatim}
>>> import turtle
>>> bob = turtle.Turtle()
\end{verbatim}

Cuando ejecutes este código, debería crearse una nueva ventana
con una flecha pequeña que representa la tortuga.  Cierra la ventana.

Crea un archivo con nombre {\tt mipoligono.py} y escribe en él las siguientes
líneas de código:

\begin{verbatim}
import turtle
bob = turtle.Turtle()
print(bob)
turtle.mainloop()
\end{verbatim}
%
El módulo {\tt turtle} (con 't' minúscula) proporciona una función
llamada {\tt Turtle} (con 'T' mayúscula) que crea un objeto Turtle,
el cual asignamos a una variable con nombre {\tt bob}.
Al imprimir {\tt bob} se muestra algo como:

\begin{verbatim}
<turtle.Turtle object at 0xb7bfbf4c>
\end{verbatim}
%
Esto significa que {\tt bob} se refiere a un objeto con tipo
{\tt Turtle}
como se define en el módulo {\tt turtle}.

\verb"mainloop" le dice a la ventana que espere a que el usuario
haga algo, aunque en este caso no hay mucho que pueda hacer
el usuario excepto cerrar la ventana.

Una vez que creas una tortuga, puedes llamar a un {\bf método} para moverla
dentro de la ventana.  Un método es similar a una función, pero
utiliza una sintaxis un poco diferente.  Por ejemplo, para mover la tortuga
hacia adelante:

\begin{verbatim}
bob.fd(100)
\end{verbatim}
%
El método, {\tt fd} ({\em forward}), está asociado con el objeto
tortuga que llamamos {\tt bob}.
Llamar a un método es como hacer una solicitud: le estás pidiendo a {\tt bob}
que se mueva hacia adelante.

El argumento de {\tt fd} es una distancia en pixeles, por lo que el tamaño
real depende de tu pantalla.

Otros métodos que puedes llamar en un objeto Turtle son {\tt bk} ({\em backward}) para
retroceder, {\tt lt} ({\em left turn}) para girar a la izquierda y {\tt rt} ({\em right turn}) para girar a la derecha.  El
argumento para {\tt lt} y {\tt rt} es un ángulo en grados.

Además, cada Turtle sostiene una pluma, que está
arriba o abajo; si la pluma está abajo, la tortuga deja
un rastro cuando se mueve.  Los métodos {\tt pu} y {\tt pd}
representan ``{\em pen up}'' y ``{\em pen down}''.

Para dibujar un ángulo recto, agrega estas líneas al programa
(después de crear a {\tt bob} y antes de llamar a \verb"mainloop"):

\begin{verbatim}
bob.fd(100)
bob.lt(90)
bob.fd(100)
\end{verbatim}
%
Cuando ejecutes este programa, deberías ver a {\tt bob} moverse al este y luego
al norte, dejando dos segmentos de línea atrás.

Ahora modifica el programa para dibujar un cuadrado.  ¡No continues hasta que
lo hayas hecho funcionar!

%\newpage

\section{Repetición simple}
\label{repetition}
\index{repetición}

Es probable que hayas escrito algo así:

\begin{verbatim}
bob.fd(100)
bob.lt(90)

bob.fd(100)
bob.lt(90)

bob.fd(100)
bob.lt(90)

bob.fd(100)
\end{verbatim}
%
Podemos hacer lo mismo de manera más concisa con una sentencia {\tt for}.
Agrega este ejemplo a {\tt mipoligono.py} y ejecútalo de nuevo:
\index{bucle!for}
\index{for, bucle}
\index{sentencia!for}

\begin{verbatim}
for i in range(4):
    print('¡Hola!')
\end{verbatim}
%
Deberías ver algo como esto:

\begin{verbatim}
¡Hola!
¡Hola!
¡Hola!
¡Hola!
\end{verbatim}
%
Este es el uso más simple de una sentencia {\tt for}; después veremos
más.  Pero eso debería ser suficiente para dejarte reescribir tu
programa que dibuja cruadrados.  No continues hasta que lo hagas.

Aquí hay una sentencia {\tt for} que dibuja un cuadrado:

\begin{verbatim}
for i in range(4):
    bob.fd(100)
    bob.lt(90)
\end{verbatim}
%
La sintaxis de una sentencia {\tt for} es similar a una definición
de función.  Tiene un encabezado que termina con el signo dos puntos y un cuerpo
con sangrías.  El cuerpo puede contener cualquier número de sentencias.

Una sentencia {\tt for} también es llamada {\bf bucle} porque
el flujo de ejecución pasa por el cuerpo y luego vuelve
hacia arriba.  Es este caso, pasa por el cuerpo cuatro veces.
\index{bucle}

Esta versión es en realidad un poco diferente del código
que dibuja cuadrados propuesto anteriormente porque hace otro giro después
de dibujar el último lado del cuadrado.  El giro extra toma
más tiempo, pero simplifica el código si hacemos lo mismo
en cada paso por el bucle.  Esta versión también tiene el efecto
de regresar a la tortuga a su posición inicial, apuntando a
la dirección inicial.

\section{Ejercicios}

Lo siguiente es una serie de ejercicios que utilizan el módulo {\tt turtle}.  Pretenden
ser divertidos, pero también tienen un punto.  Mientras trabajes en
ellos, piensa cuál es el punto.

\index{modulo@módulo!turtle}
\index{turtle, módulo}

Las siguientes secciones tienen soluciones a los ejercicios, así que
no las mires hasta que hayas terminado (o al menos intentado).

\begin{enumerate}

\item Escribe una función llamada {\tt cuadrado} que tome un parámetro
con nombre {\tt t}, que es una tortuga.  Debería utilizar la tortuga para dibujar
un cuadrado.

Escribe una llamada a función que pase a {\tt bob} como argumento de 
{\tt cuadrado}, y luego ejecuta el programa de nuevo.

\item Agrega otro parámetro, con nombre {\tt longitud}, a {\tt cuadrado}.
Modifica el cuerpo para que la longitud de los lados sea {\tt longitud}, y luego
modifica la llamada a función para poner un segundo argumento.  Ejecuta el
programa de nuevo.  Prueba tu programa con un rango de valores para {\tt
longitud}.

\item Haz una copia de {\tt cuadrado} y cambia el nombre a {\tt
  poligono}.  Agrega otro parámetro con nombre {\tt n} y modifica el cuerpo
  para que dibuje un polígono regular con n lados.  Pista: los ángulos exteriores
  de un polígono regular con n lados son de $360/n$ grados.  \index{función!polígono}
    \index{poligono, funcion@polígono, función}

\item Escribe una función llamada {\tt circulo} que tome una tortuga,
{\tt t}, y radio, {\tt r}, como parámetros y dibuje un círculo
aproximado llamando a {\tt poligono} con una longitud y
número de lados apropiado.  Prueba tu función con un rango de valores
de {\tt r}.  \index{función!círculo} \index{circulo, funcion@círculo, función}

Pista: averigua cuál es el perímetro del círculo y asegúrate de que
se cumpla que {\tt longitud * n = perimetro}.

\item Haz una versión más general de {\tt circulo} llamada {\tt arco}
que tome un parámetro adicional, {\tt angulo}, que determine
qué fracción de un círculo dibujar.  {\tt angulo} está en
grados, así que cuando {\tt angulo=360}, {\tt arco} debería dibujar un círculo
completo.
\index{función!arco}
\index{arco, función}

\end{enumerate}


\section{Encapsulamiento}

El primer ejercicio te pide poner tu código que dibuja cuadrados
dentro de una definición de función y luego llamar a la función, pasando
a la tortuga como parámetro.  Esta es la solución:

\begin{verbatim}
def cuadrado(t):
    for i in range(4):
        t.fd(100)
        t.lt(90)

cuadrado(bob)
\end{verbatim}
%
Las sentencias de más adentro, {\tt fd} y {\tt lt}, tienen doble sangría para
mostrar que están dentro del bucle {\tt for}, el cual está dentro de la
definición de función.  La siguiente línea, {\tt cuadrado(bob)}, está alineada con
el margen izquierdo, lo cual indica el término tanto del bucle {\tt for}
como de la definición de función.

Dentro de la función, {\tt t} se refiere a la misma tortuga {\tt bob}, por lo que
{\tt t.lt(90)} tiene el mismo efecto que {\tt bob.lt(90)}.  En ese
caso, ¿por qué no
llamar al parámetro {\tt bob}?  La idea es que {\tt t} puede ser cualquier
tortuga, no solo {\tt bob}, así que podrías crear una segunda tortuga y
pasarla como argumento a {\tt cuadrado}:

\begin{verbatim}
alice = turtle.Turtle()
cuadrado(alice)
\end{verbatim}
%
El acto de envolver un pedazo de código en una función se llama {\bf
encapsulamiento}.  Uno de los beneficios del encapsulamiento es que
adjunta un nombre al código, lo cual sirve como una especie de documentación.
Otra ventaja es que si reutilizas el código, ¡es más conciso
llamar a una función dos veces que copiar y pegar el cuerpo!
\index{encapsulamiento}


\section{Generalización}

El siguiente paso es agregar un parámetro {\tt longitud} a {\tt cuadrado}.
Aquí hay una solución:

\begin{verbatim}
def cuadrado(t, longitud):
    for i in range(4):
        t.fd(longitud)
        t.lt(90)

cuadrado(bob, 100)
\end{verbatim}
%
El acto de agregar un parámetro a una función se llama {\bf generalización}
porque hace que la función sea más general: en la versión
anterior, el cuadrado tiene siempre el mismo tamaño; en esta versión,
puede ser de cualquier tamaño.
\index{generalización}

El siguiente paso es también una generalización.  En lugar de dibujar
cuadrados, {\tt poligono} dibuja polígonos regulares con cualquier número de
lados.  Aquí hay una solución:

\begin{verbatim}
def poligono(t, n, longitud):
    angulo = 360 / n
    for i in range(n):
        t.fd(longitud)
        t.lt(angulo)

poligono(bob, 7, 70)
\end{verbatim}
%
Este ejemplo dibuja un polígono de 7 lados de longitud 70.

Si estás usando Python 2, el valor de {\tt angulo} podría ser incorrecto
debido a una división entera.  Una solución simple es calcular
{\tt angulo = 360.0 / n}.  Dado que el numerador es un número de
coma flotante, el resultado es de coma flotante.
\index{Python 2}

Cuando una función tiene más que unos pocos argumentos numéricos, es fácil
olvidar qué son, o en qué orden deberían estar.  En ese caso,
a menudo es una buena idea incluir los nombres de los parámetros en la
lista de argumentos:

\begin{verbatim}
poligono(bob, n=7, longitud=70)
\end{verbatim}
%
Estos se llaman {\bf argumentos de palabra clave} porque incluyen
a los nombres de parámetro tratándolos como ``palabras clave'' (no confundir con
las palabras clave de Python como {\tt while} y {\tt def}).
\index{argumento de palabra clave}
\index{palabra clave!argumento de}

Esta sintaxis hace que el programa sea más legible.  Es también un recordatorio
sobre cómo funcionan los argumentos y los parámetros: cuando llamas a una función, los
argumentos son asignados a los parámetros.


\section{Diseño de interfaz}

El siguiente paso es escribir {\tt circulo}, el cual toma un radio,
{\tt r}, como parámetro.  Aquí hay una solución simple que utiliza a
{\tt poligono} para dibujar un polígono de 50 lados:

\begin{verbatim}
import math

def circulo(t, r):
    perimetro = 2 * math.pi * r
    n = 50
    longitud = perimetro / n
    poligono(t, n, longitud)
\end{verbatim}
%
La primera línea calcula el perímetro de un círculo con radio
{\tt r} utilizando la fórmula $2 \pi r$.  Dado que utilizamos {\tt math.pi},
tenemos que importar {\tt math}.  Por convención, las sentencias {\tt import}
generalmente están al comienzo del script.

{\tt n} es el número de segmentos de línea en tu aproximación de un círculo,
por lo que {\tt longitud} es la longitud de cada segmento.  Así, {\tt poligono}
dibuja un polígono de 50 lados que aproxima un círculo con radio {\tt r}.

Una limitación de esta solución es que {\tt n} es una constante, lo cual
significa que para círculos muy grandes, los segmentos de línea son muy largos, y
para círculos pequeños, ocupamos mucho tiempo dibujando segmentos muy pequeños.  Una
solución sería generalizar la función para que tome a {\tt n} como
parámetro.  Esto le daría al usuario (quien llame a {\tt circulo})
más control, pero la interfaz sería menos limpia.
\index{interfaz}

La {\bf interfaz} de una función es un resumen de cómo esta se utiliza: ¿cuáles
son los parámetros?  ¿Qué hace la función?  ¿Y cuál es el valor de
retorno?  Una interfaz es ``limpia'' si permite a la sentencia llamadora hacer
lo que quiere sin lidiar con detalles innecesarios.

En este ejemplo, {\tt r} forma parte de la interfaz porque
especifica el círculo a dibujar.  {\tt n} es menos apropiado
porque pertenece a los detalles de {\em cómo} debería dibujarse
el círculo.

En lugar de desordenar la interfaz, es mejor
escoger un valor apropiado de {\tt n}
que dependa del {\tt perímetro}:

\begin{verbatim}
def circulo(t, r):
    perimetro = 2 * math.pi * r
    n = int(perimetro / 3) + 3
    longitud = perimetro / n
    poligono(t, n, longitud)
\end{verbatim}
%
Ahora, el número de segmentos es un entero cercano a {\tt perimetro/3},
por lo que la longitud de cada segmento es aproximadamente 3, lo cual es suficientemente
pequeño para que el círculo se vea bien, pero lo suficientemente grande para ser eficiente,
y aceptable para cualquier tamaño de círculo.

Sumar 3 a {\tt n} garantiza que el polígono tenga al menos 3 lados.


\section{Refactorización}
\label{refactoring}
\index{refactorización}

Cuando escribí {\tt circulo}, fui capaz de reutilizar {\tt poligono}
porque un polígono de muchos lados es una buena aproximación de un círculo.
Pero {\tt arco} no es tan cooperativo: no podemos utilizar {\tt poligono}
o {\tt circulo} para dibujar un arco.

Una alternativa es comenzar con una copia
de {\tt poligono} y transformarla en {\tt arco}.  El resultado
podría verse así:

\begin{verbatim}
def arco(t, r, angulo):
    longitud_arco = 2 * math.pi * r * angulo / 360
    n = int(longitud_arco / 3) + 1
    longitud_paso = longitud_arco / n
    angulo_paso = angulo / n

    for i in range(n):
        t.fd(longitud_paso)
        t.lt(angulo_paso)
\end{verbatim}
%
La segunda mitad de esta función se parece a {\tt poligono}, pero
no podemos reutilizar {\tt poligono} sin cambiar la interfaz.  Podríamos
generalizar {\tt poligono} para que tome un ángulo como tercer argumento,
¡pero entonces {\tt poligono} ya no sería un nombre apropiado!
En cambio, llamemos {\tt polilinea} a la función más general: 

\begin{verbatim}
def polilinea(t, n, longitud, angulo):
    for i in range(n):
        t.fd(longitud)
        t.lt(angulo)
\end{verbatim}
%
Ahora podemos reescribir {\tt poligono} y {\tt arco} para que utilice a {\tt polilinea}:

\begin{verbatim}
def poligono(t, n, longitud):
    angulo = 360.0 / n
    polilinea(t, n, longitud, angulo)

def arco(t, r, angulo):
    longitud_arco = 2 * math.pi * r * angulo / 360
    n = int(longitud_arco / 3) + 1
    longitud_paso = longitud_arco / n
    angulo_paso = float(angulo) / n
    polilinea(t, n, longitud_paso, angulo_paso)
\end{verbatim}
%
Finalmente, podemos reescribir {\tt circulo} para que utilice a {\tt arco}:

\begin{verbatim}
def circulo(t, r):
    arco(t, r, 360)
\end{verbatim}
%
Este proceso ---reorganizar un programa para mejorar
las interfaces y facilitar la reutilización de código--- se llama {\bf refactorización}.
En este caso, notamos que había código similar en {\tt arco} y
{\tt poligono}, así que ``lo factorizamos'' en {\tt polilinea}.
\index{refactorización}

Si hubiéramos planificado con anticipación, podríamos haber escrito {\tt polilinea} primero
y haber evitado la refactorización, pero a menudo no sabes lo suficiente al
comienzo de un proyecto como para diseñar todas las interfaces.  Una vez que comienzas
a escribir código, entiendes mejor el problema.  A veces la refactorización es una
señal de que has aprendido algo.


\section{Un plan de desarrollo}
\index{plan de desarrollo!encapsulamiento y generalización}

Un {\bf plan de desarrollo} es un proceso para escribir programas.  El
proceso que utilizamos en este estudio de caso es ``encapsulamiento y
generalización''.  Los pasos de este proceso son:

\begin{enumerate}

\item Comenzar escribiendo un programa pequeño sin definiciones de función.

\item Una vez que el programa funciona, identifica una parte
  coherente, encapsula la parte en una función y dale un nombre.

\item Generaliza la función agregando parámetros apropiados.

\item Repite los pasos 1--3 hasta que tengas un conjunto de funciones eficaces.
Copia y pega código que funcione para evitar repetir (y volver a depurar).

\item Busca oportunidades para mejorar el programa refactorizando.
Por ejemplo, si tienes código similar en muchos lugares, considera
factorizarlo dentro de una función general apropiada.

\end{enumerate}

Este proceso tiene algunos inconvenientes ---más tarde veremos alternativas--- pero
puede ser útil si no sabes de antemano cómo dividir el
programa en funciones.  Este enfoque te permite diseñar a medida que
avanzas.


\section{docstring}
\label{docstring}
\index{docstring}

Un {\bf docstring} es una cadena al comienzo de una función que
explica la interfaz (``doc'' es la abreviatura de ``documentation'').  Aquí
hay un ejemplo:

\begin{verbatim}
def polilinea(t, n, longitud, angulo):
    """Dibuja n segmentos de línea con la longitud dada
    y el ángulo (en grados) entre ellos.  t es una tortuga.
    """
    for i in range(n):
        t.fd(longitud)
        t.lt(angulo)
\end{verbatim}
%
Por convención, todos los docstrings son cadenas entre triple comillas, también conocidas
como cadenas multilínea porque las triple comillas permiten expandir
la cadena a más de una línea.
\index{comillas}
\index{cadena!entre triple comillas}
\index{triple comillas, cadena entre}
\index{cadena!multilínea}
\index{multilínea, cadena}

Es breve, pero contiene la información esencial
que alguien necesitaría para utilizar esta función.  Explica de manera concisa lo que
hace la función (sin entrar en detalles sobre cómo lo
hace).  Explica qué efecto tiene cada parámetro en el comportamiento de
la función y de qué tipo debería ser cada parámetro (si no es
obvio).

Escribir este tipo de documentación es una parte importante del diseño de la
interfaz.  Una interfaz bien diseñada debería ser simple de explicar;
si tienes dificultades al explicar una de tus funciones,
quizás la interfaz podría mejorar.


\section{Depuración}
\index{depuración}
\index{interfaz}

Una interfaz es como un contrato entre una función y la sentencia llamadora.
La llamadora acepta proporcionar ciertos argumentos y la función
acepta hacer cierto trabajo.

Por ejemplo, {\tt polilinea} requiere cuatro argumentos: {\tt t} tiene que ser
Turtle; {\tt n} tiene que ser un
entero; {\tt longitud} debería ser un número positivo; y {\tt
  angulo} tiene que ser un número, que se entiende que está en grados.

Estos requisitos se llaman {\bf precondiciones} porque
se supone que son verdaderos antes de que la función comience a ejecutarse.
De forma opuesta, las condiciones al final de la función son
{\bf postcondiciones}.  Las postcondiciones incluyen el efecto
previsto de la función (como al dibujar segmentos de línea) y cualquier
efecto secundario (como mover la tortuga o hacer otros cambios).
\index{precondición}
\index{postcondición}

Las precondiciones son responsabilidad de la llamadora.  Si la llamadora
viola una precondición (¡debidamente documentada!) y la función
no funciona de forma correcta, el error está en la llamadora, no en la función.

Si las precondiciones se satisfacen y las postcondiciones
no, el error está en la función.  Si tus pre y post condiciones
están claras, pueden ayudar con la depuración.


\section{Glosario}

\begin{description}

\item[método:] Una función que se asocia a un objeto y se llama
utilizando notación de punto.
\index{metodo@método}

\item[bucle:] Una parte de un programa que puede ejecutarse de forma repetida.
\index{bucle}

\item[encapsulamiento:] El proceso de transformar una secuencia de
sentencias en una definición de función.
\index{encapsulamiento}

\item[generalización:] El proceso de reemplazar algo
innecesariamente específico (como un número) con algo apropiadamente
general (como una variable o parámetro).
\index{generalización}

\item[argumento de palabra clave:] Un argumento que incluye el nombre del
parámetro tratándolo como ``palabra clave''.
\index{argumento de palabra clave}
\index{palabra clave!argumento de}

\item[interfaz:] Una descripción de cómo utilizar una función, incluyendo
el nombre y descripciones de los argumentos y el valor de retorno.
\index{interfaz}

\item[refactorización:] El proceso de modificar un programa que funciona para
  mejorar las interfaces de funciones y otras cualidades del código.
\index{refactorización}

\item[plan de desarrollo:] Un proceso para escribir programas.
\index{plan de desarrollo}

\item[docstring:] Una cadena que aparece en la parte superior de una definición
  de función para documentar la interfaz de la función.
\index{docstring}

\item[precondición:] Un requisito que debería satisfacer la
sentencia llamadora antes de que la función comience.
\index{precondición}

\item[postcondición:] Un requisito que debería satisfacer la
función antes de que termine.
\index{precondición}

\end{description}


\section{Ejercicios}

\begin{exercise}

Descarga el código de este capítulo en
\url{http://thinkpython.com/code/polygon.py}.

\begin{enumerate}

\item Dibuja un diagrama de pila que muestre el estado del programa
al ejecutar {\tt circulo(bob, radio)}.  Puedes hacer la
aritmética a mano o agregar sentencias {\tt print} al código.
\index{diagrama!de pila}

\item La versión de {\tt arco} en la Sección~\ref{refactoring} no es
muy precisa debido a que la aproximación lineal del
círculo está siempre afuera del verdadero círculo.  Como resultado,
la tortuga termina a unos pocos pixeles de distancia del destino
correcto.  Mi solución muestra una manera de reducir
el efecto de este error.  Lee el código y ve si
tiene sentido para ti.  Si dibujas un diagrama, podrías ver cómo trabaja.

\end{enumerate}

\end{exercise}

\begin{figure}
\centerline
{\includegraphics[scale=0.8]{figs/flowers.pdf}}
\caption{Flores tortuga.}
\label{fig.flowers}
\end{figure}

\begin{exercise}
\index{flor}

Escribe un conjunto de funciones apropiadamente generales que
puedan dibujar flores como en la Figura~\ref{fig.flowers}.

Solución: \url{http://thinkpython.com/code/flower.py},
también requiere \url{http://thinkpython.com/code/polygon.py}.

\end{exercise}

\begin{figure}
\centerline
{\includegraphics[scale=0.8]{figs/pies.pdf}}
\caption{Pasteles tortuga.}
\label{fig.pies}
\end{figure}


\begin{exercise}
\index{pastel}

Escribe un conjunto de funciones apropiadamente generales que
puedan dibujar formas como en la Figura~\ref{fig.pies}.

Solución: \url{http://thinkpython.com/code/pie.py}.

\end{exercise}

\begin{exercise}
\index{alfabeto}
\index{tortuga, máquina de escribir}
\index{maquina de escribir tortuga@máquina de escribir tortuga}

Las letras del alfabeto se pueden construir desde un número moderado
de elementos básicos, como líneas verticales y horizontales y unas pocas
curvas.  Diseña un alfabeto que pueda dibujarse con un número
mínimo de elementos básicos y luego escribe funciones que dibujen las letras.

Deberías escribir una función para cada letra, con nombres
\verb"dibujar_a", \verb"dibujar_b", etc., y poner tus funciones
en un archivo con nombre {\tt letras.py}.  Puedes descargar una
``máquina de escribir tortuga'' desde \url{http://thinkpython.com/code/typewriter.py}
para ayudarte a probar tu código.

Puedes obtener una solución en \url{http://thinkpython.com/code/letters.py};
también requiere
\url{http://thinkpython.com/code/polygon.py}.

\end{exercise}

\begin{exercise}

Lee sobre espirales en \url{http://en.wikipedia.org/wiki/Spiral}; luego
escribe un programa que dibuje una espiral arquimediana (o uno de los otros
tipos).  Solución: \url{http://thinkpython.com/code/spiral.py}.
\index{espiral}
\index{arquimediana, espiral}

\end{exercise}


\chapter{Condicionales y recursividad}

El tema principal de este capítulo es la sentencia {\tt if}, la cual
ejecuta código diferente dependiendo del estado del programa.
Pero primero quiero presentar dos operadores nuevos: división entera
y módulo.


\section{División entera y módulo}

El operador {\bf división entera}, \verb"//", divide
dos números y redondea a un entero hacia abajo.  Por ejemplo, supongamos que la
duración de una película es 105 minutos.  Quizás quieres saber
cuánto dura en horas.  La división convencional
devuelve un número de coma flotante:

\begin{verbatim}
>>> minutos = 105
>>> minutos / 60
    1.75
\end{verbatim}

Sin embargo, normalmente no escribimos las horas con decimales.  La división
entera devuelve el número entero de horas, redondeando:

\begin{verbatim}
>>> minutos = 105
>>> horas = minutos // 60
>>> horas
    1
\end{verbatim}

Para obtener el resto, podrías restar una hora en minutos:

\begin{verbatim}
>>> resto = minutos - horas * 60
>>> resto
    45
\end{verbatim}

\index{división entera}
\index{división de coma flotante}
\index{entera, división}
\index{coma flotante!división de}
\index{operador!de módulo}
\index{modulo@módulo!operador}

Una alternativa es usar el {\bf operador de módulo}, \verb"%", que
divide dos números y devuelve el resto.

\begin{verbatim}
>>> resto = minutos % 60
>>> resto
    45
\end{verbatim}
%
El operador de módulo es más útil de lo que parece.  Por
ejemplo, puedes verificar si un número es divisible por otro: si
{\tt x\%y} es cero, entonces {\tt x} es divisible por {\tt y}.
\index{divisibilidad}

Además, puedes extraer el dígito de más a la derecha
o más dígitos de un número.  Por ejemplo, {\tt x\%10} entrega el
dígito de más a la derecha de {\tt x} (en base 10).  De manera similar, {\tt x\%100}
entrega los dos últimos dígitos.

Si estás usando Python 2, la división funciona diferente.  El
operador división, \verb"/", realiza una división entera si ambos
operandos son enteros, y la división de coma flotante si cualquiera de
los dos operandos es un {\tt float}.
\index{Python 2}


\section{Expresión booleana}
\index{expresión!booleana}
\index{booleana, expresión}
\index{operador!lógico}
\index{logico, operador@lógico, operador}

Una {\bf expresión booleana} es una expresión que es verdadera
o falsa.  Los siguientes ejemplos usan el
operador {\tt ==}, el cual compara dos operandos y produce
{\tt True} si son iguales y {\tt False} si no lo son:

\begin{verbatim}
>>> 5 == 5
    True
>>> 5 == 6
    False
\end{verbatim}
%
{\tt True} y {\tt False} son valores
especiales que pertenecen al tipo {\tt bool}; no son cadenas:
\index{valor especial!True}
\index{valor especial!False}
\index{True, valor especial}
\index{False, valor especial}
\index{tipo!bool}
\index{bool, tipo}

\begin{verbatim}
>>> type(True)
    <class 'bool'>
>>> type(False)
    <class 'bool'>
\end{verbatim}
%
El operador {\tt ==} es uno de los {\bf operadores relacionales}; los
otros son:

\begin{verbatim}
      x != y               # x no es igual a y
      x > y                # x es mayor que y
      x < y                # x es menor que y
      x >= y               # x es mayor o igual que y
      x <= y               # x es menor o igual que y
\end{verbatim}
%
Aunque estas operaciones probablemente sean comunes para ti, los símbolos de Python
son diferentes a los símbolos matemáticos.  Un error común
es utilizar el signo igual simple ({\tt =}) en lugar de un signo igual doble
({\tt ==}).  Recuerda que {\tt =} es un operador de asignación y
{\tt ==} es un operador relacional.   No hay tal cosa como
{\tt =<} o {\tt =>}.
\index{operador!relacional}
\index{relacional, operador}


\section {Operadores lógicos}
\index{operador!lógico}
\index{logico, operador@lógico, operador}

Existen tres {\bf operadores lógicos}: {\tt and}, {\tt
or} y {\tt not}.  La semántica (significado) de estos operadores es
similar a su significado en inglés.  Por ejemplo,
{\tt x > 0 and x < 10} es verdadera solo si {\tt x} es mayor que 0
{\em y} menor que 10.
\index{operador!and}
\index{operador!or}
\index{operador!not}
\index{and, operador}
\index{or, operador}
\index{not, operador}

{\tt n\%2 == 0 or n\%3 == 0} es verdadera si {\em cualquiera o ambas}
condiciones son verdaderas, es decir, si el número es divisible por 2 {\em o}
3.

Finalmente, el operador {\tt not} niega una expresión
booleana, así que {\tt not (x > y)} es verdadera si {\tt x > y} es falsa,
es decir, si {\tt x} es menor o igual que {\tt y}.

Estrictamente hablando, los operandos de los operadores lógicos deberían ser
expresiones booleanas, pero Python no es muy estricto.
Cualquier número distinto de cero es interpretado como {\tt True}:

\begin{verbatim}
>>> 42 and True
    True
\end{verbatim}
%
Esta flexibilidad puede ser útil, pero hay algunas sutilezas
que podrían ser confusas.  Quizás quieras evitar esto (a menos que
sepas lo que estás haciendo).


\section{Ejecución condicional}
\label{conditional.execution}

\index{sentencia!condicional}
\index{condicional!sentencia}
\index{sentencia!if}
\index{if, sentencia}
\index{ejecución condicional}
Para escribir programas útiles, casi siempre necesitamos la capacidad
de verificar las condiciones y cambiar el comportamiento del programa
según corresponda.  Las {\bf sentencias condicionales} nos dan esta capacidad.  La
forma más simple es la sentencia {\tt if}:

\begin{verbatim}
if x > 0:
    print('x es positivo')
\end{verbatim}
%
La expresión booleana después de {\tt if} se
llama {\bf condición}.  Si es verdadera, se ejecutan las sentencias
con sangría.  Si no, no pasa nada.
\index{condición}
\index{sentencia!compuesta}
\index{compuesta, sentencia}

Las sentencias {\tt if} tienen la misma estructura que las definiciones de función:
un encabezado seguido de un cuerpo con sangrías.  Las sentencias como esta se
llaman {\bf sentencias compuestas}.

No hay límite en el número de sentencias que pueden aparecer en
el cuerpo, pero tiene que haber al menos una.
A veces, es útil tener un cuerpo sin sentencias (generalmente
para reservar lugar a código que no has escrito todavía).  En ese
caso, puedes usar la sentencia {\tt pass}, la cual no hace nada.
\index{sentencia!pass}
\index{pass, sentencia}

\begin{verbatim}
if x < 0:
    pass          # PENDIENTE: ¡falta manejar los valores negativos!
\end{verbatim}
%

\section{Ejecución alternativa}
\label{alternative.execution}
\index{ejecución alternativa}
\index{palabra clave!else}
\index{else, palabra clave}

Una segunda forma de la sentencia {\tt if} es la ``ejecución alternativa'',
en la cual hay dos posibilidades y la condición determina
cuál se ejecuta.  La sintaxis se ve así:

\begin{verbatim}
if x % 2 == 0:
    print('x es par')
else:
    print('x es impar')
\end{verbatim}
%
Si el resto de dividir {\tt x} por 2 es 0, entonces sabemos que
{\tt x} es par y el programa muestra un mensaje correspondiente.  Si
la condición es falsa, se ejecuta el segundo conjunto de sentencias.
Dado que la condición debe ser verdadera o falsa, se ejecutará exactamente
una de las alternativas.  Las alternativas se llaman {\bf
  ramas}, porque son ramas en el flujo de ejecución.
\index{rama}



\section{Condicionales encadenados}
\index{condicional encadenado}
\index{encadenado, condicional}

A veces hay más de dos posibilidades y necesitamos más de
dos ramas.  Una manera de expresar una computación como esa es un {\bf
condicional encadenado}:

\begin{verbatim}
if x < y:
    print('x es menor que y')
elif x > y:
    print('x es mayor que y')
else:
    print('x e y son iguales')
\end{verbatim}
%
{\tt elif} es una abreviación de ``else if''.  De nuevo, se ejecutará exactamente
una rama.  No hay límite en el número de sentencias {\tt
elif}.  Si hay una cláusula {\tt else}, tiene que estar
al final, pero no necesariamente tiene que haber una.
\index{palabra clave!elif}
\index{elif, palabra clave}

\begin{verbatim}
if opcion == 'a':
    dibujar_a()
elif opcion == 'b':
    dibujar_b()
elif opcion == 'c':
    dibujar_c()
\end{verbatim}
%
Cada condición es verificada en orden.  Si la primera es falsa,
se verifica la siguiente, y así sucesivamente.  Si una de ellas es
verdadera, se ejecuta la rama correspondiente y la sentencia
termina.  Incluso si más de una condición es verdadera, solo se ejecuta
la primera rama verdadera.


\section{Condicionales anidados}
\index{condicional anidado}
\index{anidado, condicional}

Un condicional puede también estar anidado dentro de otro.  Podríamos haber
escrito el ejemplo de la sección anterior de esta forma:

\begin{verbatim}
if x == y:
    print('x e y son iguales')
else:
    if x < y:
        print('x es menor que y')
    else:
        print('x es mayor que y')
\end{verbatim}
%
El condicional de más afuera contiene dos ramas.  La
primera rama contiene una sentencia simple.  La segunda rama
contiene otra sentencia {\tt if}, la cual tiene dos ramas
propias.  Aquellas dos ramas son sentencias simples,
aunque también podrían haber sido sentencias condicionales.

A pesar de que la sangría de las sentencias hacen evidente la
estructura, los {\bf condicionales anidados} se vuelven difíciles de leer
rápidamente.  Es una buena idea evitarlos cuando puedas.

Los operadores lógicos a menudo proporcionan una manera de simplificar las sentencias condicionales
anidadas.  Por ejemplo, podemos reescribir el siguiente código utilizando un
único condicional:

\begin{verbatim}
if 0 < x:
    if x < 10:
        print('x es un número positivo de un dígito.')
\end{verbatim}
%
La sentencia {\tt print} solo se ejecuta si pasamos por los dos
condicionales, así que podemos obtener el mismo efecto con el operador {\tt and}:

\begin{verbatim}
if 0 < x and x < 10:
    print('x es un número positivo de un dígito.')
\end{verbatim}

Para este tipo de condición, Python proporciona una opción más concisa:

\begin{verbatim}
if 0 < x < 10:
    print('x es un número positivo de un dígito.')
\end{verbatim}


\section{Recursividad}
\label{recursion}
\index{recursividad}

Es legal para una función llamar a otra función;
es legal también para una función llamarse a sí misma.  Puede que no sea obvia
la razón por la cual eso es una buena idea, pero resulta ser una de las cosas más
mágicas que puede hacer un programa.
Por ejemplo, mira la siguiente función:

\begin{verbatim}
def cuenta_reg(n):
    if n <= 0:
        print('¡Despegue!')
    else:
        print(n)
        cuenta_reg(n-1)
\end{verbatim}
%
Si {\tt n} es 0 o negativo, muestra la palabra, ``¡Despegue!''
De lo contrario, muestra a {\tt n} y luego llama a la función con nombre {\tt
cuenta\_reg} ---a sí misma--- pasando a {\tt n-1} como argumento.

¿Qué ocurre si llamamos a esta función así?

\begin{verbatim}
>>> cuenta_reg(3)
\end{verbatim}
%
La ejecución de {\tt cuenta\_reg} comienza con {\tt n=3}, y dado que
{\tt n} es mayor que 0, muestra el valor 3 y se llama a sí misma...

\begin{quote}
La ejecución de {\tt cuenta\_reg} comienza con {\tt n=2}, y dado que
{\tt n} es mayor que 0, muestra el valor 2 y se llama a sí misma...

\begin{quote}
La ejecución de {\tt cuenta\_reg} comienza con {\tt n=1}, y dado que
{\tt n} es mayor que 0, muestra el valor 1 y se llama a sí misma...

\begin{quote}
La ejecución de {\tt cuenta\_reg} comienza con {\tt n=0}, y dado que {\tt
n} no es mayor que 0, muestra la palabra ``¡Despegue!'' y luego
vuelve.
\end{quote}

La {\tt cuenta\_reg} que obtuvo {\tt n=1} vuelve.
\end{quote}

La {\tt cuenta\_reg} que obtuvo {\tt n=2} vuelve.
\end{quote}

La {\tt cuenta\_reg} que obtuvo {\tt n=3} vuelve.

Y luego estás de regreso en \verb"__main__".  Por lo tanto, la
salida completa se ve así:
\index{main}

\begin{verbatim}
3
2
1
¡Despegue!
\end{verbatim}
%
Una función que se llama a sí misma es {\bf recursiva}; el proceso de
ejecutarla se llama {\bf recursividad}.
\index{recursividad}
\index{función!recursiva}

Como ejemplo adicional, podemos escribir una función que imprima una
cadena {\tt n} veces.

\begin{verbatim}
def imprimir_n(s, n):
    if n <= 0:
        return
    print(s)
    imprimir_n(s, n-1)
\end{verbatim}
%
Si {\tt n <= 0}, la {\bf sentencia return} hace que se salga de la función.  El
flujo de ejecución vuelve inmediatamente a la sentencia llamadora y las líneas restantes
de la función no se ejecutan.

\index{sentencia!return}
\index{return, sentencia}

El resto de la función es similar a {\tt cuenta\_reg}: muestra a
{\tt s} y luego se llama a sí misma para mostrar a {\tt s} otras $n-1$
veces.  Entonces, el número de líneas de salida es {\tt 1 + (n - 1)}, lo cual
suma {\tt n}.

Para ejemplos simples como este, probablemente es más fácil utilizar un bucle {\tt
for}.  Sin embargo, más adelante veremos ejemplos que son difíciles de escribir
con un bucle {\tt for} y fáciles de escribir con recursividad, así que es
bueno comenzar pronto.
\index{bucle!for}
\index{for, bucle}


\section{Diagramas de pila para funciones recursivas}
\label{recursive.stack}
\index{diagrama!de pila}
\index{función!marco de}
\index{marco}

En la Sección~\ref{stackdiagram}, utilizamos un diagrama de pila para representar
el estado de un programa durante una llamada a función.  El mismo tipo de
diagrama puede ayudar a interpretar una función recursiva.

Cada vez que una función es llamada, Python crea un
marco que contiene las variables locales y los parámetros de la función.
Para una función recursiva, podría haber más de un marco en la
pila al mismo tiempo.

La Figura~\ref{fig.stack2} muestra un diagrama de pila para {\tt cuenta\_reg} llamada con
{\tt n = 3}.

\begin{figure}
\centerline
{\includegraphics[scale=0.8]{figs/stack2.pdf}}
\caption{Diagrama de pila.}
\label{fig.stack2}
\end{figure}


Como siempre, la parte de arriba de la pila es el marco para \verb"__main__".
Está vacío porque no creamos variables en
\verb"__main__" ni le pasamos argumentos.
\index{caso base}
\index{recursividad!caso base de}

Los cuatro marcos de {\tt cuenta\_reg} tienen valores diferentes para el
parámetro {\tt n}.  La parte de abajo de la pila, donde {\tt n=0}, se
llama {\bf caso base}.  No hace una llamada recursiva, así que
no hay más marcos.

Como ejercicio, dibuja un diagrama de pila para \verb"imprimir_n" llamada con
\verb"s = 'Hola'" y {\tt n=2}.
Luego, escribe una función llamada \verb"hacer_n" que tome un objeto de
función y un número, {\tt n}, como argumentos, y que llame
a dicha función {\tt n} veces.


\section{Recursividad infinita}
\index{recursividad!infinita}
\index{infinita, recursividad}
\index{error!de tiempo de ejecución}
\index{tiempo de ejecución, error de}
\index{rastreo}

Si una recursividad nunca llega a un caso base, sigue haciendo
llamadas recursivas por siempre y el programa nunca termina.  Esto se
conoce como {\bf recursividad infinita} y en general no es
una buena idea.  Aquí hay un programa mínimo con una recursividad infinita:

\begin{verbatim}
def recursivo():
    recursivo()
\end{verbatim}
%
En la mayoría de los entornos de programación, un programa con recursividad infinita
no se ejecuta realmente por siempre.  Python entrega un
mensaje de error cuando la recursividad alcanza la profundidad máxima:
\index{excepción!RuntimeError}
\index{RuntimeError}

\begin{verbatim}
  File "<stdin>", line 2, in recursivo
  File "<stdin>", line 2, in recursivo
  File "<stdin>", line 2, in recursivo
                  .
                  .
                  .
  File "<stdin>", line 2, in recursivo
RuntimeError: Maximum recursion depth exceeded
\end{verbatim}
%
Este rastreo es un poco más grande del que vimos en el
capítulo anterior.  Cuando el error ocurre, ¡hay 1000
marcos de {\tt recursivo} en la pila!

Si encuentras una recursividad infinita por accidente, revisa
tu función para confirmar que hay un caso base que no
hace una llamada recursiva.  Y si hay un caso base, verifica si
tienes garantizado alcanzarlo.


\section{Entrada de teclado}
\index{entrada de teclado}

Los programas que hemos escrito hasta ahora no admiten entradas del usuario.
Simplemente hacen siempre lo mismo.

Python proporciona una función incorporada llamada {\tt input} que
detiene el programa y
espera a que el usuario escriba algo.  Cuando el usuario presiona {\sf
  Return} o {\sf Enter}, el programa continúa e \verb"input"
devuelve como cadena lo que escribió el usuario.  En Python 2, la misma
función se llama \verb"raw_input".
\index{Python 2}
\index{función!input}
\index{input, función}

\begin{verbatim}
>>> texto = input()
    ¿Qué estás esperando?
>>> texto
    '¿Qué estás esperando?'
\end{verbatim}
%
Antes de obtener la entrada del usuario, es una buena idea imprimir
un mensaje que le diga al usuario qué escribir.  \verb"input" puede tomar un
mensaje como argumento:
\index{mensaje}

\begin{verbatim}
>>> nombre = input('¿Cuál...es tu nombre?\n')
    ¿Cuál...es tu nombre?
    ¡Arturo, Rey de los Britones!
>>> nombre
    '¡Arturo, Rey de los Britones!'
\end{verbatim}
%
La secuencia \verb"\n" al final del mensaje representa una {\bf
  nueva línea}, la cual es un carácter especial que provoca un salto de línea.
Esa es la razón por la cual la entrada del usuario aparece debajo del mensaje.  \index{nueva línea}

Si esperas que el usuario escriba un entero, puedes intentar convertir
el valor de retorno a {\tt int}:

\begin{verbatim}
>>> mensaje = '¿Cuál...es la velocidad media de una golondrina sin carga?\n'
>>> velocidad = input(mensaje)
    ¿Cuál...es la velocidad media de una golondrina sin carga?
    42
>>> int(velocidad)
    42
\end{verbatim}
%
Pero si el usuario escribe algo distinto a una cadena de dígitos,
obtienes un error:

\begin{verbatim}
>>> velocidad = input(mensaje)
    ¿Cuál...es la velocidad media de una golondrina sin carga?
    ¿De qué especie, de la africana o de la europea?
>>> int(velocidad)
    ValueError: invalid literal for int() with base 10
\end{verbatim}
%
Más adelante veremos cómo tratar este tipo de error.
\index{ValueError}
\index{excepción!ValueError}


\section{Depuración}
\label{whitespace}
\index{depuración}
\index{rastreo}

Cuando ocurre un error de sintaxis o de tiempo de ejecución, el mensaje de error contiene
mucha información, pero esto puede ser abrumador.  Las partes
más útiles suelen ser:

\begin{itemize}

\item Qué tipo de error fue y

\item Dónde ocurrió.

\end{itemize}

Los errores de sintaxis son generalmente fáciles de encontrar, pero hay algunas
trampas.  Los errores de espacio en blanco pueden ser complicados porque los espacios y
las sangrías son invisibles y estamos acostumbrados a ignorarlos.
\index{espacio en blanco}

\begin{verbatim}
>>> x = 5
>>>  y = 6
    File "<stdin>", line 1
       y = 6
       ^
    IndentationError: unexpected indent
\end{verbatim}
%
En este ejemplo, el problema es que la segunda línea está desajustada por
un espacio.  Pero el mensaje de error señala a {\tt y}, lo cual es
engañoso.  En general, los mensajes de error indican dónde fue descubierto
el problema, pero el error real podría estar antes en el código,
a veces en una línea anterior. Lo mismo ocurre con los errores de tiempo de ejecución. 
\index{tiempo de ejecución, error de}
\index{error!de tiempo de ejecución}

Supongamos que estás intentando
calcular una relación señal/ruido en decibeles.  La fórmula
es $RSR_{db} = 10 \log_{10} (P_{se\tilde{n}al} / P_{ruido})$.  En Python,
podrías escribir algo así:

\begin{verbatim}
import math
potencia_senal = 9
potencia_ruido = 10
relacion = potencia_senal // potencia_ruido
decibeles = 10 * math.log10(relacion)
print(decibeles)
\end{verbatim}
%
Cuando ejecutas este programa, obtienes una excepción:
%
\index{excepción!OverflowError}
\index{OverflowError}

\begin{verbatim}
Traceback (most recent call last):
  File "snr.py", line 5, in ?
    decibeles = 10 * math.log10(relacion)
ValueError: math domain error
\end{verbatim}
%
El mensaje de error indica la línea 5, pero no hay nada
malo con esa línea.  Para encontrar el error real, podría ser
útil imprimir el valor de {\tt relacion}, que resulta
ser 0.  El problema está en la línea 4, que utiliza división entera
en lugar de división de coma flotante.
\index{división entera}
\index{entera, división}

Deberías tomarte el tiempo de leer cuidadosamente los mensajes de error, pero no
supongas que todo lo que dice es correcto.


\section{Glosario}

\begin{description}

\item[división entera:] Un operador, denotado por {\tt //}, que divide dos
  números y redondea a un entero hacia abajo (en sentido hacia el infinito negativo).
  \index{división entera}
  \index{entera, división}

\item[operador de módulo:]  Un operador, denotado con un signo de porcentaje
({\tt \%}), que trabaja con enteros y devuelve el resto de
dividir un número por otro.
\index{operador!de módulo}
\index{modulo@módulo!operador}

\item[expresión booleana:]  Una expresión cuyo valor es
{\tt True} o {\tt False}.
\index{expresión!booleana}
\index{booleana, expresión}

\item[operador relacional:] Uno de los operadores que comprara
sus operandos: {\tt ==}, {\tt !=}, {\tt >}, {\tt <}, {\tt >=} y {\tt <=}.

\item[operador lógico:] Uno de los operadores que combina expresiones
booleanas: {\tt and}, {\tt or} y {\tt not}.

\item[sentencia condicional:]  Una sentencia que controla el flujo de
ejecución dependiendo de una condición.
\index{sentencia!condicional}
\index{condicional!sentencia}

\item[condición:] La expresión booleana en una sentencia condicional
que determina cuál rama se ejecuta.
\index{condición}

\item[sentencia compuesta:]  Una sentencia que consiste en un encabezado
y un cuerpo.  El encabezado termina con un signo de dos puntos (:).  El cuerpo tiene sangrías
relativas al encabezado.
\index{sentencia!compuesta}

\item[rama:] Una de las secuencias de sentencias alternativas en
una sentencia condicional.
\index{rama}

\item[condicional encadenado:]  Una sentencia condicional con una serie
de ramas alternativas.
\index{condicional encadenado}
\index{encadenado, condicional}

\item[condicional anidado:]  Una sentencia condicional que aparece
en una de las ramas de otra sentencia condicional.
\index{condicional anidado}
\index{anidado, condicional}

\item[sentencia return:] Una sentencia que provoca que una función
termine inmediatamente y vuelva a la sentencia llamadora.

\item[recursividad:]  El proceso de llamar a la función que ya se está
ejecutando.
\index{recursividad}

\item[caso base:]  Una rama condicional de una
función recursiva que no hace una llamada recursiva.
\index{caso base}

\item[recursividad infinita:]  Una recursividad que no tiene un
caso base, o nunca lo alcanza.  Eventualmente, una recursividad infinita
provoca un error de tiempo de ejecución.
\index{recursividad!infinita}

\end{description}

\section{Ejercicios}

\begin{exercise}

El módulo {\tt time} proporciona una función, con el mismo nombre {\tt time}, que
devuelve el tiempo transcurrido desde la Hora Media de Greenwich (GMT) en ``la época'' ({\em epoch}), que es
un momento arbitrario usado como punto de referencia.  En sistemas UNIX, la
época es el 1 de enero de 1970.

\begin{verbatim}
>>> import time
>>> time.time()
    1437746094.5735958
\end{verbatim}

Escribe un script que lea el tiempo actual y lo convierta a
una hora del día en horas, minutos y segundos, además del número de
días desde la época.

\end{exercise}


\begin{exercise}
\index{Ultimo Teorema de Fermat@Último Teorema de Fermat}

El Último Teorema de Fermat dice que no hay enteros positivos
$a$, $b$ y $c$ tales que
\[ a^n + b^n = c^n \]
%
para cualquier valor de $n$ mayor que 2.

\begin{enumerate}

\item Escribe una función con nombre \verb"comprobar_fermat" que tome cuatro
parámetros ---{\tt a}, {\tt b}, {\tt c} y {\tt n}--- y
compruebe si se cumple el teorema de Fermat.  Si
$n$ es mayor que 2 y
\[a^n + b^n = c^n \]
%
el programa debería imprimir ``¡Oh, no, Fermat se equivocó!''.
De lo contrario, el programa debería imprimir ``No, eso no funciona.''

\item Escribe una función que permita al usuario ingresar valores
para {\tt a}, {\tt b}, {\tt c} y {\tt n}, los convierta a
enteros y utilice la función \verb"comprobar_fermat" para comprobar si
violan el teorema de Fermat.

\end{enumerate}

\end{exercise}


\begin{exercise}
\index{triángulo}

Si te dan tres palos, podrías ser capaz o no de formar
un triángulo.  Por ejemplo, si uno de los palos mide 12 pulgadas
y los otros dos miden una pulgada, no serás
capaz de hacer que los palos cortos se encuentren en el medio.  Para tres
longitudes cualesquiera, hay una prueba simple para ver si es posible formar
un triángulo:

\begin{quotation}
Si cualquiera de las tres longitudes es mayor que la suma de las otras
  dos, entonces no puedes formar un triángulo.  De lo contrario, sí
  puedes.  (Si la suma de dos longitudes es igual a la tercera, forman
    lo que llaman un triángulo ``degenerado''.)
\end{quotation}

\begin{enumerate}

\item Escribe una función con nombre \verb"es_triangulo" que tome tres
  enteros como argumentos e imprima ``Sí'' o ``No'', dependiendo
  de si puedes o no formar un triángulo con palos cuyas longitudes sean
  los enteros dados.

\item Escribe una función que permita al usuario ingresar tres longitudes de
  palos, los convierta a enteros y utilice la función \verb"es_triangulo" para
  comprobar si los palos con las longitudes dadas pueden formar un triángulo.

\end{enumerate}

\end{exercise}

\begin{exercise}
¿Cuál es la salida del siguiente programa?
Dibuja un diagrama de pila que muestre el estado del programa
cuando imprime el resultado.

\begin{verbatim}
def recursivo(n, s):
    if n == 0:
        print(s)
    else:
        recursivo(n-1, n+s)

recursivo(3, 0)
\end{verbatim}

\begin{enumerate}

\item ¿Qué ocurriría si llamas a esta función así: {\tt
  recursivo(-1, 0)}?

\item Escribe un docstring que explique todo lo que alguien necesitaría
  saber para utilizar esta función (y nada más).

\end{enumerate}

\end{exercise}


Los siguientes ejercicios utilizan el módulo {\tt turtle}, descrito en el
Capítulo~\ref{turtlechap}:
%\index{Turtle}

\begin{exercise}

Lee la siguiente función y mira si puedes averiguar
lo que hace (mira los ejemplos en el Capítulo~\ref{turtlechap}).  Luego ejecútala
y mira si la entendiste bien.

\begin{verbatim}
def dibujar(t, longitud, n):
    if n == 0:
        return
    angulo = 50
    t.fd(longitud*n)
    t.lt(angulo)
    dibujar(t, longitud, n-1)
    t.rt(2*angulo)
    dibujar(t, longitud, n-1)
    t.lt(angulo)
    t.bk(longitud*n)
\end{verbatim}

\end{exercise}


\begin{figure}
\centerline
{\includegraphics[scale=0.8]{figs/koch.pdf}}
\caption{Una curva de Koch.}
\label{fig.koch}
\end{figure}

\begin{exercise}
\index{Koch, curva de}\index{curva de Koch}

La curva de Koch es un fractal que se ve más o menos como en la
Figura~\ref{fig.koch}.  Para dibujar una curva de Koch con longitud $x$, todo lo que
tienes que hacer es

\begin{enumerate}

\item Dibujar una curva de Koch con longitud $x/3$.

\item Girar 60 grados a la izquierda.

\item Dibujar una curva de Koch con longitud $x/3$.

\item Girar 120 grados a la derecha.

\item Dibujar una curva de Koch con longitud $x/3$.

\item Girar 60 grados a la izquierda.

\item Dibujar una curva de Koch con longitud $x/3$.

\end{enumerate}

La excepción es si $x$ es menor que 3: en ese caso,
puedes simplemente dibujar una línea recta con longitud $x$.

\begin{enumerate}

\item Escribe una función llamada {\tt koch} que tome una tortuga y
una longitud como parámetros y que use a la tortuga para dibujar una curva de Koch
con la longitud dada.

\item Escribe una función llamada {\tt copo\_de\_nieve} que dibuje tres
curvas de Koch que hagan el contorno de un copo de nieve.

Solución: \url{http://thinkpython.com/code/koch.py}.

\item La curva de Koch se puede generalizar en muchas formas.  Mira
\url{http://en.wikipedia.org/wiki/Koch_snowflake} para ejemplos e
implementa tu favorito.

\end{enumerate}
\end{exercise}


\chapter{Funciones productivas}
\label{fruitchap}

Muchas de las funciones de Python que hemos utilizado, tales como las funciones
matemáticas, producen valores de retorno. Sin embargo, las funciones que hemos escrito
son todas nulas: tienen un efecto, como imprimir un valor
o mover una tortuga, pero no tienen un valor de retorno.  En
este capítulo aprenderás a escribir funciones productivas.


\section{Valores de retorno}
\index{valor de retorno}

Al llamar a una función se genera un valor de
retorno, que usualmente asignamos a una variable o utilizamos como parte de una
expresión.

\begin{verbatim}
e = math.exp(1.0)
altura = radio * math.sin(radianes)
\end{verbatim}
%
Las funciones que hemos escrito hasta ahora son todas nulas.  Dicho en términos vagos,
no tienen valor de retorno; de manera más precisa,
su valor de retorno es {\tt None}.

En este capítulo, vamos a escribir (finalmente) funciones productivas.
El primer ejemplo es {\tt area}, que devuelve el área de un círculo
con un radio dado:

\begin{verbatim}
def area(radio):
    a = math.pi * radio**2
    return a
\end{verbatim}
%
Hemos visto la sentencia {\tt return} antes, pero en una función
productiva la sentencia {\tt return} incluye
una expresión.  Esta sentencia significa: ``Sal inmediatamente de
esta función y utiliza la siguiente expresión como valor de retorno.''
La expresión puede ser arbitrariamente complicada, por lo que podríamos
haber escrito esta función de manera más concisa:
\index{sentencia!return}
\index{return, sentencia}

\begin{verbatim}
def area(radio):
    return math.pi * radio**2
\end{verbatim}
%
Por otra parte, las {\bf variables temporales} como {\tt a} pueden hacer
más fácil la depuración.
\index{variable temporal}
\index{temporal, variable}

A veces es útil tener múltiples sentencias return, una en cada
rama de un condicional:

\begin{verbatim}
def valor_absoluto(x):
    if x < 0:
        return -x
    else:
        return x
\end{verbatim}
%
Dado que estas sentencias {\tt return} están en un condicional alternativo,
solo se ejecuta una.

Tan pronto como se ejecute una sentencia return, la función
termina sin ejecutar ninguna de las sentencias posteriores.
El código que aparece después de una sentencia {\tt return}, o cualquier otro lugar
que el flujo de ejecución nunca puede alcanzar, se llama
{\bf código muerto}.\index{codigo muerto@código muerto}

En una función productiva, es una buena idea asegurarse de
que cada camino posible a través del programa llegue a una
sentencia {\tt return}.  Por ejemplo:

\begin{verbatim}
def valor_absoluto(x):
    if x < 0:
        return -x
    if x > 0:
        return x
\end{verbatim}
%
Esta función es incorrecta porque si {\tt x} es 0,
ninguna condición es verdadera, y la función termina sin llegar a una
sentencia {\tt return}.  Si el flujo de ejecución llega al final
de una función, el valor de retorno es {\tt None}, lo cual no es
el valor absoluto de 0.
\index{valor especial!None}
\index{None, valor especial}

\begin{verbatim}
>>> print(valor_absoluto(0))
    None
\end{verbatim}
%
Por cierto, Python proporciona una función incorporada llamada
{\tt abs} que calcula valores absolutos.
\index{función!abs}
\index{abs, función}

Como ejercicio, escribe una función {\tt comparar} que
tome dos valores, {\tt x} e {\tt y}, y devuelva {\tt 1} si {\tt x > y},
devuelva {\tt 0} si {\tt x == y} o devuelva {\tt -1} si {\tt x < y}.
\index{función!comparar}
\index{comparar, función}


\section{Desarrollo incremental}
\label{incremental.development}
\index{plan de desarrollo!incremental}

A medida que vayas escribiendo funciones más grandes, podrías encontrarte
pasando más tiempo depurando.

Para lidiar con programas cada vez más complejos,
tal vez quieras intentar un proceso llamado
{\bf desarrollo incremental}.  El objetivo del desarrollo incremental
es evitar largas sesiones de depuración agregando y probando solo
un pedazo de código a la vez.
\index{pruebas en desarrollo incremental}
\index{Teorema de Pitágoras}

Como ejemplo, supongamos que quieres encontrar la distancia entre dos
puntos, dados por las coordenadas $(x_1, y_1)$ y $(x_2, y_2)$.
Por el teorema de Pitágoras, la distancia es:
\begin{displaymath}
\mathrm{distancia} = \sqrt{(x_2 - x_1)^2 + (y_2 - y_1)^2}
\end{displaymath}
%
El primer paso es considerar cómo debería ser una función de {\tt distancia}
en Python.  En otras palabras, ¿cuáles son las entradas (parámetros)
y cuál es la salida (valor de retorno)?

En este caso, las entradas son dos puntos, que puedes representar
utilizando cuatro números.  El valor de retorno es la distancia representada por
un valor de coma flotante.

Inmediatamente puedes escribir un esbozo de la función:

\begin{verbatim}
def distancia(x1, y1, x2, y2):
    return 0.0
\end{verbatim}
%
Obviamente, esta versión no calcula distancias: siempre devuelve
cero.  Pero es sintácticamente correcta, y funciona, lo cual significa que
puedes probarla antes de que la hagas más complicada.

Para probar la nueva función, llámala con argumentos de prueba:

\begin{verbatim}
>>> distancia(1, 2, 4, 6)
    0.0
\end{verbatim}
%
Escojo estos valores para que la distancia horizontal sea 3 y la
distancia vertical sea 4; de esa forma, el resultado es 5, la hipotenusa
de un triángulo rectángulo 3-4-5. Al probar una función, es
útil saber la respuesta correcta.
\index{pruebas sabiendo la respuesta}

En este punto hemos confirmado que la función es sintácticamente
correcta y podemos comenzar agregando código al cuerpo.
Un siguiente paso razonable es encontrar las diferencias
$x_2 - x_1$ e $y_2 - y_1$.  La siguiente versión almacena esos valores en
variables temporales y las imprime.

\begin{verbatim}
def distancia(x1, y1, x2, y2):
    dx = x2 - x1
    dy = y2 - y1
    print('dx es', dx)
    print('dy es', dy)
    return 0.0
\end{verbatim}
%
Si la función está bien, debería mostrar \verb"dx es 3" y
\verb"dy es 4".  Si es así, sabemos que la función obtiene los argumentos
correctos y realiza el primer cálculo de manera correcta.  Si no,
solo hay que revisar unas pocas líneas.

Luego calculamos la suma de los cuadrados de {\tt dx} y {\tt dy}:

\begin{verbatim}
def distance(x1, y1, x2, y2):
    dx = x2 - x1
    dy = y2 - y1
    dcuadrado = dx**2 + dy**2
    print('dcuadrado es: ', dcuadrado)
    return 0.0
\end{verbatim}
%
De nuevo, tendrías que ejecutar el programa en este punto y verificar la salida
(que debería ser 25).
Finalmente, puedes usar {\tt math.sqrt} para calcular y devolver el resultado:
\index{sqrt}
\index{función!sqrt}

\begin{verbatim}
def distancia(x1, y1, x2, y2):
    dx = x2 - x1
    dy = y2 - y1
    dcuadrado = dx**2 + dy**2
    resultado = math.sqrt(dcuadrado)
    return resultado
\end{verbatim}
%
Si eso funciona de manera correcta, estás listo.  De lo contrario, tal vez
quieras imprimir el valor de {\tt resultado} antes de la sentencia
return.

La versión final de la función no muestra nada cuando se
ejecuta: solo devuelve un valor.  Las sentencias {\tt print} que escribimos
son útiles para depurar, pero una vez que la función esté bien, deberías
borrarlas.  Un código como ese se llama {\bf andamiaje} (en inglés, {\em scaffolding})
porque es útil para construir el programa pero no es parte del
producto final.
\index{scaffolding}\index{andamiaje}

Cuando empieces, deberías agregar solo una o dos líneas de código a la
vez.  A medida que ganes experiencia, podrías encontrarte escribiendo
y depurando partes más grandes.  De cualquier manera, el desarrollo incremental
puede ahorrarte mucho tiempo de depuración.

Los aspectos clave del proceso son:

\begin{enumerate}

\item Comienza con un programa que funcione y haz pequeños cambios incrementales.
En cualquier punto, si hay un error, deberías tener una buena idea de
dónde está.

\item Utiliza variables que guarden valores intermedios de tal manera que puedas
mostrarlos y verificarlos.

\item Una vez que el programa funciona, tal vez quieras borrar algo del
andamiaje o consolidar varias sentencias en una expresión
compuesta, pero solo si no hace al programa difícil de
leer.

\end{enumerate}

Como ejercicio, utiliza desarrollo incremental para escribir una función
llamada {\tt hipotenusa} que devuelva el largo de la hipotenusa de un
triángulo rectánculo, dadas las longitudes de los otros dos lados como argumentos.
Guarda cada etapa del proceso de desarrollo a medida que avances.
\index{hipotenusa}



\section{Composición}
\index{composición}
\index{composición de funciones}

Como ya deberías esperar, puedes llamar a una función desde dentro de
otra.  Como ejemplo, escribiremos una función que tome dos puntos,
el centro de un círculo y un punto de su perímetro, y calcule
el área del círculo.

Supongamos que el punto central se almacena en las variables {\tt xc} e
{\tt yc}, y el punto del perímetro está en {\tt xp} e {\tt yp}. El
primer paso es encontrar el radio del círculo, que es la distiancia
entre los dos puntos.  Acabamos de escribir una función, {\tt
distancia}, que hace eso:

\begin{verbatim}
radio = distancia(xc, yc, xp, yp)
\end{verbatim}
%
El siguiente paso es encontrar el área de un círculo con ese radio;
también escribimos eso:

\begin{verbatim}
resultado = area(radio)
\end{verbatim}
%
Encapsulando estos pasos en una función, obtenemos:
\index{encapsulamiento}

\begin{verbatim}
def area_circulo(xc, yc, xp, yp):
    radio = distancia(xc, yc, xp, yp)
    resultado = area(radio)
    return resultado
\end{verbatim}
%
Las variables temporales {\tt radio} y {\tt resultado} son útiles para
el desarrollo y la depuración, pero una vez que el programa funciona, podemos
hacerlo más conciso componiendo las llamadas a funciones:

\begin{verbatim}
def area_circulo(xc, yc, xp, yp):
    return area(distancia(xc, yc, xp, yp))
\end{verbatim}
%

\section{Funciones booleanas}
\label{boolean}

Las funciones pueden devolver booleanos, lo cual es a menudo conveniente para ocultar
pruebas complicadas dentro de las funciones.  \index{función!booleana}
Por ejemplo:

\begin{verbatim}
def es_divisible(x, y):
    if x % y == 0:
        return True
    else:
        return False
\end{verbatim}
%
Es común darle a las funciones booleanas nombres que suenen como preguntas
sí/no; \verb"es_divisible" devuelve {\tt True} o {\tt False}
para indicar si {\tt x} es divisible por {\tt y}.

Aquí hay un ejemplo:

\begin{verbatim}
>>> es_divisible(6, 4)
    False
>>> es_divisible(6, 3)
    True
\end{verbatim}
%
El resultado del operador {\tt ==} es un booleano, así que podemos escribir la
función de manera más concisa devolviéndolo directamente:

\begin{verbatim}
def es_divisible(x, y):
    return x % y == 0
\end{verbatim}
%
Las funciones booleanas se utilizan a menudo en las sentencias condicionales:
\index{sentencia!condicional}
\index{condicional!sentencia}

\begin{verbatim}
if es_divisible(x, y):
    print('x es divisible por y')
\end{verbatim}
%
Puede ser tentador escribir algo como:

\begin{verbatim}
if es_divisible(x, y) == True:
    print('x es divisible por y')
\end{verbatim}
%
Pero la comparación extra es innecesaria.

Como ejercicio, escribe una función \verb"esta_entre(x, y, z)" que
devuelva {\tt True} si $x \le y \le z$ o {\tt False} si no.


\section{Más recursividad}
\label{more.recursion}
\index{recursividad}
\index{lenguaje!Turing completo}
\index{Turing completo, lenguaje}
\index{Turing, Alan}
\index{Tesis de Turing}

Hemos cubierto solo un pequeño subconjunto de Python, pero tal vez
te interese saber que este subconjunto es un lenguaje de programación {\em completo},
lo cual significa que cualquier cosa que pueda ser
calculada se puede expresar en este lenguaje.  Cualquier programa alguna vez escrito
puede ser reescrito utilizando solo las características del lenguaje que has aprendido
hasta ahora (en realidad, necesitarías unos pocos comandos para controlar dispositivos
como el mouse, discos, etc., pero eso es todo).

Probar ese aserto es un ejercicio no trivial realizado por primera vez por Alan
Turing, uno de los primeros informáticos (algunos discuten que
fue un matemático, pero muchos de los primeros informáticos comenzaron como
matemáticos).  En consecuencia, se conoce como Tesis de Turing.
Para un tratamiento más completo (y preciso) de la Tesis de Turing,
recomiendo el libro de Michael Sipser {\em Introduction to the
Theory of Computation}.

Para darte una idea de lo que puedes hacer con las herramientas que has aprendido
hasta ahora, evaluaremos algunas funciones matemáticas definidas de manera
recursiva.  Una definición recursiva es similar a una definición
circular, en el sentido de que la definición contiene una referencia a lo que
se está definiendo.  Una definición verdaderamente circular no es muy
útil:

\begin{description}

\item[vorpal:] Un adjetivo utilizado para describir algo que es vorpal.
\index{vorpal}
\index{definición circular}
\index{circular, definición}

\end{description}

Si vieras esa definición en el diccionario, quizás te moleste. Por
otra parte, si buscaras la definición de la función
factorial, denotada con el símbolo $!$, podrías obtener algo
así:
%
\begin{eqnarray*}
&&  0! = 1 \\
&&  n! = n (n-1)!
\end{eqnarray*}
%
Esta definición dice que el factorial de 0 es 1 y el factorial de
cualquier otro valor, $n$, es $n$ multiplicado por el factorial de $n-1$.

Así que $3!$ es 3 por $2!$, lo cual es 2 por $1!$, lo cual es 1 por
$0!$. Poniéndolo todo junto, $3!$ es igual a 3 por 2 por 1 por 1,
lo cual es 6.
\index{función!factorial}
\index{factorial!función}
\index{recursiva, definición}

Si puedes escribir una definición recursiva de algo, puedes
escribir un programa en Python para evaluarla.  El primer paso es decidir
cuáles deberían ser los parámetros.  En este caso, debería estar claro
que {\tt factorial} toma un entero:

\begin{verbatim}
def factorial(n):
\end{verbatim}
%
Si el argumento es 0, todo lo que tenemos que hacer es devolver 1:

\begin{verbatim}
def factorial(n):
    if n == 0:
        return 1
\end{verbatim}
%
De lo contrario, y esta es la parte interesante, tenemos que hacer una
llamada recursiva para encontrar el factorial de $n-1$ y luego multiplicarlo por
$n$:

\begin{verbatim}
def factorial(n):
    if n == 0:
        return 1
    else:
        recur = factorial(n-1)
        resultado = n * recur
        return resultado
\end{verbatim}
%
El flujo de ejecución para este programa es similar al flujo de {\tt
cuenta\_reg} de la Sección~\ref{recursion}.  Si llamamos a {\tt factorial}
con el valor 3:

Dado que 3 no es 0, tomamos la segunda rama y calculamos el factorial
de {\tt n-1}...

\begin{quote}
Dado que 2 no es 0, tomamos la segunda rama y calculamos el factorial de
{\tt n-1}...


  \begin{quote}
  Dado que 1 no es 0, tomamos la segunda rama y calculamos el factorial
  de {\tt n-1}...


    \begin{quote}
    Dado que 0 es igual a 0, tomamos la primera rama y devolvemos 1
    sin hacer más llamadas recursivas.
    \end{quote}


  El valor de retorno, 1, se multiplica por $n$, que es 1, y se
  devuelve el resultado.
  \end{quote}


El valor de retorno, 1, se multiplica por $n$, que es 2, y se
devuelve el resultado.
\end{quote}


El valor de retorno (2) se multiplica por $n$, que es 3, y el resultado, 6,
se convierte en el valor de retorno de la llamada a función que comenzó todo
el proceso.
\index{diagrama!de pila}

La Figura~\ref{fig.stack3} muestra cómo se ve el diagrama de pila
para esta sucesión de llamadas a función.

\begin{figure}
\centerline
{\includegraphics[scale=0.8]{figs/stack3.pdf}}
\caption{Diagrama de pila.}
\label{fig.stack3}
\end{figure}

Los valores de retorno se muestran volviendo hacia arriba en la pila.  En cada
marco, el valor de retorno es el valor de {\tt resultado}, que es el
producto de {\tt n} y {\tt recur}.
\index{función!marco de una}
\index{marco}

En el último marco, las variables
locales {\tt recur} y {\tt resultado} no existen, porque
la rama que las crea no se ejecuta.


\section{Salto de fe}
\index{recursividad}
\index{salto de fe}

Seguir el flujo de ejecución es una manera de leer programas, pero
puede volverse abrumador rápidamente.  Una
alternativa es lo que yo llamo ``salto de fe''.  Cuando llegas a una
llamada a función, en lugar de seguir el flujo de ejecución, {\em
supones} que la función es correcta y que devuelve el resultado
correcto.

De hecho, ya estás practicando este salto de fe cuando utilizas
funciones incorporadas.  Cuando llamas a {\tt math.cos} o {\tt math.exp},
no examinas los cuerpos de esas funciones.  Sólo
supones que funcionan porque las personas que escribieron las funciones
incorporadas eran buenos programadores.

Lo mismo es verdad cuando llamas a una de tus propias funciones.  Por
ejemplo, en la Sección~\ref{boolean}, escribimos una función llamada
\verb"es_divisible" que determina si un número es divisible por
otro.  Una vez que nos hemos convencido de que esta función es
correcta ---examinando el código y probándolo--- podemos utilizar la función
sin mirar el cuerpo otra vez.
\index{prueba!de salto de fe}

Lo mismo es verdad para las funciones recursivas.  Cuando llegues a la llamada
recursiva, en lugar de seguir el flujo de ejecución, deberías suponer
que la llamada recursiva funciona (devuelve el resultado correcto) y luego
preguntarte: ``suponiendo que puedo encontrar el factorial de $n-1$, ¿puedo
calcular el factorial de $n$?''  Está claro que
puedes, multiplicando por $n$.

Desde luego, es un poco extraño suponer que la función hace lo
correcto cuando no has terminado de escribirla, ¡pero es por eso
que se llama salto de fe!


\section{Un ejemplo más}
\label{one.more.example}

\index{función!de fibonacci}
\index{fibonacci, función de}
Después de {\tt factorial}, el ejemplo más común de una función
matemática definida de manera recursiva es {\tt fibonacci}, que tiene la
siguiente definición (ver
  \url{http://en.wikipedia.org/wiki/Fibonacci_number}):
%
\begin{eqnarray*}
&& \mathrm{fibonacci}(0) = 0 \\
&& \mathrm{fibonacci}(1) = 1 \\
&& \mathrm{fibonacci}(n) = \mathrm{fibonacci}(n-1) + \mathrm{fibonacci}(n-2)
\end{eqnarray*}
%
Traducido a Python, se ve así:

\begin{verbatim}
def fibonacci(n):
    if n == 0:
        return 0
    elif  n == 1:
        return 1
    else:
        return fibonacci(n-1) + fibonacci(n-2)
\end{verbatim}
%
Si intentas seguir el flujo de ejecución aquí, incluso para valores
de $n$ bastante pequeños, tu cabeza estalla.  Sin embargo, de acuerdo al
salto de fe, si supones que las dos llamadas recursivas
funcionan correctamente, entonces está claro que obtienes
el resultado correcto al sumarlas.
\index{flujo de ejecución}


\section{Verificar tipos}
\label{guardian}

¿Qué ocurre si llamamos a {\tt factorial} y le entregamos 1.5 como argumento?
\index{verificación de tipos}
\index{error!verificación de}\index{verificación de errores}
\index{factorial!función}\index{función!factorial}
\index{RuntimeError}

\begin{verbatim}
>>> factorial(1.5)
    RuntimeError: Maximum recursion depth exceeded
\end{verbatim}
%
Se ve como una recursividad infinita.  ¿Cómo puede ser?  La función
tiene un caso base: cuando {\tt n == 0}.  Pero si {\tt n} no es un entero,
podemos {\em perder} el caso base y seguir con la recursividad sin parar.
\index{recursividad!infinita}
\index{infinita, recursividad}

En la primera llamada recursiva, el valor de {\tt n} es 0.5.
En la siguiente, es -0.5.  Desde allí, se hace menor
(más negativo), pero nunca será 0.

Tenemos dos opciones.  Podemos intentar generalizar la función {\tt factorial}
para que funcione con números de coma flotante o podemos hacer que {\tt
  factorial} verifique el tipo de sus argumentos.  La primera opción se
llama la función gamma y está
un poco más allá del alcance de este libro.  Entonces iremos por la segunda.
\index{función!gamma}

Podemos utilizar la función incorporada {\tt isinstance} para verificar el tipo
del argumento.  Mientras estemos en ello, podemos también asegurarnos de que el
argumento sea positivo:
\index{función!isinstance}
\index{isinstance, función}

\begin{verbatim}
def factorial(n):
    if not isinstance(n, int):
        print('El factorial solo está definido para enteros.')
        return None
    elif n < 0:
        print('El factorial no está definido para enteros negativos.')
        return None
    elif n == 0:
        return 1
    else:
        return n * factorial(n-1)
\end{verbatim}
%
El primer caso base se encarga de los no enteros; el
segundo se  encarga de los enteros negativos.  En ambos casos, el programa imprime
un mensaje de error y devuelve {\tt None} para indicar que algo
anduvo mal:

\begin{verbatim}
>>> print(factorial('fred'))
    El factorial solo está definido para enteros.
    None
>>> print(factorial(-2))
    El factorial no está definido para enteros negativos.
    None
\end{verbatim}
%
Si pasamos ambas verificaciones, sabemos que $n$ es un entero no negativo, por lo que podemos probar que la recursividad termina.
\index{patrón!guardián}
\index{guardián, patrón}

Este programa demuestra un patrón a veces llamado {\bf guardián}.
Los primeros dos condicionales actúan como guardianes, protegiendo el código que
sigue de valores que podrían provocar un error.  Los guardianes hacen
posible probar la exactitud del código.

En la Sección~\ref{raise} veremos una alternativa más flexible que imprime
un mensaje de error: plantear una excepción.


\section{Depuración}
\label{factdebug}

Separar un programa grande en funciones pequeñas crea puntos de control
naturales para la depuración.  Si una función no está
funcionando, hay tres posibilidades a considerar:
\index{depuración}

\begin{itemize}

\item Hay algo mal en los argumentos que obtiene
la función: se viola una precondición.

\item Hay algo mal en la función: se viola una
postcondición.

\item Hay algo mal en el valor de retorno o la
manera en que se utiliza.

\end{itemize}

Para descartar la primera posibilidad, puedes agregar una sentencia {\tt print}
al comienzo de la función y mostrar los valores de los
parámetros (y quizás sus tipos).  O bien puedes escribir código
que verifique las precondiciones de manera explícita.
\index{precondición}
\index{postcondición}

Si los parámetros se ven bien, agrega una sentencia {\tt print} antes de cada
sentencia {\tt return} y muestra el valor de retorno.  Si es
posible, verifica el resultado a mano.  Considera llamar a la
función con valores que faciliten la verificación del resultado
(como en la Sección~\ref{incremental.development}).

Si la función parece funcionar, mira la llamada a función
para asegurarte de que el valor de retorno se esté utilizando correctamente (¡o si al
menos se está utilizando!).
\index{flujo de ejecución}

Agregar sentencias print al comienzo y al final de una función
puede ayudar a hacer más visible el flujo de ejecución.
Por ejemplo, aquí hay una versión de {\tt factorial} con
sentencias print:

\begin{verbatim}
def factorial(n):
    espacio = ' ' * (4 * n)
    print(espacio, 'factorial', n)
    if n == 0:
        print(espacio, 'devolviendo 1')
        return 1
    else:
        recursivo = factorial(n-1)
        resultado = n * recursivo
        print(espacio, 'devolviendo', resultado)
        return resultado
\end{verbatim}
%
{\tt espacio} es una cadena de caracteres de espacio que controla la
sangría de la salida.  Este es el resultado de {\tt factorial(4)} :

\begin{verbatim}
                 factorial 4
             factorial 3
         factorial 2
     factorial 1
 factorial 0
 devolviendo 1
     devolviendo 1
         devolviendo 2
             devolviendo 6
                 devolviendo 24
\end{verbatim}
%
Si estás confundido acerca del flujo de ejecución, este tipo de
salidas puede ser útil.  Desarrollar andamiaje eficaz toma algo de
tiempo, pero un poco de andamiaje puede ahorrar mucha depuración.


\section{Glosario}

\begin{description}

\item[variable temporal:]  Una variable utilizada para almacenar un valor intermedio en
una computación compleja.
\index{variable temporal}
\index{temporal, variable}

\item[código muerto:]  Parte de un programa que nunca se ejecuta, a menudo debido a que
aparece después de una sentencia {\tt return}.
\index{codigo muerto@código muerto}

\item[desarrollo incremental:]  Un plan de desarrollo de programa destinado a
evitar la depuración agregando y probando solo
un pedazo de código a la vez.
\index{desarrollo incremental}

\item[andamiaje:]  Código que se utiliza durante el desarrollo de un programa pero
no es parte de la versión final.
\index{scaffolding}\index{andamiaje}

\item[guardián:]  Un patrón de programación que utiliza una sentencia
condicional para verificar y encargarse de circunstancias que
podrían provocar un error.
\index{patrón!guardián}
\index{guardián, patrón}

\end{description}


\section{Ejercicios}

\begin{exercise}

Dibuja un diagrama de pila para el siguiente programa.  ¿Qué imprime el programa?
\index{diagrama!de pila}

\begin{verbatim}
def b(z):
    prod = a(z, z)
    print(z, prod)
    return prod

def a(x, y):
    x = x + 1
    return x * y

def c(x, y, z):
    total = x + y + z
    cuadrado = b(total)**2
    return cuadrado

x = 1
y = x + 1
print(c(x, y+3, x+y))
\end{verbatim}

\end{exercise}


\begin{exercise}
\label{ackermann}

La función de Ackermann, $A(m, n)$, se define:
\begin{eqnarray*}
A(m, n) = \begin{cases}
              n+1 & \mbox{si } m = 0 \\
        A(m-1, 1) & \mbox{si } m > 0 \mbox{ y } n = 0 \\
A(m-1, A(m, n-1)) & \mbox{si } m > 0 \mbox{ y } n > 0.
\end{cases}
\end{eqnarray*}
%
Ver \url{http://en.wikipedia.org/wiki/Ackermann_function}.
Escribe una función con nombre {\tt ack} que evalúe la función de Ackermann.
Utiliza tu función para evaluar {\tt ack(3, 4)}, que debería ser 125.
¿Qué ocurre para valores más grandes de {\tt m} y {\tt n}?
Solución: \url{http://thinkpython.com/code/ackermann.py}.
\index{Ackermann, función de}
\index{función!ack}

\end{exercise}


\begin{exercise}
\label{palindrome}

Un palíndromo es una palabra que se deletrea igual hacia atrás y
hacia adelante, como ``noon'' y ``redivider''.  De manera recursiva, una palabra
es un palíndromo si la primera y la última letra son la misma
y el medio es un palíndromo.
\index{palíndromo}

Las siguientes son funciones que toman una cadena como argumento y
devuelven las letras primera, última y del medio:

\begin{verbatim}
def primera(palabra):
    return palabra[0]

def ultima(palabra):
    return palabra[-1]

def medio(palabra):
    return palabra[1:-1]
\end{verbatim}
%
Veremos cómo funcionan en el Capítulo~\ref{strings}.

\begin{enumerate}

\item Escribe estas funciones en un archivo con nombre {\tt palindromo.py}
y pruébalas.  ¿Qué ocurre si llamas a {\tt medio} con
una cadena de dos letras?  ¿De una letra?  ¿Qué pasa con la cadena
vacía, la cual se escribe \verb"''" y no contiene letras?

\item Escribe una función llamada \verb"es_palindromo" que tome
una cadena como argumento y devuelva {\tt True} si es un palíndromo
y {\tt False} si no.  Recuerda que puedes utilizar la
función incorporada {\tt len} para verificar la longitud de una cadena.

\end{enumerate}

Solución: \url{http://thinkpython.com/code/palindrome_soln.py}.

\end{exercise}

\begin{exercise}

Un número, $a$, es potencia de $b$ si es divisible por $b$
y además $a/b$ es potencia de $b$.  Escribe una función llamada
\verb"es_potencia" que tome parámetros {\tt a} y {\tt b}
y devuelva {\tt True} si {\tt a} es potencia de {\tt b}.
Nota: tendrás que pensar en el caso base.

\end{exercise}


\begin{exercise}
\index{maximo comun divisor (MCD)@máximo común divisor (MCD)}
\index{MCD (máximo común divisor)}

El máximo común divisor (MCD) de $a$ y $b$ es el número más grande
que divide a ambos sin obtener resto.

Una manera de encontrar el MCD de dos números está basada en la observación
de que si $r$ es el resto de dividir $a$ por $b$, entonces $mcd(a,
b) = mcd(b, r)$.  Como caso base, podemos utilizar $mcd(a, 0) = a$.

Escribe una función llamada
\verb"mcd" que tome parámetros {\tt a} y {\tt b}
y devuelva su máximo común divisor.

Crédito: Este ejercicio está basado en un ejemplo de
{\em Structure and Interpretation of Computer Programs} de Abelson y Sussman.

\end{exercise}


\chapter{Iteración}

Este capítulo trata sobre la iteración, que es la capacidad de ejecutar
un bloque de sentencias de forma repetida.  Vimos un tipo de iteración,
utilizando recursividad, en la Sección~\ref{recursion}.
Vimos otro tipo, utilizando un bucle {\tt for},
en la Sección~\ref{repetition}.  Es este capítulo veremos otro
tipo, utilizando una sentencia {\tt while}.
Pero primero quiero contar un poco más sobre la asignación de variables.


\section{Reasignación}
\index{asignación}
\index{sentencia!de asignación}
\index{reasignación}

Como habrás descubierto, es legal hacer más de una
asignación a la misma variable.  Una nueva asignación hace que una variable existente
se refiera a un nuevo valor (y deje de referirse al valor antiguo).

\begin{verbatim}
>>> x = 5
>>> x
    5
>>> x = 7
>>> x
    7
\end{verbatim}
%
La primera vez que mostramos
{\tt x}, su valor es 5; la segunda vez, su
valor es 7.

La Figura~\ref{fig.assign2} muestra cómo se ve una {\bf reasignación}
en un diagrama de estado. \index{diagrama!de estado} \index{estado, diagrama de}

En este punto quiero abordar un origen común de
confusión.
Debido a que Python utiliza el signo igual ({\tt =}) para la asignación, es
tentador interpretar una sentencia del tipo {\tt a = b} como una
proposición
matemática de igualdad, es decir, la afirmación de que {\tt a} y
{\tt b} son iguales.  Sin embargo, esta interpretación es incorrecta.
\index{igualdad y asignación}

Primero, la igualdad es una relación simétrica y la asignación no lo es.  Por
ejemplo, en matemáticas, si $a=7$ entonces $7=a$.  Pero en Python, la
sentencia {\tt a = 7} es legal y {\tt 7 = a} no lo es.

Además, en matemáticas, una proposición de igualdad es verdadera o
falsa todo el tiempo.  Si $a=b$ ahora, entonces $a$ siempre será igual a $b$.
En Python, una sentencia de asignación puede igualar dos variables, pero
no tienen que permanecer iguales:

\begin{verbatim}
>>> a = 5
>>> b = a    # a y b son iguales ahora
>>> a = 3    # a y b ya no son iguales
>>> b
    5
\end{verbatim}
%
La tercera línea cambia el valor de {\tt a} pero no cambia el
valor de {\tt b}, por lo que ya no son iguales.

La reasignación de variables es a menudo útil, pero deberías utilizarla
con precaución.  Si los valores de las variables cambian frecuentemente, puede
hacer que el código sea difícil de leer y depurar.

\begin{figure}
\centerline
{\includegraphics[scale=0.8]{figs/assign2.pdf}}
\caption{Diagrama de estado.}
\label{fig.assign2}
\end{figure}



\section{Actualizar variables}
\label{update}

\index{actualizar}
\index{variable!actualizar}

Un tipo común de reasignación es la {\bf actualización},
donde el nuevo valor de la variable depende del antiguo.

\begin{verbatim}
>>> x = x + 1
\end{verbatim}
%
Esto significa ``obten el valor actual de {\tt x}, suma uno, y luego
actualiza a {\tt x} con el nuevo valor.''

Si intentas actualizar una variable que no existe, obtienes un
error, debido a que Python evalúa el lado derecho antes de asignar
un valor a {\tt x}:

\begin{verbatim}
>>> x = x + 1
    NameError: name 'x' is not defined
\end{verbatim}
%
Antes de que puedas actualizar una variable, la tienes que {\bf inicializar},
generalmente con una asignación simple:
\index{inicialización (antes de actualizar)}

\begin{verbatim}
>>> x = 0
>>> x = x + 1
\end{verbatim}
%
Actualizar una variable sumando 1 se llama {\bf incremento};
restar 1 se llama {\bf decremento}.
\index{incremento}
\index{decremento}




\section{La sentencia {\tt while}}
\index{sentencia!while}
\index{bucle while}
\index{while, bucle}
\index{iteración}

Los computadores a menudo se utilizan para automatizar tareas repetitivas.  Repetir
tareas idénticas o similares sin cometer errores es algo que
los computadores hacen bien y las personas hacen mal.  En un programa de computador,
la repetición también se llama {\bf iteración}.

Ya hemos visto dos funciones, {\tt cuenta\_reg} e
\verb"imprimir_n", que iteran utilizando recursividad.  Debido a que la iteración es tan
común, Python proporciona características del lenguaje que la hacen más fácil.
Una es la sentencia {\tt for} que vimos en la Sección~\ref{repetition}.
Volveremos a eso más adelante.

Otra es la sentencia {\tt while}.  Aquí hay una versión de {\tt
cuenta\_reg} que utiliza una sentencia {\tt while}:

\begin{verbatim}
def cuenta_reg(n):
    while n > 0:
        print(n)
        n = n - 1
    print('¡Despegue!')
\end{verbatim}
%
Casi puedes leer la sentencia {\tt while} como si fuera inglés.
Significa, ``Mientras {\tt n} sea mayor que 0,
muestra el valor de {\tt n} y luego decrementa
{\tt n}.  Cuando llegues a 0, muestra la palabra {\tt ¡Despegue!}''
\index{flujo de ejecución}

De manera más formal, aquí está el flujo de ejecución para una sentencia {\tt while}:

\begin{enumerate}

\item Determinar si la condición es verdadera o falsa.

\item Si es falsa, salir de la sentencia {\tt while}
y continuar con la ejecución de la siguiente sentencia.

\item Si la condición es verdadera, ejecutar el
cuerpo y luego volver al paso 1.

\end{enumerate}

Este tipo de flujo se llama bucle porque el tercer paso
hace que vuelva hacia arriba.
\index{condición}
\index{bucle}
\index{cuerpo}

El cuerpo del bucle debería cambiar el valor de una o más variables
de modo que la condición se vuelva falsa eventualmente y el bucle
termine.  De lo contrario, el bucle se repetirá por siempre, lo cual se llama
{\bf bucle infinito}.  Una fuente interminable de diversión para
informáticos es la observación de que las instrucciones en un champú,
``Enjabonar, enjuagar, repetir'', son un bucle infinito.
\index{bucle!infinito}
\index{infinito, bucle}

En el caso de {\tt cuenta\_reg}, podemos probar que el bucle
termina: si {\tt n} es cero o negativo, el bucle nunca se ejecuta.
De lo contrario, {\tt n} se hace más pequeño cada vez que se pasa por
el bucle, por lo que eventualmente tenemos que llegar a 0.

Para otros bucles, no es tan fácil decirlo.  Por ejemplo:

\begin{verbatim}
def sucesion(n):
    while n != 1:
        print(n)
        if n % 2 == 0:        # n es par
            n = n / 2
        else:                 # n es impar
            n = n*3 + 1
\end{verbatim}
%
La condición para este bucle es {\tt n != 1}, así que el bucle continuará
hasta que {\tt n} sea {\tt 1}, lo cual hace que la condición sea falsa.

En cada paso por el bucle, el programa muestra el valor de {\tt n}
y luego verifica si es par o impar.  Si es par, {\tt n} se
divide por 2.  Si es impar, el valor de {\tt n} se reemplaza por
{\tt n*3 + 1}. Por ejemplo, si el argumento pasado a {\tt sucesion}
es 3, los valores resultantes de {\tt n} son 3, 10, 5, 16, 8, 4, 2, 1.

Dado que {\tt n} a veces aumenta y a veces disminuye, no hay
demostración obvia de que {\tt n} alcanzará el 1 alguna vez, o de que el programa
termina.  Para algunos valores particulares de {\tt n}, podemos probar que
termina.  Por ejemplo, si el valor inicial es una potencia de dos,
{\tt n} será par cada vez que se pase por el bucle
hasta que alcance el 1. El ejemplo anterior termina con tal sucesión,
comenzando con 16.
\index{Collatz, conjetura de}\index{conjetura de Collatz}

La pregunta difícil es si podemos probar que este programa termina
para {\em todos} los valores positivos de {\tt n}.  Hasta ahora, ¡nadie ha
sido capaz de probarlo {\em o} refutarlo!  (Ver
  \url{http://en.wikipedia.org/wiki/Collatz_conjecture}.)

Como ejercicio, reescribe la función \verb"print_n" de la
Sección~\ref{recursion} utilizando iteración en lugar de recursividad.


\section{{\tt break}}
\index{sentencia!break}
\index{break, sentencia}

A veces no sabes que es momento de terminar un bucle hasta que llegas a la mitad
del cuerpo.  En ese caso puedes utilizar la sentencia {\tt break}
para saltar hacia afuera del bucle.

Por ejemplo, supongamos que quieres tomar la entrada del usuario hasta que
se escriba {\tt listo}.    Podrías escribir:

\begin{verbatim}
while True:
    linea = input('> ')
    if linea == 'listo':
        break
    print(linea)

print('¡Listo!')
\end{verbatim}
%
La condición del bucle es {\tt True}, lo cual siempre es verdadero, así que el
bucle se ejecuta hasta que llega a la sentencia {\tt break}.

En cada paso, solicita la entrada del usuario con un paréntesis angular.
Si el usuario escribe {\tt listo}, la sentencia {\tt break} hace que se salga
del bucle.  De lo contrario, el programa repite lo que escriba el usuario
y regresa a la parte superior del bucle.  Aquí hay una ejecución de muestra:

\begin{verbatim}
> no listo
no listo
> listo
¡Listo!
\end{verbatim}
%
Esta forma de escribir bucles {\tt while} es común porque puedes
verificar la condición en cualquier lugar del bucle (no solo en la
parte superior) y puedes expresar la condición de detención de manera afirmativa
(``detente cuando esto ocurra'') en lugar de negativa (``continúa
hasta que eso no ocurra'').


\section{Raíces cuadradas}
\label{squareroot}
\index{raiz cuadrada@raíz cuadrada}

Los bucles se utilizan a menudo en programas que calculan
resultados numéricos iniciando con una respuesta aproximada y
mejorándola iterativamente.
\index{metodo@método!de Newton}

Por ejemplo, una manera de calcular raíces cuadradas es el método de Newton.
Supongamos que quieres saber la raíz cuadrada de $a$.  Si comienzas
con casi cualquier estimación, $x$, puedes calcular una mejor
estimación con la siguiente fórmula:

\[ y = \frac{x + a/x}{2} \]
%
Por ejemplo, si $a$ es 4 y $x$ es 3:

\begin{verbatim}
>>> a = 4
>>> x = 3
>>> y = (x + a/x) / 2
>>> y
    2.16666666667
\end{verbatim}
%
El resultado está más cerca de la respuesta correcta ($\sqrt{4} = 2$).  Si
repetimos el proceso con una nueva estimación, se acerca aún más:

\begin{verbatim}
>>> x = y
>>> y = (x + a/x) / 2
>>> y
    2.00641025641
\end{verbatim}
%
Después de algunas actualizaciones más, la estimación es casi exacta:
\index{actualizar}

\begin{verbatim}
>>> x = y
>>> y = (x + a/x) / 2
>>> y
    2.00001024003
>>> x = y
>>> y = (x + a/x) / 2
>>> y
    2.00000000003
\end{verbatim}
%
En general, no sabemos de antemano cuántos pasos toma
llegar a la respuesta correcta, pero sabemos cuándo la obtenemos
porque la estimación
deja de cambiar:

\begin{verbatim}
>>> x = y
>>> y = (x + a/x) / 2
>>> y
    2.0
>>> x = y
>>> y = (x + a/x) / 2
>>> y
    2.0
\end{verbatim}
%
Cuando {\tt y == x}, podemos parar.  Aquí hay un bucle que comienza
con una estimación inicial, {\tt x}, y la mejora hasta que
deja de cambiar:

\begin{verbatim}
while True:
    print(x)
    y = (x + a/x) / 2
    if y == x:
        break
    x = y
\end{verbatim}
%
Para la mayoría de los valores de {\tt a} esto funciona bien, pero en general es
peligroso probar la igualdad de números {\tt float}.
Los valores de coma flotante son solo aproximadamente correctos:
la mayoría de los números racionales, como $1/3$, y los números irracionales, como
$\sqrt{2}$, no se pueden representar de manera exacta con un {\tt float}.
\index{coma flotante}
\index{epsilon}

En lugar de verificar si {\tt x} e {\tt y} son exactamente iguales, es
más seguro utilizar la función {\tt abs} para calcular el
valor absoluto, o magnitud, de la diferencia entre estos:

\begin{verbatim}
    if abs(y-x) < epsilon:
        break
\end{verbatim}
%
donde \verb"epsilon" tiene un valor como {\tt 0.0000001} que
determina qué tan cerca es lo suficientemente cerca.


\section{Algoritmos}
\index{algoritmo}

El método de Newton es un ejemplo de {\bf algoritmo}: es un
proceso mecánico para resolver una categoría de problemas (en este
caso, calcular raíces cuadradas).

Para entender qué es un algoritmo, quizás ayude comenzar con
algo que no es un algoritmo.  Cuando aprendiste a multiplicar
números de un solo dígito, probablemente memorizaste la tabla de multiplicar.
En realidad, memorizaste 100 soluciones específicas.  Esa clase de
conocimiento no es algorítmica.

Pero si eras ``perezoso'', podrías haber aprendido algunos
trucos.  Por ejemplo, para encontrar el producto de $n$ y 9, puedes
escribir $n-1$ como el primer dígito y $10-n$ como el segundo
dígito.  Este truco es una solución general para multiplicar cualquier
número de un solo dígito por 9.  ¡Eso es un algoritmo!
\index{suma con reserva}
\index{reserva, suma con}
\index{resta con préstamo}
\index{prestamo, resta con@préstamo, resta con}

Del mismo modo, las técnicas que aprendiste para la suma con reserva,
resta con préstamo y división larga son todas algoritmos.  Una
de las características de los algoritmos es que no requieren ninguna
inteligencia para realizarlos.  Son procesos mecánicos donde
cada paso sigue al último de acuerdo a un conjunto simple de reglas.

Ejecutar algoritmos es aburrido, pero diseñarlos es interesante,
intelectualmente desafiante y una parte central de las ciencias de la computación.

Algunas de las cosas que las personas hacen de manera natural, sin dificultad o
pensamiento consciente, son las más difíciles de expresar de manera algorítmica.
Entender un lenguaje natural es un buen ejemplo.  Todos lo hacemos, pero
hasta ahora nadie ha sido capaz de explicar {\em cómo} lo hacemos, al menos
no en la forma de un algoritmo.


\section{Depuración}
\label{bisectbug}

A medida que comiences a escribir programas más grandes, podrías encontrarte
ocupando más tiempo en la depuración.  Más código significa más posibilidades de
cometer un error y más lugares para esconder errores de programación.
\index{depuración!por bisección}
\index{bisección, depuración por}

Una manera de acortar tu tiempo de depuración es la ``depuración por bisección''.
Por ejemplo, si hay 100 líneas en tu programa y las
revisas una a la vez, tomaría 100 pasos.

En cambio, intenta separar el problema por la mitad.  Mira la mitad
del programa, o cerca de esta, para un valor intermedio que
puedas verificar.  Agrega una sentencia {\tt print} (o algo más
que tenga un efecto verificable) y ejecuta el programa.

Si la verificación en el punto medio es incorrecta, debe haber un problema en la
primera mitad del programa.  Si es correcta, el problema está
en la segunda mitad.

Cada vez que hagas una verificación como esta, reduces a la mitad el número
de líneas que tienes que buscar.  Después de seis pasos (lo cual es menos que 100),
estarías revisando una o dos líneas de código, al menos en teoría.

En la práctica, no siempre está claro cuál es
``la mitad del programa'' y no siempre es posible
verificarla.  No tiene sentido contar líneas y encontrar el
punto medio exacto.  En cambio, piensa en lugares
del programa donde podría haber errores y lugares donde es
fácil poner una verificación.  Luego, escoge un sitio donde
creas que las posibilidades de que el error esté antes o después
de la verificación son casi las mismas.




\section{Glosario}

\begin{description}

\item[reasignación:] Asignar un nuevo valor a una variable que
ya existe.
\index{reasignación}

\item[actualización:] Una asignación donde el nuevo valor de una variable
depende del antiguo.
\index{actualizar}

\item[inicialización:] Una asignación que le da un valor inicial a
una variable que será actualizada.
\index{inicialización de variable}

\item[incremento:] Una actualización que aumenta el valor de una variable
(generalmente en una unidad).
\index{incremento}

\item[decremento:] Una actualización que disminuye el valor de una variable.
\index{decremento}

\item[iteración:] Ejecución repetida de un conjunto de sentencias utilizando
una llamada a función recursiva o un bucle.
\index{iteración}

\item[bucle infinito:] Un bucle cuya condición para terminar nunca
se satisface.
\index{bucle!infinito}

\item[algoritmo:]  Un proceso general para resolver una categoría de
problemas.
\index{algoritmo}

\end{description}


\section{Ejercicios}

\begin{exercise}
\index{algoritmo de raíz cuadrada}

Copia el bucle de la Sección~\ref{squareroot}
y encapsúlalo en una función llamada
\verb"mi_sqrt" que tome a {\tt a} como parámetro, escoja un
valor razonable de {\tt x} y devuelva una estimación de la raíz
cuadrada de {\tt a}.  \index{encapsulamiento}

Para probarla, escribe una función con nombre \verb"probar_raiz_cuadrada"
que imprima una tabla como esta:

\begin{verbatim}
a   mi_sqrt(a)    math.sqrt(a)  diferencia
-   ----------    ------------  ----------
1.0 1.0           1.0           0.0
2.0 1.41421356237 1.41421356237 2.22044604925e-16
3.0 1.73205080757 1.73205080757 0.0
4.0 2.0           2.0           0.0
5.0 2.2360679775  2.2360679775  0.0
6.0 2.44948974278 2.44948974278 0.0
7.0 2.64575131106 2.64575131106 0.0
8.0 2.82842712475 2.82842712475 4.4408920985e-16
9.0 3.0           3.0           0.0
\end{verbatim}
%
La primera columna es un número, $a$; la segunda columna es la raíz
cuadrada de $a$ calculada con \verb"mi_sqrt"; la tercera columna es la
raíz cuadrada calculada por {\tt math.sqrt}; la cuarta columna es el
valor absoluto de la diferencia entre las dos estimaciones.
\end{exercise}


\begin{exercise}
\index{función!eval}
\index{eval, función}

La función incorporada {\tt eval} toma una cadena y la evalúa
utilizando el intérprete de Python.  Por ejemplo:

\begin{verbatim}
>>> eval('1 + 2 * 3')
    7
>>> import math
>>> eval('math.sqrt(5)')
    2.2360679774997898
>>> eval('type(math.pi)')
    <class 'float'>
\end{verbatim}
%
Escribe una función llamada \verb"bucle_eval" que, de manera iterativa,
solicite la entrada del usuario, tome la entrada resultante y la evalúe
utilizando {\tt eval}, e imprima el resultado.

Debería continuar hasta que el usuario ingrese \verb"'listo'" y luego
devuelva el valor de la última expresión que evaluó.

\end{exercise}


\begin{exercise}
\index{Ramanujan, Srinivasa}

El matemático Srinivasa Ramanujan encontró una
serie infinita
que se puede utilizar para generar una aproximación
numérica de $1 / \pi$:
\index{pi}

\[ \frac{1}{\pi} = \frac{2\sqrt{2}}{9801}
\sum^\infty_{k=0} \frac{(4k)!(1103+26390k)}{(k!)^4 396^{4k}} \]

Escribe una función llamada \verb"estimacion_pi" que utilice esta fórmula
para calcular y devolver una estimación de $\pi$.  Debería utilizar un bucle {\tt while}
para calcular términos de la sumatoria hasta que el último término sea
más pequeño que {\tt 1e-15} (que es la notación de Python para $10^{-15}$).
Puedes verificar el resultado comparándolo con {\tt math.pi}.

Solución: \url{http://thinkpython.com/code/pi.py}.

\end{exercise}


\chapter{Cadenas}
\label{strings}

Las cadenas no son como los enteros, los números de coma flotante y los booleanos.  Una cadena
es una {\bf secuencia}, lo cual significa que es
una colección ordenada de valores.  En este capítulo verás
cómo acceder a los caracteres que forman una cadena y
aprenderás sobre algunos de los métodos que proporcionan las cadenas.
\index{secuencia}


\section{Una cadena es una secuencia}

\index{secuencia}
\index{caracter@carácter}
\index{operador!de corchetes}
\index{corchetes!operador de}
Una cadena es una secuencia de caracteres.
Puedes acceder a los caracteres uno a la vez con el
operador de corchetes:

\begin{verbatim}
>>> fruta = 'banana'
>>> letra = fruta[1]
\end{verbatim}
%
La segunda sentencia selecciona el carácter número 1 de {\tt
fruta} y lo asigna a {\tt letra}.
\index{indice@índice}

La expresión en los corchetes se llama {\bf índice}.
El índice indica cuál carácter en la secuencia
quieres (de ahí el nombre).

Sin embargo, lo que obtienes podría no ser lo que esperas:

\begin{verbatim}
>>> letra
    'a'
\end{verbatim}
%
Para la mayoría de las personas, la primera letra de \verb"'banana'" es {\tt b}, no
{\tt a}.  Pero para los informáticos, el índice es un desplazamiento desde el
comienzo de la cadena, y el desplazamiento de la primera letra es cero.

\begin{verbatim}
>>> letra = fruta[0]
>>> letra
    'b'
\end{verbatim}
%
Entonces {\tt b} es la 0-ésima letra  (``cero-ésima'') de \verb"'banana'", {\tt
  a} es la 1-ésima letra (``uno-ésima'') y {\tt n} es la 2-ésima letra
(``dos-ésima'').  \index{indice@índice!a partir de cero} \index{cero, índice
  a partir de}

Como índice, puedes utilizar una expresión que contenga variables y
operadores:
\index{indice@índice}

\begin{verbatim}
>>> i = 1
>>> fruta[i]
    'a'
>>> fruta[i+1]
    'n'
\end{verbatim}
%

Pero el valor del índice tiene que ser un entero.  De lo contrario,
obtienes:
\index{excepción!TypeError}
\index{TypeError}

\begin{verbatim}
>>> letra = fruta[1.5]
    TypeError: string indices must be integers
\end{verbatim}
%

\section{{\tt len}}
\index{función!len}
\index{len, función}

{\tt len} es una función incorporada que devuelve el número de caracteres
en una cadena:

\begin{verbatim}
>>> fruta = 'banana'
>>> len(fruta)
    6
\end{verbatim}
%
Para obtener la última letra de una cadena, quizás te tientes a intentar algo
así:
\index{excepción!IndexError}
\index{IndexError}

\begin{verbatim}
>>> longitud = len(fruta)
>>> ultima = fruta[longitud]
    IndexError: string index out of range
\end{verbatim}
%
El motivo del {\tt IndexError} es que no hay letra en {\tt
'banana'} con el índice 6.  Dado que comenzamos a contar desde cero, las
seis letras se enumeran del 0 al 5.  Para obtener el último carácter, tienes
que restar 1 a {\tt longitud}:

\begin{verbatim}
>>> ultima = fruta[longitud-1]
>>> ultima
    'a'
\end{verbatim}
%
O bien puedes utilizar índices negativos, que cuentan hacia atrás desde
el final de la cadena.  La expresión {\tt fruta[-1]} entrega la última
letra, {\tt fruta[-2]} entrega la penúltima, y así sucesivamente.
\index{indice@índice!negativo}
\index{negativo, índice}


\section{Recorrido con un bucle {\tt for}}
\label{for}
\index{recorrer}
\index{bucle!recorrido en}
\index{bucle!for}
\index{for, bucle}
\index{sentencia!for}
\index{recorrer}

Muchas computaciones implican procesar una cadena trabajando un carácter a la
vez.  Generalmente parten al comienzo, seleccionan cada carácter en
turno, le hacen algo, y continúan hasta el final.  Este patrón de
procesamiento se llama {\bf recorrido}.  Una manera de escribir un recorrido
es con un bucle {\tt while}:

\begin{verbatim}
indice = 0
while indice < len(fruta):
    letra = fruta[indice]
    print(letra)
    indice = indice + 1
\end{verbatim}
%
Este bucle recorre la cadena y muestra cada letra en una
línea.  La condición del bucle es {\tt indice < len(fruta)}, entonces
cuando {\tt indice} es igual a la longitud de la cadena, la
condición es falsa y el cuerpo del bucle no se ejecuta.  El
último carácter accedido es el que tiene índice {\tt len(fruta)-1},
que es el último carácter en la cadena.

Como ejercicio, escribe una función que tome una cadena como argumento
y muestre las letras hacia atrás, una por línea.

Otra manera de escribir un recorrido es con un bucle {\tt for}:

\begin{verbatim}
for letra in fruta:
    print(letra)
\end{verbatim}
%
En cada paso por el bucle, el siguiente carácter en la cadena se asigna
a la variable {\tt letra}.  El bucle continúa hasta que no queden
caracteres.
\index{concatenación}
\index{abecedario}
\index{McCloskey, Robert}

El siguiente ejemplo muestra cómo utilizar la concatenación (suma de cadenas)
y un bucle {\tt for} para generar una serie abecedaria (es decir, en
orden alfabético).  En el libro de Robert McCloskey {\em Abran
paso a los patitos}, los nombres de los patitos son Jack, Kack, Lack,
Mack, Nack, Ouack, Pack, and Quack.  Este bucle muestra dichos nombres en
orden:

\begin{verbatim}
prefijos = 'JKLMNOPQ'
sufijo = 'ack'

for letra in prefijos:
    print(letra + sufijo)
\end{verbatim}
%
La salida es:

\begin{verbatim}
Jack
Kack
Lack
Mack
Nack
Oack
Pack
Qack
\end{verbatim}
%
Desde luego, eso no es del todo correcto porque ``Ouack'' y ``Quack'' están
mal escritos.  Como ejercicio, modifica el programa para arreglar este error.



\section{Trozos de cadena}
\label{slice}
\index{operador!de trozo} \index{trozo!operador} \index{indice@índice!de trozo}
\index{cadena!trozo de} \index{trozo de cadena} \index{slice}

Un segmento de una cadena se llama {\bf trozo} (en inglés, {\em slice}).  Seleccionar un trozo es
similar a seleccionar un carácter:

\begin{verbatim}
>>> s = 'Monty Python'
>>> s[0:5]
    'Monty'
>>> s[6:12]
    'Python'
\end{verbatim}
%
El operador {\tt [n:m]} devuelve la parte de la cadena desde el 
``n-ésimo'' carácter al ``m-ésimo'' carácter, incluyendo el primero pero
excluyendo el último.  Este comportamiento es contraintuitivo, pero tal vez
ayude imaginar los índices apuntando {\em entre} los
caracteres, como en la Figura~\ref{fig.banana}.

\begin{figure}
\centerline
{\includegraphics[scale=0.8]{figs/banana.pdf}}
\caption{Índices de trozo.}
\label{fig.banana}
\end{figure}

Si omites el primer índice (antes del signo de dos puntos), el trozo comienza al
principio de la cadena.  Si omites el segundo índice, el trozo
llega hasta el final de la cadena:

\begin{verbatim}
>>> fruta = 'banana'
>>> fruta[:3]
    'ban'
>>> fruta[3:]
    'ana'
\end{verbatim}
%
Si el primer índice es mayor o igual al segundo, el resultado
es una {\bf cadena vacía}, representada por dos comillas:
\index{comillas}

\begin{verbatim}
>>> fruta = 'banana'
>>> fruta[3:3]
    ''
\end{verbatim}
%
Una cadena vacía no contiene caracteres y tiene longitud 0, pero aparte
de eso, es lo mismo que cualquier otra cadena.

Continuando con este ejemplo, ¿qué crees que significa
{\tt fruta[:]}?  Pruébalo y mira.
\index{trozo!copia de}
\index{copia!de trozo}



\section{Las cadenas son inmutables}
\index{mutabilidad}
\index{inmutabilidad}
\index{cadena!inmutable}

Es tentador utilizar el operador {\tt []} en el lado izquierdo de una
asignación, con la intención de cambiar un carácter en una cadena.
Por ejemplo:
\index{TypeError}
\index{excepción!TypeError}

\begin{verbatim}
>>> saludo = 'Hola, mundo'
>>> saludo[0] = 'J'
    TypeError: 'str' object does not support item assignment
\end{verbatim}
%
El ``objeto'' en este caso es la cadena y el ``ítem'' es
el carácter que intentaste asignar.  Por ahora, un objeto es
lo mismo que un valor, pero refinaremos esa definición
más adelante (Sección~\ref{equivalence}).
\index{objeto}
\index{item@ítem}
\index{asignación de ítem}
\index{item@ítem!asignación de}
\index{inmutabilidad}

La razón del error es que
las cadenas son {\bf inmutables}, lo cual significa que no puedes cambiar una
cadena que ya existe.  Lo mejor que puedes hacer es crear una nueva cadena
que sea una variación de la original:

\begin{verbatim}
>>> saludo = 'Hola, mundo'
>>> nuevo_saludo = 'J' + saludo[1:]
>>> nuevo_saludo
    'Jola, mundo'
\end{verbatim}
%
Este ejemplo concatena una nueva primera letra con
un trozo de {\tt saludo}.  No tiene efecto en
la cadena original.
\index{concatenación}


\section{Buscar}
\label{find}

¿Qué hace la siguiente función?
\index{función!find}
\index{find, función}

\begin{verbatim}
def encontrar(palabra, letra):
    indice = 0
    while indice < len(palabra):
        if palabra[indice] == letra:
            return indice
        indice = indice + 1
    return -1
\end{verbatim}
%
En un sentido, {\tt encontrar} es la inversa del operador {\tt []}.
En lugar de tomar un índice y extraer el carácter correspondiente,
toma un carácter y encuentra el índice donde aparece ese
carácter.  Si el carácter no se encuentra, la función devuelve {\tt
-1}.

Este es el primer ejemplo que hemos visto de una sentencia {\tt return}
dentro de un bucle.  Si {\tt palabra[indice] == letra}, la función se sale
del bucle y devuelve inmediatamente.

Si el carácter no aparece en la cadena, el programa
termina el bucle de manera normal y devuelve {\tt -1}.

Este patrón de computación ---recorrer una secuencia y devolver
cuando encontramos lo que buscamos--- se llama {\bf búsqueda}.
\index{recorrer}
\index{patrón!de búsqueda}
\index{busqueda@búsqueda!patrón de}

Como ejercicio, modifica {\tt encontrar} para que tenga un
tercer parámetro: el índice en {\tt palabra} donde debería comenzar la
búsqueda.


\section{Bucles y conteo}
\label{counter}
\index{contador}
\index{conteo y bucles}
\index{bucles y conteo}
\index{bucle!con cadenas}

El siguiente programa cuenta el número de veces que aparece la letra {\tt a}
en una cadena:

\begin{verbatim}
palabra = 'banana'
contar = 0
for letra in palabra:
    if letra == 'a':
        contar = contar + 1
print(contar)
\end{verbatim}
%
Este programa demuestra otro patrón de computación llamado {\bf
contador}.  La variable {\tt contar} se inicializa en 0 y luego
se incrementa cada vez que se encuentra una {\tt a}.
Cuando el bucle termina, {\tt contar}
contiene el resultado: el número total de letras {\tt a}.

\index{encapsulamiento}
Como ejercicio, encapsula este código en una función con nombre {\tt
contar} y generalízalo para que acepte la cadena y la
letra como argumentos.

Luego reescribe la función de modo que, en lugar de
recorrer la cadena, utilice la versión de tres parámetros de {\tt
encontrar} de la sección anterior.


\section{Métodos de cadena}
\label{optional}

Las cadenas proporcionan métodos que realizan una variedad de operaciones útiles.
Un método es similar a una función ---toma argumentos y
devuelve un valor--- pero la sintaxis es diferente.  Por ejemplo, el
método {\tt upper} toma una cadena y devuelve una nueva cadena con
todas las letras mayúsculas.
\index{metodo@método}
\index{cadena!método de}

En lugar de la sintaxis de función {\tt upper(palabra)}, utiliza
la sintaxis de método {\tt palabra.upper()}.

\begin{verbatim}
>>> palabra = 'banana'
>>> nueva_palabra = palabra.upper()
>>> nueva_palabra
    'BANANA'
\end{verbatim}
%
Esta forma de notación de punto especifica el nombre del método, {\tt
upper}, y el nombre de la cadena a la cual se le aplica el método, {\tt
palabra}.  Los paréntesis vacíos indican que este método no toma
argumentos.
\index{paréntesis!vacíos}
\index{notación de punto}

Una llamada a un método se llama {\bf invocación}; en este caso,
diríamos que estamos invocando a {\tt upper} en {\tt palabra}.
\index{invocación}

Resulta que hay un método de cadena con nombre {\tt find} que
es notablemente similar a la función {\tt encontrar} que escribimos:

\begin{verbatim}
>>> palabra = 'banana'
>>> indice = palabra.find('a')
>>> indice
    1
\end{verbatim}
%
En este ejemplo, invocamos a {\tt find} en {\tt palabra} y pasamos
la letra que buscamos como parámetro.

En realidad, el método {\tt find} es más general que nuestra función;
puede encontrar subcadenas, no solo caracteres:

\begin{verbatim}
>>> palabra.find('na')
    2
\end{verbatim}
%
Por defecto, {\tt find} comienza al principio de la cadena, pero
puede tomar un segundo argumento, el índice donde debería comenzar:
\index{argumento opcional}
\index{opcional!argumento}

\begin{verbatim}
>>> palabra.find('na', 3)
    4
\end{verbatim}
%
Este es un ejemplo de {\bf argumento opcional};
{\tt find} puede
tomar también un tercer argumento, el índice donde debería detenerse:

\begin{verbatim}
>>> nombre = 'bob'
>>> nombre.find('b', 1, 2)
    -1
\end{verbatim}
%
Esta búsqueda falla porque {\tt b} no
aparece en el rango de índices desde {\tt 1} hasta {\tt 2}, sin incluir el {\tt
2}.  Buscar hasta el segundo índice, sin incluirlo, hace a
{\tt find} consistente con el operador de trozo.



\section{El operador {\tt in}}
\label{inboth}
\index{operador!in}
\index{in, operador}
\index{operador!booleano}
\index{booleano, operador}

La palabra {\tt in} es un operador booleano que toma dos cadenas y
devuelve {\tt True} si la primera aparece como una subcadena en la segunda:

\begin{verbatim}
>>> 'a' in 'banana'
    True
>>> 'semilla' in 'banana'
    False
\end{verbatim}
%
Por ejemplo, la siguiente función imprime todas las
letras de {\tt palabra1} que también aparecen en {\tt palabra2}:

\begin{verbatim}
def en_ambas(palabra1, palabra2):
    for letra in palabra1:
        if letra in palabra2:
            print(letra)
\end{verbatim}
%
Con nombres de variables bien escogidos,
Python a veces se lee como el inglés.  Podrías leer
este bucle, ``{\em para} (cada) letra {\em en} (la primera) palabra, {\em si} (la) letra
(aparece) {\em en} (la segunda) palabra, {\em imprimir} (la) letra.''

Esto es lo que obtienes si comparas uvas y tunas:

\begin{verbatim}
>>> en_ambas('uvas', 'tunas')
    u
    a
    s
\end{verbatim}
%

\section{Comparación de cadenas}
\index{cadenas, comparación de}
\index{comparación de cadenas}

Los operadores relacionales funcionan en las cadenas.  Para ver si dos cadenas son iguales:

\begin{verbatim}
if palabra == 'banana':
    print('Todo bien, bananas.')
\end{verbatim}
%
Otras operaciones relacionales son útiles para poner palabras en orden
alfabético:

\begin{verbatim}
if palabra < 'banana':
    print('Tu palabra, ' + palabra + ', viene antes de banana.')
elif palabra > 'banana':
    print('Tu palabra, ' + palabra + ', viene después de banana.')
else:
    print('Todo bien, bananas.')
\end{verbatim}
%
Python no maneja letras mayúsculas y minúsculas de la misma manera en que
lo hacen las personas.  Todas las letras mayúsculas vienen antes de todas las
letras minúsculas, entonces:

\begin{verbatim}
Tu palabra, Piña, viene antes de banana.
\end{verbatim}
%
Una manera común de abordar este problema es convertir las cadenas a un
formato estándar, por ejemplo todas minúsculas, antes de realizar
la comparación.  Ten eso en mente en caso de que tengas que defenderte
de un hombre armado con una Piña.


\section{Depuración}
\index{depuración}
\index{recorrer}

Cuando utilizas índices para recorrer los valores en una secuencia,
es difícil obtener el comienzo y final de un recorrido
de manera correcta.  Aquí hay una función que se supone que compara dos
palabras y devuelve {\tt True} si una de las palabras es el inverso
de la otra, pero contiene dos errores:

\begin{verbatim}
def es_inverso(palabra1, palabra2):
    if len(palabra1) != len(palabra2):
        return False

    i = 0
    j = len(palabra2)

    while j > 0:
        if palabra1[i] != palabra2[j]:
            return False
        i = i+1
        j = j-1

    return True
\end{verbatim}
%
La primera sentencia {\tt if} verifica si las palabras tienen la
misma longitud.  Si no, podemos devolver {\tt False} inmediatamente.
De lo contrario, para el resto de la función, podemos suponer que las palabras
tienen la misma longitud.  Este es un ejemplo del patrón guardián
de la Sección~\ref{guardian}.
\index{patrón!guardián}
\index{guardián, patrón}
\index{indice@índice}

{\tt i} y {\tt j} son índices: {\tt i} recorre a {\tt palabra1}
hacia adelante mientras {\tt j} recorre a {\tt palabra2} hacia atrás.  Si encontramos
dos letras que no coinciden, podemos devolver {\tt False} inmediatamente.
Si terminamos todo el bucle y todas las letras coinciden,
devolvemos {\tt True}.

Si probamos esta función con las palabras ``pots'' y ``stop'',
esperamos el valor de retorno {\tt True}, pero obtenemos un IndexError:
\index{IndexError}
\index{excepción!IndexError}

\begin{verbatim}
>>> es_inverso('pots', 'stop')
    ...
      File "inverso.py", line 15, in es_inverso
        if palabra1[i] != palabra2[j]:
    IndexError: string index out of range
\end{verbatim}
%
Para depurar este tipo de error, mi primer movimiento es
imprimir los valores de los índices inmediatamente antes de la línea
donde aparece el error.

\begin{verbatim}
    while j > 0:
        print(i, j)        # imprimir aquí

        if palabra1[i] != palabra2[j]:
            return False
        i = i+1
        j = j-1
\end{verbatim}
%
Ahora cuando ejecuto el programa de nuevo, obtengo más información:

\begin{verbatim}
>>> es_inverso('pots', 'stop')
    0 4
...
IndexError: string index out of range
\end{verbatim}
%
En el primer paso por el bucle, el valor de {\tt j} es 4,
lo cual está fuera de rango para la cadena \verb"'pots'".
El índice del último carácter es 3, por lo que el
valor inicial para {\tt j} debería ser {\tt len(palabra2)-1}.

Si arreglo este error y ejecuto el programa de nuevo, obtengo:

\begin{verbatim}
>>> es_inverso('pots', 'stop')
    0 3
    1 2
    2 1
    True
\end{verbatim}
%
Esta vez obtenemos la respuesta correcta, pero se ve como si el bucle solo se ejecutara
tres veces, lo cual es sospechoso.  Para obtener una mejor idea de lo que está
ocurriendo, es útil dibujar un diagrama de estado.  Durante la primera
iteración, el marco para \verb"es_inverso" se muestra en la
Figura~\ref{fig.state4}.  \index{diagrama!se estado} \index{estado, diagrama de}

\begin{figure}
\centerline
{\includegraphics[scale=0.8]{figs/state4.pdf}}
\caption{Diagrama de estado.}
\label{fig.state4}
\end{figure}

Me tomé la licencia de organizar las variables en el marco
y agregar líneas punteadas para mostrar que los valores de {\tt i} y
{\tt j} indican caracteres en {\tt palabra1} y {\tt palabra2}.

Comenzando con este diagrama, ejecuta el programa en papel, cambiando los
valores de {\tt i} y {\tt j} durante cada iteración.  Encuentra y arregla el
segundo error en esta función.
\label{isreverse}


\section{Glosario}

\begin{description}

\item[objeto:] Algo a lo cual una variable puede referirse.  Por ahora,
puedes utilizar ``objeto'' y ``valor'' indistintamente.
\index{objeto}

\item[secuencia:] Una colección ordenada de
valores donde cada valor se identifica por un índice entero.
\index{secuencia}

\item[ítem:] Uno de los valores en una secuencia.
\index{item@ítem}

\item[índice:] Un valor entero utilizado para seleccionar un ítem en
una secuencia, tal como un carácter en una cadena.  En Python,
los índices parten desde 0.
\index{indice@índice}

\item[trozo ({\em slice}):] Una parte de una cadena especificada por un rango de índices.
\index{slice}\index{trozo}

\item[cadena vacía:] Una cadena sin caracteres y con longitud 0, representada
por dos comillas.
\index{cadena!vacía}

\item[inmutable:] La propiedad de una secuencia cuyos ítems no pueden
cambiarse.
\index{inmutabilidad}

\item[recorrer:] Iterar a través de los ítems en una secuencia,
realizando una operación similar en cada uno de estos.
\index{recorrer}

\item[búsqueda:] Un patrón de un recorrido que se detiene
cuando encuentra lo que busca.
\index{patrón!de búsqueda}
\index{busqueda@búsqueda!patrón de}

\item[contador:] Una variable utilizada para contar algo, generalmente inicializada
en cero y luego incrementada.
\index{contador}

\item[invocación:] Una sentencia que llama a un método.
\index{invocación}

\item[argumento opcional:] Un argumento de función o de método que no es
obligatorio.
\index{argumento opcional}
\index{opcional!argumento}

\end{description}


\section{Ejercicios}

\begin{exercise}
\index{metodo@método!de cadena}
\index{cadena!método de}

Lee la documentación de los métodos de cadena en
\url{http://docs.python.org/3/library/stdtypes.html#string-methods}.
Tal vez quieras experimentar con algunos para asegurarte de que
entiendes cómo funcionan.  {\tt strip} y {\tt replace} son
particularmente útiles.

La documentación utiliza una sintaxis que podría confundir.
Por ejemplo, en \verb"find(sub[, start[, end]])", los corchetes
indican argumentos opcionales.  Entonces {\tt sub} es obligatorio, pero
{\tt start} es opcional, y si incluyes {\tt start},
entonces {\tt end} es opcional.
\index{argumento opcional}
\index{opcional!argumento}

\end{exercise}


\begin{exercise}
\index{metodo@método!count}
\index{count, método}

Hay un método de cadena llamado {\tt count} que es similar
a la función de la Sección~\ref{counter}.  Lee la documentación
de este método
y escribe una invocación que cuente el número de letras {\tt a}
en \verb"'banana'".
\end{exercise}


\begin{exercise}
\index{tamaño de paso}
\index{operador!de trozo}
\index{trozo!operador}

Un trozo de cadena puede tomar un tercer índice que especifique el ``tamaño
de paso'', es decir, el número de espacios entre caracteres sucesivos.
Un tamaño de paso de 2 significa cada dos caracteres, 3 significa cada tres,
etc.

\begin{verbatim}
>>> fruta = 'banana'
>>> fruta[0:5:2]
    'bnn'
\end{verbatim}

Un tamaño de paso de -1 pasa a través de la palabra hacia atrás, por lo que
el trozo \verb"[::-1]" genera una cadena invertida.
\index{palíndromo}

Utiliza esta notación para escribir una versión de una línea de \verb"es_palindromo"
del Ejercicio~\ref{palindrome}.
\end{exercise}


\begin{exercise}

Las siguientes funciones tienen la {\em intención} de verificar si una
cadena contiene al menos una letra minúscula, pero algunas son
incorrectas.  Para cada función, describe qué hace realmente la función
(suponiendo que el parámetro es una cadena).

\begin{verbatim}
def contiene_minuscula1(s):
    for c in s:
        if c.islower():
            return True
        else:
            return False

def contiene_minuscula2(s):
    for c in s:
        if 'c'.islower():
            return 'True'
        else:
            return 'False'

def contiene_minuscula3(s):
    for c in s:
        flag = c.islower()
    return flag

def contiene_minuscula4(s):
    flag = False
    for c in s:
        flag = flag or c.islower()
    return flag

def contiene_minuscula5(s):
    for c in s:
        if not c.islower():
            return False
    return True
\end{verbatim}

\end{exercise}


\begin{exercise}
\index{rotación de letras}
\index{letras, rotación de}

\label{exrotate}
Un cifrado César es una forma débil de encriptación que implica la ``rotación'' de cada
letra en un número fijo de lugares.  Rotar una letra significa
desplazarla a través del alfabeto, volviendo al comienzo si
es necesario, por lo que 'A' rotada en 3 es 'D' y 'Z' rotada en 1 es 'A'.

Para rotar una palabra, rota cada letra en la misma cantidad.
Por ejemplo, ``cheer'' rotada en 7 es ``jolly'' y ``melon'' rotada
en -10 es ``cubed''.  En la película {\em 2001: Odisea del espacio}, el
computador de la nave se llama HAL, que es IBM rotada en -1.

%For example ``sleep''
%rotated by 9 is ``bunny'' and ``latex'' rotated by 7 is ``shale''.

Escribe una función llamada \verb"rotar_palabra"
que tome una cadena y un entero como parámetros y devuelva
una cadena nueva que contenga las letras de la cadena original
rotadas en la cantidad entregada.

Tal vez quieras utilizar la función incorporada {\tt ord}, que convierte
un carácter en un código numérico, y {\tt chr}, que convierte códigos
numéricos en caracteres.  Las letras del alfabeto están codificadas en orden alfabético,
así por ejemplo:

\begin{verbatim}
>>> ord('c') - ord('a')
    2
\end{verbatim}

Debido a que \verb"'c'" es la dos-ésima letra del alfabeto.  Pero ten
cuidado: los códigos numéricos para las letras mayúsculas son diferentes.

Los chistes potencialmente ofensivos en internet a veces están codificados en
ROT13, que es un cifrado César con rotación 13.  Si no te
ofendes fácilmente, encuentra y decodifica algunos.  Solución:
\url{http://thinkpython.com/code/rotate.py}.

\end{exercise}


\chapter{Estudio de caso: juego de palabras}
\label{wordplay}

Este capítulo presenta el segundo estudio de caso, el cual involucra
resolver puzles de palabras buscando palabras que tengan ciertas
propiedades.  Por ejemplo, encontraremos los palíndromos más largos
en inglés y buscaremos palabras cuyas letras aparezcan en
orden alfabético.  Además, presentaré otro plan de desarrollo de
programa: reducción a un problema previamente resuelto.


\section{Leer listas de palabras}
\label{wordlist}

Para los ejercicios de este capítulo necesitamos una lista de palabras en inglés.
Hay muchas listas de palabras disponibles en la web, pero la más
adecuada para nuestro propósito es una de las listas de palabras recopiladas y
contribuidas al dominio público por Grady Ward como parte del proyecto léxico Moby
(ver \url{http://wikipedia.org/wiki/Moby_Project}).  Es
una lista de 113,809 palabras de crucigrama oficiales, es decir, palabras que se
consideran válidas en crucigramas y otros juegos de palabras.  En la
colección Moby, el nombre del archivo es {\tt 113809of.fic}; puedes descargar
una copia, con el nombre más simple {\tt words.txt}, en
\url{http://thinkpython.com/code/words.txt}.
\index{Moby Project}
\index{crucigramas}

Este archivo está en texto plano, así que puedes abrirlo con un editor
de texto, pero también puedes leerlo desde Python.  La función incorporada
{\tt open} toma el nombre del archivo como parámetro
y devuelve un {\bf objeto de archivo} que puedes utilizar para leer dicho archivo.

\index{función!open}
\index{open, función}
\index{texto plano}
\index{plano, texto}
\index{objeto!de archivo}
\index{archivo!objeto de}

\begin{verbatim}
>>> fin = open('words.txt')
\end{verbatim}
%
{\tt fin} es un nombre común para un objeto de archivo utilizado para la entrada ({\em file input}).  El objeto
de archivo proporciona varios métodos para la lectura, incluyendo {\tt readline},
que lee caracteres desde un archivo hasta que llega a una nueva línea y
devuelve el resultado como una cadena: \index{metodo@método!readline}
\index{readline, método}

\begin{verbatim}
>>> fin.readline()
    'aa\n'
\end{verbatim}
%
La primera palabra de esta particular lista es ``aa'', que es un tipo de
lava.  La secuencia \verb"\n" representa el carácter nueva línea que
separa esta palabra de la siguiente.

El objeto de archivo hace un seguimiento del lugar del archivo en donde este se encuentra en un instante determinado, así que
si llamas a {\tt readline} de nuevo, obtienes la palabra siguiente:

\begin{verbatim}
>>> fin.readline()
    'aah\n'
\end{verbatim}
%
La palabra siguiente es ``aah'', que es una palabra perfectamente legítima,
así que deja de mirarme así.
O bien, si es el carácter nueva línea lo que te molesta,
podemos deshacernos de este con el método de cadena {\tt strip}:
\index{metodo@método!strip}
\index{strip, método}

\begin{verbatim}
>>> linea = fin.readline()
>>> palabra = linea.strip()
>>> palabra
    'aahed'
\end{verbatim}
%
Puedes también utilizar un objeto de archivo como parte de un bucle {\tt for}.
Este programa lee {\tt words.txt} e imprime cada palabra, una
por línea:
\index{función!open}
\index{open, función}

\begin{verbatim}
fin = open('words.txt')
for linea in fin:
    palabra = linea.strip()
    print(palabra)
\end{verbatim}
%

\section{Ejercicios}

Hay soluciones a estos ejercicios en la siguiente sección.
Deberías al menos intentar cada uno antes de leer las soluciones.

\begin{exercise}
Escribe un programa que lea {\tt words.txt} e imprima solo las
palabras con más de 20 caracteres (sin contar espacios en blanco).
\index{espacio en blanco}

\end{exercise}

\begin{exercise}

En 1939, Ernest Vincent Wright publicó una novela de 50.000 palabras llamada
{\em Gadsby}, la cual no contiene la letra ``e''.  Dado que la ``e'' es
la letra más común en el idioma inglés, no es una tarea fácil.

Sin duda, solo imaginar una oración sin utilizar dicho símbolo tan común
implica una actividad difícil.  Si tardas mucho al principio, hazlo con cuidado
y trabaja horas para adquirir la habilidad poco a poco.

Basta, ya no sigo más.

Escribe una función llamada \verb"no_tiene_e" que devuelva {\tt True} si
la palabra dada no incluye la letra ``e''.

Escribe un programa que lea {\tt words.txt} e imprima solo las palabras
que no tienen ``e''.  Calcula el porcentaje de palabras en la lista
que no tienen ``e''.
\index{lipograma}

\end{exercise}


\begin{exercise}

Escribe una función con nombre {\tt excluye}
que tome una palabra y una cadena de letras prohibidas y
devuelva {\tt True} si la palabra no utiliza ninguna de las letras
prohibidas.

Escribe un programa que solicite al usuario ingresar una cadena
de letras prohibidas y luego imprima el número de palabras que
no contienen ninguna de estas.
¿Puedes encontrar una combinación de 5 letras prohibidas que
excluya al menor número de palabras?

\end{exercise}



\begin{exercise}

Escribe una función con nombre \verb"usa_solo" que tome una palabra y una
cadena de letras y devuelva {\tt True} si la palabra contiene
solo letras de la lista.  ¿Puedes crear una oración en inglés utilizando solo las
letras {\tt acefhlo}?  ¿Una distinta a ``Hoe alfalfa''?

\end{exercise}


\begin{exercise}

Escribe una función con nombre \verb"usa_todas" que tome una palabra y una
cadena de letras requeridas y devuelva {\tt True} si la palabra
utiliza todas las letras requeridas al menos una vez.  ¿Cuántas palabras
que utilizan todas las vocales {\tt aeiou} existen?  ¿Qué pasa con {\tt aeiouy}?

\end{exercise}


\begin{exercise}

Escribe una función llamada \verb"es_abecedario" que devuelva
{\tt True} si las letras en una palabra aparecen en orden alfabético
(las letras dobles están permitidas).
¿Cuántas palabras abecedarias existen?

\index{abecedario}

\end{exercise}



\section{Búsqueda}
\label{search}
\index{patrón!de búsqueda}
\index{busqueda@búsqueda!patrón de}

Todos los ejercicios de la sección anterior tienen algo en
común: pueden ser resueltos con el patrón de búsqueda que vimos
en la Sección~\ref{find}.  El ejemplo más simple es:

\begin{verbatim}
def no_tiene_e(palabra):
    for letra in palabra:
        if letra == 'e':
            return False
    return True
\end{verbatim}
%
El bucle {\tt for} recorre los caracteres en {\tt palabra}.  Si encontramos
la letra ``e'', podemos devolver {\tt False} inmediatamente; de lo contrario,
tenemos que ir a la siguiente letra.  Si terminamos el bucle de manera normal,
significa que no encontramos una ``e'', por lo cual devolvemos {\tt True}.
\index{recorrer}

\index{operador!in}
\index{in, operador}
Podrías haber escrito esta función de manera más concisa utilizando el operador {\tt in},
pero comencé con esta versión porque
demuestra la lógica del patrón de búsqueda.

\index{generalización}
{\tt excluye} es una versión más general de \verb"no_tiene_e" pero
tiene la misma estructura:

\begin{verbatim}
def excluye(palabra, prohibidas):
    for letra in palabra:
        if letra in prohibidas:
            return False
    return True
\end{verbatim}
%
Podemos devolver {\tt False} apenas encontremos una letra prohibida;
si llegamos al final del bucle, devolvemos {\tt True}.

\verb"usa_solo" es similar, excepto que el sentido de la condición
es inverso:

\begin{verbatim}
def usa_solo(palabra, disponibles):
    for letra in palabra:
        if letra not in disponibles:
            return False
    return True
\end{verbatim}
%
En lugar de una lista de letras prohibidas, tenemos una lista de letras
disponibles.  Si encontramos una letra en {\tt palabra} que no está en
{\tt disponibles}, podemos devolver {\tt False}.

\verb"usa_todas" es similar, excepto que invertimos el rol
de la palabra y la cadena de letras:

\begin{verbatim}
def usa_todas(palabra, requeridas):
    for letra in requeridas:
        if letra not in palabra:
            return False
    return True
\end{verbatim}
%
En lugar de recorrer las letras en {\tt palabra}, el bucle
recorre las letras requeridas.  Si alguna de las letras requeridas
no aparece en la palabra, podemos devolver {\tt False}.
\index{recorrer}

Si realmente estuvieras pensando como un informático,
habrías reconocido que \verb"usa_todas" era una instancia de un
problema previamente resuelto, y habrías escrito:

\begin{verbatim}
def usa_todas(palabra, requeridas):
    return usa_solo(requeridas, palabra)
\end{verbatim}
%
Este es un ejemplo de un plan de desarrollo de programa llamado {\bf
  reducción a un problema previamente resuelto}, lo cual significa que
reconoces el problema en el que estás trabajando como una instancia de un
problema resuelto y aplicas una solución existente.
\index{reducción a un problema previamente resuelto} \index{plan de desarrollo!reducción}


\section{Bucles con índices}
\index{bucle!con índices}
\index{indice@índice!bucle con}

Escribí las funciones de la sección anterior con bucles {\tt for}
porque solo necesitaba los caracteres en las cadenas; no
tenía que hacer nada con los índices.

Para \verb"es_abecedario" tenemos que comparar letras adyacentes,
lo cual es un poco difícil con un bucle {\tt for}:

\begin{verbatim}
def es_abecedario(palabra):
    anterior = palabra[0]
    for c in palabra:
        if c < anterior:
            return False
        anterior = c
    return True
\end{verbatim}

Una alternativa es usar recursividad:

\begin{verbatim}
def es_abecedario(palabra):
    if len(palabra) <= 1:
        return True
    if palabra[0] > palabra[1]:
        return False
    return es_abecedario(palabra[1:])
\end{verbatim}

Otra opción es usar un bucle {\tt while}:

\begin{verbatim}
def es_abecedario(palabra):
    i = 0
    while i < len(palabra)-1:
        if palabra[i+1] < palabra[i]:
            return False
        i = i+1
    return True
\end{verbatim}
%
Este bucle comienza con {\tt i=0} y termina cuando {\tt i=len(palabra)-1}.  En cada
paso por el bucle, compara al $i$-ésimo carácter (que puedes
pensarlo como el carácter actual) con el $i+1$-ésimo carácter (que puedes
pensarlo como el siguiente).

Si el siguiente carácter es menor (alfabéticamente anterior) que el
actual, entonces hemos descubierto que se rompe la tendencia abecedaria y
devolvemos {\tt False}.

Si llegamos al final del bucle sin encontrar una falla, entonces la
palabra pasa la prueba.  Para convencerte de que el bucle termina
de manera correcta, considera un ejemplo como \verb"'flossy'".  La
longitud de la palabra es 6, por lo cual
la última vez que el bucle se ejecuta es cuando {\tt i} es 4, que es el
índice del penúltimo carácter.  En la última iteración,
compara el penúltimo carácter con el último, que es
lo que queremos.
\index{palíndromo}

Aquí hay una versión de \verb"es_palindromo" (ver
Ejercicio~\ref{palindrome}) que utiliza dos índices: uno comienza al
principio y aumenta, el otro comienza al final y disminuye.

\begin{verbatim}
def es_palindromo(palabra):
    i = 0
    j = len(palabra)-1

    while i<j:
        if palabra[i] != palabra[j]:
            return False
        i = i+1
        j = j-1

    return True
\end{verbatim}

O bien podríamos reducir a un problema previamente
resuelto y escribir
\index{reducción a un problema previamente resuelto}
\index{plan de desarrollo!reducción}

\begin{verbatim}
def es_palindromo(palabra):
    return es_inverso(palabra, palabra)
\end{verbatim}
%
utilizando \verb"es_inverso" de la Sección~\ref{isreverse}.


\section{Depuración}
\index{depuración}
\index{prueba!es difícil}
\index{prueba!de programa}

Probar programas es difícil.  Las funciones de este capítulo son
relativamente fáciles de probar porque puedes verificar los resultados a mano.
Aun así, escoger un conjunto de palabras que pruebe todos los errores posibles
está en algún lugar entre difícil e imposible.

Tomando a \verb"no_tiene_e" como ejemplo, hay dos casos obvios
para verificar: las palabras que tienen una `e' deberían devolver {\tt False} y
las palabras que no la tienen deberían devolver {\tt True}.  No deberías
tener problemas para proponer una palabra de cada caso.

Dentro de cada caso, hay algunos subcasos menos obvios.  Entre las
palabras que tienen una ``e'', deberías probar palabras con una ``e'' al
principio, al final y en algún lugar del medio.  Deberías probar palabras
largas, palabras cortas y palabras muy cortas, como la cadena vacía.  La
cadena vacía es un ejemplo de un {\bf caso especial}, que es uno de
los casos no obvios donde los errores a menudo acechan.
\index{caso especial}

Además de los casos de prueba que generes, puedes también probar
tu programa con una lista de palabras como {\tt words.txt}.  Escudriñando
la salida, podrías ser capaz de captar los errores, pero ten cuidado:
podrías captar un tipo de error (palabras que no deberían estar
incluidas, pero lo están) y otro no (palabras que deberían estar incluidas,
pero no lo están).

En general, las pruebas pueden ayudarte a encontrar errores, pero no es fácil
generar un buen conjunto de casos de prueba, e incluso si lo haces, no puedes
asegurarte de que tu programa está correcto.
De acuerdo al legendario informático:
\index{prueba!y ausencia de errores}

\begin{quote}
Se pueden probar programas para mostrar la presencia de errores, ¡pero nunca para
mostrar su ausencia!

--- Edsger W. Dijkstra
\end{quote}
\index{Dijkstra, Edsger}


\section{Glosario}

\begin{description}

\item[objeto de archivo:] Un valor que representa un archivo abierto.
\index{objeto!de archivo}
\index{archivo!objeto de}

\item[reducción a un problema previamente resuelto:] Una manera de resolver un
  problema expresándolo como una instancia de un problema previamente
  resuelto.  \index{reducción a un problema previamente resuelto}
  \index{plan de desarrollo!reducción}

\item[caso especial:] Un caso de prueba que es atípico o no obvio
(y menos probable de abordar correctamente).
\index{caso especial}

\end{description}


\section{Ejercicios}

\begin{exercise}
\index{Car Talk}
\index{Puzzler}
\index{letras dobles}

Esta pregunta está basada en un Puzzler que fue transmitido en el programa
de radio {\em Car Talk}
(\url{http://www.cartalk.com/content/puzzlers}):

\begin{quote}
Dame una palabra con tres letras dobles consecutivas. Te daré un
par de palabras que casi califican, pero no. Por ejemplo, la palabra
committee, c-o-m-m-i-t-t-e-e. Estaría excelente, si no fuera por la `i' que
se cuela allí. O Mississippi: M-i-s-s-i-s-s-i-p-p-i. Si pudieras
sacar esas `i' funcionaría. Pero hay una palabra que tiene tres
pares de letras consecutivos y, por lo que sé, puede ser
la única palabra. Por supuesto que probablemente hay 500 más pero solo puedo
pensar en una. ¿Cuál es la palabra?
\end{quote}

Escribe un programa que la encuentre.
Solución: \url{http://thinkpython.com/code/cartalk1.py}.

\end{exercise}


\begin{exercise}
Aquí hay otro Puzzler de {\em Car Talk}
(\url{http://www.cartalk.com/content/puzzlers}):
\index{Car Talk}
\index{Puzzler}
\index{odómetro}
\index{palíndromo}

\begin{quote}
``Estaba conduciendo en la carretera el otro día y me fijé en
mi odómetro. Al igual que la mayoría de los odómetros, muestra seis dígitos,
solo en millas enteras. Entonces, si mi automóvil tenía 300.000
millas, por ejemplo, veía 3-0-0-0-0-0.

``Ahora, lo que vi ese día fue muy interesante. Me di cuenta de que los
últimos 4 dígitos eran palíndromo, es decir, se leían igual hacia adelante
y hacia atrás. Por ejemplo, 5-4-4-5 es un palíndromo, por lo que mi odómetro
podría haber leído 3-1-5-4-4-5.

``Una milla más adelante, los últimos 5 números eran palíndromo. Por ejemplo,
podría haber leído 3-6-5-4-5-6.  Una milla después de eso, los 4 números que están
al medio de los 6 eran palíndromo.  ¿Y estás listo para esto? Una milla más adelante,
¡los 6 eran palíndromo!

``La pregunta es, ¿qué había en mi odómetro cuando miré por primera vez?''
\end{quote}

Escribe un programa en Python que pruebe todos los números de seis dígitos e imprima
aquellos números que satisfagan estos requisitos.
Solución: \url{http://thinkpython.com/code/cartalk2.py}.

\end{exercise}


\begin{exercise}
Aquí hay otro Puzzler de {\em Car Talk} que puedes resolver con una
búsqueda (\url{http://www.cartalk.com/content/puzzlers}):
\index{Car Talk}
\index{Puzzler}
\index{palíndromo}

\begin{quote}
``Recientemente tuve una visita con mi mamá y me di cuenta de que
los dos dígitos que componen mi edad cuando se invierten resulta en su
edad. Por ejemplo, si ella tiene 73, yo tengo 37. Nos preguntamos cuán a menudo ha
ocurrido esto a través de los años pero nos desviamos a otros temas y
nunca dimos con una respuesta.

``Cuando llegué a casa descubrí que los dígitos de nuestras edades han sido
reversibles seis veces hasta ahora. También descubrí que, si tenemos suerte,
ocurriría de nuevo en unos años, y si realmente tenemos suerte
ocurriría una vez más después de eso. En otras palabras, habría
ocurrido 8 veces en total. Entonces la pregunta es, ¿qué edad tengo ahora?''

\end{quote}

Escribe un programa en Python que busque soluciones a este Puzzler.
Pista: podrías encontrar útil el método de cadena {\tt zfill}.

Solución: \url{http://thinkpython.com/code/cartalk3.py}.

\end{exercise}



\chapter{Listas}

Este capítulo presenta uno de los tipos incorporados más útiles de Python: las listas.
Además, aprenderás más sobre objetos y lo que puede ocurrir cuando tienes
más de un nombre para el mismo objeto.


\section{Una lista es una secuencia}
\label{sequence}

Al igual que una cadena, una {\bf lista} es una secuencia de valores.  En una cadena, los
valores son caracteres; en una lista, pueden ser de cualquier tipo.  Los valores en
una lista se llaman {\bf elementos} o a veces {\bf ítems}.
\index{lista}
\index{tipo!lista}
\index{elemento}
\index{secuencia}
\index{item@ítem}

Hay varias maneras de crear una lista nueva; la más simple es
encerrar los elementos en corchetes (\verb"[" y \verb"]"):

\begin{verbatim}
[10, 20, 30, 40]
['crunchy frog', 'ram bladder', 'lark vomit']
\end{verbatim}
%
El primer ejemplo es una lista de cuatro enteros.  El segundo es una lista de
tres cadenas.  Los elementos de una lista no tienen que ser del mismo tipo.
La siguiente lista contiene una cadena, un número de coma flotante, un entero y
(¡atención!) otra lista:

\begin{verbatim}
['spam', 2.0, 5, [10, 20]]
\end{verbatim}
%
Una lista dentro de otra lista está {\bf anidada}.
\index{lista!anidada}
\index{anidada, lista}

Una lista que no contiene elementos se
llama lista vacía; puedes crear una con corchetes
vacíos, \verb"[]".
\index{lista!vacía}
\index{vacía!lista}

Como podrías esperar, puedes asignar valores de lista a variables:

\begin{verbatim}
>>> quesos = ['Cheddar', 'Edam', 'Gouda']
>>> numeros = [42, 123]
>>> vacio = []
>>> print(quesos, numeros, vacio)
    ['Cheddar', 'Edam', 'Gouda'] [42, 123] []
\end{verbatim}
%
\index{asignación}


\section{Las listas son mutables}
\label{mutable}
\index{lista!elemento}
\index{acceso}
\index{indice@índice}
\index{operador!de corchetes}
\index{corchetes!operador de}

La sintaxis para acceder a los elementos de una lista es la misma que para
acceder a los caracteres de una cadena: el operador de corchetes.  La
expresión dentro de los corchetes especifica el índice.  Recuerda que los
índices comienzan en 0:

\begin{verbatim}
>>> quesos[0]
    'Cheddar'
\end{verbatim}
%
A diferencia de las cadenas, las listas son mutables.  Cuando el operador de corchetes aparece
en el lado izquierdo de una asignación, este identifica el elemento de la
lista que será asignado.
\index{mutabilidad}

\begin{verbatim}
>>> numeros = [42, 123]
>>> numeros[1] = 5
>>> numeros
    [42, 5]
\end{verbatim}
%
El uno-ésimo elemento de {\tt numeros}, que
solía ser 123, ahora es 5.
\index{indice@índice!comenzando en cero}
\index{cero, índice comenzando en}

La Figura~\ref{fig.liststate} muestra
el diagrama de estado para {\tt
quesos}, {\tt numeros} y {\tt vacio}.
\index{diagrama!de estado}
\index{estado, diagrama de}

\begin{figure}
\centerline
{\includegraphics[scale=0.8]{figs/liststate.pdf}}
\caption{Diagrama de estado.}
\label{fig.liststate}
\end{figure}

Las listas se representan por cajas con la palabra ``list'' por fuera
y los elementos de la lista por dentro.  {\tt quesos} se refiere a
una lista con tres elementos con índices 0, 1 y 2.
{\tt numeros} contiene dos elementos; el diagrama muestra que el
valor del segundo elemento ha sido reasignado de 123 a 5.
{\tt vacio} se refiere a una lista sin elementos.
\index{asignación de ítem}
\index{item@ítem!asignación de}
\index{reasignación}

Los índices de las listas funcionan de la misma manera que los índices de las cadenas:

\begin{itemize}

\item Cualquier expresión entera se puede utilizar como índice.

\item Si intentas leer o escribir un elemento que no existe,
obtienes un {\tt IndexError}.
\index{excepción!IndexError}
\index{IndexError}

\item Si un índice tiene un valor negativo, se cuenta hacia atrás desde el
final de la lista.

\end{itemize}
\index{lista!indice de@índice de}

\index{lista!pertenencia}
\index{pertenencia!lista}
\index{operador!in}
\index{in, operador}

El operador {\tt in} también funciona en las listas.

\begin{verbatim}
>>> quesos = ['Cheddar', 'Edam', 'Gouda']
>>> 'Edam' in quesos
    True
>>> 'Brie' in quesos
    False
\end{verbatim}


\section{Recorrer una lista}
\index{recorrer!lista}
\index{lista!recorrer}
\index{bucle!for}
\index{for, bucle}
\index{sentencia!for}

La manera más común de recorrer los elementos de una lista es
con un ciclo {\tt for}.  La sintaxis es la misma que para las cadenas:

\begin{verbatim}
for queso in quesos:
    print(queso)
\end{verbatim}
%
Esto funciona bien si solo necesitas leer los elementos de la
lista.  Pero si quieres escribir o actualizar los elementos,
necesitas los índices.  Una manera común de hacer eso es combinar
las funciones incorporadas {\tt range} y {\tt len}:
\index{bucle!con índices}
\index{indice@índice!bucle con}

\begin{verbatim}
for i in range(len(numeros)):
    numeros[i] = numeros[i] * 2
\end{verbatim}
%
Este bucle recorre la lista y actualiza cada elemento.  {\tt len}
devuelve el número de elementos en la lista.  {\tt range} devuelve
una lista de índices de 0 a $n-1$, donde $n$ es el largo de
la lista.  En cada paso por el bucle, {\tt i} obtiene el índice
del siguiente elemento.  La sentencia de asignación en el cuerpo usa
{\tt i} para leer el valor antiguo del elemento y asignar el
nuevo valor.
\index{actualizar!item@ítem}
\index{item@ítem!actualizar}

Un bucle {\tt for} a través de una lista vacía nunca ejecuta el cuerpo:

\begin{verbatim}
for x in []:
    print('Esto nunca ocurre.')
\end{verbatim}
%
A pesar de que una lista puede contener otra lista, la lista
anidada aún cuenta como un solo elemento.  La longitud de esta lista es
cuatro:
\index{lista!anidada}
\index{anidada, lista}

\begin{verbatim}
['spam', 1, ['Brie', 'Roquefort', 'Pol le Veq'], [1, 2, 3]]
\end{verbatim}



\section{Operaciones de lista}
\index{lista!operación}

El operador {\tt +} concatena listas:
\index{lista!concatenación}
\index{concatenación!lista}

\begin{verbatim}
>>> a = [1, 2, 3]
>>> b = [4, 5, 6]
>>> c = a + b
>>> c
    [1, 2, 3, 4, 5, 6]
\end{verbatim}
%
El operador {\tt *} repite una lista un número dado de veces:
\index{repetición!lista}
\index{lista!repetición}

\begin{verbatim}
>>> [0] * 4
    [0, 0, 0, 0]
>>> [1, 2, 3] * 3
    [1, 2, 3, 1, 2, 3, 1, 2, 3]
\end{verbatim}
%
El primer ejemplo repite {\tt [0]} cuatro veces.  El segundo ejemplo
repite la lista {\tt [1, 2, 3]} tres veces.


\section{Trozos de lista}
\index{operador!de trozo}
\index{trozo!operador}
\index{indice@índice!de trozo}
\index{lista!trozo}
\index{trozo!lista}\index{slice}

El operador de trozo también funciona en las listas:

\begin{verbatim}
>>> t = ['a', 'b', 'c', 'd', 'e', 'f']
>>> t[1:3]
    ['b', 'c']
>>> t[:4]
    ['a', 'b', 'c', 'd']
>>> t[3:]
    ['d', 'e', 'f']
\end{verbatim}
%
Si omites el primer índice, el trozo comienza al principio.
Si omites el segundo, el trozo llega al final.  Entonces, si
omites ambos, el trozo es una copia de la lista completa.
\index{lista!copia}
\index{trozo!copia de}
\index{copia!de trozo}

\begin{verbatim}
>>> t[:]
    ['a', 'b', 'c', 'd', 'e', 'f']
\end{verbatim}
%
Dado que las listas son mutables, a menudo es útil crear una copia
antes de realizar operaciones que modifiquen listas.
\index{mutabilidad}

Un operador de trozo en el lado izquierdo de una asignación
puede actualizar múltiples elementos:
\index{trozo!actualizar}\index{slice}
\index{actualizar!trozo}

\begin{verbatim}
>>> t = ['a', 'b', 'c', 'd', 'e', 'f']
>>> t[1:3] = ['x', 'y']
>>> t
    ['a', 'x', 'y', 'd', 'e', 'f']
\end{verbatim}
%

% You can add elements to a list by squeezing them into an empty
% slice:

% % \begin{verbatim}
% >>> t = ['a', 'd', 'e', 'f']
% >>> t[1:1] = ['b', 'c']
% >>> print t
%     ['a', 'b', 'c', 'd', 'e', 'f']
% \end{verbatim}
% \afterverb
%
% And you can remove elements from a list by assigning the empty list to
% them:

% % \begin{verbatim}
% >>> t = ['a', 'b', 'c', 'd', 'e', 'f']
% >>> t[1:3] = []
% >>> print t
%     ['a', 'd', 'e', 'f']
% \end{verbatim}
% \afterverb
%
% But both of those operations can be expressed more clearly
% with list methods.


\section{Métodos de lista}
\index{lista!método de}
\index{metodo@método!de lista}

Python proporciona métodos que operan en listas.  Por ejemplo,
{\tt append} agrega un nuevo elemento al final de la lista:
\index{metodo@método!append}
\index{append, método}

\begin{verbatim}
>>> t = ['a', 'b', 'c']
>>> t.append('d')
>>> t
    ['a', 'b', 'c', 'd']
\end{verbatim}
%
{\tt extend} toma una lista como argumento y anexa todos
los elementos:
\index{metodo@método!extend}
\index{extend, método}

\begin{verbatim}
>>> t1 = ['a', 'b', 'c']
>>> t2 = ['d', 'e']
>>> t1.extend(t2)
>>> t1
    ['a', 'b', 'c', 'd', 'e']
\end{verbatim}
%
Este ejemplo deja a {\tt t2} sin modificar.

{\tt sort} ordena los ejementos de la lista de menor a mayor:
\index{metodo@método!sort}
\index{sort, método}

\begin{verbatim}
>>> t = ['d', 'c', 'e', 'b', 'a']
>>> t.sort()
>>> t
    ['a', 'b', 'c', 'd', 'e']
\end{verbatim}
%
La mayoría de los métodos de lista son nulos: modifican la lista y devuelven {\tt None}.
Si por casualidad escribes {\tt t = t.sort()}, te decepcionará
el resultado.
\index{metodo@método!nulo}
\index{nulo, método}
\index{valor especial!None}
\index{None, valor especial}


\section{Mapa, filtro y reducción}
\label{filter}

Para sumar todos los números de una lista, puedes utilizar un bucle como este:

% see add.py

\begin{verbatim}
def sumar_todos(t):
    total = 0
    for x in t:
        total += x
    return total
\end{verbatim}
%
{\tt total} se inicializa en 0.  En cada paso por el bucle,
{\tt x} obtiene un elemento de la lista.  El operador {\tt +=}
proporciona una manera corta de actualizar una variable.  Esta
{\bf sentencia de asignación aumentada},
\index{operador!de actualización}
\index{actualización, operador de}
\index{aumentada, asignación}
\index{asignación!aumentada}

\begin{verbatim}
    total += x
\end{verbatim}
%
es equivalente a

\begin{verbatim}
    total = total + x
\end{verbatim}
%
A medida que el bucle se ejecuta, {\tt total} acumula la suma de los
elementos; una variable que se utiliza de esta manera a veces se llama
{\bf acumulador}.
\index{acumulador de suma}

Sumar los elementos de una lista es una operación tan común
que Python la facilita como función incorporada, {\tt sum}:

\begin{verbatim}
>>> t = [1, 2, 3]
>>> sum(t)
    6
\end{verbatim}
%
Una operación como esta que combina una secuencia de elementos en
un solo valor a veces se llama {\bf reducción}.
\index{patrón!de reducción}
\index{reducción, patrón de}
\index{recorrer}

A veces quieres recorrer una lista mientras construyes
otra.  Por ejemplo, la siguiente función toma una lista de cadenas
y devuelve una nueva lista que contiene cadenas que comienzan con mayúscula:

\begin{verbatim}
def todas_con_mayuscula(t):
    res = []
    for s in t:
        res.append(s.capitalize())
    return res
\end{verbatim}
%
{\tt res} se inicializa con una lista vacía; en cada paso por el bucle,
anexamos el elemento siguiente.  Entonces {\tt res} es otro
tipo de acumulador.
\index{acumulador lista}

Una operación como \verb"todas_con_mayuscula" a veces es llamada {\bf
mapa} porque ``mapea'' una función (en este caso el método {\tt
capitalize}) sobre cada uno de los elementos en una secuencia.
\index{patrón!de mapa}
\index{mapa, patrón de}
\index{patrón!de filtro}
\index{filtro, patrón de}

Otra operación común es seleccionar algunos de los elementos de
una lista y devolver una sublista.  Por ejemplo, la siguiente
función toma una lista de cadenas y devuelve una lista que contiene
solo las cadenas escritas con mayúsculas:

\begin{verbatim}
def solo_mayusculas(t):
    res = []
    for s in t:
        if s.isupper():
            res.append(s)
    return res
\end{verbatim}
%
{\tt isupper} es un método de cadena que devuelve {\tt True} si
la cadena solo contiene letras mayúsculas.

Una operación como \verb"solo_mayusculas" se llama {\bf filtro} porque
selecciona algunos de los elementos y filtra los otros.

La mayoría de las operaciones de lista se pueden expresar como una combinación
de mapa, filtro y reducción.


\section{Eliminar elementos}
\index{eliminación de elementos}
\index{elemento!eliminación}

Hay varias maneras de eliminar elementos de una lista.  Si conoces
el índice del elemento que quieres, puedes utilizar
{\tt pop}:
\index{metodo@método!pop}
\index{pop, método}

\begin{verbatim}
>>> t = ['a', 'b', 'c']
>>> x = t.pop(1)
>>> t
    ['a', 'c']
>>> x
    'b'
\end{verbatim}
%
{\tt pop} modifica la lista y devuelve el elemento que se eliminó.
Si no entregas un índice, elimina y devuelve el
último elemento.

Si no necesitas el valor eliminado, puedes utilizar el
operador {\tt del}:
\index{operador!del}
\index{del, operador}

\begin{verbatim}
>>> t = ['a', 'b', 'c']
>>> del t[1]
>>> t
    ['a', 'c']
\end{verbatim}
%
Si conoces el elemento que quieres eliminar (pero no el índice),
puedes utilizar {\tt remove}:
\index{metodo@método!remove}
\index{remove, método}

\begin{verbatim}
>>> t = ['a', 'b', 'c']
>>> t.remove('b')
>>> t
    ['a', 'c']
\end{verbatim}
%
El valor de retorno de {\tt remove} es {\tt None}.
\index{valor especial!None}
\index{None, valor especial}

Para eliminar más de un elemento, puedes utilizar {\tt del} con
índices de trozo:

\begin{verbatim}
>>> t = ['a', 'b', 'c', 'd', 'e', 'f']
>>> del t[1:5]
>>> t
    ['a', 'f']
\end{verbatim}
%
Como siempre, el trozo selecciona todos los elementos hasta el segundo índice
pero sin incluirlo.



\section{Listas y cadenas}
\index{lista}
\index{cadena}
\index{secuencia}

Una cadena es una secuencia de caracteres y una lista es una secuencia
de valores, pero una lista de caracteres no es lo mismo que una
cadena.  Para convertir una cadena en una lista de caracteres,
puedes utilizar {\tt list}:
\index{list, función}
\index{función!list}

\begin{verbatim}
>>> s = 'spam'
>>> t = list(s)
>>> t
    ['s', 'p', 'a', 'm']
\end{verbatim}
%
Dado que {\tt list} es el nombre de una función incorporada, deberías
evitar utilizarlo como nombre de variable.  Yo además evito {\tt l} porque
se parece mucho a {\tt 1}.  Entonces por eso uso {\tt t}.

La función {\tt list} separa la cadena en letras individuales.  Si
quieres separar una cadena en palabras, puedes utilizar
el método {\tt split}:
\index{metodo@método!split}
\index{split, método}

\begin{verbatim}
>>> s = 'pining for the fjords'
>>> t = s.split()
>>> t
    ['pining', 'for', 'the', 'fjords']
\end{verbatim}
%
Un argumento opcional llamado {\bf delimitador} especifica qué
caracteres usar como separador de palabras.
El siguiente ejemplo
usa un guión como delimitador:
\index{argumento opcional}
\index{opcional!argumento}
\index{delimitador}

\begin{verbatim}
>>> s = 'spam-spam-spam'
>>> delimitador = '-'
>>> t = s.split(delimitador)
>>> t
    ['spam', 'spam', 'spam']
\end{verbatim}
%
{\tt join} es el inverso de {\tt split}.
Toma una lista de cadenas y
concatena los elementos.  {\tt join} es un método de cadena,
por lo que tienes que invocarlo en el delimitador y pasarle la
lista como parámetro:
\index{metodo@método!join}
\index{join, método}
\index{concatenación}

\begin{verbatim}
>>> t = ['pining', 'for', 'the', 'fjords']
>>> delimitador = ' '
>>> s = delimitador.join(t)
>>> s
    'pining for the fjords'
\end{verbatim}
%
En este caso el delimitador es un carácter de espacio, por lo que
{\tt join} pone un espacio entre las palabras.  Para concatenar
cadenas sin espacios, puedes usar la cadena vacía,
\verb"''", como delimitador.
\index{cadena!vacía}
\index{vacía!cadena}


\section{Objetos y valores}
\label{equivalence}
\index{objeto}
\index{valor}

Si ejecutamos estas sentencias de asignación:

\begin{verbatim}
a = 'banana'
b = 'banana'
\end{verbatim}
%
sabemos que {\tt a} y {\tt b} se refieren a una
cadena, pero no
sabemos si se refieren a la {\em misma} cadena.
Hay dos estados posibles, mostrados en la Figura~\ref{fig.list1}.
\index{alias}

\begin{figure}
\centerline
{\includegraphics[scale=0.8]{figs/list1.pdf}}
\caption{Diagrama de estado.}
\label{fig.list1}
\end{figure}

En el primer caso, {\tt a} y {\tt b} se refieren a dos objetos diferentes que
tienen el mismo valor.  En el segundo caso, se refieren al mismo
objeto.
\index{operador!is}
\index{is, operador}

Para verificar si dos variables se refieren al mismo objeto, puedes
usar el operador {\tt is}.

\begin{verbatim}
>>> a = 'banana'
>>> b = 'banana'
>>> a is b
    True
\end{verbatim}
%
En este ejemplo, Python solo crea un objeto de cadena y tanto {\tt
  a} como {\tt b} se refieren a este.  Pero cuando creas dos listas, obtienes
dos objetos:

\begin{verbatim}
>>> a = [1, 2, 3]
>>> b = [1, 2, 3]
>>> a is b
    False
\end{verbatim}
%
Entonces el diagrama de estado se ve como la Figura~\ref{fig.list2}.
\index{diagrama!de estado}
\index{estado, diagrama de}

\begin{figure}
\centerline
{\includegraphics[scale=0.8]{figs/list2.pdf}}
\caption{Diagrama de estado.}
\label{fig.list2}
\end{figure}

En este caso diríamos que las dos listas son {\bf equivalentes},
porque tienen los mismos elementos, pero no {\bf idénticos}, porque
no son el mismo objeto.  Si dos objetos son idénticos, son
también equivalentes, pero si son equivalentes, no necesariamente son
idénticos.
\index{equivalencia}
\index{identidad}

Hasta ahora, hemos estado utilizando ``objeto'' y ``valor''
indistintamente, pero es más preciso decir que un objeto tiene un
valor.  Si evalúas {\tt [1, 2, 3]}, obtienes un objeto de
lista cuyo valor es una secuencia de enteros.  Si otra
lista tiene los mismos elementos, decimos que tiene el mismo valor, pero
no es el mismo objeto.
\index{objeto}
\index{valor}


\section{Alias}
\index{alias}
\index{referencia!alias}

Si {\tt a} se refiere a un objeto y asignas {\tt b = a},
entonces ambas variables se refieren al mismo objeto:

\begin{verbatim}
>>> a = [1, 2, 3]
>>> b = a
>>> b is a
    True
\end{verbatim}
%
El diagrama de estado se ve como la Figura~\ref{fig.list3}.
\index{diagrama!de estado}
\index{estado, diagrama de}

\begin{figure}
\centerline
{\includegraphics[scale=0.8]{figs/list3.pdf}}
\caption{Diagrama de estado.}
\label{fig.list3}
\end{figure}

La asociación de una variable con un objeto se llama {\bf
referencia}.  En este ejemplo, hay dos referencias al mismo
objeto.
\index{referencia}

Un objeto con más de una referencia tiene más
de un nombre, por lo que decimos que el objeto tiene un {\bf alias}.
\index{mutabilidad}

Si el objeto con alias es mutable, los cambios realizados con un alias afectan
al otro:

\begin{verbatim}
>>> b[0] = 42
>>> a
    [42, 2, 3]
\end{verbatim}
%
Aunque este comportamiento puede ser útil, es propenso a errores.  En general,
es más seguro evitar los alias cuando trabajes con objetos
mutables.
\index{inmutabilidad}

Para los objetos inmutables como las cadenas, los alias no son tan
problemáticos.  En este ejemplo:

\begin{verbatim}
a = 'banana'
b = 'banana'
\end{verbatim}
%
casi nunca hace una diferencia si {\tt a} y {\tt b} se refieren
a la misma cadena o no.


\section{Argumentos de lista}
\label{list.arguments}
\index{lista!como argumento}
\index{argumento}
\index{argumento de lista}
\index{referencia}
\index{parámetro}

Cuando pasas una lista a una función, la función obtiene una referencia a
la lista.  Si la función modifica la lista, la sentencia llamadora ve
el cambio.  Por ejemplo, \verb"sin_cabeza" quita el primer elemento
de una lista:

\begin{verbatim}
def sin_cabeza(t):
    del t[0]
\end{verbatim}
%
Se utiliza de la siguiente manera:

\begin{verbatim}
>>> letras = ['a', 'b', 'c']
>>> sin_cabeza(letras)
>>> letras
    ['b', 'c']
\end{verbatim}
%
El parámetro {\tt t} y la variable {\tt letras} son
alias para el mismo objeto.  El diagrama de pila se ve como la
Figura~\ref{fig.stack5}.
\index{diagrama!de pila}
\index{pila, diagrama de}

\begin{figure}
\centerline
{\includegraphics[scale=0.8]{figs/stack5.pdf}}
\caption{Diagrama de pila.}
\label{fig.stack5}
\end{figure}

Dado que la lista es compartida por dos marcos, la dibujé
entre estos.

Es importante distinguir entre operaciones que
modifican listas y operaciones que crean nuevas listas.  Por
ejemplo, el método {\tt append} modifica una lista, pero el
operador {\tt +} crea una nueva lista.
\index{metodo@método!append}
\index{append, método}
\index{lista!concatenación}
\index{concatenación!lista}

Aquí hay un ejemplo que utiliza {\tt append}:
%
\begin{verbatim}
>>> t1 = [1, 2]
>>> t2 = t1.append(3)
>>> t1
    [1, 2, 3]
>>> t2
    None
\end{verbatim}
%
El valor de retorno de {\tt append} es {\tt None}.

Aquí hay un ejemplo que utiliza el operador {\tt +}:
%
\begin{verbatim}
>>> t3 = t1 + [4]
>>> t1
    [1, 2, 3]
>>> t3
    [1, 2, 3, 4]
\end{verbatim}
%
El resultado del operador es una lista nueva y la lista original no ha
cambiado.

Esta diferencia es importante cuando escribes funciones que
se supone que modifican listas.  Por ejemplo, esta función
{\em no} elimina la cabeza de una lista:
%
\begin{verbatim}
def sin_cabeza_mal(t):
    t = t[1:]              # ¡INCORRECTO!
\end{verbatim}
%
El operador de trozo crea una nueva lista y la asignación
hace que {\tt t} se refiera a esta, pero eso no afecta a la llamadora.
\index{operador!de trozo}\index{slice}
\index{trozo!operador}
%
\begin{verbatim}
>>> t4 = [1, 2, 3]
>>> sin_cabeza_mal(t4)
>>> t4
    [1, 2, 3]
\end{verbatim}
%
Al principio de \verb"sin_cabeza_mal", {\tt t} y {\tt t4}
se refieren a la misma lista.  Al final, {\tt t} se refiere a una nueva lista,
pero {\tt t4} aún se refiere a la original, la lista sin modificar.

Una alternativa es escribir una función que cree y
devuelva una nueva lista.  Por
ejemplo, {\tt cola} devuelve todos los elementos
de una lista excepto el primero:

\begin{verbatim}
def cola(t):
    return t[1:]
\end{verbatim}
%
Esta función deja a la lista original sin modificar.
Se utiliza de la siguiente manera:

\begin{verbatim}
>>> letra = ['a', 'b', 'c']
>>> resto = cola(letras)
>>> resto
    ['b', 'c']
\end{verbatim}



\section{Depuración}
\index{depuración}

El uso descuidado de las listas (y otros objetos mutables)
puede llevar a largas horas de depuración.  Aquí hay algunas
trampas comunes y maneras de evitarlas:

\begin{enumerate}

\item La mayoría de los métodos de lista modifican el argumento y
  devuelven {\tt None}.  Esto es lo opuesto a los métodos de cadena,
  que devuelven una nueva cadena y dejan sola a la original.

Si te acostumbraste a escribir código de cadena como este:

\begin{verbatim}
palabra = palabra.strip()
\end{verbatim}

Es tentador escribir código de lista como este:

\begin{verbatim}
t = t.sort()           # ¡INCORRECTO!
\end{verbatim}
\index{metodo@método!sort}
\index{sort, método}

Dado que {\tt sort} devuelve {\tt None}, es probable que la
siguiente operación que realices con {\tt t} falle.

Antes de utilizar métodos y operadores de lista, deberías leer la
documentación cuidadosamente y luego probarlos en modo interactivo.

\item Escoge una forma y quédate con esa.

Parte del problema con las listas es que hay muchas
maneras de hacer las cosas.  Por ejemplo, para eliminar un elemento de
una lista, puedes utilizar {\tt pop}, {\tt remove}, {\tt del},
o incluso una asignación de trozo.

Para agregar un elemento, puedes utilizar el método {\tt append} o
el operador {\tt +}.  Suponiendo que {\tt t} es una lista y
{\tt x} es un elemento de lista, estas líneas son correctas:

\begin{verbatim}
t.append(x)
t = t + [x]
t += [x]
\end{verbatim}

Y estas son incorrectas:

\begin{verbatim}
t.append([x])          # ¡INCORRECTO!
t = t.append(x)        # ¡INCORRECTO!
t + [x]                # ¡INCORRECTO!
t = t + x              # ¡INCORRECTO!
\end{verbatim}

Prueba cada uno de estos ejemplos en modo interactivo para asegurarte
de que entiendes lo que haces.  Nota que solo el último
provoca un error de tiempo de ejecución; los otros tres son legales, pero
hacen lo incorrecto.


\item Crea copias para evitar los alias.
\index{alias!copiar para evitar}
\index{copia!para evitar alias}

Si quieres utilizar un método como {\tt sort} que modifique
el argumento, pero necesitas mantener la lista original
también, puedes crear una copia.

\begin{verbatim}
>>> t = [3, 1, 2]
>>> t2 = t[:]
>>> t2.sort()
>>> t
    [3, 1, 2]
>>> t2
    [1, 2, 3]
\end{verbatim}

En este ejemplo podrías utilizar también la función incorporada {\tt sorted},
que devuelve una nueva lista ordenada y deja sola a la original.
\index{sorted, función}
\index{función!sorted}

\begin{verbatim}
>>> t2 = sorted(t)
>>> t
    [3, 1, 2]
>>> t2
    [1, 2, 3]
\end{verbatim}

\end{enumerate}



\section{Glosario}

\begin{description}

\item[lista:] Una secuencia de valores.
\index{lista}

\item[elemento:] Uno de los valores en una lista (u otra secuencia),
también llamados ítems.
\index{elemento}

\item[lista anidada:] Una lista que es un elemento de otra lista.
\index{lista!anidada}

\item[acumulador:] Una variable utilizada en un bucle para sumar o
acumular un resultado.
\index{acumulador}

\item[asignación aumentada:] Una sentencia que actualiza el valor
de una variable utilizando un operador como \verb"+=".
\index{aumentada, asignación}
\index{asignación aumentada}
\index{recorrer}

\item[reducción:] Un patrón de procesamiento que recorre una secuencia
y acumula los elementos en un solo resultado.
\index{patrón!de reducción}
\index{reducción, patrón de}

\item[mapa:] Un patrón de procesamiento que recorre una secuencia y
realiza una operación en cada elemento.
\index{patrón!de mapa}
\index{mapa, patrón de}

\item[filtro:] Un patrón de procesamiento que recorre una lista y
selecciona los elementos que satisfacen algún criterio.
\index{patrón!de filtro}
\index{filtro, patrón de}

\item[objeto:] Algo a lo cual una variable puede referirse.  Un objeto
tiene un tipo y un valor.
\index{objeto}

\item[equivalente:] Que tiene el mismo valor.
\index{equivalencia}

\item[idéntico:] Que es el mismo objeto (lo cual implica equivalencia).
\index{identidad}

\item[referencia:] La asociación entre una variable y su valor.
\index{referencia}

\item[alias:] Una circunstancia donde dos o más variables se refieren al mismo
objeto.
\index{alias}

\item[delimitador:] Un carácter o cadena utilizado para indicar dónde
debería separarse una cadena.
\index{delimitador}

\end{description}


\section{Ejercicios}

Puedes descargar las soluciones a estos ejercicios en
\url{http://thinkpython.com/code/list_exercises.py}.

\begin{exercise}

Escribe una función llamada \verb"suma_anidada" que tome una lista de listas
de enteros y sume los elementos de todas las listas anidadas.
Por ejemplo:

\begin{verbatim}
>>> t = [[1, 2], [3], [4, 5, 6]]
>>> suma_anidada(t)
    21
\end{verbatim}

\end{exercise}

\begin{exercise}
\label{cumulative}
\index{suma acumulativa}

Escribe una función llamada {\tt cumsum} que tome una lista de números y
devuelva la suma acumulativa, es decir, una lista nueva donde el $i$-ésimo
elemento es la suma de los primeros $i+1$ elementos de la lista original.
Por ejemplo:

\begin{verbatim}
>>> t = [1, 2, 3]
>>> cumsum(t)
    [1, 3, 6]
\end{verbatim}

\end{exercise}

\begin{exercise}

Escribe una función llamada \verb"medio" que tome una lista y
devuelva una nueva lista que contenga todos los elementos excepto el primero
y el último.  Por ejemplo:

\begin{verbatim}
>>> t = [1, 2, 3, 4]
>>> medio(t)
    [2, 3]
\end{verbatim}

\end{exercise}

\begin{exercise}

Escribe una función llamada \verb"acortar" que tome una lista, la modifique
eliminando el primer y último elemento, y devuelva {\tt None}.
Por ejemplo:

\begin{verbatim}
>>> t = [1, 2, 3, 4]
>>> acortar(t)
>>> t
    [2, 3]
\end{verbatim}

\end{exercise}


\begin{exercise}
Escribe una función llamada \verb"esta_ordenada" que tome una lista como
parámetro y devuelva {\tt True} si la lista está ordenada de manera
ascendente y {\tt False} si no.  Por ejemplo:

\begin{verbatim}
>>> esta_ordenada([1, 2, 2])
    True
>>> esta_ordenada(['b', 'a'])
    False
\end{verbatim}

\end{exercise}


\begin{exercise}
\label{anagram}
\index{anagrama}

Dos palabras son anagramas si puedes reordenar las letras de una
para escribir la otra.  Escribe una función llamada \verb"es_anagrama"
que tome dos cadenas y devuelva {\tt True} si son anagramas.
\end{exercise}



\begin{exercise}
\label{duplicate}
\index{duplicado}
\index{unicidad}

Escribe una función llamada \verb"tiene_duplicados" que tome
una lista y devuelva {\tt True} si hay algún elemento que
aparece más de una vez.  No debería modificar la lista
original.

\end{exercise}


\begin{exercise}

Este ejercicio está relacionado con la denominada ``Paradoja del cumpleaños'', de la cual
puedes leer en \url{http://en.wikipedia.org/wiki/Birthday_paradox}.
\index{paradoja del cumpleaños}

Si hay 23 estudiantes en tu clase, ¿cuáles son las posibilidades
de que dos de ellos estén de cumpleaños el mismo día?  Puedes estimar esta
probabilidad generando muestras al azar de 23 cumpleaños
y verificar coincidencias.  Pista: puedes generar cumpleaños aleatorios
con la función {\tt randint} del módulo {\tt random}.
\index{modulo@módulo!random}
\index{random, módulo}
\index{función!randint}
\index{randint, función}

Puedes descargar mi
solución en \url{http://thinkpython.com/code/birthday.py}.

\end{exercise}



\begin{exercise}
\index{metodo@método!append}
\index{append, método}
\index{lista!concatenación}
\index{concatenación!lista}

Escribe una función que lea el archivo {\tt words.txt} y construya
una lista con un elemento por palabra.  Escribe dos versiones de
esta función, una utilizando el método {\tt append} y la
otra utilizando la notación {\tt t = t + [x]}.  ¿Cuál toma más
tiempo en ejecutar?  ¿Por qué?

Solución: \url{http://thinkpython.com/code/wordlist.py}.
\index{modulo@módulo!time}
\index{time, módulo}

\end{exercise}


\begin{exercise}
\label{wordlist1}
\label{bisection}
\index{pertenencia!búsqueda de bisección}
\index{busqueda@búsqueda!de bisección}
\index{bisección, búsqueda de}
\index{pertenencia!búsqueda binaria}
\index{busqueda@búsqueda!binaria}
\index{binaria, búsqueda}

Para verificar si una palabra está en la lista de palabras, podrías utilizar
el operador {\tt in}, pero sería lento porque busca
en orden a través de las palabras.

Debido a que las palabras están en orden alfabético, podemos acelerar las cosas
con una búsqueda de bisección (también conocida como búsqueda binaria), que es
similar a lo que haces cuando buscas una palabra en el diccionario (el libro, no la estructura de datos).
Comienzas en el medio y verificas si la palabra que buscas
viene antes de la palabra en el medio de la lista.  Si es así,
buscas en la primera mitad de la lista de la misma manera.  De lo contrario, buscas
la segunda mitad.

De cualquier manera, cortas por la mitad el espacio de búsqueda que queda.  Si la
lista de palabras tiene 113,809 palabras, tomará alrededor de 17 pasos
para encontrar la palabra o concluir que no está.

Escribe una función llamada \verb"in_bisect" que tome una lista ordenada
y un valor objetivo, y devuelva {\tt True} si la palabra está
en la lista y {\tt False} si no está.
\index{modulo@módulo!bisect}
\index{bisect, módulo}

¡O bien podrías leer la documentación del módulo {\tt bisect}
y utilizarlo!  Solución: \url{http://thinkpython.com/code/inlist.py}.

\end{exercise}

\begin{exercise}
\index{par de palabras inversas}

Dos palabras son un ``par inverso'' si cada una es el inverso de la
otra.  Escribe un programa que encuentre todos los pares inversos en la
lista de palabras.  Solución: \url{http://thinkpython.com/code/reverse_pair.py}.

\end{exercise}

\begin{exercise}
\index{palabras entrelazadas}

Dos palabras se ``entrelazan'' si tomando letras alternas de cada una se forma
una nueva palabra.  Por ejemplo, ``shoe'' y ``cold''
se entrelazan para formar ``schooled''.
Solución: \url{http://thinkpython.com/code/interlock.py}.
Crédito: Este ejercicio está inspirado en un ejemplo de \url{http://puzzlers.org}.

\begin{enumerate}

\item Escribe un programa que encuentre todos los pares de palabras que se entrelazan.
  Pista: ¡no revises todos los pares!

\item ¿Puedes encontrar alguna palabra que sea un triple entrelazado? Es decir,
  una palabra que si se lee cada tres letras, comenzando con la primera, la segunda o
  la tercera letra, se forma una nueva palabra.

\end{enumerate}
\end{exercise}


\chapter{Diccionarios}

Este capítulo presenta otro tipo incorporado llamado diccionario.
Los diccionarios son una de las mejores características de Python: son los
bloques de construcción de muchos algoritmos eficientes y elegantes.


\section{Un diccionario es un mapeo}

\index{diccionario}
\index{tipo!dict}
\index{clave}
\index{par clave-valor}
\index{indice@índice}
Un {\bf diccionario} es como una lista, pero más general.  En una lista,
los índices tienen que ser enteros; en un diccionario pueden ser
(casi) de cualquier tipo.

Un diccionario contiene una colección de índices, que se llaman {\bf
  claves}, y una colección de valores.  Cada clave está asociada a un
valor único.  La asociación de una clave y un valor se llama {\bf
  par clave-valor}, o a veces {\bf ítem}.  \index{item@ítem}

En lenguaje matemático, un diccionario representa un {\bf mapeo}
de las claves a los valores, por tanto puedes decir también que cada clave
``mapea a'' un valor.
Como ejemplo, construiremos un diccionario que mapea de palabras en inglés
a palabras en español, así las claves y los valores son todas cadenas.

La función {\tt dict} crea un nuevo diccionario sin ítems.
Como {\tt dict} es el nombre de una función incorporada,
deberías evitar utilizarla como nombre de variable.
\index{función!dict}
\index{dict, función}

\begin{verbatim}
>>> ing_esp = dict()
>>> ing_esp
    {}
\end{verbatim}

Las llaves, \verb"{}", representan un diccionario vacío.
Para agregar ítems al diccionario, puedes utilizar corchetes:
\index{llaves}
%\index{bracket!squiggly}

\begin{verbatim}
>>> ing_esp['one'] = 'uno'
\end{verbatim}
%
Esta línea crea un ítem que mapea de la clave
\verb"'one'" al valor \verb"'uno'".  Si imprimimos el
diccionario de nuevo, vemos un par clave-valor con un signo de dos puntos
entre la clave y el valor:

\begin{verbatim}
>>> ing_esp
    {'one': 'uno'}
\end{verbatim}
%
Este formato de salida es también un formato de entrada.  Por ejemplo,
puedes crear un nuevo diccionario con tres ítems:

\begin{verbatim}
>>> ing_esp = {'one': 'uno', 'two': 'dos', 'three': 'tres'}
\end{verbatim}
%
Pero si imprimes {\tt ing\_esp}, quizás te sorprendas:

\begin{verbatim}
>>> ing_esp
    {'one': 'uno', 'three': 'tres', 'two': 'dos'}
\end{verbatim}
%
El orden de los pares clave-valor podría no ser el mismo.  Si
escribiste el mismo ejemplo en tu computador, podrías obtener un
resultado diferente.  En general, el orden de los ítems en
un diccionario es impredecible.

Sin embargo, eso no es un problema porque
los elementos de un diccionario nunca se indexan con índices enteros.
En cambio, utilizas las claves para buscar los valores correspondientes:

\begin{verbatim}
>>> ing_esp['two']
    'dos'
\end{verbatim}
%
La clave \verb"'two'" siempre mapea al valor \verb"'dos'", entonces el orden
de los ítems no importa.

Si la clave no está en el diccionario, obtienes una excepción:
\index{excepción!KeyError}
\index{KeyError}

\begin{verbatim}
>>> ing_esp['four']
    KeyError: 'four'
\end{verbatim}
%
La función {\tt len} funciona con diccionarios: devuelve el
número de pares clave-valor.
\index{función!len}
\index{len, función}

\begin{verbatim}
>>> len(ing_esp)
    3
\end{verbatim}
%
El operador {\tt in} también funciona con diccionarios: te dice si
algo aparece como una {\em clave} en el diccionario (aparecer
como un valor no basta).
\index{pertenencia!diccionario}
\index{in, operador}
\index{operador!in}

\begin{verbatim}
>>> 'one' in ing_esp
    True
>>> 'uno' in ing_esp
    False
\end{verbatim}
%
Para ver si algo aparece como un valor en un diccionario,
puedes utilizar el método {\tt values}, el cual devuelve una colección de
valores, y entonces utilizar el operador {\tt in}:
\index{metodo@método!values}
\index{values, método}

\begin{verbatim}
>>> valores = ing_esp.values()
>>> 'uno' in valores
    True
\end{verbatim}
%
El operador {\tt in} utiliza diferentes algoritmos para las listas y los
diccionarios.  Para las listas, busca los elementos de la lista en
orden, como en la Sección~\ref{find}.  A medida que la lista se vuelve más larga,
el tiempo de búsqueda se hace más largo en proporción directa.

Los diccionarios de Python utilizan una estructura de datos
llamada {\bf tabla hash} que tiene una propiedad notable: el
operador {\tt in} toma casi la misma cantidad de tiempo sin importar
cuántos ítems hay en el diccionario.  Explico cómo eso es posible
en la Sección~\ref{hashtable}, pero la explicación podría no tener
sentido hasta que hayas leído algunos capítulos más.


\section{El diccionario como colección de contadores}
\label{histogram}
\index{contador}

Supongamos que te dan una cadena y quieres contar cuántas
veces aparece cada letra.  Hay varias maneras en que podrías hacerlo:

\begin{enumerate}

\item Podrías crear 26 variables, una para cada letra del
alfabeto.  Luego podrías recorrer la cadena y, para cada
carácter, incrementar el contador correspondiente, probablemente utilizando
un condicional encadenado.

\item Podrías crear una lista de 26 elementos.  Luego podrías
convertir cada carácter a un número (usando la función incorporada
{\tt ord}), utilizar el número como un índice dentro de la lista e incrementar
el contador apropiado.

\item Podrías crear un diccionario con caracteres como claves
y contadores como los valores correspondientes.  La primera vez que
veas un carácter, añadirías un ítem al diccionario.  Después
de eso incrementarías el valor de un ítem existente.

\end{enumerate}

Cada una de estas opciones realiza la misma computación, pero cada una
de ellas implementa esa computación de una manera diferente.
\index{implementación}

Una {\bf implementación} es una manera de realizar una computación;
algunas implementaciones son mejores que otras.  Por ejemplo,
una ventaja de la implementación con diccionario es que no
tenemos que saber de antemano qué letras aparecen en la cadena
y solo tenemos que hacer espacio para las letras que sí aparecen.

Así es como se vería el código:

\begin{verbatim}
def histograma(s):
    d = dict()
    for c in s:
        if c not in d:
            d[c] = 1
        else:
            d[c] += 1
    return d
\end{verbatim}
%
El nombre de la función es {\tt histograma}, que es un término
estadístico para una colección de contadores (o frecuencias).
\index{histograma}
\index{frecuencia}
\index{recorrer}

La primera línea de la
función crea un diccionario vacío.  El bucle {\tt for} recorre
la cadena.  En cada paso por el bucle, si el carácter {\tt c} no
está en el diccionario, creamos un nuevo ítem con clave {\tt c} y
valor inicial 1 (dado que hemos visto esta letra una vez).  Si {\tt c}
ya está en el diccionario, incrementamos {\tt d[c]}.
\index{histograma}

Funciona así:

\begin{verbatim}
>>> h = histograma('brontosaurus')
>>> h
    {'a': 1, 'b': 1, 'o': 2, 'n': 1, 's': 2, 'r': 2, 'u': 2, 't': 1}
\end{verbatim}
%
El histograma indica que las letras \verb"'a'" y \verb"'b'"
aparecen una vez; \verb"'o'" aparece dos veces, y así sucesivamente.


\index{metodo@método!get}
\index{get, método}
Los diccionarios tienen un método llamado {\tt get} que toma una clave
y un valor por defecto.  Si la clave aparece en el diccionario,
{\tt get} devuelve el valor correspondiente; de lo contrario, devuelve
el valor por defecto.  Por ejemplo:

\begin{verbatim}
>>> h = histograma('a')
>>> h
    {'a': 1}
>>> h.get('a', 0)
    1
>>> h.get('c', 0)
    0
\end{verbatim}
%
Como ejercicio, usa {\tt get} para escribir {\tt histograma} de manera más concisa.
Deberías ser capaz de eliminar la sentencia {\tt if}.


\section{Bucles y diccionarios}
\index{diccionario!bucles con}
\index{bucle!con diccionarios}
\index{recorrer}

Si utilizas un diccionario en una sentencia {\tt for}, este recorre
las claves del diccionario.  Por ejemplo, \verb"imprimir_hist"
imprime cada clave y el valor correspondiente:

\begin{verbatim}
def imprimir_hist(h):
    for c in h:
        print(c, h[c])
\end{verbatim}
%
La salida se ve así:

\begin{verbatim}
>>> h = histograma('parrot')
>>> imprimir_hist(h)
    a 1
    p 1
    r 2
    t 1
    o 1
\end{verbatim}
%
Nuevamente, las claves no estan en un orden particular.  Para recorrer las claves
en orden, puedes utilizar la función incorporada {\tt sorted}:
\index{sorted, función}
\index{función!sorted}

\begin{verbatim}
>>> for clave in sorted(h):
...     print(clave, h[clave])
    a 1
    o 1
    p 1
    r 2
    t 1
\end{verbatim}

%TODO: get this on Atlas


\section{Consulta inversa}
\label{raise}
\index{diccionario!consulta}
\index{diccionario!consulta inversa}
\index{consulta, diccionario}
\index{consulta inversa, diccionario}

Dado un diccionario {\tt d} y una clave {\tt k}, es fácil
encontrar el valor correspondiente {\tt v = d[k]}.  Esta operación
se llama {\bf consulta} (en inglés, {\em lookup}).

¿Y qué pasa si tienes {\tt v} y quieres encontrar {\tt k}?
Tienes dos problemas: primero, podría haber más de una
clave que mapee al valor {\tt v}.  Dependiendo de la aplicación,
quizás puedas elegir una o tengas que hacer
una lista que las contenga a todas.  Segundo, no hay
una sintaxis sencilla para hacer una {\bf consulta inversa}: tienes que buscar.

Aquí hay una función que toma un valor y devuelve la primera
clave que mapea a ese valor:

\begin{verbatim}
def consulta_inversa(d, v):
    for k in d:
        if d[k] == v:
            return k
    raise LookupError()
\end{verbatim}
%
Esta función es otro ejemplo de patrón de búsqueda pero que
utiliza una característica que no hemos visto antes, {\tt raise}.  La
{\bf sentencia raise} causa una excepción; en este caso causa un
{\tt LookupError}, que es una excepción incorporada que se utiliza para indicar
que falló una operación de consulta.
\index{busqueda@búsqueda}
\index{busqueda@búsqueda!patrón de} \index{sentencia!raise} \index{raise, sentencia}
\index{excepción!LookupError} \index{LookupError}

Si se llega al final del bucle, significa que {\tt v}
no aparece en el diccionario como valor, por lo que plantea una
excepción.

Aquí hay un ejemplo de una consulta inversa eficaz:

\begin{verbatim}
>>> h = histograma('parrot')
>>> clave = consulta_inversa(h, 2)
>>> clave 
    'r'
\end{verbatim}
%
Y una ineficaz:

\begin{verbatim}
>>> clave = consulta_inversa(h, 3)
    Traceback (most recent call last):
      File "<stdin>", line 1, in <module>
      File "<stdin>", line 5, in consulta_inversa
    LookupError
\end{verbatim}
%
El efecto de cuando levantas una excepción mediante {\tt raise} es el mismo que cuando
Python levanta una: imprime un rastreo y un mensaje de error.
\index{rastreo}
\index{argumento opcional}
\index{opcional!argumento}

Cuando levantes una excepción, puedes proporcionar un mensaje de error detallado como argumento opcional.  Por ejemplo:

\begin{verbatim}
>>> raise LookupError('el valor no aparece en el diccionario')
    Traceback (most recent call last):
      File "<stdin>", line 1, in ?
    LookupError: el valor no aparece en el diccionario
\end{verbatim}
%
Una consulta inversa es mucho más lenta que una consulta directa; si
tienes que hacerlo a menudo, o si el diccionario se vuelve grande, el rendimiento
de tu progama se verá afectado.


\section{Diccionarios y listas}
\label{invert}

Las listas pueden aparecer como valores en un diccionario.  Por ejemplo, si
te dan un diccionario que mapea de letras a frecuencias,
quizás quieras invertirlo, es decir, crear un diccionario que mapee
de frecuencias a letras.  Dado que podría haber muchas letras
con la misma frecuencia, cada valor en el diccionario invertido
debería ser una lista de letras.
\index{invertir diccionario}
\index{diccionario!invertir}

Aquí hay una función que invierte un diccionario:

\begin{verbatim}
def invertir_dict(d):
    inverso = dict()
    for clave in d:
        valor = d[clave]
        if valor not in inverso:
            inverso[valor] = [clave]
        else:
            inverso[valor].append(clave)
    return inverso
\end{verbatim}
%
En cada paso por el bucle, {\tt clave} obtiene una clave de {\tt d} y
{\tt valor} obtiene el valor correspondiente.  Si {\tt valor} no está en {\tt
  inverso}, lo cual significa que no lo hemos visto antes, entonces creamos un nuevo
ítem y lo inicializamos con un {\bf singleton} (una lista que contiene un
único elemento).  De lo contrario, hemos visto este valor antes, por lo cual
anexamos a la lista la clave correspondiente.  \index{singleton}

Aquí hay un ejemplo:

\begin{verbatim}
>>> hist = histograma('parrot')
>>> hist
    {'a': 1, 'p': 1, 'r': 2, 't': 1, 'o': 1}
>>> inv = invertir_dict(hist)
>>> inv
    {1: ['a', 'p', 't', 'o'], 2: ['r']}
\end{verbatim}

\begin{figure}
\centerline
{\includegraphics[scale=0.8]{figs/dict1.pdf}}
\caption{Diagrama de estado.}
\label{fig.dict1}
\end{figure}

La Figura~\ref{fig.dict1} es un diagrama de estado que muestra a {\tt hist} e {\tt inv}.
Un diccionario se representa como una caja con el tipo {\tt dict} arriba suyo
y el par clave-valor adentro.  Si los valores son enteros, números de coma flotante o
cadenas, los dibujo adentro de la caja, pero usualmente dibujo las listas
afuera de la caja, solo para mantener simple al diagrama.
\index{diagrama!de estado}
\index{estado, diagrama de}

Las listas pueden ser valores en un diccionario, tal como muestra este ejemplo simple, pero
no pueden ser claves.  Esto es lo que ocurre si lo intentas:
\index{TypeError}
\index{excepción!TypeError}


\begin{verbatim}
>>> t = [1, 2, 3]
>>> d = dict()
>>> d[t] = 'ups'
    Traceback (most recent call last):
      File "<stdin>", line 1, in ?
    TypeError: list objects are unhashable
\end{verbatim}
%
Anteriormente mencioné que un diccionario se implementa utilizando
una tabla hash y eso significa que las claves tienen que ser {\bf hashables}.
\index{función!hash}
\index{hashable}

Un {\bf hash} es una función que toma un valor (de cualquier tipo)
y devuelve un entero.  Los diccionarios utilizan estos enteros,
llamados valores hash, para almacenar y consultar pares clave-valor.
\index{inmutabilidad}

Este sistema funciona bien si las claves son inmutables.  Pero si las
claves son mutables, como las listas, ocurren cosas malas.  Por ejemplo,
cuando creas un par clave-valor, Python hashea la clave y
la almacena en la ubicación correspondiente.  Si modificas la
clave y luego la hasheas de nuevo, iría a una ubicación diferente.
En ese caso podrías tener dos entradas para la misma clave,
o quizás no seas capaz de encontrar una clave.  De cualquier manera, el
diccionario no funcionaría de manera correcta.

Por eso es que las claves tienen que ser hashables y, por la misma razón, los tipos mutables como
las listas no lo son.  La manera más simple de evitar esta limitación es
utilizar tuplas, las cuales veremos en el capítulo siguiente.

Dado que los diccionarios son mutables, estos no pueden utilizarse como claves,
pero {\em pueden} utilizarse como valores.


\section{Memos}
\label{memoize}

Si jugaste con la función {\tt fibonacci} de la
Sección~\ref{one.more.example}, quizás has notado que mientras más grande
es el argumento que entregues, más tarda la función en ejecutarse.
Además, el tiempo de ejecución aumenta rápidamente. 
\index{fibonacci, función de}
\index{función!fibonacci}

Para entender el por qué, considera la Figura~\ref{fig.fibonacci}, que muestra
el {\bf gráfico de llamadas} para {\tt fibonacci} con {\tt n=4}.

\begin{figure}
\centerline
{\includegraphics[scale=0.7]{figs/fibonacci.pdf}}
\caption{Gráfico de llamadas.}
\label{fig.fibonacci}
\end{figure}

Un gráfico de llamadas muestra un conjunto de marcos de funciones, con líneas que conectan cada
marco a los marcos de las funciones que están siendo llamadas.  En la parte de arriba del
gráfico, {\tt fibonacci} con {\tt n=4} llama a {\tt fibonacci} con {\tt
n=3} y {\tt n=2}.  A su vez, {\tt fibonacci} con {\tt n=3} llama a
{\tt fibonacci} con {\tt n=2} y {\tt n=1}.  Y así sucesivamente.
\index{marco de una función}
\index{marco}
\index{grafico de llamadas@gráfico de llamadas}

Cuenta cuántas veces se llama a {\tt fibonacci(0)} y {\tt fibonacci(1)}.
Esta es una solución ineficiente para el probema y se pone peor
a medida que el argumento se hace más grande.

\index{memo}

Una solución es hacer un seguimiento de los valores que ya han sido
calculados almacenándolos en un diccionario.   Un valor previamente calculado
que se almacena para un uso posterior se llama {\bf memo}.  Aquí hay una versión
``memoizada'' de {\tt fibonacci}:

\begin{verbatim}
conocidos = {0:0, 1:1}

def fibonacci(n):
    if n in conocidos:
        return conocidos[n]

    resultado = fibonacci(n-1) + fibonacci(n-2)
    conocidos[n] = resultado
    return resultado
\end{verbatim}
%
{\tt conocidos} es un diccionario que hace un seguimiento de los números de Fibonacci
que ya conocemos.  Comienza con
dos ítems: 0 mapea a 0 y 1 mapea a 1.

Cada vez que se llama a {\tt fibonacci}, revisa a {\tt conocidos}.
Si el resultado ya está ahí, puede devolverlo
inmediatamente.  De lo contrario, tiene que
calcular el nuevo valor, agregarlo al diccionario y devolverlo.

Si ejecutas esta versión de {\tt fibonacci} y la comparas con
la original, encontrarás que esta es mucho más rápida.



\section{Variables globales}
\index{variable global}
\index{global, variable}

En los ejemplos anteriores, {\tt conocidos} se crea fuera de la función,
por lo que pertenece al marco especial llamado \verb"__main__".
Las variables en \verb"__main__" a veces se llaman {\bf globales}
porque se puede acceder a ellos desde cualquier función.  A diferencia de las variables
locales, que desaparecen cuando su función termina, las variables globales 
persisten de una llamada a función a la siguiente.
\index{flag}\index{bandera}
\index{main}

Es común utilizar variables globales para las {\bf banderas} (en inglés, {\em flags}), es decir,
variables booleanas que indican si una condición
es verdadera.  Por ejemplo, algunos programas utilizan
una bandera llamada {\tt verbose} para controlar el nivel de detalle en la
salida:

\begin{verbatim}
verbose = True

def ejemplo1():
    if verbose:
        print('Ejecutando ejemplo1')
\end{verbatim}
%
Si intentas reasignar una variable global, quizás te sorprendas.
El siguiente ejemplo se supone que hace seguimiento de si la
función ha sido llamada:
\index{reasignación}

\begin{verbatim}
fue_llamada = False

def ejemplo2():
    fue_llamada = True         # INCORRECTO 
\end{verbatim} 
% 
Pero si lo ejecutas verás que el valor de \verb"fue_llamada"
no cambia. El problema es que {\tt ejemplo2} crea una nueva variable local
con nombre \verb"fue_llamada".  La variable local se va cuando 
la función termina y no tiene efecto en la variable global.
\index{sentencia!global}
\index{global, sentencia}
\index{declaración}

Para reasignar una variable global dentro de una función, tienes que
{\bf declarar} la variable global antes de utilizarla:

\begin{verbatim}
fue_llamada = False

def ejemplo2():
    global fue_llamada
    fue_llamada = True
\end{verbatim}
%
La {\bf sentencia global} le dice al intérprete
algo como ``En esta función, cuando digo \verb"fue_llamada",
me refiero a la variable global; no crees una local.''
\index{actualizar!variable global}
\index{variable global!actializar}

Aquí hay un ejemplo que intenta actualizar una variable global:

\begin{verbatim}
contar = 0

def ejemplo3():
    contar = contar + 1          # INCORRECTO
\end{verbatim}
%
Si lo ejecutas obtienes:
\index{UnboundLocalError}
\index{excepción!UnboundLocalError}

\begin{verbatim}
UnboundLocalError: local variable 'contar' referenced before assignment
\end{verbatim}
%
Python supone que {\tt contar} es local, y bajo ese supuesto
lo estás leyendo antes de escribirlo.  La solución, nuevamente,
es declarar {\tt contar} como global.
\index{contador}

\begin{verbatim}
def ejemplo3():
    global contar
    contar += 1
\end{verbatim}
%
Si una variable global se refiere a un valor mutable, puedes modificar
el valor sin declarar la variable:
\index{mutabilidad}

\begin{verbatim}
conocidos = {0:0, 1:1}

def ejemplo4():
    conocidos[2] = 1
\end{verbatim}
%
Entonces puedes agregar, eliminar y reemplazar elementos de una lista global o diccionario
global, pero si quieres reasignar la variable,
tienes que declararla:

\begin{verbatim}
def ejemplo5():
    global conocidos
    conocidos = dict()
\end{verbatim}
%
Las variables globales pueden ser útiles, pero si tienes muchas,
y las modificas frecuentemente, pueden hacer que los programas sean
difíciles de depurar.


\section{Depuración}
\index{depuración}

A medida que trabajes con conjuntos de datos más grandes, depurar imprimiendo y
verificando la salida a mano puede volverse algo difícil de manejar.  Aquí hay algunas
sugerencias para depurar conjuntos grandes de datos.

\begin{description}

\item[Reduce la escala:] Si es posible, reduce el tamaño del
conjunto de datos.  Por ejemplo, si el programa lee un archivo de texto, comienza 
solo con las primeras 10 líneas, o con el ejemplo más pequeño que puedas encontrar.
Puedes editar aquellos archivos o (mejor) modificar el
programa de manera que este lea solo las primeras {\tt n} líneas.

Si hay un error, puedes reducir {\tt n} al valor más pequeño
que muestre el error y luego incrementarlo gradualmente
mientras encuentras y corriges los errores.

\item[Revisa resúmenes y tipos:] En lugar de imprimir y verificar el conjunto
de datos completo, considera imprimir resúmenes de los datos: por ejemplo,
el número de ítems en el diccionario o el total de una lista de números.

Una causa común de errores de tiempo de ejecución es un valor que no es del tipo
correcto.  Para depurar esta clase de errores, a menudo es suficiente imprimir
el tipo del valor.

\item[Escribe verificaciones automáticas:]  A veces puedes escribir código para verificar
errores de manera automática. Por ejemplo, si estás calculando el
promedio de una lista de números, podrías verificar que el resultado no
sea mayor que el elemento más grande de la lista ni menor que
el más pequeño. A esto se le llama ``prueba de cordura'' (en inglés, {\em sanity check})
porque detecta resultados que son ``locos''.
\index{sanity check}\index{prueba!de cordura}
\index{prueba!de consistencia}

Otro tipo de prueba compara los resultados de dos computaciones
diferentes para ver si son consistentes. A esto se le llama
``prueba de consistencia''.

\item[Dale formato a la salida:] Dar formato a la salida de la depuración
puede hacer más fácil detectar un error.  Vimos un ejemplo en la 
Sección~\ref{factdebug}.  Otra herramienta que quizás encuentres útil es el módulo {\tt pprint},
el cual proporciona una función {\tt pprint} que muestra tipos incorporados en
un formato más legible por humanos ({\tt pprint} significa
``pretty print'').
\index{pretty print}
\index{modulo@módulo!pprint}
\index{pprint, módulo}

\end{description}

Nuevamente, el tiempo que pasas construyendo andamiaje puede reducir
el tiempo que pasas depurando.
\index{scaffolding}\index{andamiaje}


\section{Glosario}

\begin{description}

\item[mapeo:] Una relación en la cual cada elemento de un conjunto
corresponde a un elemento de otro conjunto.
\index{mapeo}

\item[diccionario:] Un mapeo de las claves a sus
valores correspondienes.
\index{diccionario}

\item[par clave-valor:] La representación del mapeo de
una clave a un valor.
\index{par clave-valor}

\item[ítem:] En un diccionario, otro nombre para un par
clave-valor.
\index{item@ítem!de un diccionario}

\item[clave:] Un objeto que aparece en un diccionario como la
primera parte de un par clave-valor.
\index{clave}

\item[valor:] Un objeto que aparece en un diccionario como la
segunda parte de un par clave-valor.  Esto es más específico que
nuestro uso anterior de la palabra ``valor''.
\index{valor}

\item[implementación:] Una manera de realizar una computación.
\index{implementación}

\item[tabla hash:] El algoritmo utilizado para implementar los
diccionarios de Python.
\index{tabla hash}

\item[función hash:] Una función utilizada por una tabla hash para calcular
la ubicación de una clave.
\index{función!hash}

\item[hashable:] Un tipo que tiene una función hash.  Los tipos inmutables
como los enteros, números de
coma flotante y cadenas son hashables; los tipos mutables como las listas y
diccionarios no lo son.
\index{hashable}

\item[consulta:] Una operación de diccionario que toma una clave y encuentra
el valor correspondiente.
\index{consulta}

\item[consulta inversa:] Una operación de diccionario que toma un valor y encuentra
una o más claves que mapean a este.
\index{consulta inversa}

\item[sentencia raise:]  Una sentencia que (deliberadamente) levanta una excepción.
\index{sentencia!raise}
\index{raise, sentencia}

\item[singleton:] Una lista (u otra secuencia) con un solo elemento.
\index{singleton}

\item[gráfico de llamadas:] Un diagrama que muestra cada marco creado durante
la ejecución de un programa, con una flecha desde cada llamador
hacia cada llamado.
\index{grafico de llamadas@gráfico de llamadas}
\index{diagrama!gráfico de llamadas}

\item[memo:] Un valor calculado que se almacena para evitar una futura
computación innecesaria.
\index{memo}

\item[variable global:]  Una variable definida fuera de la función. Las
variables globales pueden ser accesibles desde cualquier función.
\index{variable global}

\item[sentencia global:]  Una sentencia que declara global a un nombre
de variable.
\index{sentencia!global}
\index{global, sentencia}

\item[bandera:] Una variable booleana que se utiliza para indicar si una condición
es verdadera.
\index{bandera}

\item[declaración:] Una sentencia como {\tt global} que le dice al
intérprete algo sobre una variable.
\index{declaración}

\end{description}


\section{Ejercicios}

\begin{exercise}
\label{wordlist2}
\index{pertenencia!a conjunto}
\index{conjunto!pertenencia a}

Escribe una función que lea las palabras en {\tt words.txt} y
las almacene como claves en un diccionario.  No importa cuáles sean
los valores.  Luego puedes utilizar el operador {\tt in}
como una manera rápida de verificar si una cadena está en
el diccionario.

Si hiciste el Ejercicio~\ref{wordlist1}, puedes comparar la velocidad
de esta implementación con el operador {\tt in} de lista y la
búsqueda de bisección.

\end{exercise}


\begin{exercise}
\label{setdefault}

Lee la documentación del método de diccionario {\tt setdefault}
y utilízalo para escribir una versión más concisa de \verb"invertir_dict".
Solución: \url{http://thinkpython.com/code/invert_dict.py}.
\index{setdefault!método}
\index{metodo@método!setdefault}

\end{exercise}


\begin{exercise}
Memoiza la función de Ackermann del Ejercicio~\ref{ackermann} y ve si
la memoización permite evaluar la función con argumentos
más grandes.  Pista: no.
Solución: \url{http://thinkpython.com/code/ackermann_memo.py}.
\index{Ackermann, función de}
\index{función!ack}

\end{exercise}



\begin{exercise}
\index{duplicado}

Si hiciste el Ejercicio~\ref{duplicate}, ya tienes
una función con nombre \verb"tiene_duplicados" que toma una lista
como parámetro y devuelve {\tt True} si hay algún objeto
que aparece más de una vez en la lista.

Utiliza un diccionario para escribir una versión más rápida y simple de
\verb"tiene_duplicados".
Solución: \url{http://thinkpython.com/code/has_duplicates.py}.

\end{exercise}


\begin{exercise}
\label{exrotatepairs}
\index{rotación de letras}
\index{letras, rotación de}

Dos palabras son ``pares rotativos'' si puedes rotar una de ellas
y obtener la otra (ver \verb"rotar_palabra" en el Ejercicio~\ref{exrotate}).

Escribe un programa que lea una lista de palabras y encuentre todos los pares
rotativos.  Solución: \url{http://thinkpython.com/code/rotate_pairs.py}.

\end{exercise}


\begin{exercise}
\index{Car Talk}
\index{Puzzler}

Aquí hay otro Puzzler de {\em Car Talk}
(\url{http://www.cartalk.com/content/puzzlers}):

\begin{quote}
Este fue enviado por un compañero llamado Dan O'Leary. Se encontró hace poco con una palabra
monosílaba de cinco letras común, que tiene la siguiente propiedad
única. Cuando quitas la primera letra, las demás letras forman
un homófono de la palabra original, es decir, una palabra que suena exactamente
igual. Reemplaza la primera letra, es decir, ponla de nuevo y quita
la segunda letra y el resultado es otro homófono de la
palabra original. Y la pregunta es, ¿cuál es la palabra?

Ahora voy a darte un ejemplo que no funciona. Veamos la palabra
de cinco letras, `wrack.' W-R-A-C-K, ya sabes, como `wrack with
pain.' Si quito la primera letra, me quedo con una palabra de cuatro
letras, 'R-A-C-K.' Como en `Holy cow, did you see the rack on that buck!
It must have been a nine-pointer!' Es un homófono perfecto. Si
vuelves a poner la `w' y quitas la `r' en su lugar, te quedas con la
palabra `wack' que es una palabra real, que simplemente no es un homófono de las
otras dos palabras.

Pero hay, de todas maneras, al menos una palabra de la cual Dan y nosotros sabemos,
que entregará dos homófonos, si quitas cualquiera de las primeras dos
letras para hacer dos nuevas palabras de cuatro letras. La pregunta es, ¿cuál
es la palabra?
\end{quote}
\index{homófono}
\index{palabra reducible}
\index{reducible, palabra}

Puedes utilizar el diccionario del Ejercicio~\ref{wordlist2} para verificar
si una cadena está en la lista de palabras.

Para verificar si dos palabras son homófonas, puedes utilizar el
Diccionario de Pronunciación de la CMU.  Lo puedes descargar en
\url{http://www.speech.cs.cmu.edu/cgi-bin/cmudict} o en
\url{http://thinkpython.com/code/c06d} y también puedes descargar
\url{http://thinkpython.com/code/pronounce.py}, que proporciona una función
con nombre \verb"read_dictionary" que lee el diccionario de pronunciación y
devuelve un diccionario de Python que mapea de cada palabra a una cadena que 
describe su pronunciación primaria.

Escribe un programa que haga una lista de todas las palabras que resuelven el Puzzler.
Solución: \url{http://thinkpython.com/code/homophone.py}.

\end{exercise}



\chapter{Tuplas}
\label{tuplechap}

Este capítulo presenta otro tipo incorporado más, la tupla, y luego
muestra la manera en que las listas, los diccionarios y las tuplas trabajan juntos.
Además, presento una característica útil para listas de argumentos de longitud variable:
los operadores de reunión y dispersión.

% One note: there is no consensus on how to pronounce ``tuple''.
% Some people say ``tuh-ple'', which rhymes with ``supple''.  But
% in the context of programming, most people say ``too-ple'', which
% rhymes with ``quadruple''.


\section{Las tuplas son inmutables}
\index{tupla}
\index{tipo!tupla}
\index{secuencia}

Una tupla es una secuencia de valores.  Los valores pueden ser de cualquier tipo y 
están indexados por enteros, por lo que en ese sentido las tuplas se parecen mucho
a las listas.  La diferencia importante es que las tuplas son inmutables.
\index{mutabilidad}
\index{inmutabilidad}

Sintácticamente, una tupla es una lista de valores separados por comas:

\begin{verbatim}
>>> t = 'a', 'b', 'c', 'd', 'e'
\end{verbatim}
%
Aunque no es necesario, es común encerrar las tuplas en
paréntesis:
\index{paréntesis!tuplas en}

\begin{verbatim}
>>> t = ('a', 'b', 'c', 'd', 'e')
\end{verbatim}
%
Para crear una tupla con un solo elemento, tienes que incluir una coma
final:
\index{singleton}
\index{tupla!singleton}

\begin{verbatim}
>>> t1 = 'a',
>>> type(t1)
    <class 'tuple'>
\end{verbatim}
%
Un valor en paréntesis no es una tupla:

\begin{verbatim}
>>> t2 = ('a')
>>> type(t2)
    <class 'str'>
\end{verbatim}
%
Otra manera de crear una tupla es la función incorporada {\tt tuple}.
Sin argumentos, esta crea una tupla vacía:
\index{función!tuple}
\index{tuple, función}

\begin{verbatim}
>>> t = tuple()
>>> t
    ()
\end{verbatim}
%
Si el argumento es una secuencia (cadena, lista o tupla), el resultado
es una tupla con los elementos de la secuencia:

\begin{verbatim}
>>> t = tuple('lupins')
>>> t
    ('l', 'u', 'p', 'i', 'n', 's')
\end{verbatim}
%
Dado que {\tt tuple} es el nombre de una función incorporada, deberías
evitar usarlo como nombre de una variable.

La mayoría de los operadores de lista funcionan también en tuplas.  El operador de corchetes
indexa un elemento:
\index{operador!de corchetes}
\index{corchetes!operador de}

\begin{verbatim}
>>> t = ('a', 'b', 'c', 'd', 'e')
>>> t[0]
    'a'
\end{verbatim}
%
Y el operador de trozo selecciona un rango de elementos.
\index{operador!de trozo}\index{slice}
\index{trozo!operador}
\index{tupla!trozo}
\index{trozo!tupla}

\begin{verbatim}
>>> t[1:3]
    ('b', 'c')
\end{verbatim}
%
Sin embargo, si intentas modificar uno de los elementos de la tupla, obtienes
un error:
\index{excepción!TypeError}
\index{TypeError}
\index{asignación de ítem}
\index{item@ítem!asignación de}

\begin{verbatim}
>>> t[0] = 'A'
    TypeError: object doesn't support item assignment
\end{verbatim}
%
Dado que las tuplas son inmutables, no puedes modificar sus elementos.  Pero
puedes reemplazar una tupla con otra:

\begin{verbatim}
>>> t = ('A',) + t[1:]
>>> t
    ('A', 'b', 'c', 'd', 'e')
\end{verbatim}
%
Esta sentencia crea una nueva tupla y luego hace que {\tt t} se refiera a esta.

Los operadores relacionales funcionan con las tuplas y otras secuencias;
Python comienza comparando el primer elemento de cada
secuencia.  Si son iguales, avanza a los siguientes elementos,
y así sucesivamente, hasta que encuentre elementos que sean diferentes.  Los elementos
posteriores no se consideran (incluso si son realmente grandes).
\index{tupla!comparación}
\index{comparación de tupla}

\begin{verbatim}
>>> (0, 1, 2) < (0, 3, 4)
    True
>>> (0, 1, 2000000) < (0, 3, 4)
    True
\end{verbatim}



\section{Asignación de tupla}
\label{tuple.assignment}
\index{tupla!asignación de}
\index{asignación!de tupla}
\index{patrón!de intercambio}
\index{intercambio, patrón de}

A menudo es útil intercambiar los valores de dos variables.
Con asignaciones convencionales, tienes que utilizar una variable
temporal.  Por ejemplo, para intercambiar {\tt a} y {\tt b}:

\begin{verbatim}
>>> temp = a
>>> a = b
>>> b = temp
\end{verbatim}
%
Esta solución es incómoda; la {\bf asignación de tupla} es más elegante:

\begin{verbatim}
>>> a, b = b, a
\end{verbatim}
%
El lado izquierdo es una tupla de variables; el lado derecho es una tupla de
expresiones.  Cada valor es asignado a su respectiva variable.
Todas las expresiones en el lado derecho son evaluadas antes que cualquiera
de las asignaciones.

La cantidad de variables en el lado izquierdo y la cantidad de valores del
lado derecho tienen que ser la misma:
\index{excepción!ValueError}
\index{ValueError}

\begin{verbatim}
>>> a, b = 1, 2, 3
    ValueError: too many values to unpack
\end{verbatim}
%
De manera más general, el lado derecho puede ser cualquier tipo de secuencia
(cadena, lista o tupla).  Por ejemplo, para separar una dirección de email
en nombre de usuario y dominio, podrías intentar:
\index{metodo@método!split}
\index{split, método}
\index{dirección de correo electrónico}

\begin{verbatim}
>>> direccion = 'monty@python.org'
>>> nombre_usuario, dominio = direccion.split('@')
\end{verbatim}
%
El valor de retorno de {\tt split} es una lista con dos elementos:
el primer elemento se asigna a {\tt nombre\_usuario}, el segundo a
{\tt dominio}.

\begin{verbatim}
>>> nombre_usuario
    'monty'
>>> dominio
    'python.org'
\end{verbatim}
%

\section{Tuplas como valores de retorno}
\index{tupla}
\index{valor!tupla}
\index{valor de retorno!tupla}
\index{función, tupla como valor de retorno}

Estrictamente hablando, una función puede devolver solo un valor de retorno, pero
si el valor es una tupla, el efecto es el mismo que devolver
múltiples valores.  Por ejemplo, si quieres dividir dos enteros
y calcular el cociente y el resto, es ineficiente
calcular {\tt x//y} y luego {\tt x\%y}.  Es mejor calcular
ambos al mismo tiempo.
\index{divmod}

La función incorporada {\tt divmod} toma dos argumentos y
devuelve una tupla de dos valores: el cociente y el resto.
Puedes almacenar el resultado como una tupla:

\begin{verbatim}
>>> t = divmod(7, 3)
>>> t
    (2, 1)
\end{verbatim}
%
O bien, utilizar asignación de tupla para almacenar los elementos por separado:
\index{asignación!de tupla}
\index{tupla!asignación de}

\begin{verbatim}
>>> cociente, resto = divmod(7, 3)
>>> cociente
    2
>>> resto
    1
\end{verbatim}
%
Aquí hay un ejemplo de una función que devuelve una tupla:

\begin{verbatim}
def min_max(t):
    return min(t), max(t)
\end{verbatim}
%
{\tt max} y {\tt min} son funciones incorporadas que encuentran
los elementos más grande y más pequeño en una secuencia.  \verb"min_max"
calcula ambos y devuelve una tupla de dos valores.
\index{max, función}
\index{función!max}
\index{min, función}
\index{función!min}


\section{Tuplas de argumentos de longitud variable}
\label{gather}
\index{tupla!de argumentos de longitud variable}
\index{argumento!tupla de longitud variable}
\index{reunion@reunión}
\index{parámetro!de reunión}
\index{argumento!reunión de}

Las funciones pueden tomar una cantidad variable de argumentos.  Un nombre de parámetro
que comienza con {\tt *} hace una {\bf reunión} de argumentos en
una tupla.  Por ejemplo, {\tt printall}
toma cualquier cantidad de argumentos y los imprime:

\begin{verbatim}
def printall(*args):
    print(args)
\end{verbatim}
%
El parámetro de reunión puede tener cualquier nombre que quieras, pero {\tt args} es
convencional.  Aquí se muestra cómo opera la función:

\begin{verbatim}
>>> printall(1, 2.0, '3')
    (1, 2.0, '3')
\end{verbatim}
%
El complemento de la reunión es la {\bf dispersión}.  Si tienes una
secuencia de valores y quieres pasarlo a una función
como multiples argumentos, puedes utilizar el operador {\tt *}.
Por ejemplo, {\tt divmod} toma exactamente dos argumentos;
no funciona con una tupla:
\index{dispersión}
\index{argumento!dispersión de}
\index{TypeError}
\index{excepción!TypeError}

\begin{verbatim}
>>> t = (7, 3)
>>> divmod(t)
    TypeError: divmod expected 2 arguments, got 1
\end{verbatim}
%
Sin embargo, si dispersas la tupla, funciona:

\begin{verbatim}
>>> divmod(*t)
    (2, 1)
\end{verbatim}
%
Muchas de las funciones incorporadas utilizan
tuplas de argumentos de longitud variable.  Por ejemplo, {\tt max}
y {\tt min} pueden tomar cualquier cantidad de argumentos:
\index{max, función}
\index{función!max}
\index{min, función}
\index{función!min}

\begin{verbatim}
>>> max(1, 2, 3)
    3
\end{verbatim}
%
Pero {\tt sum} no puede.
\index{sum, función}
\index{función!sum}

\begin{verbatim}
>>> sum(1, 2, 3)
    TypeError: sum expected at most 2 arguments, got 3
\end{verbatim}
%
Como ejercicio, escribe una función llamada \verb"sumar_todos" que tome cualquier cantidad
de argumentos y devuelva su suma.


\section{Listas y tuplas}
\index{zip!función}
\index{función!zip}

{\tt zip} es una función incorporada que toma dos o más secuencias y
las intercala.  El nombre de la función se refiere a
una cremallera ({\em zipper}), ya que esta intercala dos filas de dientes.

Este ejemplo le hace zip a una cadena y una lista:

\begin{verbatim}
>>> s = 'abc'
>>> t = [0, 1, 2]
>>> zip(s, t)
    <zip object at 0x7f7d0a9e7c48>
\end{verbatim}
%
El resultado es un {\bf objeto zip} que sabe cómo iterar a través
de los pares.  El uso más común de {\tt zip} es en un bucle {\tt for}:

\begin{verbatim}
>>> for par in zip(s, t):
...     print(par)
...
    ('a', 0)
    ('b', 1)
    ('c', 2)
\end{verbatim}
%
Un objeto zip es una especie de {\bf iterador}, que es cualquier objeto
que itera a través de una secuencia.  Los iteradores son similares a las listas de alguna
manera, pero a diferencia de las listas, no puedes utilizar un índice para seleccionar un elemento de
un iterador.
\index{iterador}

Si quieres utilizar operadores de lista y métodos, puedes
utilizar un objeto zip para crear una lista:

\begin{verbatim}
>>> list(zip(s, t))
    [('a', 0), ('b', 1), ('c', 2)]
\end{verbatim}
%
El resultado es una lista de tuplas; en este ejemplo, cada tupla contiene
un carácter de la cadena y el elemento correspondiente de
la lista.
\index{lista!de tuplas}

Si las secuencias no tienen la misma longitud, el resultado tiene la
longitud de la secuencia más corta.

\begin{verbatim}
>>> list(zip('Anne', 'Elk'))
    [('A', 'E'), ('n', 'l'), ('n', 'k')]
\end{verbatim}
%
Puedes utilizar asignación de tupla en un bucle {\tt for} para recorrer una lista de
tuplas:
\index{recorrer}
\index{asignación!de tupla}
\index{tupla!asignación de}

\begin{verbatim}
t = [('a', 0), ('b', 1), ('c', 2)]
for letra, numero in t:
    print(numero, letra)
\end{verbatim}
%
En cada paso por el bucle, Python selecciona la siguiente tupla en
la lista y asigna los elementos a {\tt letra} y
{\tt numero}.  La salida de este bucle es:
\index{bucle}

\begin{verbatim}
0 a
1 b
2 c
\end{verbatim}
%
Si combinas {\tt zip}, {\tt for} y asignación de tupla, obtienes una
manera útil para recorrer dos (o más) secuencias al mismo
tiempo.  Por ejemplo, \verb"tiene_coincidencia" toma dos secuencias, {\tt t1} y
{\tt t2}, y devuelve {\tt True} si hay un índice {\tt i}
tal que {\tt t1[i] == t2[i]}:
\index{bucle!for}

\begin{verbatim}
def tiene_coincidencia(t1, t2):
    for x, y in zip(t1, t2):
        if x == y:
            return True
    return False
\end{verbatim}
%
Si necesitas recorrer los elementos de una secuencia y sus
índices, puedes utilizar la función incorporada {\tt enumerate}:
\index{recorrer}
\index{función!enumerate}
\index{enumerate!función}

\begin{verbatim}
for indice, elemento in enumerate('abc'):
    print(indice, elemento)
\end{verbatim}
%
El resultado de {\tt enumerate} es un objeto enumerate, que
itera una secuencia de pares; cada par contiene un índice (comenzando
desde 0) y un elemento de la secuencia dada.
En este ejemplo, la salida es

\begin{verbatim}
0 a
1 b
2 c
\end{verbatim}
%
otra vez.
\index{iterador}
\index{objeto!enumerate}
\index{enumerate!objeto}


\section{Diccionarios y tuplas}
\label{dictuple}
\index{diccionario}
\index{metodo@método!items}
\index{items, método}
\index{par clave-valor}

Los diccionarios tienen un método llamado {\tt items} que devuelve una secuencia de
tuplas, donde cada tupla es un par clave-valor.

\begin{verbatim}
>>> d = {'a':0, 'b':1, 'c':2}
>>> t = d.items()
>>> t
    dict_items([('c', 2), ('a', 0), ('b', 1)])
\end{verbatim}
%
El resultado es un objeto \verb"dict_items", que es un iterador que
itera los pares clave-valor.  Puedes utilizarlo en un bucle {\tt for}
así:
\index{iterador}

\begin{verbatim}
>>> for clave, valor in d.items():
...     print(clave, valor)
...
    c 2
    a 0
    b 1
\end{verbatim}
%
Tal como esperarías de un diccionario, los ítems no están en un
orden particular.

En sentido contrario, puedes utilizar una lista de tuplas para
inicializar un nuevo diccionario: \index{diccionario!inicializar}

\begin{verbatim}
>>> t = [('a', 0), ('c', 2), ('b', 1)]
>>> d = dict(t)
>>> d
    {'a': 0, 'c': 2, 'b': 1}
\end{verbatim}

Combinando {\tt dict} con {\tt zip} se produce una manera concisa
de crear un diccionario:
\index{zip!usar con dict}

\begin{verbatim}
>>> d = dict(zip('abc', range(3)))
>>> d
    {'a': 0, 'c': 2, 'b': 1}
\end{verbatim}
%
El método de diccionario {\tt update} además toma una lista de tuplas
y las agrega, como pares clave-valor, a un diccionario existente.
\index{metodo@método!update}
\index{update, método}
\index{recorrer!diccionario}
\index{diccionario!recorrer}

Es común utilizar tuplas como claves en diccionarios (principalmente porque
no puedes utilizar listas).  Por ejemplo, un directorio telefónico podría mapear
de pares apellido-nombre a números telefónicos.  Suponiendo
que hemos definido {\tt apellido}, {\tt nombre} y {\tt numero},
podríamos escribir:
\index{tupla!como clave en diccionario}
\index{hashable}

\begin{verbatim}
directorio[apellido, nombre] = numero
\end{verbatim}
%
La expresión en corchetes es una tupla.  Podríamos utilizar asignación
de tupla para recorrer este diccionario.
\index{tupla!en corchetes}

\begin{verbatim}
for apellido, nombre in directorio:
    print(nombre, apellido, directorio[apellido,nombre])
\end{verbatim}
%
Este bucle recorre las claves en {\tt directorio}, que son tuplas.  Asigna
los elementos de cada tupla a {\tt apellido} y {\tt nombre}, luego
imprime el nombre completo y el número telefónico correspondiente.

Hay dos maneras de representar tuplas en un diagrama de estado.  La versión
más detallada muestra los índices y elementos tal como aparecen en
una lista.  Por ejemplo, la tupla \verb"('Cleese', 'John')" se mostraría
como en la Figura~\ref{fig.tuple1}.
\index{diagrama!de estado}
\index{estado, diagrama de}

\begin{figure}
\centerline
{\includegraphics[scale=0.8]{figs/tuple1.pdf}}
\caption{Diagrama de estado.}
\label{fig.tuple1}
\end{figure}

Pero en un diagrama más grande quizás quieras omitir los
detalles.  Por ejemplo, un diagrama del directorio telefónico podría
mostrarse como en la Figura~\ref{fig.dict2}.

\begin{figure}
\centerline
{\includegraphics[scale=0.8]{figs/dict2.pdf}}
\caption{Diagrama de estado.}
\label{fig.dict2}
\end{figure}

Aquí, las tuplas se muestran utilizando la notación de la sintáxis de Python.
El número de teléfono del diagrama es la línea de reclamos
de la BBC, así que por favor no llames.


\section{Secuencias de secuencias}
\index{secuencia}

Me he enfocado en listas de tuplas, pero casi todos los ejemplos de
este capítulo funcionan también con listas de listas, tuplas de tuplas y
tuplas de listas.  Para evitar enumerar las posibles combinaciones, a veces
es más fácil hablar de secuencias de secuencias.

En muchos contextos, los diferentes tipos de secuencias (cadenas, listas y
tuplas) se pueden utilizar indistintamente.  Entonces, ¿cómo deberías escoger uno
por sobre los otros?
\index{cadena}
\index{lista}
\index{tupla}
\index{mutabilidad}
\index{inmutabilidad}

Para comenzar con lo obvio, las cadenas son más limitadas que las otras
secuencias porque los elementos tienen que ser caracteres.  Además,
son inmutables.  Si necesitas la posibilidad de cambiar los caracteres
en una cadena (en contraposición a crear una nueva cadena), quizás quieras
utilizar una lista de caracteres en su lugar.

Las listas son más comunes que las tuplas, principalmente porque son mutables.
Sin embargo, hay algunos casos donde podrías preferir tuplas:

\begin{enumerate}

\item En algunos contextos, como una sentencia {\tt return}, es
sintácticamente más simple crear una tupla que una lista.

\item Si quieres utilizar una secuencia como clave de diccionario,
tienes que utilizar un tipo inmutable como una tupla o una cadena.

\item Si pasas una secuencia como argumento a una función,
utilizar tuplas reduce las posibilidades de comportamiento inesperado
debido a los alias.

\end{enumerate}

Dado que las tuplas son inmutables, no proporcionan métodos como {\tt
  sort} y {\tt reverse}, que modifican listas existentes.  Sin embargo, Python
proporciona la función incorporada {\tt sorted}, que toma cualquier secuencia
y devuelve una lista nueva con los mismos elementos en orden, y
{\tt reversed}, que toma una secuencia y devuelve un iterador que
recorre la lista en orden invertido.
\index{función!sorted}
\index{sorted, función} \index{función!reversed}
\index{reversed, función}
\index{iterador}


\section{Depuración}
\index{depuración}
\index{estructura de datos}
\index{error!de forma}
\index{forma!error de}

Las listas, diccionarios y tuplas son ejemplos de {\bf estructuras
  de datos}; en este capítulo comenzamos a ver estructuras de datos
combinadas, como listas de tuplas, o diccionarios que contienen tuplas
como claves y listas como valores.  Las estructuras de datos combinadas son útiles,
pero son propensas a lo que yo llamo {\bf errores de forma}, es decir, errores
causados cuando una estructura de datos tiene el tipo, tamaño o estructura
incorrecta. Por ejemplo, si estás esperando una lista que contiene un entero y yo
te doy un simple y viejo entero (no en una lista), no funcionará.
\index{modulo@módulo!structshape}
\index{structshape, módulo}

Para ayudar a depurar esta clase de errores, he escrito un módulo
llamado {\tt structshape} que proporciona una función, también llamada
{\tt structshape}, que toma cualquier tipo de estructura de datos como
argumento y devuelve una cadena que resume su forma.
Puedes descargarlo en \url{http://thinkpython.com/code/structshape.py}

Aquí hay un resultado para una lista simple:

\begin{verbatim}
>>> from structshape import structshape
>>> t = [1, 2, 3]
>>> structshape(t)
    'list of 3 int'
\end{verbatim}
%
Un programa más elegante podría escribir ``list of 3 int{\em s}'', pero
fue más fácil no lidiar con los plurales.  Aquí hay una lista de listas:

\begin{verbatim}
>>> t2 = [[1,2], [3,4], [5,6]]
>>> structshape(t2)
    'list of 3 list of 2 int'
\end{verbatim}
%
Si los elementos de la lista no son del mismo tipo,
{\tt structshape} los agrupa, en orden, por tipo:

\begin{verbatim}
>>> t3 = [1, 2, 3, 4.0, '5', '6', [7], [8], 9]
>>> structshape(t3)
    'list of (3 int, float, 2 str, 2 list of int, int)'
\end{verbatim}
%
Aquí hay una lista de tuplas:

\begin{verbatim}
>>> s = 'abc'
>>> lt = list(zip(t, s))
>>> structshape(lt)
    'list of 3 tuple of (int, str)'
\end{verbatim}
%
Y aquí hay un diccionario con 3 ítems que mapean enteros a cadenas.

\begin{verbatim}
>>> d = dict(lt)
>>> structshape(d)
    'dict of 3 int->str'
\end{verbatim}
%
Si tienes problemas al hacer seguimiento de tus estructuras de datos,
{\tt structshape} puede ayudar.


\section{Glosario}

\begin{description}

\item[tupla:] Una secuencia inmutable de elementos.
\index{tupla}

\item[asignación de tupla:] Una asignación con una secuencia en el
lado derecho y una tupla de variables en el lado izquierdo.  El lado
derecho es evaluado y luego sus elementos son asignados a las
variables en el lado izquierdo.
\index{asignación!de tupla}
\index{tupla!asignación de}

\item[reunión:] Una operación que junta múltiples argumentos en una tupla.
\index{reunion@reunión}

\item[dispersión:] Una operación que hace que una secuencia se comporte como múltiples argumentos.
\index{dispersión}

\item[objeto zip:] El resultado de llamar a la función incorporada {\tt zip};
un objeto que itera a través de una secuencia de tuplas.
\index{objeto!zip}
\index{zip!objeto}

\item[iterador:] Un objeto que itera a través de una secuencia, pero
que no proporciona operadores ni métodos de lista.
\index{iterador}

\item[estructura de datos:] Una colección de valores relacionados, a menudo
organizada en listas, diccionarios, tuplas, etc.
\index{estructura de datos}

\item[error de forma:] Un error causado debido a que un valor tiene la forma
incorrecta, es decir, el tipo o tamaño incorrecto.
\index{forma}

\end{description}


\section{Ejercicios}

\begin{exercise}

Escribe una función llamada \verb"mas_frecuente" que tome una cadena e
imprima las letras en orden de frecuencia descendente.  Encuentra ejemplos
de texto en varios idiomas diferentes y ve cómo varía la frecuencia de letras
entre los idiomas.  Compara tus resultados con las tablas en
\url{http://en.wikipedia.org/wiki/Letter_frequencies}.  Solución:
\url{http://thinkpython.com/code/most_frequent.py}.  \index{frecuencia de
  letras} \index{letras, frecuencia}

\end{exercise}


\begin{exercise}
\label{anagrams}
\index{conjunto de anagramas}
\index{anagramas, conjunto de}

¡Más anagramas!

\begin{enumerate}

\item Escribe un programa
que lea una lista de palabras desde un archivo (ver Sección~\ref{wordlist}) e
imprima todos los conjuntos de palabras que son anagramas.

Aquí hay un ejemplo de cómo se vería la salida:

\begin{verbatim}
['deltas', 'desalt', 'lasted', 'salted', 'slated', 'staled']
['retainers', 'ternaries']
['generating', 'greatening']
['resmelts', 'smelters', 'termless']
\end{verbatim}
%
Pista: quizás quieras construir un diccionario que mapee de una
colección de letras a una lista de palabras que se puedan escribir con esas
letras.  La pregunta es, ¿cómo puedes representar la colección de
letras de manera que se pueda utilizar como clave?

\item Modifica el programa anterior para que imprima la lista de anagramas
más grande primero, seguido de la segunda más grande, y así sucesivamente.
\index{Scrabble}
\index{bingo}

\item En Scrabble, un ``bingo'' es cuando juegas todas las siete fichas de
tu atril, junto con una letra en el tablero, para formar una palabra de ocho
letras.  ¿Qué colección de 8 letras forma la mayor cantidad de bingos posible?

% (7, ['angriest', 'astringe', 'ganister', 'gantries', 'granites',
% 'ingrates', 'rangiest'])

Solución: \url{http://thinkpython.com/code/anagram_sets.py}.

\end{enumerate}
\end{exercise}

\begin{exercise}
\index{metatesis@metátesis}

Dos palabras forman un ``par de metátesis'' si puedes transformar una en la
otra intercambiando dos letras; por ejemplo, ``converse'' y
``conserve''.  Escribe un programa que encuentre todos los pares de metátesis
en el diccionario.  Pista: no pruebes todos los pares de palabras ni
pruebes todos los posibles intercambios.  Solución:
\url{http://thinkpython.com/code/metathesis.py}.  Crédito: Este
ejercicio está inspirado por un ejemplo de \url{http://puzzlers.org}.

\end{exercise}


\begin{exercise}
\index{Car Talk}
\index{Puzzler}

Aquí hay otro Puzzler de Car Talk
(\url{http://www.cartalk.com/content/puzzlers}):

\begin{quote}
¿Cuál es la palabra en inglés más larga que, a medida que eliminas sus letras
una a la vez, sigue siendo una palabra en inglés válida?

A ver, las letras se pueden eliminar desde cualquier extremo, o del medio, pero
no puedes reorganizar ninguna de las letras. Cada vez que retiras una letra,
terminas con otra palabra en inglés. Si haces eso, eventualmente
vas a terminar con una letra y esa también va a ser una
palabra en inglés, una que se encuentra en el diccionario. Quiero saber:
¿cuál es la palabra más larga y cuantas letras
tiene?

Voy a darte un pequeño ejemplo modesto: Sprite. ¿De acuerdo? Inicias
con sprite, quitas una letra, una del interior de la
palabra, te llevas la r, y nos quedamos con la palabra spite, luego
quitamos la e del final, nos quedamos con spit, quitamos la s, nos
quedamos con pit, it, y por último I.
\end{quote}
\index{palabra reducible}
\index{reducible, palabra}

Escribe un programa que encuentre todas las palabras que se pueden reducir de
esta manera, y luego encuentra la más larga.

Este ejercicio es uno de los más desafiantes, así que aquí hay
algunas sugerencias:

\begin{enumerate}

\item Quizás quieras escribir una función que tome una palabra y
  obtenga una lista de todas las palabras que se pueden formar eliminando una
  letra.  Estas son las ``hijas'' de la palabra.
\index{definición recursiva}
\index{recursiva, definición}

\item De manera recursiva, una palabra es reducible si cualquiera de sus hijas
es reducible.  Como caso base, puedes considerar la cadena vacía
reducible.

\item La lista de palabras que proporciono, {\tt words.txt}, no
contiene palabras de una sola letra.  Entonces quizás quieras agregar
``I'', ``a'' y la cadena vacía.

\item Para mejorar el desempeño de tu programa, quizás quieras
memoizar las palabras que se sabe que son reducibles.

\end{enumerate}

Solución: \url{http://thinkpython.com/code/reducible.py}.

\end{exercise}




%\begin{exercise}
%\url{http://en.wikipedia.org/wiki/Word_Ladder}
%\end{exercise}




\chapter{Estudio de caso: selección de estructura de datos}

En este punto has aprendido sobre las estructuras de datos esenciales de Python,
y has visto algunos de los algoritmos que los utilizan.
Si te gustaría saber más sobre algoritmos, este podría ser un buen
momento para leer el Apéndice~\ref{algorithms}.
Sin embargo, no tienes que leerlo antes de continuar; puedes leerlo
cuando te interese.

Este capítulo presenta un estudio de caso con ejercicios que te hacen
pensar sobre la elección de estructuras de datos y practicar su uso.


\section{Análisis de frecuencia de palabras}
\label{analysis}

Como siempre, deberías al menos intentar los ejercicios
antes de que leas mis soluciones.

\begin{exercise}

Escribe un programa que lea un archivo, separe cada línea en
palabras, quite los espacios en blanco y la puntuación de las palabras,
y las convierta en minúsculas.
\index{modulo@módulo!string}
\index{string!módulo}

Pista: El módulo {\tt string} proporciona una cadena con nombre {\tt whitespace},
que contiene los caracteres de espacio, tabulación, nueva línea, etc.,
y {\tt punctuation} que contiene los caracteres de puntuación.  Veamos
si podemos hacer que Python diga groserías:

\begin{verbatim}
>>> import string
>>> string.punctuation
    '!"#$%&\'()*+,-./:;<=>?@[\\]^_`{|}~'
\end{verbatim}
%
Además, podrías considerar el uso de los métodos de cadena {\tt strip},
{\tt replace} y {\tt translate}.
\index{metodo@método!strip}
\index{strip, método}
\index{metodo@método!replace}
\index{replace, método}
\index{metodo@método!translate}
\index{translate, método}

\end{exercise}


\begin{exercise}
\index{Project Gutenberg}

Ve a Project Gutenberg (\url{http://gutenberg.org}) y descarga
tu libro sin derechos de autor favorito en formato de texto plano.
\index{texto plano}
\index{plano!texto}

Modifica tu programa del ejercicio anterior para que lea el libro
que descargaste, se salte la información del encabezado al principio
del archivo y procese el resto de las palabras como antes.

Luego, modifica el programa para que cuente la cantidad total de palabras en
el libro y el número de veces que se utiliza cada palabra.
\index{frecuencia de palabras}
\index{palabras, frecuencia de}

Imprime el número de palabras diferentes utilizadas en el libro.  Compara
libros diferentes de autores diferentes, escritos en épocas diferentes.
¿Qué autor utiliza el vocabulario más extenso?
\end{exercise}


\begin{exercise}

Modifica el programa del ejercicio anterior para que imprima las
20 palabras utilizadas con mayor frecuencia en el libro.

\end{exercise}


\begin{exercise}

Modifica el programa anterior para que lea una lista de palabras (ver
Sección~\ref{wordlist}) y luego imprima todas las palabras del libro que
no están en la lista de palabras.  ¿Cuántas de ellas son errores tipográficos?
¿Cuántas de ellas son palabras comunes que {\em deberían} estar en la lista
de palabras y cuántas de ellas son realmente oscuras?

\end{exercise}


\section{Números aleatorios}
\index{numero aleatorio@número aleatorio}
\index{aleatorio!número}
\index{determinista}
\index{pseudoaleatorio}

Dadas las mismas entradas, la mayoría de los programas de computador generan
las mismas salidas cada vez, por lo que se dice que son {\bf deterministas}.
El determinismo es usualmente algo bueno, ya que esperamos que los mismos
cálculos produzcan el mismo resultado.  Para algunas aplicaciones, sin embargo,
queremos que el computador sea impredecible.  Los juegos son un ejemplo
obvio, pero hay más.

Hacer un programa verdaderamente no determinista resulta difícil,
pero hay maneras de hacer que al menos parezca no determinista.  Una de
ellas es utilizar algoritmos que generen números {\bf pseudoaleatorios}.
Los números pseudoaleatorios no son verdaderamente aleatorios porque se generan
por una computación determinista, pero solo mirando a dichos números
es casi imposible distinguirlos de los aleatorios.
\index{modulo@módulo!random}
\index{random, módulo}

El módulo {\tt random} proporciona funciones que generan
números pseudoaleatorios (que simplemente llamaré ``aleatorios'' desde
aquí en adelante).
\index{función!random}
\index{random, función}

La función {\tt random} devuelve un número de coma flotante aleatorio
entre 0.0 y 1.0 (incluyendo 0.0 pero no 1.0).  Cada vez que
llamas a {\tt random}, obtienes el siguiente número de una larga serie.
Para ver una muestra, ejecuta este bucle:

\begin{verbatim}
import random

for i in range(10):
    x = random.random()
    print(x)
\end{verbatim}
%
La función {\tt randint} toma los parámetros {\tt low} y
{\tt high} y devuelve un entero entre {\tt low} y
{\tt high} (incluyendo a ambos).
\index{función!randint}
\index{randint, función}

\begin{verbatim}
>>> random.randint(5, 10)
    5
>>> random.randint(5, 10)
    9
\end{verbatim}
%
Para escoger un elemento de una secuencia de manera aleatoria, puedes usar
{\tt choice}:
\index{función!choice}
\index{choice, función}

\begin{verbatim}
>>> t = [1, 2, 3]
>>> random.choice(t)
    2
>>> random.choice(t)
    3
\end{verbatim}
%
El módulo {\tt random} también proporciona funciones para generar
valores aleatorios a partir de distribuciones continuas, incluyendo
la gaussiana, exponencial, gamma y algunas más.

\begin{exercise}
\index{histograma!elección aleatoria}

Escribe una función con nombre \verb"escoger_de_hist" que tome
un histograma como se definió en la Sección~\ref{histogram} y devuelva un
valor aleatorio del histograma, esogido con probabilidad
proporcional a la frecuencia.  Por ejemplo, para este histograma:

\begin{verbatim}
>>> t = ['a', 'a', 'b']
>>> hist = histograma(t)
>>> hist
    {'a': 2, 'b': 1}
\end{verbatim}
%
Tu función debería devolver \verb"'a'" con probabilidad $2/3$ y \verb"'b'"
con probabilidad $1/3$.
\end{exercise}


\section{Histograma de palabras}

Deberías intentar los ejemplos anteriores antes de continuar.
Puedes descargar mi solución en
 \url{http://thinkpython.com/code/analyze_book1.py}.  Además,
necesitarás \url{http://thinkpython.com/code/emma.txt}.

Aquí hay un programa que lee un archivo y construye un histograma de las
palabras en el archivo:
\index{histograma!frecuencia de palabras}

\begin{verbatim}
import string

def procesar_archivo(nombre_archivo):
    hist = dict()
    fp = open(nombre_archivo)
    for linea in fp:
        procesar_linea(linea, hist)
    return hist

def procesar_linea(linea, hist):
    linea = linea.replace('-', ' ')

    for palabra in linea.split():
        palabra = palabra.strip(string.punctuation + string.whitespace)
        palabra = palabra.lower()
        hist[palabra] = hist.get(palabra, 0) + 1

hist = procesar_archivo('emma.txt')
\end{verbatim}
%
Este programa lee a {\tt emma.txt}, que contiene el texto de {\em
  Emma} de Jane Austen.
\index{Austen, Jane}

\verb"procesar_archivo" recorre las líneas del archivo,
pasándolas una a la vez a \verb"procesar_linea".  El histograma
{\tt hist} se utiliza como acumulador.
\index{acumulador!histograma}
\index{recorrer}

\verb"procesar_linea" utiliza el método de cadena {\tt replace} para reemplazar
los guiones con espacios antes de utilizar {\tt split} para separar la línea en una
lista de cadenas.  Recorre la lista de palabras y utiliza {\tt strip}
y {\tt lower} para eliminar la puntuación y convertir a minúsculas.  (Es
una abreviatura decir que las cadenas se ``convierten''; recuerda que
las cadenas son inmutables, por lo que los métodos como {\tt strip} y {\tt lower}
devuelven cadenas nuevas.)

Finalmente, \verb"procesar_linea" actualiza el histograma creando un nuevo
ítem o incrementando uno existente.
\index{actualizar!histograma}

Para contar el número total de palabras en el archivo, podemos sumar
las frecuencias del histograma:

\begin{verbatim}
def total_palabras(hist):
    return sum(hist.values())
\end{verbatim}
%
El número de palabras diferentes es solo el número de ítems en
el diccionario:

\begin{verbatim}
def palabras_diferentes(hist):
    return len(hist)
\end{verbatim}
%
Aquí hay algo de código para imprimir los resultados:

\begin{verbatim}
print('Número total de palabras:', total_palabras(hist))
print('Número de palabras diferentes:', palabras_diferentes(hist))
\end{verbatim}
%
Y los resultados:

\begin{verbatim}
Número total de palabras: 161080
Número de palabras diferentes: 7214
\end{verbatim}
%

\section{Palabras más comunes}

Para encontrar las palabras más comunes, podemos crear una lista de tuplas,
donde cada tupla contenga una palabra y su frecuencia,
y ordenarla.

La siguiente función toma un histograma y devuelve una lista de
tuplas frecuencia-palabra:

\begin{verbatim}
def mas_comunes(hist):
    t = []
    for clave, valor in hist.items():
        t.append((valor, clave))

    t.sort(reverse=True)
    return t
\end{verbatim}

En cada tupla, la frecuencia aparece primero, por lo que la lista resultante está
ordenada por frecuencia.  Aquí hay un bucle que imprime las diez palabras más
comunes:

\begin{verbatim}
t = mas_comunes(hist)
print('Las palabras más comunes son:')
for frec, palabra in t[:10]:
    print(palabra, frec, sep='\t')
\end{verbatim}
%
Utilizo el argumento de palabra clave {\tt sep} para decirle a {\tt print} que utilice un
carácter de tabulación como ``separador'', en lugar de un espacio, así la segunda
columna está alineada.  Estos son los resultados de {\em Emma}:

\begin{verbatim}
Las palabras más comunes son:
to      5242
the     5205
and     4897
of      4295
i       3191
a       3130
it      2529
her     2483
was     2400
she     2364
\end{verbatim}
%
Este código se puede simplificar utilizando el parámetro {\tt key} de
la función {\tt sort}.  Si sientes curiosidad, puedes leer sobre este
en \url{https://wiki.python.org/moin/HowTo/Sorting}.


\section{Parámetros opcionales}
\index{parámetro!opcional}
\index{opcional!parámetro}

Hemos visto funciones y métodos que toman argumentos
opcionales.  Es posible, además, escribir funciones definidas por el programador
con argumentos opcionales.  Por ejemplo, aquí hay una función que
imprime las palabras más comunes de un histograma:
\index{programador, función definida por el}
\index{función!definida por el programador}

\begin{verbatim}
def imprime_mas_comunes(hist, num=10):
    t = mas_comunes(hist)
    print('Las palabras más comunes son:')
    for frec, palabra in t[:num]:
        print(palabra, frec, sep='\t')
\end{verbatim}

El primer parámetro es obligatorio, el segundo es opcional.
El {\bf valor por defecto} de {\tt num} es 10.
\index{valor por defecto}
\index{por defecto, valor}

Si solo entregas un argumento:

\begin{verbatim}
imprimir_mas_comunes(hist)
\end{verbatim}

{\tt num} obtiene el valor por defecto.  Si entregas dos argumentos:

\begin{verbatim}
imprimir_mas_comunes(hist, 20)
\end{verbatim}

{\tt num} obtiene el valor del argumento en su lugar.  En otras
palabras, el argumento opcional {\bf anula} al valor por defecto.
\index{anular}

Si una función tiene parámetros tanto obligatorios como opcionales, todos
los parámetros obligatorios tienen que ir primero, seguido por los
opcionales.


\section{Diferencia de diccionarios}
\label{dictsub}
\index{diccionario!diferencia de}
\index{diferencia de diccionarios}

Encontrar las palabras del libro que no están en la lista de palabras
de {\tt words.txt} es un problema que podrías reconocer como diferencia
de conjuntos, es decir, queremos encontrar todas las palabras de un
conjunto (las palabras en el libro) que no están en el otro (las
palabras en la lista).

{\tt diferencia} toma dos diccionarios, {\tt d1} y {\tt d2}, y devuelve un
nuevo diccionario que contiene todas las claves de {\tt d1} que no están
en {\tt d2}.  Dado que en realidad no nos importan los valores,
los ponemos todos como None.

\begin{verbatim}
def diferencia(d1, d2):
    res = dict()
    for clave in d1:
        if clave not in d2:
            res[clave] = None
    return res
\end{verbatim}
%
Para encontrar las palabras en el libro que no están en {\tt words.txt},
podemos utlizar \verb"procesar_archivo" para construir un histograma para
{\tt words.txt}, y luego obtener la diferencia:

\begin{verbatim}
palabras = procesar_archivo('words.txt')
dif = diferencia(hist, palabras)

print("Palabras en el libro que no están en la lista de palabras:")
for palabra in dif:
    print(palabra, end=' ')
\end{verbatim}
%
Aquí hay algunos de los resultados para {\em Emma}:

\begin{verbatim}
Palabras en el libro que no están en la lista de palabras:
rencontre jane's blanche woodhouses disingenuousness
friend's venice apartment ...
\end{verbatim}
%
Algunas de esas palabras son nombres y posesivos.  Otras, como
``rencontre'', ya no son de uso común.  Sin embargo, ¡algunas son
palabras comunes que realmente deberían estar en la lista!

\begin{exercise}
\index{conjunto}\index{set}
\index{tipo!set}

Python proporciona una estructura de datos llamada {\tt set} que provee muchas
operaciones de conjunto comunes.  Puedes leer sobre estas en la Sección~\ref{sets},
o leer la documentación en
\url{http://docs.python.org/3/library/stdtypes.html#types-set}.

Escribe un programa que utilice la diferencia de conjuntos para encontrar palabras en el libro
que no están en la lista.  Solución:
\url{http://thinkpython.com/code/analyze_book2.py}.

\end{exercise}


\section{Palabras aleatorias}
\label{randomwords}
\index{histograma!elección aleatoria}

Para escoger una palabra de forma aleatoria del histograma, el algoritmo más simple
es construir una lista con múltiples copias de cada palabra, según
la frecuencia observada, y luego escoger de la lista:

\begin{verbatim}
def palabra_aleatoria(h):
    t = []
    for palabra, frec in h.items():
        t.extend([palabra] * frec)

    return random.choice(t)
\end{verbatim}
%
La expresión {\tt [palabra] * frec} crea una lista con {\tt frec}
copias de la cadena {\tt palabra}.  El método {\tt extend}
es similar a {\tt append}, excepto que el argumento es
una secuencia.

Este algoritmo funciona, pero no es muy eficiente: cada vez que
escoges una palabra aleatoria, reconstruye la lista, que es tan grande como
el libro original.  Una mejora obvia es construir la lista
una vez y luego hacer múltiples selecciones, pero la lista es grande aún.

Una alternativa es:

\begin{enumerate}

\item Utilizar {\tt keys} para obtener una lista de las palabras del libro.

\item Construir una lista que contenga la suma acumulativa de las frecuencias
  de las palabras (ver Ejercicio~\ref{cumulative}).  El último ítem
  en esta lista es el número total de palabras en el libro: $n$.

\item Escoger un número de 1 a $n$.  Utilizar búsqueda de bisección
  (ver Ejercicio~\ref{bisection}) para encontrar el índice donde el número
  aleatorio sería insertado en la suma acumulativa.

\item Utilizar el índice para encontrar la palabra correspondiente en la lista de palabras.

\end{enumerate}

\begin{exercise}
\label{randhist}
\index{algoritmo}

Escribe un programa que utilice este algoritmo para escoger una palabra aleatoria del
libro.  Solución:
\url{http://thinkpython.com/code/analyze_book3.py}.

\end{exercise}



\section{Análisis de Markov}
\label{markov}
\index{analisis de markov@Análisis de Markov}

Si escoges palabras del libro de manera aleatoria, puedes obtener una
idea del vocabulario, pero probablemente no obtendrás una oración:

\begin{verbatim}
this the small regard harriet which knightley's it most things
\end{verbatim}
%
Una serie de palabras aleatorias rara vez tiene sentido porque
no hay relación entre palabras sucesivas.  Por ejemplo, en
una oración real esperarías que un artículo como ``the'' esté
seguida por un adjetivo o un sustantivo, y probablemente no por un verbo
o un adverbio.

Una manera de medir estas relaciones es el análisis de
Markov, que
caracteriza, para una secuencia de palabras dada, la probabilidad de las
palabras que podrían venir después.  Por ejemplo, la canción {\em Eric, the Half a
  Bee} comienza:

\begin{quote}
Half a bee, philosophically, \\
Must, ipso facto, half not be. \\
But half the bee has got to be \\
Vis a vis, its entity. D'you see? \\
\\
But can a bee be said to be \\
Or not to be an entire bee \\
When half the bee is not a bee \\
Due to some ancient injury? \\
\end{quote}
%
En este texto,
la frase ``half the'' está siempre seguida por la palabra ``bee'',
pero la frase ``the bee'' podría estar seguida por
``has'' o ``is''.
\index{prefijo}
\index{sufijo}
\index{mapeo}

El resultado del análisis de Markov es un mapeo de cada prefijo
(como ``half the'' y ``the bee'') a todos los posibles sufijos
(como ``has'' e ``is'').
\index{texto aleatorio}
\index{aleatorio!texto}

Dado este mapeo, puedes generar un texto aleatorio
comenzando con cualquier prefijo y escogiendo de manera aleatoria
en los posibles sufijos.  Después, puedes combinar el final del
prefijo y el nuevo sufijo para formar el nuevo prefijo, y repetir.

Por ejemplo, si comienzas con el prefijo ``Half a'', entonces la
siguiente palabra tiene que ser ``bee'', porque el prefijo solo aparece
una vez en el texto.  El siguiente prefijo es ``a bee'', por lo que el
siguiente sufijo podría ser ``philosophically'', ``be'' o ``due''.

En este ejemplo, la longitud del prefijo es siempre dos, pero
puedes hacer análisis de Markov con cualquier longitud de prefijo.

\begin{exercise}

Análisis de Markov:

\begin{enumerate}

\item Escribe un programa que lea un texto desde un archivo y realice análisis
de Markov.  El resultado podría ser un diccionario que mapee de
prefijos a una colección de posibles sufijos.  La colección
podría ser una lista, tupla o diccionario; depende de ti hacer
una elección apropiada.  Puedes probar tu programa con un prefijo
de largo dos, pero podrías escribir el programa de una manera en que resulte
fácil intentar otras longitudes.

\item Agrega una función al programa anterior que genere texto aleatorio
basado en el análisis de Markov.  Aquí hay un ejemplo de {\em Emma}
con prefijo de largo 2:

\begin{quote}
He was very clever, be it sweetness or be angry, ashamed or only
amused, at such a stroke. She had never thought of Hannah till you
were never meant for me?" "I cannot make speeches, Emma:" he soon cut
it all himself.
\end{quote}

Para este ejemplo, dejé la puntuación unida a las palabras.
El resultado es casi sintácticamente correcto, pero no del todo.
Semánticamente, casi tiene sentido, pero no del todo.

¿Qué ocurre si aumentas la longitud del prefijo?  ¿Tiene más sentido
el texto aleatorio?

\item Una vez que tu programa funcione, quizás quieras probar una mezcla:
si combinas texto de dos o más libros, el texto aleatorio
que generes mezclará el vocabulario y las frases de
las fuentes de maneras interesantes.
\index{mash-up}

\end{enumerate}

Crédito: Este estudio de caso está basado en un ejemplo de Kernighan and
Pike, {\em The Practice of Programming}, Addison-Wesley, 1999.

\end{exercise}

Deberías intentar este ejercicio antes de continuar; luego puedes
descargar mi solución en \url{http://thinkpython.com/code/markov.py}.
Además, necesitarás \url{http://thinkpython.com/code/emma.txt}.


\section{Estructuras de datos}
\index{estructura de datos}

Utilizar análisis de Markov para generar texto aleatorio es divertido, pero también
hay un punto en este ejercicio: la selección de la estructura de datos.  En tu
solución al ejercicio anterior, tuviste que escoger:

\begin{itemize}

\item Cómo representar los prefijos.

\item Cómo representar la colección de posibles sufijos.

\item Cómo representar el mapeo de cada prefijo a
la colección de posibles sufijos.

\end{itemize}

El último es fácil: un diccionario es la elección obvia
para mapear de claves a valores correspondientes.

Para los prefijos, las opciones más obvias son cadenas,
lista de cadenas o tupla de cadenas.

Para los sufijos,
una opción es una lista, otra es un histograma (diccionario).
\index{implementación}

¿Cómo deberías escoger?  El primer paso es pensar en
las operaciones que necesitarás implementar para cada esturcura de datos.
Para prefijos, necesitamos poder eliminar palabras del
principio y agregar al final.  Por ejemplo, si el prefijo actual
es ``Half a'', y la siguiente palabra es ``bee'', necesitas
poder formar el siguiente prefijo, ``a bee''.
\index{tupla!como clave en diccionario}

Tu primera elección podría ser una lista, dado que es fácil agregar y
eliminar elementos, pero también necesitamos poder utilizar los
prefijos y claves en un diccionario, así que eso descarta las listas.
Con las tuplas, no puedes anexar o eliminar, pero puedes utilizar
el operador suma para formar una nueva tupla:

\begin{verbatim}
def cambiar(prefijo, palabra):
    return prefijo[1:] + (palabra,)
\end{verbatim}
%
{\tt cambiar} toma una tupla de palabras, {\tt prefijo}, y una cadena,
{\tt palabra}, y forma una nueva tupla que tiene todas las palabras
en {\tt prefijo} excepto la primera, y {\tt palabra} agregada al
final.

Para la colección de sufijos, las operaciones que necesitamos
realizar incluyen agregar un nuevo sufijo (o aumentar la frecuencia
de uno existente) y escoger un sufijo aleatorio.

Agregar un nuevo sufijo es igual de fácil con lista o histograma
en cuanto a implementación.  Escoger un elemento aleatorio de una lista
es fácil; escoger del histograma es más díficil de hacer 
de manera eficiente (ver Ejercicio~\ref{randhist}).

Hasta ahora hemos estado hablando principalmente sobre la facilidad de la implementación,
pero hay otros factores a considerar al escoger estructuras de datos.
Uno es el tiempo de ejecución.  A veces hay una razón teórica para esperar
que una estructura de datos sea más rápida que otra; por ejemplo, mencioné
que el operador {\tt in} es más rápido para diccionarios que para listas,
al menos cuando el número de elementos es grande.

Sin embargo, a menudo no sabes de antemano cuál implementación
será más rápida.  Una opción es implementar ambas y ver cuál
es mejor.  Este enfoque se llama {\bf evaluación comparativa} (en inglés, {\em benchmarking}).
Una alternativa práctica es escoger la estructura de datos que sea
la más fácil de implementar, y luego ver si es lo suficientemente rápida para
la aplicación en cuestión.  Si es así, no hay necesidad de seguir.  Si no,
hay herramientas, como el módulo {\tt profile}, que puede identificar
los lugares en un programa que toman la mayor parte del tiempo.
\index{evaluación comparativa}\index{benchmarking}
\index{modulo@módulo!profile}
\index{profile, módulo}

El otro factor a considerar es el espacio de almacenamiento.  Por ejemplo, utilizar un
histograma para la colección de sufijos podría ocupar menos espacio porque
solo tienes que almacenar cada palabra una vez, sin importar cuántas veces
aparezca en el texto.  En algunos casos, ahorrar espacio puede también hacer que tu
programa se ejecute más rápido, y en el caso extremo, tu programa podría no ejecutarse
en absoluto si te quedas sin memoria.  Sin embargo, para muchas aplicaciones el espacio es
una consideración secundaria después del tiempo de ejecución.

Una útlima reflexión: en esta discusión, he insinuado que
deberíamos utilizar una estructura de datos tanto para el análisis como para la generación.
Pero dado que estas son fases separadas, sería posible también utilizar una
estructura para el análisis y luego convertirla a otra estructura para
la generación.  Esto sería una ganancia neta si el tiempo ahorrado durante
la generación excediera al tiempo ocupado en la conversión.


\section{Depuración}
\index{depuración}

Cuando estés depurando un programa, y especialmente si estás
trabajando en un error de programación difícil, hay cinco cosas para probar:

\begin{description}

\item[Lectura:] Examina tu código, léelo de nuevo a ti mismo y
verifica que dice lo que querías decir.

\item[Ejecución:] Experimenta haciendo cambios y ejecutando versiones
diferentes.  A menudo, si muestras en pantalla lo correcto en el lugar correcto
del programa, el problema se vuelve obvio, pero a veces tienes que
construir andamiaje.

\item[Rumiación:] ¡Tómate un tiempo para pensar!  ¿Qué tipo de error
es: de sintaxis, de tiempo de ejecución o semántico?  ¿Qué información puedes obtener a partir
de los mensajes de error o de la salida del programa?  ¿Qué tipo de
error podría causar el problema que estás viendo?  ¿Qué cambiaste
últimamente, antes de que el problema apareciera?

\item[Patito de goma:] Si le explicas el problema a alguien más,
  a veces encuentras la respuesta antes de terminar la pregunta.
  A menudo no necesitas a la otra persona; podrías simplemente hablarle a un
  patito de goma.  Y ese es el origen de la conocida estrategia llamada {\bf
  método de depuración del patito de goma} (en inglés, {\em rubber duck debugging}).
  No estoy inventando; ver \url{https://en.wikipedia.org/wiki/Rubber_duck_debugging}.

\item[Retroceso:] En algún punto, lo mejor que se puede hacer es
retroceder, deshacer los cambios recientes, hasta que regreses a un programa que
funcione y que entiendas.  Luego puedes comenzar a reconstruir.

\end{description}

Los programadores principiantes a veces se atascan en una de estas actividades
y olvidan las otras.  Cada actividad viene con su propio modo
de fallo.
\index{error!tipográfico}

Por ejemplo, leer tu código podría ayudar si el problema es un
error tipográfico, pero no si el problema es un malentendido
conceptual.  Si no entiendes lo que tu programa hace,
puedes leerlo 100 veces y nunca ver el error, porque el error está en
tu cabeza.
\index{depuración!experimental}\index{experimental, depuración}

Ejecutar experimentos puede ayudar, especialmente si ejecutas pruebas
pequeñas y simples.  Pero si ejecutas experimentos sin pensar ni leer tu
código, podrías caer en un patrón que yo llamo ``programación de camino aleatorio'',
que es el proceso de hacer cambios aleatorios hasta que el programa
haga lo correcto.  No hace falta decir que la programación de camino aleatorio
puede tomar mucho tiempo.
\index{programación de camino aleatorio}
\index{plan de desarrollo!programación de camino aleatorio}

Tienes que tomarte el tiempo de pensar.  La depuración es como una
ciencia experimental: deberías tener al menos una hipótesis acerca de
cuál es el problema.  Si hay dos o más posibilidades, intenta
pensar en una prueba que eliminaría una de ellas.

Sin embargo, incluso las mejores técnicas de depuración fallarán si hay muchos
errores, o si el código que intentas arreglar es muy grande y
complicado.  A veces la mejor opción es retroceder, simplificando el
programa hasta que obtengas algo que funcione y que
entiendas.

Los programadores principiantes son a menudo reacios a retroceder porque
no pueden soportar eliminar una línea de códiigo (incluso si es incorrecta).
Si te hace sentir mejor, copia tu programa en otro archivo
antes de comenzar a recortarlo.  Luego puedes volver a copiar los
pedazos uno a la vez.

Encontrar un error de programación difícil requiere lectura, ejecución, rumiación y
a veces retroceso.  Si te atascas en una de estas actividades,
intenta las otras.


\section{Glosario}

\begin{description}

\item[determinista:] Dicho de un programa que hace lo
mismo cada vez que se ejecuta, dadas las mismas entradas.
\index{determinista}

\item[pseudoaleatorio:] Dicho de una secuencia de números que aparenta
ser aleatorio, pero se genera por un programa determinista.
\index{pseudoaleatorio}

\item[valor por defecto:] El valor dado a un parámetro opcional si
no se le entrega un argumento.

\index{valor por defecto}

\item[anular:] Reemplazar un valor por defecto con un argumento.
\index{anular}

\item[evaluación comparativa:] El proceso de escoger entre estructuras de datos
implementando alternativas y probándolas en una muestra de
posibles entradas.
\index{evaluación comparativa}

\item[método de depuración del patito de goma:] Depurar explicando tu problema
a un objeto inanimado tal como un patito de goma.  Articular el
problema puede ayudarte a resolverlo, incluso si el patito de goma no sabe
Python.
\index{rubber duck debugging}\index{patito de goma, depuración}
\index{depuración!patito de goma}

\end{description}


\section{Ejercicios}

\begin{exercise}
\index{frecuencia de palabras}
\index{palabras, frecuencia de}
\index{Ley de Zipf}

El ``rango'' de una palabra es su posición en una lista de palabras
ordenadas por frecuencia: la palabra más común tiene rango 1, la
segunda más común tiene rango 2, etc.

La ley de Zipf describe una relación entre los rangos y frecuencias de
palabras en lenguajes naturales
(\url{http://en.wikipedia.org/wiki/Zipf's_law}).  Específicamente,
predice que la frecuencia, $f$, de cada palabra con rango $r$ es:
\[ f = c r^{-s} \]
%
donde $s$ y $c$ son parámetros que dependen del lenguaje y el
texto.  Si tomas el logaritmo de ambos lados de esta ecuación,
obtienes:
\index{logaritmo}
\[ \log f = \log c - s \log r \]
%
Entonces, si graficas log $f$ versus log $r$, deberías obtener
una línea recta con pendiente $-s$ e intercepto log $c$.

Escribe un programa que lea un texto de un archivo, cuente
las frecuencias de palabras e imprima una línea
para cada palabra, en orden de frecuencia descendiente, con
log $f$ y log $r$.  Utiliza el progmara de gráficos de tu
elección para graficar los resultados y verificar si forman
una línea recta.  ¿Puedes estimar el valor de $s$?

Solución: \url{http://thinkpython.com/code/zipf.py}.
Para ejecutar mi solución, necesitas el módulo de gráficos {\tt matplotlib}.
Si instalaste Anaconda, ya tienes {\tt matplotlib};
de lo contrario, quizás tengas que instalarlo.
\index{matplotlib}

\end{exercise}



\chapter{Archivos}

Este capítulo presenta la idea de programas ``persistentes'' que
mantienen los datos en almacenamiento permanente y muestra cómo utilizar diferentes
tipos de almacenamiento permanente, tales como archivos y bases de datos.


\section{Persistencia}
\index{archivo}
\index{tipo!archivo}
\index{persistencia}

La mayoría de los programas que hemos visto hasta ahora son transitorios en el
sentido de que se ejecutan por un tiempo corto y producen alguna salida,
pero cuando terminan sus datos desaparecen.  Si ejecutas el programa
de nuevo, comienza con una pizarra en blanco.

Otros programas son {\bf persistentes}: se ejecutan por un tiempo largo
(o todo el tiempo), mantienen al menos algunos de sus datos
en almacenamiento permanente (un disco duro, por ejemplo) y,
si se apagan y reinician, continúan donde estaban.

Ejemplos de programas persistentes son los sistemas operativos, que se
se ejecutan casi siempre que un computador está encendido, y los servidores web,
que se ejecutan todo el tiempo, esperando solicitudes para entrar en
la red.

Una de las maneras más simples que tienen los programas para mantener sus datos 
es leyendo y escribiendo archivos de texto.  Ya hemos visto
programas que leen archivos de texto; en este capítulo veremos programas
que los escriban.

Una alternativa es almacenar el estado del programa en una base de datos.
En este capítulo presentaré una base de datos simple y un módulo,
{\tt pickle}, que facilita el almacenamiento de datos del programa.
\index{modulo@módulo!pickle}
\index{pickle!módulo}


\section{Leer y escribir}
\index{archivo!leer y escribir}

Un archivo de texto es una secuencia de caracteres almacenados en un medio
permanente como un disco duro, memoria flash o CD-ROM.  Vimos cómo
abrir y leer un archivo en la Sección~\ref{wordlist}.
\index{función!open}
\index{open, función}

Para escribir un archivo, tienes que abrirlo con el modo \verb"'w'" como segundo
parámetro:

\begin{verbatim}
>>> fout = open('output.txt', 'w')
\end{verbatim}
%
Si el archivo existe, abrirlo en modo escritura elimina
los datos antiguos y comienza de nuevo, ¡así que ten cuidado!
Si el archivo no existe, se crea uno nuevo.

{\tt open} devuelve un objeto de archivo que proporciona métodos para trabajar
con el archivo.
El método {\tt write} pone datos en el archivo.

\begin{verbatim}
>>> linea1 = "He aquí el junco,\n"
>>> fout.write(linea1)
    18
\end{verbatim}
%
El valor de retorno es la cantidad de caracteres que se escribieron.
El objeto de archivo hace un seguimiento del lugar en donde está, por lo cual si
llamas a {\tt write} de nuevo, agrega los nuevos datos al final del
archivo.

\begin{verbatim}
>>> linea2 = "emblema de nuestra tierra.\n"
>>> fout.write(linea2)
    27
\end{verbatim}
%
Cuando hayas terminado de escribir, deberías cerrar el archivo.

\begin{verbatim}
>>> fout.close()
\end{verbatim}
%
\index{metodo@método!close}
\index{close, método}
%
Si no cierras el archivo, se cierra cuando el
programa termina.


\section{Operador de formato}
\index{operador!de formato}
\index{formato, operador de}

El argumento de {\tt write} tiene que ser una cadena, por lo cual si queremos
poner valores en un archivo, tenemos que convertirlos a
cadenas.  La manera más fácil de hacer eso es con {\tt str}:

\begin{verbatim}
>>> x = 52
>>> fout.write(str(x))
\end{verbatim}
%
Una alternativa es utilizar el {\bf operador de formato}: {\tt \%}.  Cuando
se aplica a enteros, {\tt \%} es el operador de módulo.  Pero
cuando el primer operando es una cadena, {\tt \%} es el operador de formato.
\index{format string}

El primer operando es la {\bf cadena de formato}, que contiene
una o más {\bf secuencias de formato}, las cuales
especifican la manera en que
se da formato al segundo operando.  El resultado es una cadena.
\index{secuencia de formato}

Por ejemplo, la secuencia de formato \verb"'%d'" significa que
el segundo operando debería formatearse como un entero
decimal:

\begin{verbatim}
>>> camellos = 42
>>> '%d' % camellos
    '42'
\end{verbatim}
%
El resultado es la cadena \verb"'42'", que no debe confundirse
con el valor entero {\tt 42}.

Una secuencia de formato puede aparecer en cualquier lugar de la cadena,
así que puedes incrustar un valor en una oración:

\begin{verbatim}
>>> 'He visto %d camellos.' % camellos
    'He visto 42 camellos.'
\end{verbatim}
%
Si hay más de una secuencia de formato en la cadena,
el segundo argumento tiene que ser una tupla.  Cada secuencia de formato es
emparejada con un elemento de la tupla, en orden.

El siguiente ejemplo utiliza \verb"'%d'" para dar formato a un entero,
\verb"'%g'" para dar formato a un número de coma flotante y
\verb"'%s'" para dar formato a una cadena:

\begin{verbatim}
>>> 'En %d años he visto %g %s.' % (3, 0.1, 'camellos')
    'En 3 años he visto 0.1 camellos.'
\end{verbatim}
%
El número de elementos en la tupla tiene que coincidir con el número
de secuencias de formato en la cadena.  Además, los tipos de los
elementos tienen que coincidir con las secuencias de formato:
\index{excepción!TypeError}
\index{TypeError}

\begin{verbatim}
>>> '%d %d %d' % (1, 2)
    TypeError: not enough arguments for format string
>>> '%d' % 'dólares'
    TypeError: %d format: a number is required, not str
\end{verbatim}
%
En el primer ejemplo, no hay suficientes elementos; en el
segundo, el elemento tiene tipo incorrecto.

Para más información sobre el operador de formato, ver
\url{https://docs.python.org/3/library/stdtypes.html#printf-style-string-formatting}.  Una alternativa más poderosa es
el método de cadena format, del cual puedes leer en
\url{https://docs.python.org/3/library/stdtypes.html#str.format}.

% You can specify the number of digits as part of the format sequence.
% For example, the sequence \verb"'%8.2f'"
% formats a floating-point number to be 8 characters long, with
% 2 digits after the decimal point:

% % \begin{verbatim}
% >>> '%8.2f' % 3.14159
%     '    3.14'
% \end{verbatim}
% \afterverb
% %
% The result takes up eight spaces with two
% digits after the decimal point.


\section{Nombres de archivo y rutas}
\label{paths}
\index{nombre de archivo}
\index{path}\index{ruta}
\index{directorio}
\index{carpeta}

Los archivos se organizan en {\bf directorios} (también llamados ``carpetas'').
Cada programa que se ejecuta tiene un ``directorio actual'', que es el
directorio por defecto para la mayor parte de las operaciones.
Por ejemplo, cuando abres un archivo para lectura, Python lo busca en el
directorio actual.
\index{modulo@módulo!os}
\index{os, módulo}

El módulo {\tt os} proporciona funciones para trabajar con archivos y
directorios (``os'' significa ``operating system'').  {\tt os.getcwd}
devuelve el nombre del directorio actual:
\index{función!getcwd}
\index{getcwd, función}

\begin{verbatim}
>>> import os
>>> cwd = os.getcwd()
>>> cwd
    '/home/dinsdale'
\end{verbatim}
%
{\tt cwd} significa ``current working directory''.  El resultado en
este ejemplo es {\tt /home/dinsdale}, que es el directorio principal de un
usuario con nombre {\tt dinsdale}.
\index{directorio de trabajo}
\index{trabajo, directorio de}

Una cadena como \verb"'/home/dinsdale'" que identifica un archivo o un
directorio se llama {\bf ruta} (en inglés, {\em path}).

Un nombre de archivo simple, como {\tt memo.txt}, también se considera una ruta,
pero es una {\bf ruta relativa} porque se relaciona con el directorio
actual.  Si el directorio actual es {\tt /home/dinsdale}, el
nombre de archivo {\tt memo.txt} haría referencia a {\tt /home/dinsdale/memo.txt}.
\index{ruta relativa} \index{relativa, ruta}
\index{ruta absoluta} \index{absoluta, ruta}

Una ruta que comienza con {\tt /} no depende del directorio
actual: se llama {\bf ruta absoluta}.  Para encontrar la ruta
absoluta de un archivo, puedes utilizar {\tt os.path.abspath}:

\begin{verbatim}
>>> os.path.abspath('memo.txt')
    '/home/dinsdale/memo.txt'
\end{verbatim}
%
{\tt os.path} proporciona otras funciones para trabajar con nombres de archivo
y rutas.  Por ejemplo,
{\tt os.path.exists} verifica
si un archivo o directorio existe:
\index{función!exists}
\index{exists, función}

\begin{verbatim}
>>> os.path.exists('memo.txt')
    True
\end{verbatim}
%
Si existe, {\tt os.path.isdir} verifica si es un directorio:

\begin{verbatim}
>>> os.path.isdir('memo.txt')
    False
>>> os.path.isdir('/home/dinsdale')
    True
\end{verbatim}
%
Del mismo modo, {\tt os.path.isfile} verifica si es un archivo.

{\tt os.listdir} devuelve una lista de los archivos (y otros directorios)
en el directorio dado:

\begin{verbatim}
>>> os.listdir(cwd)
    ['music', 'photos', 'memo.txt']
\end{verbatim}
%
Para demostrar estas funciones, el siguiente ejemplo
``recorre'' un directorio, imprime
los nombres de todos los archivos y se llama a sí mismo de manera recursiva en
todos los directorios.
\index{directorio!recorrer}
\index{recorrer!directorio}

\begin{verbatim}
def walk(nombre_dir):
    for nombre in os.listdir(nombre_dir):
        ruta = os.path.join(nombre_dir, nombre)

        if os.path.isfile(ruta):
            print(ruta)
        else:
            walk(ruta)
\end{verbatim}
%
{\tt os.path.join} toma un directorio y un archivo y los une
en una ruta completa.

El módulo {\tt os} proporciona una función llamada {\tt walk} que es
similar a esta pero más versátil.  Como ejercicio, lee la
documentación y utiliza esta función para imprimir los nombres de los archivos
en un directorio dado y sus subdirectorios.  Puedes descargar mi solución en
\url{http://thinkpython.com/code/walk.py}.


\section{Capturar excepciones}
\label{catch}

Muchas cosas pueden salir mal cuando intentas leer y escribir
archivos.  Si intentas abrir un archivo que no existe, obtienes un
{\tt FileNotFoundError}:
\index{función!open}
\index{open, función}
\index{excepción!FileNotFoundError}
\index{FileNotFoundError}

\begin{verbatim}
>>> fin = open('archivo_malo')
    FileNotFoundError: [Errno 2] No such file or directory: 'archivo_malo'
\end{verbatim}
%
Si no tienes permisos de acceso a un archivo:
\index{permisos de archivo}
\index{archivo!permisos de}

\begin{verbatim}
>>> fout = open('/etc/passwd', 'w')
    PermissionError: [Errno 13] Permission denied: '/etc/passwd'
\end{verbatim}
%
Y si intentas abrir un directorio para lectura, obtienes

\begin{verbatim}
>>> fin = open('/home')
    IsADirectoryError: [Errno 21] Is a directory: '/home'
\end{verbatim}
%
Para evitar estos errores, podrías utilizar funciones como {\tt os.path.exists}
y {\tt os.path.isfile}, pero tomaría mucho tiempo y código
verificar todas las posibilidades (si ``{\tt Errno 21}'' indica
algo, hay al menos 21 cosas que pueden salir mal).
\index{excepción, detectar}
\index{sentencia!try}
\index{try, sentencia}

Es mejor continuar y avanzar ---y lidiar con los problemas, si
ocurren--- que es exactamente lo que hace la sentencia {\tt try}.  La
sintaxis es similar a una sentencia {\tt if...else}:

\begin{verbatim}
try:
    fin = open('archivo_malo')
except:
    print('Algo salió mal.')
\end{verbatim}
%
Python comienza ejecutando la cláusula {\tt try}.  Si todo sale
bien, se salta la cláusula {\tt except} y continúa.  Si ocurre
una excepción, salta hacia afuera de la cláusula {\tt try} y
ejecuta la cláusula {\tt except}.

Manejar una excepción con una sentencia {\tt try} se llama {\bf
capturar} una excepción.  En este ejemplo, la cláusula {\tt except}
imprime un mensaje de error que no es de mucha ayuda.  En general,
capturar una excepción te da una oportunidad de arreglar el problema, o intentar
de nuevo, o al menos terminar el programa de manera elegante.


\section{Bases de datos}
\index{base de datos}

Una {\bf base de datos} es un archivo que está organizado para almacenar datos.  Muchas
bases de datos están organizadas como un diccionario en el sentido de que mapean
de claves a valores.  La diferencia más grande entre una base de datos y un
diccionario es que la base de datos está en un disco (u otro almacenamiento
permanente), por lo cual persiste después de que el programa termina.  \index{modulo@módulo!dbm}
\index{dbm, módulo}

El módulo {\tt dbm} proporciona una interfaz para crear
y actualizar archivos de base de datos.
Como ejemplo, crearé una base de datos
que contiene títulos para archivos de imagen.
\index{función!open}
\index{open, función}

Abrir una base de datos es similar a abrir otros archivos:

\begin{verbatim}
>>> import dbm
>>> db = dbm.open('títulos', 'c')
\end{verbatim}
%
El modo \verb"'c'" significa que la base de datos debería ser creada si
no existe ya.  El resultado es un objeto de base de datos
que puede ser utilizado (para la mayoría de las operaciones) como un diccionario.
\index{objeto!de base de datos}
\index{base de datos!objeto de}

Cuando creas un nuevo ítem, {\tt dbm} actualiza el archivo de base de datos.
\index{actualizar!base de datos}

\begin{verbatim}
>>> db['cleese.png'] = 'Foto de John Cleese.'
\end{verbatim}
%
Cuando accedes a uno de los ítems, {\tt dbm} lee el archivo:

\begin{verbatim}
>>> db['cleese.png']
    b'Foto de John Cleese.'
\end{verbatim}
%
El resultado es un {\bf objeto de bytes}, que es la razón por la cual comienza con {\tt
  b}.  Un objeto de bytes es similar a una cadena en muchos sentidos.  A medida que llegas más lejos en Python, la diferencia se vuelve importante, pero por ahora
podemos ignorarla.
\index{objeto!de bytes}
\index{bytes, objeto de}

Si haces otra asignación a una clave existente, {\tt dbm} reemplaza
el valor antiguo:

\begin{verbatim}
>>> db['cleese.png'] = 'Foto de John Cleese haciendo un tonto paseo.'
>>> db['cleese.png']
    b'Foto de John Cleese haciendo un tonto paseo.'
\end{verbatim}
%

Algunos métodos de diccionario, como {\tt keys} e {\tt items}, no
funcionan con objetos de base de datos.  Pero la iteración con un bucle {\tt for}
funciona:
\index{modulo@módulo!dbm}

\begin{verbatim}
for clave in db.keys():
    print(clave, db[clave])
\end{verbatim}
%
Al igual que los otros archivos, deberías cerrar la base de datos cuando
termines:

\begin{verbatim}
>>> db.close()
\end{verbatim}
%
\index{metodo@método!close}
\index{close, método}


\section{Uso de pickle}
\index{pickle}

Una limitación de {\tt dbm} es que las claves y valores tienen que ser
cadenas o bytes.  Si intentas utilizar cualquier otro tipo, obtienes un error.
\index{modulo@módulo!pickle} \index{pickle!módulo}

El módulo {\tt pickle} puede ayudar.  Este módulo traduce
casi cualquier tipo de objeto en una cadena adecuada para almacenar
en una base de datos y también traduce cadenas para que vuelvan a ser objetos.

{\tt pickle.dumps} toma un objeto como parámetro y devuelve
una representación de cadena ({\tt dumps} es una abreviatura de ``dump string''):

\begin{verbatim}
>>> import pickle
>>> t = [1, 2, 3]
>>> pickle.dumps(t)
    b'\x80\x03]q\x00(K\x01K\x02K\x03e.'
\end{verbatim}
%
El formato no es obvio para lectores humanos: está hecho para que
{\tt pickle} lo encuentre fácil de interpretar.  {\tt pickle.loads}
(``load string'') reconstituye el objeto:

\begin{verbatim}
>>> t1 = [1, 2, 3]
>>> s = pickle.dumps(t1)
>>> t2 = pickle.loads(s)
>>> t2
    [1, 2, 3]
\end{verbatim}
%
Aunque el nuevo objeto tiene el mismo valor que el antiguo, no es
(en general) el mismo objeto:

\begin{verbatim}
>>> t1 == t2
    True
>>> t1 is t2
    False
\end{verbatim}
%
En otras palabras, ``picklear'' y luego ``despicklear'' tiene el mismo efecto
que copiar el objeto.

Puedes utilizar {\tt pickle} para almacenar objetos que no sean cadena en una base de datos.
De hecho, esta combinación es tan común que ha sido
encapsulada en un módulo llamado {\tt shelve}.
\index{modulo@módulo!shelve}
\index{shelve, módulo}


\section{Tuberías}
\index{shell}
\index{tubería}

La mayoría de los sistemas operativos proporcionan una interfaz de línea de comandos,
también conocida como {\bf shell}.  Las shells generalmente proporcionan comandos
para navegar en el sistema de archivos e iniciar aplicaciones.  Por
ejemplo, en Unix puedes cambiar directorios con {\tt cd},
mostrar los contenidos de un directorio con {\tt ls} e iniciar
un navegador web escribiendo (por ejemplo) {\tt firefox}.
\index{ls (comando Unix)}
\index{comando Unix ls}

Cualquier programa que puedes iniciar desde la shell puede también
iniciarse desde Python utilizando un {\bf objeto de tubería} (en inglés, {\em pipe object}), que
representa un programa en ejecución.

Por ejemplo, el comando Unix {\tt ls -l} normalmente muestra los
contenidos del directorio actual en formato largo.  Puedes
ejecutar {\tt ls} con {\tt os.popen}\footnote{{\tt popen} ahora está
obsoleto, lo cual significa que se supone que debemos dejar de utilizarla y comenzar a utilizar
el módulo {\tt subprocess}.  Pero para casos simples, encuentro a
{\tt subprocess} más complicado que necesario.  Entonces voy a
seguir utilizando {\tt popen} hasta que lo quiten.}:
\index{función!popen}
\index{popen, función}

\begin{verbatim}
>>> cmd = 'ls -l'
>>> fp = os.popen(cmd)
\end{verbatim}
%
El argumento es una cadena que contiene un comando de shell.  El
valor de retorno es un objeto que se comporta como un archivo
abierto.  Puedes leer la salida del proceso {\tt ls} una
línea a la vez con {\tt readline} u obtener todo de una vez
con {\tt read}:
\index{metodo@método!readline}
\index{readline, método}
\index{metodo@método!read}
\index{read, método}

\begin{verbatim}
>>> res = fp.read()
\end{verbatim}
%
Cuando termines, cierras la tubería como un archivo:
\index{metodo@método!close}
\index{close, método}

\begin{verbatim}
>>> stat = fp.close()
>>> print(stat)
    None
\end{verbatim}
%
El valor de retorno es el estado final del proceso {\tt ls};
{\tt None} significa que termina de manera normal (sin errores).

Por ejemplo, la mayoría de los sistemas Unix proporcionan un comando llamado {\tt md5sum}
que lee los contenidos de un archivo y calcula una ``suma de verificación''.
Puedes leer sobre MD5 en \url{http://en.wikipedia.org/wiki/Md5}.  Este
comando proporciona una manera eficiente de verificar que dos archivos
tengan los mismos contenidos.  La probabilidad de que diferentes contenidos
entreguen la misma suma de verificación es muy pequeña (es decir, improbable que ocurra
antes de que el universo colapse).
\index{md5}
\index{suma de verificación}

Puedes utilizar una tubería para ejecutar {\tt md5sum} desde Python y obtener el resultado:

\begin{verbatim}
>>> nombre_archivo = 'book.tex'
>>> cmd = 'md5sum ' + nombre_archivo
>>> fp = os.popen(cmd)
>>> res = fp.read()
>>> stat = fp.close()
>>> print(res)
    1e0033f0ed0656636de0d75144ba32e0  book.tex
>>> print(stat)
    None
\end{verbatim}



\section{Escribir módulos}
\label{modules}
\index{modulo, escribir@módulo, escribir}
\index{contar palabras}

Cualquier archivo que contiene código Python puede importarse como módulo.
Por ejemplo, supongamos que tienes un archivo con nombre {\tt wc.py} con el siguiente
código:

\begin{verbatim}
def contar_lineas(nombre_archivo):
    contar = 0
    for linea in open(nombre_archivo):
        contar += 1
    return contar

print(contar_lineas('wc.py'))
\end{verbatim}
%
Si ejecutas este programa, se lee a sí mismo e imprime el número
de líneas en el archivo, el cual es 7.
Puedes también importarlo así:

\begin{verbatim}
>>> import wc
    7
\end{verbatim}
%
Ahora tienes un objeto de módulo {\tt wc}:
\index{objeto!de módulo}
\index{modulo@módulo!objeto}

\begin{verbatim}
>>> wc
    <module 'wc' from 'wc.py'>
\end{verbatim}
%
El objeto de módulo proporciona \verb"contar_lineas":

\begin{verbatim}
>>> wc.contar_lineas('wc.py')
    7
\end{verbatim}
%
Entonces así es como escribes módulos en Python.

El único problema con este ejemplo es que, cuando importas
el módulo, ejecuta el código de prueba de la parte final.  Normalmente,
cuando importas un módulo, este define nuevas funciones pero
no las ejecuta.
\index{sentencia!import}
\index{import, sentencia}

Los programas que serán importados como módulos a menudo
utilizan la siguiente forma:

\begin{verbatim}
if __name__ == '__main__':
    print(contar_lineas('wc.py'))
\end{verbatim}
%
\verb"__name__" es una variable incorporada que se establece cuando
se inicia el programa.  Si el programa se está ejecutando como un script,
\verb"__name__" tiene el valor \verb"'__main__'"; en ese
caso, el código de prueba se ejecuta.  De lo contrario,
si el módulo se está importando, se salta el código de prueba.

\index{name, variable incorporada}
\index{main}

Como ejercicio, escribe este ejemplo en un archivo con nombre {\tt wc.py} y ejecútalo
como un script.  Luego ejecuta el intérprete de Python y
haz {\tt import wc}.  ¿Cuál es el valor de \verb"__name__"
cuando el módulo se está importando?

Advertencia: si importas un módulo que ya ha sido importado,
Python no hace nada.  No vuelve a leer el archivo, incluso si ha
cambiado.
\index{modulo@módulo!reload}
\index{función!reload}
\index{reload, función}

Si quieres volver a cargar un módulo, puedes utilizar la función incorporada
{\tt reload}, pero puede ser complicada, por lo cual lo más seguro es
reiniciar el intérprete y luego importar el módulo de nuevo.


\section{Depuración}
\index{depuración}
\index{espacio en blanco}

Cuando lees y escribes archivos, podrías tener problemas
con el espacio en blanco.  Estos errores pueden ser difíciles de depurar porque los espacios,
sangrías y nuevas líneas son normalmente invisibles:

\begin{verbatim}
>>> s = '1 2\t 3\n 4'
>>> print(s)
    1 2	 3
    4
\end{verbatim}
\index{función!repr}
\index{repr, función}
\index{representación de cadena}

La función incorporada {\tt repr} puede ayudar.  Toma cualquier objeto como
argumento y devuelve una representación de cadena del objeto.  Para
las cadenas, representa los caracteres de
espacio en blanco con secuencias de barras invertidas:

\begin{verbatim}
>>> print(repr(s))
    '1 2\t 3\n 4'
\end{verbatim}

Esto puede ser útil para depurar.

Otro problema que podrías encontrar es que sistemas diferentes
utilizan caracteres diferentes para indicar el fin de una línea.  Algunos
sistemas utilizan una nueva línea, representada por \verb"\n".  Otros utilizan
un carácter ``return'', representado por \verb"\r".  Algunos utilizan ambos.
Si mueves archivos entre sistemas diferentes, estas inconsistencias
pueden causar problemas.

\index{caracter de fin de linea@carácter de fin de línea}

Para la mayoría de los sistemas, hay aplicaciones para convertir de
un formato a otro.  Puedes encontrarlas (y leer más sobre este
tema) en \url{http://en.wikipedia.org/wiki/Newline}.  O, por supuesto,
puedes escribir una por tu cuenta.


\section{Glosario}

\begin{description}

\item[persistente:] Dicho de un programa que se ejecuta indefinidamente
y mantiene al menos alguno de sus datos en almacenamiento permanente.
\index{persistencia}

\item[operador de formato:] Un operador, {\tt \%}, que toma una cadena
de formato y una tupla y genera una cadena que incluye
los elementos de la tupla con el formato especificado por la cadena de formato.
\index{operador!de formato}
\index{formato, operador de}

\item[cadena de formato:] Una cadena, utilizada con el operador de formato, que
contiene secuencias de formato.
\index{cadena!de formato}

\item[secuencia de formato:] Una secuencia de caracteres en una cadena de formato,
como {\tt \%d}, que especifica cómo debería ser el formato de un valor.
\index{secuencia de formato}

\item[archivo de texto:] Una secuencia de caracteres guardada en almacenamiento
permanente, como en un disco duro.
\index{archivo!de texto}

\item[directorio:] Una colección de archivos que tiene un nombre, también llamado carpeta.
\index{directorio}

\item[ruta:] Una cadena que identifica un archivo.
\index{ruta}

\item[ruta relativa:] Una ruta que comienza desde el directorio actual.
\index{ruta relativa}

\item[ruta absoluta:] Una ruta que comienza desde el directorio más alto
en el sistema de archivos.
\index{ruta absoluta}

\item[capturar]: Evitar que una excepción termine
un programa, utilizando las sentencias {\tt try}
y {\tt except}.
\index{capturar}

\item[base de datos:] Un archivo cuyo contenido está organizado como un diccionario
con claves y sus correspondientes valores.
\index{base de datos}

\item[objeto de bytes:] Un objeto similar a una cadena.
\index{objeto!de bytes}
\index{bytes, objeto de}

\item[shell:] Un programa que permite a los usuarios escribir comandos y luego
ejecutarlos iniciando otros programas.
\index{shell}

\item[objeto de tubería:] Un objeto que representa un programa en ejecución, permitiendo
a un programa de Python ejecutar comandos y leer los resultados.
\index{objeto!de tubería}
\index{tubería!objeto}

\end{description}


\section{Ejercicios}

\begin{exercise}

Escribe una función llamada {\tt sed} que tome como argumentos una cadena de patrón,
una cadena de reemplazo y dos nombres de archivo; debería leer el primer archivo
y escribir los contenidos en el segundo archivo (creándolo si es
necesario).  Si la cadena de patrón aparece en algún lugar en el archivo,
debería reemplazarse con la cadena de reemplazo.

Si ocurre un error mientras se abren, leen, escriben o cierran los archivos,
tu programa debería capturar la excepción, imprimir un mensaje de error y
salir.  Solución: \url{http://thinkpython.com/code/sed.py}.

\end{exercise}


\begin{exercise}
\index{conjunto de anagramas}
\index{anagramas, conjunto de}

Si descargas mi solución al Ejercicio~\ref{anagrams} de
\url{http://thinkpython.com/code/anagram_sets.py}, verás que crea
un diccionario que mapea de una cadena ordenada de letras a la lista de
palabras que se pueden formar con esas letras.  Por ejemplo,
\verb"'opst'" mapea a la lista
\verb"['opts', 'post', 'pots', 'spot', 'stop', 'tops']".

Escribe un módulo que importe \verb"anagram_sets" y proporcione
dos nuevas funciones: \verb"store_anagrams" debería almacenar el
diccionario de anagrama en un ``estante'' (ver módulo {\tt shelve}); \verb"read_anagrams" debería
buscar una palabra y devolver una lista de sus anagramas.
Solución: \url{http://thinkpython.com/code/anagram_db.py}.

\end{exercise}


\begin{exercise}
\label{checksum}
\index{MP3}

En una gran coleccioń de archivos MP3, puede haber más de una
copia de la misma canción, almacenados en directorios diferentes o con
nombres de archivo diferentes.  El objetivo de este ejercicio es buscar
duplicados.

\begin{enumerate}

\item Escribe un programa que busque un directorio y todos sus
subdirectorios, de manera recursiva, y devuelva una lista de rutas completas
para todos los archivos con un sufijo dado (como {\tt .mp3}).
Pista: {\tt os.path} proporciona varias funciones útiles para
manipular nombres de archivo y de rutas.
\index{duplicado}
\index{algoritmo MD5}
\index{MD5, algoritmo}
\index{suma de verificación}

\item Para reconocer duplicados, puedes utilizar {\tt md5sum}
para calcular una ``suma de verificación'' de cada archivo.  Si dos archivos tienen
la misma suma de verificación, probablemente tienen los mismos contenidos.
\index{md5sum}

\item Para una doble verificación, puedes utilizar el comando Unix {\tt diff}.
\index{diff}

\end{enumerate}

Solución: \url{http://thinkpython.com/code/find_duplicates.py}.

\end{exercise}



\chapter{Clases y objetos}
\label{clobjects}

En este punto sabes cómo utilizar
funciones para organizar código y
tipos incorporados para organizar datos.  El siguiente paso es aprender
``programación orientada a objetos'', que utiliza tipos definidos por el programador
para organizar tanto código como datos.  La programación orientada a objetos es
un gran tema; tomará un par de capítulos para llegar allí.
\index{programación orientada a objetos}

Los ejemplos de código de este capítulo están disponibles en
\url{http://thinkpython.com/code/Point1.py}; las soluciones
a los ejercicios están disponibles en
\url{http://thinkpython.com/code/Point1_soln.py}.


\section{Tipos definidos por el programador}
\label{point}
\index{tipo!definido por el programador}
\index{programador, tipo definido por el}

Hemos utilizado muchos tipos incorporados de Python; ahora vamos
a definir un tipo nuevo.  Como ejemplo, crearemos un tipo
llamado {\tt Punto} que representa un punto en el espacio
bidimensional.
\index{punto matemático}

En notación matemática, los puntos son a menudo escritos en
paréntesis con una coma que separa las coordenadas.  Por ejemplo,
$(0,0)$ representa el origen, y $(x,y)$ representa el
punto que está $x$ unidades hacia la derecha e $y$ unidades hacia arriba, a partir del origen.

Hay muchas maneras en las cuales podríamos representar puntos en Python:

\begin{itemize}

\item Podríamos almacenar las coordenadas de manera separada en dos
variables, {\tt x} e {\tt y}.

\item Podríamos almacenar las coordenadas como elementos en una lista
o tupla.

\item Podríamos crear un tipo nuevo para representar puntos como
objetos.

\end{itemize}
\index{representación}

Crear un tipo nuevo
es más complicado que las otras opciones, pero
tiene ventajas que pronto serán aparentes.

Un tipo definido por el programador también se llama {\bf clase}.
Una definición de clase se ve así:
\index{clase}
\index{objeto!de clase}
\index{definición de clase}
\index{clase!definición}

\begin{verbatim}
class Punto:
    """Representa un punto en un espacio 2-D."""
\end{verbatim}
%
El encabezado indica que la clase nueva se llama {\tt Punto}.
El cuerpo es un docstring que explica para qué es la clase.
Puedes definir variables y métodos dentro de una definición de clase,
pero volveremos a eso más adelante.
\index{clase!Punto}
\index{Punto, clase}
\index{docstring}

Definir una clase con nombre {\tt Punto} crea un {\bf objeto de clase}.

\begin{verbatim}
>>> Punto
    <class '__main__.Punto'>
\end{verbatim}
%
Dado que {\tt Punto} se define en el nivel más alto, su ``nombre
completo'' es \verb"__main__.Punto".
\index{objeto!de clase}
\index{clase!objeto}

El objeto de clase es como una fábrica para crear objetos.  Para crear un
Punto, llamas a {\tt Punto} como si fuera una función.

\begin{verbatim}
>>> vacio = Punto()
>>> vacio
    <__main__.Punto object at 0xb7e9d3ac>
\end{verbatim}
%
El valor de retorno es una referencia a un objeto Punto, que
asignamos a {\tt vacio}.

El acto de crear un objeto nuevo se llama
{\bf instanciación}, y el objeto es una {\bf instancia} de
la clase.
\index{instancia}
\index{instanciación}

Cuando imprimes una instancia, Python te dice a qué clase
pertenece y dónde se almacena en la memoria (el prefijo
{\tt 0x} significa que el siguiente número es un hexadecimal).
\index{hexadecimal}

Cada objeto es una instancia de alguna clase, por tanto ``objeto'' e
``instancia'' son intercambiables.  Sin embargo, en este capítulo utilizo
``instancia'' para indicar que estoy hablando de un tipo definido por el
programador.


\section{Atributos}
\label{attributes}
\index{atributo de instancia}
\index{instancia!atributo de}
\index{notación de punto}

Puedes asignar valores a una instancia utilizando notación de punto:

\begin{verbatim}
>>> vacio.x = 3.0
>>> vacio.y = 4.0
\end{verbatim}
%
Esta sintaxis es similar a la sintaxis para seleccionar una variable de un
módulo, tal como {\tt math.pi} o {\tt string.whitespace}.  En este caso,
sin embargo, estamos asignando valores a elementos que tienen nombre y pertenecen a un objeto.
Estos elementos se llaman {\bf atributos}.

%As a noun, ``AT-trib-ute'' is pronounced with emphasis on the first
%syllable, as opposed to ``a-TRIB-ute'', which is a verb.

La Figura~\ref{fig.point} es un diagrama de estado que muestra el resultado de estas asignaciones.
Un diagrama de estado que muestra un objeto con sus atributos se
llama {\bf diagrama de objeto}.

\index{diagrama!de estado}
\index{estado, diagrama de}
\index{diagrama!de objeto}
\index{objeto!diagrama de}

\begin{figure}
\centerline
{\includegraphics[scale=0.8]{figs/point.pdf}}
\caption{Diagrama de objeto.}
\label{fig.point}
\end{figure}

La variable {\tt vacio} se refiere a un objeto Punto, que
contiene dos atributos.  Cada atributo se refiere a un
número de coma flotante.

Puedes leer el valor de un atributo utilizando la misma sintaxis:

\begin{verbatim}
>>> vacio.y
    4.0
>>> x = vacio.x
>>> x
    3.0
\end{verbatim}
%
La expresión {\tt vacio.x} significa ``Ve al objeto al cual {\tt vacio}
se refiere y obten el valor de {\tt x}.''  En este ejemplo, asignamos ese
valor a una variable con nombre {\tt x}.  No hay conflicto entre
la variable {\tt x} y el atributo {\tt x}.

Puedes utilizar la notación de punto como parte de una expresión.  Por ejemplo:

\begin{verbatim}
>>> '(%g, %g)' % (vacio.x, vacio.y)
    '(3.0, 4.0)'
>>> distancia = math.sqrt(vacio.x**2 + vacio.y**2)
>>> distancia
    5.0
\end{verbatim}
%
Puedes pasar una instancia como argumento en la manera usual.
Por ejemplo:
\index{instancia!como argumento}

\begin{verbatim}
def imprimir_punto(p):
    print('(%g, %g)' % (p.x, p.y))
\end{verbatim}
%
\verb"imprimir_punto" toma un punto como argumento y lo muestra en
notación matemática.  Para invocarla, puedes pasar a {\tt vacio} como
argumento:

\begin{verbatim}
>>> imprimir_punto(vacio)
    (3.0, 4.0)
\end{verbatim}
%
Dentro de la función, {\tt p} es un alias para {\tt vacio}, por tanto si
la función modifica a {\tt p}, {\tt vacio} cambia.
\index{alias}

Como ejercicio, escribe una función llamada \verb"distancia_entre_puntos"
que tome dos Puntos como argumentos y devuelva la distancia entre
ellos.


\section{Rectángulos}
\label{rectangles}

A veces es obvio cuáles deberían ser los atributos de un objeto,
pero otras veces tienes que tomar decisiones.  Por ejemplo, imagina que
estás diseñando una clase para representar rectángulos. ¿Qué atributos
utilizarías para especificar la ubicación y tamaño de un rectángulo?  Puedes
ignorar el ángulo; para mantener las cosas simples, supongamos que el rectángulo es
vertical u horizontal.
\index{representación}

Hay al menos dos posibilidades:

\begin{itemize}

\item Podrías especificar una esquina del rectángulo
(o el centro), la anchura y la altura.

\item Podrías especificar dos esquinas opuestas.

\end{itemize}

En este punto es difícil decir si una alternativa es mejor que
la otra, así que implementaremos la primera, solo como ejemplo.
\index{clase!Rectángulo}
\index{Rectángulo, clase}

Esta es la definición de la clase:

\begin{verbatim}
class Rectangulo:
    """Representa un rectángulo.

    atributos: anchura, altura, esquina.
    """
\end{verbatim}
%
El docstring contiene una lista de los atributos:  {\tt anchura} y
{\tt altura} son números; {\tt esquina} es un objeto Punto que
especifica la esquina inferior izquierda.

Para representar un rectángulo, tienes que instanciar un objeto Rectángulo
y asignar valores a los atributos:

\begin{verbatim}
caja = Rectangulo()
caja.anchura = 100.0
caja.altura = 200.0
caja.esquina = Punto()
caja.esquina.x = 0.0
caja.esquina.y = 0.0
\end{verbatim}
%
La expresión {\tt caja.esquina.x} significa
``Ve al objeto al cual {\tt caja} se refiere y seleccionea el atributo con nombre
{\tt esquina}; luego ve a ese objeto y selecciona el atributo con nombre
{\tt x}.''

\begin{figure}
\centerline
{\includegraphics[scale=0.8]{figs/rectangle.pdf}}
\caption{Diagrama de objeto.}
\label{fig.rectangle}
\end{figure}


La Figura~\ref{fig.rectangle} muestra el estado de este objeto.
Un objeto que es un atributo de otro objeto está {\bf incrustado}.
\index{diagrama!de estado}
\index{estado, diagrama de}
\index{diagrama!de objeto}
\index{objeto!diagrama de}
\index{objeto!incrustado}
\index{incrustado, objeto}


\section{Instancias como valores de retorno}
\index{instancia!como valor de retorno}
\index{valor de retorno}

Las funciones pueden devolver instancias.  Por ejemplo, \verb"encontrar_centro"
toma un {\tt Rectangulo} como argumento y devuelve un {\tt Punto}
que contiene las coordenadas del centro del {\tt Rectangulo}:

\begin{verbatim}
def encontrar_centro(rect):
    p = Punto()
    p.x = rect.esquina.x + rect.anchura/2
    p.y = rect.esquina.y + rect.altura/2
    return p
\end{verbatim}
%
Aquí hay un ejemplo que pasa a {\tt caja} como argumento y asigna
el Punto resultante a {\tt centro}:

\begin{verbatim}
>>> centro = entontrar_centro(caja)
>>> imprimir_punto(centro)
    (50, 100)
\end{verbatim}
%

\section{Los objetos son mutables}
\index{objeto!mutable}
\index{mutabilidad}

Puedes cambiar el estado de un objeto haciendo una asignación a uno de
sus atributos.  Por ejemplo, para cambiar el tamaño de un rectángulo
sin cambiar su posición, puedes modificar los valores de {\tt
anchura} y {\tt altura}:

\begin{verbatim}
caja.anchura = caja.anchura + 50
caja.altura = caja.altura + 100
\end{verbatim}
%
Puedes también escribir funciones que modifiquen objetos.  Por ejemplo,
\verb"crecer_rectangulo" toma un objeto Rectángulo y dos números,
{\tt d\_anchura} y {\tt d\_altura}, y suma los números a la
anchura y altura del rectángulo:

\begin{verbatim}
def crecer_rectangulo(rect, d_anchura, d_altura):
    rect.anchura += d_anchura
    rect.altura += d_altura
\end{verbatim}
%
Aquí hay un ejemplo que demuestra el efecto:

\begin{verbatim}
>>> caja.anchura, caja.altura
    (150.0, 300.0)
>>> crecer_rectangulo(caja, 50, 100)
>>> caja.anchura, caja.altura
    (200.0, 400.0)
\end{verbatim}
%
Dentro de la función, {\tt rect} es un
alias para {\tt caja}, por tanto cuando la función modifica a {\tt rect},
{\tt caja} cambia.

Como ejercicio, escribe una función con nombre \verb"mover_rectangulo" que tome
un Rectángulo y dos números con nombre {\tt dx} y {\tt dy}.  Debería
cambiar la ubicación del rectángulo sumando {\tt dx}
a la coordenada {\tt x} de {\tt esquina} y sumando {\tt dy}
a la coordenada {\tt y} de {\tt esquina}.


\section{Copiar}
\label{copying}
\index{alias}

Los alias pueden hacer que un programa sea difícil de leer porque los cambios
en un lugar podrían tener efectos inesperados en otro lugar.
Es difícil hacer un seguimiento de todas las variables que podrían referirse
a un objeto dado.
\index{copiar objetos}
\index{objetos, copiar}
\index{modulo@módulo!copy}
\index{copy, módulo}

Copiar un objeto es a menudo una alternativa a los alias.
El módulo {\tt copy} contiene una función llamada {\tt copy} que
puede duplicar cualquier objeto:

\begin{verbatim}
>>> p1 = Punto()
>>> p1.x = 3.0
>>> p1.y = 4.0

>>> import copy
>>> p2 = copy.copy(p1)
\end{verbatim}
%
{\tt p1} y {\tt p2} contienen los mismos datos, pero no son
el mismo Punto.

\begin{verbatim}
>>> imprimir_punto(p1)
    (3, 4)
>>> imprimir_punto(p2)
    (3, 4)
>>> p1 is p2
    False
>>> p1 == p2
    False
\end{verbatim}
%
El operador {\tt is} indica que {\tt p1} y {\tt p2} no son el
mismo objeto, que es lo que esperábamos.  Sin embargo, tal vez hayas esperado
que {\tt ==} entregue {\tt True} porque estos puntos contienen los mismos
datos.  En ese caso, te decepcionará aprender que, para
instancias, el comportamiento por defecto del operador {\tt ==} es el mismo
que el operador {\tt is}: verifica identidad de objeto, no equivalencia
de objeto.  Eso ocurre debido a que para tipos definidos por el programador, Python no
sabe qué debería considerarse equivalente.  Al menos, no todavía.

\index{is, operador}
\index{operador!is}
\index{identidad}
\index{equivalencia}

Si usas {\tt copy.copy} para duplicar un Rectángulo, encontrarás
que copia el objeto Rectángulo pero no el Punto incrustado.
\index{objeto incrustado, copiar}

\begin{verbatim}
>>> caja2 = copy.copy(caja)
>>> caja2 is caja
    False
>>> caja2.esquina is caja.esquina
    True
\end{verbatim}

\begin{figure}
\centerline
{\includegraphics[scale=0.8]{figs/rectangle2.pdf}}
\caption{Diagrama de objeto.}
\label{fig.rectangle2}
\end{figure}

La Figura~\ref{fig.rectangle2} muestra cómo se ve el diagrama de objeto.
\index{diagrama!de estado}
\index{estado, diagrama de}
\index{diagrama!de objeto}
\index{objeto!diagrama de}
Esta operación se llama {\bf copia superficial} porque copia al
objeto y cualquier referencia que contenga, pero no los objetos incrustados.
\index{copia!superficial}
\index{superficial, copia}

Para la mayoría de las aplicaciones, esto no es lo que quieres.  En este ejemplo,
invocar a \verb"crecer_rectangulo" en uno de los Rectángulos no
afectaría al otro, ¡pero invocar a \verb"mover_rectangulo" en cualquiera
afectaría a ambos!  Este comportamiento es confuso y propenso a errores.
\index{copia!profunda}
\index{profunda, copia}

Afortunadamente, el módulo {\tt copy} proporciona un método con nombre {\tt
deepcopy} que copia no solo el objeto sino también
los objetos a los cuales este se refiere, y los objetos a los cuales {\em estos} se refieren,
y así sucesivamente.
No te sorprenderá saber que esta operación se
llama {\bf copia profunda}.
\index{función!deepcopy}
\index{deepcopy, función}

\begin{verbatim}
>>> caja3 = copy.deepcopy(caja)
>>> caja3 is caja
    False
>>> caja3.esquina is caja.esquina
    False
\end{verbatim}
%
{\tt caja3} y {\tt caja} son objetos completamente separados.

Como ejercicio, escribe una versión de \verb"mover_rectangulo" que cree y
devuelva un Rectángulo nuevo en lugar de modificar el antiguo.


\section{Depuración}
\label{hasattr}
\index{depuración}

Cuando comienzas a trabajar con objetos, es probable que encuentres
algunas excepciones nuevas.  Si intentas acceder a un atributo
que no existe, obtienes un {\tt AttributeError}:
\index{excepción!AttributeError}
\index{AttributeError}

\begin{verbatim}
>>> p = Punto()
>>> p.x = 3
>>> p.y = 4
>>> p.z
    AttributeError: Point instance has no attribute 'z'
\end{verbatim}
%
Si no sabes bien de qué tipo es un objeto, puedes consultar:
\index{función!type}
\index{type, función}

\begin{verbatim}
>>> type(p)
    <class '__main__.Punto'>
\end{verbatim}
%
Puedes también utilizar {\tt isinstance} para verificar si un objeto
es una instancia de una clase:
\index{función!isinstance}
\index{isinstance, función}

\begin{verbatim}
>>> isinstance(p, Punto)
    True
\end{verbatim}
%
Si no sabes bien si un objeto tiene un atributo en particular,
puedes utilizar la función incorporada {\tt hasattr}:
\index{función!hasattr}
\index{hasattr, función}

\begin{verbatim}
>>> hasattr(p, 'x')
    True
>>> hasattr(p, 'z')
    False
\end{verbatim}
%
El primer argumento puede ser cualquier objeto; el segundo argumento es una {\em
cadena} que contiene el nombre del atributo.
\index{atributo}

Puedes también utilizar una sentencia {\tt try} para ver si el objeto tiene los
atributos que necesitas:
\index{sentencia!try}
\index{try, sentencia}

\begin{verbatim}
try:
    x = p.x
except AttributeError:
    x = 0
\end{verbatim}

Este enfoque puede hacer más fácil escribir funciones que trabajen con
tipos diferentes; se verá más sobre este tema
en la Sección~\ref{polymorphism}.


\section{Glosario}

\begin{description}

\item[clase:] Un tipo definido por el programador.  Una definición de clase crea un nuevo
objeto de clase.
\index{clase}
\index{tipo!definido por el programador}
\index{definido por el programador!tipo}

\item[objeto de clase:] Un objeto que contiene información acerca del
tipo definido por el programador.  El objeto de clase se puede utilizar para crear instancias
del tipo.
\index{objeto!de clase}
\index{clase!objeto}

\item[instancia:] Un objeto que pertenece a una clase.
\index{instancia}

\item[instanciar:] Crear un nuevo objeto.
\index{instanciar}

\item[atributo:] Uno de los valores con nombre que están asociados a un objeto.
\index{instancia!atributo}
\index{atributo de instancia}

\item[objeto incrustado:] Un objeto que se almacena como atributo
de otro objeto.
\index{objeto!incrustado}
\index{incrustado, objeto}

\item[copia superficial:] Copiar los contenidos de un objeto, incluyendo
cualquier referencia a los objetos incrustados;
implementada por la función {\tt copy} del módulo {\tt copy}.
\index{copia!superficial}

\item[copia profunda:] Copiar los contenidos de un objeto así como cualquier
objeto incrustado, y cualquier objeto incrustado en estos, y así sucesivamente;
implementado por la función {\tt deepcopy} del módulo {\tt copy}.
\index{copia!profunda}

\item[diagrama de objeto:] Un diagrama que muestra objetos, sus
atributos y los valores de los atributos.
\index{diagrama!de objeto}
\index{objeto!diagrama de}

\end{description}


\section{Ejercicios}

\begin{exercise}

Escribe una definición de una clase con nombre {\tt Circulo} cuyos atributos sean
{\tt centro} y {\tt radio}, donde {\tt centro} es un objeto Punto
y {\tt radio} es un número.

Instancia un objeto Círculo que represente un círculo con su centro
en $(150, 100)$ y radio 75.

Escribe una función con nombre \verb"punto_en_circulo" que tome un Círculo y
un Punto, y devuelva True si el Punto está dentro o en el borde del
círculo.

Escribe una función con nombre \verb"rect_en_circulo" que tome un Círculo y un
Rectángulo y devuelva True si el Rectángulo está completamente dentro o en el borde
del círculo.

Escribe una función con nombre \verb"rect_circ_traslapan" que tome un Círculo
y un Rectángulo y devuelva True si alguna de las esquinas del Rectángulo cae
dentro del círculo.  O como versión más desafiante, que devuelva True si
cualquier parte del Rectángulo cae dentro del círculo.

Solución: \url{http://thinkpython.com/code/Circle.py}.

\end{exercise}


\begin{exercise}

Escribe una función llamada \verb"dibujar_rectangulo" que tome un objeto Turtle
y un Rectángulo y utilice al Turtle para dibujar el Rectángulo.  Ver
Capítulo~\ref{turtlechap} para ejemplos que utilizan objetos Turtle.

Escribe una función llamada \verb"dibujar_circulo" que tome un Turtle y
un Círculo y dibuje el Círculo.

Solución: \url{http://thinkpython.com/code/draw.py}.

\end{exercise}



\chapter{Clases y funciones}
\label{time}

Ahora que sabemos cómo crear tipos nuevos, el siguiente
paso es escribir funciones que tomen objetos definidos por el programador
como parámetros y los devuelvan como resultados.  En este capítulo
presento también el ``estilo de programación funcional'' y dos nuevos
planes de desarrollo de programas.

Los ejemplos de código de este capítulo están disponibles en
\url{http://thinkpython.com/code/Time1.py}.
Las soluciones a los ejercicios están en
\url{http://thinkpython.com/code/Time1_soln.py}.


\section{Tiempo}
\label{isafter}

Como otro ejemplo de tipo definido por el programador, definiremos una clase
llamada {\tt Tiempo} que registre la hora del día.  La definición de la clase
se ve así: \index{tipo!definido por el programador}
\index{definido por el programador!tipo} \index{clase!Tiempo} \index{Tiempo, clase}

\begin{verbatim}
class Tiempo:
    """Representa la hora del día.

    atributos: hora, minuto, segundo
    """
\end{verbatim}
%
Podemos crear un nuevo objeto {\tt Tiempo} y asignar
atributos para horas, minutos y segundos:

\begin{verbatim}
tiempo = Tiempo()
tiempo.hora = 11
tiempo.minuto = 59
tiempo.segundo = 30
\end{verbatim}
%
El diagrama de estado para el objeto {\tt Tiempo} se ve como en la Figura~\ref{fig.time}.
\index{diagrama!de estado}
\index{estado, diagrama de}
\index{diagrama!de objeto}
\index{objeto!diagrama}

Como ejercicio, escribe una función llamada \verb"imprimir_tiempo" que tome un
objeto Tiempo y lo imrpima en la forma {\tt hora:minuto:segundo}.
Pista: la secuencia de formato \verb"'%.2d'" imprime un entero utilizando
al menos dos dígitos, incluyendo un cero a la izquierda si es necesario.

Escribe una función booleana llamada \verb"esta_despues" que
tome dos objetos Tiempo, {\tt t1} y {\tt t2}, y
devuelva {\tt True} si {\tt t1} va cronológicamente después de {\tt t2} y
{\tt False} en caso contrario.  Desafío: no utilices una sentencia {\tt if}.

\begin{figure}
\centerline
{\includegraphics[scale=0.8]{figs/time.pdf}}
\caption{Diagrama de objeto.}
\label{fig.time}
\end{figure}


\section{Funciones puras}
\index{prototipo y parche}
\index{plan de desarrollo!prototipo y parche}

En las siguientes secciones, escribiremos dos funciones que suman valores de
tiempo.  Estas demuestran dos tipos de funciones: funciones puras y
modificadores.  Demuestran también un plan de desarrollo que yo llamo {\bf
  prototipo y parche}, que es una manera de abordar un problema complejo
comenzando con un prototipo simple y lidiando con las complicaciones
de manera incremental.

Aquí hay un prototipo simple de \verb"sumar_tiempo":

\begin{verbatim}
def sumar_tiempo(t1, t2):
    suma = Tiempo()
    suma.hora = t1.hora + t2.hora
    suma.minuto = t1.minuto + t2.minuto
    suma.segundo = t1.segundo + t2.segundo
    return suma
\end{verbatim}
%
La función crea un nuevo objeto {\tt Tiempo}, inicializa sus
atributos y devuelve una referencia al nuevo objeto.  Esto se llama
una {\bf función pura} porque no modifica ninguno de los objetos
que se le pasan como argumentos y no tiene un efecto,
como mostrar un valor u obtener entrada del usuario,
además de devolver un valor.
\index{función!pura}
\index{pura, tipo de función}

Para probar esta función, crearé dos objetos Tiempo: {\tt comienzo}
contiene el tiempo de inicio de una película, como {\em Monty Python and the
Holy Grail}, y {\tt duracion} contiene el tiempo de duración de la película,
que es una hora 35 minutos.
\index{Monty Python and the Holy Grail}

\verb"sumar_tiempo" averigua cuándo terminará la película.

\begin{verbatim}
>>> comienzo = Tiempo()
>>> comienzo.hora = 9
>>> comienzo.minuto = 45
>>> comienzo.segundo =  0

>>> duracion = Tiempo()
>>> duracion.hora = 1
>>> duracion.minuto = 35
>>> duracion.segundo = 0

>>> termino = sumar_tiempo(comienzo, duracion)
>>> imprimir_tiempo(termino)
    10:80:00
\end{verbatim}
%
El resultado, {\tt 10:80:00}, tal vez no sea lo que
esperas.  El problema es que esta función no lidia con los casos donde el
número de segundos o minutos suma más de sesenta.  Cuando eso
ocurre, tenemos que ``acarrear'' los segundos extra a la columna de minutos
o los minutos extra a la columna de horas.
\index{acarreo, suma con}

Aquí hay una versión mejorada:

\begin{verbatim}
def sumar_tiempo(t1, t2):
    suma = Tiempo()
    suma.hora = t1.hora + t2.hora
    suma.minuto = t1.minuto + t2.minuto
    suma.segundo = t1.segundo + t2.segundo

    if suma.segundo >= 60:
        suma.segundo -= 60
        suma.minuto += 1

    if suma.minuto >= 60:
        suma.minuto -= 60
        suma.hora += 1

    return suma
\end{verbatim}
%
Aunque esta función es correcta, está comenzando a ponerse grande.
Después veremos una alternativa más corta.


\section{Modificadores}
\label{increment}
\index{modificador}
\index{función!modificadora}

A veces es útil para una función modificar los objetos que obtiene como
parámetros.  En ese caso, los cambios son visibles para la llamadora.
Las funciones que trabajan de esta manera se llaman {\bf modificadores}.
\index{aumentar}

{\tt aumentar}, que agrega un número dado de segundos al objeto {\tt Tiempo},
se puede escribir de manera natural como
modificador.  Aquí hay un borrador:

\begin{verbatim}
def aumentar(tiempo, segundos):
    tiempo.segundo += segundos

    if tiempo.segundo >= 60:
        tiempo.segundo -= 60
        tiempo.minuto += 1

    if tiempo.minuto >= 60:
        tiempo.minuto -= 60
        tiempo.hora += 1
\end{verbatim}
%
La primera línea realiza la operación básica; el resto se ocupa
de los casos especiales que vimos antes.
\index{caso especial}

¿Esta función es correcta?  ¿Qué ocurre si {\tt segundos}
es mucho más grande que sesenta?

En ese caso, no es suficiente acarrear una vez; tenemos que seguir haciéndolo
hasta que {\tt tiempo.segundo} sea menor que sesenta.  Una solución es
reemplazar las sentencias {\tt if} con sentencias {\tt while}.  Eso
haría que la función sea correcta, pero no muy eficiente.  Como
ejercicio, escribe una versión correcta de {\tt aumentar} que no
contenga ningún ciclo.

Cualquier cosa que se pueda hacer con modificadores también se puede hacer con funciones
puras.  De hecho, algunos lenguajes de programación solo permiten funciones
puras.  Existe evidencia de que los programas que utilizan funciones puras
son más rápidos de desarrollar y menos propensos a errores que los programas
que utilizan modificadores.  Sin embargo, los modificadores son convenientes en ocasiones,
y los programas funcionales tienden a ser menos eficientes.

En general, recomiendo que escribas funciones puras siempre que sea
razonable y recurras a los modificadores solo si hay una ventaja
convincente.  Este enfoque podría ser llamado {\bf estilo de
programación funcional}.
\index{estilo de programación funcional}

Como ejercicio, escribe una versión ``pura'' de {\tt aumentar} que
cree y devuelva un nuevo objeto Tiempo en lugar de modificar el
parámetro.


\section{Prototipos versus planificación}
\label{prototype}
\index{prototipo y parche}
\index{plan de desarrollo!prototipo y parche}
\index{desarrollo planificado}
\index{plan de desarrollo!diseñado}

El plan de desarrollo que estoy demostrando se llama ``prototipo y
parche''.  Para cada función, escribí un prototipo que realiza los
cálculos básicos y luego lo probé, parchando errores en el
camino.

Este enfoque puede ser efectivo, especialmente si todavía no tienes
un entendimiento profundo del problema.  Sin embargo, las correcciones incrementales pueden
generar código innecesariamente complicado ---dado que lidia con muchos
casos especiales--- y no confiable ---dado que es difícil saber si
has encontrado todos los errores.

Una alternativa es el {\bf desarrollo diseñado}, en el cual la visión
de alto nivel del problema puede hacer mucho más fácil la programación.  En
este caso, la visión es que un objeto Tiempo es en realidad un número
de tres dígitos en base 60 (ver \url{http://en.wikipedia.org/wiki/Sexagesimal}).  El
atributo {\tt segundo} es la ``columna de unidades'', el atributo {\tt minuto}
es la ``columna de sesentenas'' y el atributo {\tt hora} es
la ``columna de centenas de treinta y seis''.
\index{sexagesimal}

Cuando escribimos \verb"sumar_tiempo" y {\tt aumentar}, efectivamente estábamos
haciendo suma en base 60, motivo por el cual tuvimos que acarrear de una
columna a la siguiente.
\index{acarreo, suma con}

Esta observación sugiere otro enfoque al problema completo: podemos
convertir objetos Tiempo a enteros y tomar ventaja del hecho de
que el computador sabe cómo hacer aritmética con enteros.

Aquí hay una función que convierte Tiempos a enteros:

\begin{verbatim}
def tiempo_a_int(tiempo):
    minutos = tiempo.hora * 60 + tiempo.minuto
    segundos = minutos * 60 + tiempo.segundo
    return segundos
\end{verbatim}
%
Y aquí hay una función que convierte un entero a Tiempo
(recuerda que {\tt divmod} divide el primer argumento por el segundo
y devuelve el cociente y el resto como una tupla).
\index{divmod}

\begin{verbatim}
def int_a_tiempo(segundos):
    tiempo = Tiempo()
    minutos, tiempo.segundo = divmod(segundos, 60)
    tiempo.hora, tiempo.minuto = divmod(minutos, 60)
    return tiempo
\end{verbatim}
%
Quizás tengas que pensar un poco, y ejecutar algunas pruebas, para convencerte
de que estas funciones son correctas.  Una manera de probarlas es
verificar que \verb"tiempo_a_int(int_a_tiempo(x)) == x" para muchos valores de
{\tt x}.  Este es un ejemplo de prueba de consistencia.
\index{prueba!de consistencia}

Una vez que te convenzas de que son correctas, puedes utilizarlas para
reescribir \verb"sumar_tiempo":

\begin{verbatim}
def sumar_tiempo(t1, t2):
    segundos = tiempo_a_int(t1) + tiempo_a_int(t2)
    return int_a_tiempo(segundos)
\end{verbatim}
%
Esta versión es más corta que la original, y más fácil de verificar.  Como
ejercicio, reescribe {\tt aumentar} utilizando \verb"tiempo_a_int" e
\verb"int_a_tiempo".

En algunas formas, convertir de base 60 a base 10 y viceversa es más difícil
que solo tratar con tiempos.  La conversión de base es más abstracta; nuestra
intuición para lidiar con valores de tiempo es mejor.

Sin embargo, si tenemos la visión para tratar tiempos como números en base 60 y nos
dedicamos a escribir las funciones de conversión (\verb"tiempo_a_int"
e \verb"int_a_tiempo"), obtenemos un programa más corto, más fácil de
leer y depurar, y más confiable.

Es más fácil también añadir características más adelante.  Por ejemplo, imagina
restar dos Tiempos para encontrar la duración entre ellos.  El
enfoque ingenuo sería implementar resta con préstamo.
Utilizando funciones de conversión sería más fácil y con más probabilidades de ser
correcto.
\index{resta con préstamo}
\index{prestamo, resta con@préstamo, resta con}
\index{generalización}

Irónicamente, a veces hacer que un programa sea más difícil (o más general) lo hace
más fácil (porque hay menos casos especiales y menos oportunidades
de error).


\section{Depuración}
\index{depuración}

Un objeto Tiempo está bien formado si los valores de {\tt minuto} y {\tt
segundo} están entre 0 y 60 (incluyendo a 0 pero no a 60) y si
{\tt hora} es positivo.  {\tt hora} y {\tt minuto} deberían ser
valores enteros, pero podríamos permitir que {\tt segundo} tenga una
parte de fracción.
\index{invariante}

Requisitos como estos se llaman {\bf invariantes} porque
deberían ser siempre verdaderos.  Visto de manera diferente, si no
son verdaderos es porque algo salió mal.

Escribir código para verificar invariantes puede ayudar a detectar errores
y encontrar sus causas.  Por ejemplo, quizás tengas una función
como \verb"tiempo_valido" que tome un objeto Tiempo y devuelva
{\tt False} si viola un invariante:

\begin{verbatim}
def tiempo_valido(tiempo):
    if tiempo.hora < 0 or tiempo.minuto < 0 or tiempo.segundo < 0:
        return False
    if tiempo.minuto >= 60 or tiempo.segundo >= 60:
        return False
    return True
\end{verbatim}
%
Al principio de cada función podrías verificar los
argumentos para asegurarte de que son válidos:
\index{sentencia!raise}
\index{raise, sentencia}

\begin{verbatim}
def sumar_tiempo(t1, t2):
    if not tiempo_valido(t1) or not tiempo_valido(t2):
        raise ValueError('objeto Tiempo no válido en sumar_tiempo')
    segundos = tiempo_a_int(t1) + tiempo_a_int(t2)
    return int_a_tiempo(segundos)
\end{verbatim}
%
O bien podrías utilizar una {\bf sentencia assert}, que verifica un invariante dado
y levanta una excepción si este falla:
\index{sentencia!assert}
\index{assert, sentencia}

\begin{verbatim}
def sumar_tiempo(t1, t2):
    assert tiempo_valido(t1) and tiempo_valido(t2)
    segundos = tiempo_a_int(t1) + tiempo_a_int(t2)
    return int_a_tiempo(segundos)
\end{verbatim}
%
Las sentencias {\tt assert} son útiles porque distinguen
código que funciona en condiciones normales de código
que verifica errores.


\section{Glosario}

\begin{description}

\item[prototipo y parche:] Un plan de desarrollo que involucra
escribir un borrador de un programa, probarlo y corregir errores a medida que
se encuentran.
\index{prototipo y parche}

\item[desarrollo diseñado:] Un plan de desarrollo que involucra una
visión de alto nivel del problema y más planificación que el desarrollo
incremental o el desarrollo de prototipo.\index{desarrollo diseñado}

\item[función pura:] Una función que no modifica ninguno de los objetos que
recibe como argumentos.  La mayoría de las funciones puras son productivas.
\index{función!pura}

\item[modificador:] Una función que cambia uno o más de los objetos que
  recibe como argumentos.  La mayoría de los modificadores son funciones nulas, es decir,
  devuelven {\tt None}.\index{modificador}

\item[estilo de programación funcional:] Un estilo de diseño de programa en el cual la
mayoría de las funciones son puras.
\index{estilo de programación funcional}

\item[invariante:] Una condición que debería ser siempre verdadera durante la
ejecución de un programa.
\index{invariante}

\item[sentencia assert:] Una sentencia que verifica una condición y levanta
una excepción si esta falla.
\index{sentencia!assert}
\index{assert, sentencia}

\end{description}


\section{Ejercicios}

Los ejemplos de código de este capítulo están disponibles en
\url{http://thinkpython.com/code/Time1.py}; las soluciones a los
ejercicios están disponibles en \url{http://thinkpython.com/code/Time1_soln.py}.

\begin{exercise}

Escribe una función llamada \verb"mul_tiempo" que tome un objeto Tiempo
y un número y devuelva un nuevo objeto Tiempo que contenga
el producto del Tiempo original y el número.

Luego, utiliza \verb"mul_tiempo" para escribir una función que tome un objeto
Tiempo que represente el tiempo de término en una carrera, y un número
que represente la distancia, y devuelva un objeto Tiempo que represente
el ritmo de carrera promedio (tiempo por cada milla).
\index{ritmo de carrera}

\end{exercise}


\begin{exercise}
\index{modulo@módulo!datetime}
\index{datetime, módulo}

El módulo {\tt datetime} proporciona objetos {\tt time}
que son similares a los objetos Tiempo de este capítulo, pero
proporcionan un abundante conjunto de métodos y operadores.  Lee la
documentación en \url{http://docs.python.org/3/library/datetime.html}.

\begin{enumerate}

\item Utiliza el módulo {\tt datetime} para escribir un programa que obtenga la
  fecha actual e imprima el día de la semana.

\item Escribe un programa que tome un día de nacimiento como entrada e imprima la
  edad del usuario y el número de días, horas y segundos que faltan para
  el siguiente cumpleaños.
\index{cumpleaños}

\item Para dos personas nacidas en días diferentes, hay un día en que una
  tiene el doble de edad que la otra.  Ese es su Día Doble.  Escribe un
  programa que tome dos días de nacimiento y calcule su Día Doble.

\item Para ponerlo un poco más desafiante, escribe una versión más general que
  calcule el día en que una persona es $n$ veces mayor que la otra.
\index{Dia Doble@Día Doble}

\end{enumerate}

Solución: \url{http://thinkpython.com/code/double.py}

\end{exercise}


\chapter{Clases y métodos}

Aunque estamos utilizando algunas de las características orientadas a objetos de Python,
los programas de los últimos dos capítulos no están realmente
orientados a objetos debido a que no representan las relaciones
entre los tipos definidos por el programador y las funciones que operan
en estos.  El siguiente paso es transformar esas funciones en
métodos que hagan explícitas las relaciones.

Los ejemplos de código de este capítulo están disponibles en
\url{http://thinkpython.com/code/Time2.py}, y las soluciones
a los ejercicios están en \url{http://thinkpython.com/code/Point2_soln.py}.


\section{Características orientadas a objetos}
\index{programación orientada a objetos}

Python es un {\bf lenguaje de programación orientado a objetos}, lo cual significa
que proporciona características que admiten programación
orientada a objetos, que tiene las siguientes características que la definen:

\begin{itemize}

\item Los programas incluyen definiciones de clases y métodos.

\item La mayoría de los cálculos están expresados en términos de operaciones sobre
  obtetos.

\item Los objetos a menudo representan cosas
del mundo real, y los métodos a menudo
corresponden a las maneras en las cuales interactúan las cosas del mundo real.

\end{itemize}

Por ejemplo, la clase {\tt Tiempo} definida en el Capítulo~\ref{time}
corresponde a la manera en que las personas registran la hora del día, y las
funciones que definimos corresponden a los tipos de cosas que las personas hacen con
las horas del día.  De manera similar, las clases {\tt Punto} y {\tt Rectángulo}
en el Capítulo~\ref{clobjects}
corresponden a los conceptos matemáticos de punto y rectángulo.

Hasta aquí, no hemos tomado ventaja de las características que Python proporciona para
admitir programación orientada a objetos.  Estas características
no son estrictamente necesarias: la mayoría de estas proporciona
una sintaxis alternativa para cosas que ya hemos hecho.  Sin embargo, en muchos casos,
la alternativa es más concisa y comunica la estructura del programa
de manera más precisa.

Por ejemplo, en {\tt Time1.py} no hay conexión
obvia entre la definición de clase y las definiciones de funciones
que siguen.  Con algo de revisión, es evidente que cada función
toma al menos un objeto {\tt Tiempo} como argumento.
\index{metodo@método}
\index{función}

Esta observación es la motivación para los {\bf métodos}; un método es
una función que está asociada a una clase en particular.
Hemos visto métodos para cadenas, listas, diccionarios y tuplas.
En este capítulo, definiremos métodos para tipos definidos por el programador.
\index{sintaxis}
\index{semántica}
\index{tipo!definido por el programador}
\index{definido por el programador!tipo}

Los métodos son semánticamente lo mismo que las funciones, pero hay
dos diferencias sintácticas:

\begin{itemize}

\item Los métodos se definen dentro de una definición de clase para
hacer explícita la relación entre la clase y el método.

\item La sintaxis para invocar a un método es diferente a la
sintaxis para llamar a una función.

\end{itemize}

En las siguientes secciones, tomaremos las funciones de los dos
capítulos anteriores y los transformaremos en métodos.  Esta transformación es
puramente mecánica: puedes hacerla siguiendo una secuencia de
pasos.  Si te acostumbras a convertir de una forma a la otra,
podrás escoger la mejor forma para lo que sea que estés haciendo.


\section{Imprimir objetos}
\index{objeto, imprimir}

En el Capítulo~\ref{time}, definimos una clase con nombre
{\tt Tiempo} y en la Sección~\ref{isafter}, escribiste
una función con nombre \verb"imprimir_tiempo":

\begin{verbatim}
class Tiempo:
    """Representa la hora del día."""

def imprimir_tiempo(tiempo):
    print('%.2d:%.2d:%.2d' % (tiempo.hora, tiempo.minuto, tiempo.segundo))
\end{verbatim}
%
Para llamar a esta función, tienes que pasar el objeto {\tt Tiempo} como
argumento:

\begin{verbatim}
>>> comienzo = Tiempo()
>>> comienzo.hora = 9
>>> comienzo.minuto = 45
>>> comienzo.segundo = 00
>>> imprimir_tiempo(comienzo)
    09:45:00
\end{verbatim}
%
Para convertir \verb"imprimir_tiempo" a método, todo lo que tenemos que hacer es
mover la definición de función al interior de la definición de clase.  Nota
el cambio en la sangría.
\index{sangría}

\begin{verbatim}
class Tiempo:
    def imprimir_tiempo(tiempo):
        print('%.2d:%.2d:%.2d' % (tiempo.hora, tiempo.minuto, tiempo.segundo))
\end{verbatim}
%
Ahora hay dos maneras de llamar a \verb"imprimir_tiempo".  La primera manera
(y menos común) es utilizar sintaxis de función:
\index{sintaxis!de función}
\index{notación de punto}

\begin{verbatim}
>>> Tiempo.imprimir_tiempo(comienzo)
    09:45:00
\end{verbatim}
%
En este uso de notación de punto, {\tt Tiempo} es el nombre de la clase
e \verb"imprimir_tiempo" es el nombre del método.  {\tt comienzo} se
pasa como parámetro.

La segunda manera (y más concisa) es utilizar la sintaxis de método:
\index{sintaxis!de método}

\begin{verbatim}
>>> comienzo.imprimir_tiempo()
    09:45:00
\end{verbatim}
%
En este uso de notación de punto, \verb"imprimir_tiempo" es el nombre del
método (nuevamente) y {\tt comienzo} es el objeto en el cual se invoca
al método, el cual se llama {\bf sujeto}.  Tal como el
sujeto de una oración es de lo que trata la oración, el sujeto
de una invocación a método es de lo que trata el método.
\index{sujeto}

Dentro del método, el sujeto se asigna al primer
parámetro, por tanto en este caso {\tt comienzo} se asigna
a {\tt tiempo}.
\index{self (nombre de parámetro)}
\index{parámetro!self}

Por convención, el primer parámetro de un método se
llama {\tt self}, por lo cual sería más común escribir
\verb"imprimir_tiempo" así:

\begin{verbatim}
class Tiempo:
    def imprimir_tiempo(self):
        print('%.2d:%.2d:%.2d' % (self.hora, self.minuto, self.segundo))
\end{verbatim}
%
La razón para esta convención es una metáfora implícita:
\index{metafora, invocacion a metodo@metáfora, invocación a método}

\begin{itemize}

\item La sintaxis para una llamada a función, \verb"imprimir_tiempo(comienzo)",
  sugiere que la función es el agente activo.  Dice algo
  como, ``¡Oye \verb"imprimir_tiempo"!  Aquí hay un objeto para que imprimas.''

\item En la programación orientada a objetos, los objetos son los agentes
  activos.  Una invocación a método como \verb"comienzo.imprimir_tiempo()" dice
  ``¡Oye {\tt comienzo}!  Por favor imprímete.''

\end{itemize}

Este cambio de perspectiva quizás sea más cortés, pero su utilidad
no es obvia.  En los ejemplos que hemos visto hasta ahora, podría no
serlo.  Sin embargo, a veces cambiar de funciones a objetos de manera
responsable hace posible escribir funciones (o métodos) más versátiles,
y hace más fácil mantener y reutilizar código.

Como ejercicio, reescribe \verb"tiempo_a_int" (de la
Sección~\ref{prototype}) como método.  Quizás te tientes a
reescribir \verb"int_a_tiempo" como método también, pero eso realmente
no tiene sentido porque no habría objeto donde
invocarlo.


\section{Otro ejemplo}
\index{aumentar}

Aquí hay una versión de {\tt aumentar} (de la Sección~\ref{increment})
reescrita como método:

\begin{verbatim}
# dentro de class Tiempo:

    def aumentar(self, segundos):
        segundos += self.tiempo_a_int()
        return int_a_tiempo(segundos)
\end{verbatim}
%
Esta versión supone que \verb"tiempo_a_int" está escrito
como método.  Además, notar que
esta es una función pura, no un modificador.

Así es como invocarías a {\tt aumentar}:

\begin{verbatim}
>>> comienzo.imprimir_tiempo()
    09:45:00
>>> termino = comienzo.aumentar(1337)
>>> termino.imprimir_tiempo()
    10:07:17
\end{verbatim}
%
El sujeto, {\tt comienzo}, se asigna al primer parámetro,
{\tt self}.  El argumento, {\tt 1337}, se asigna al
segundo parámetro, {\tt segundos}.

Este mecanismo puede ser confuso, especialmente si cometes un error.
Por ejemplo, si invocas a {\tt aumentar} con dos argumentos,
obtienes:
\index{excepción!TypeError}
\index{TypeError}

\begin{verbatim}
>>> termino = comienzo.aumentar(1337, 460)
    TypeError: aumentar() takes 2 positional arguments but 3 were given
\end{verbatim}
%
El mensaje de error es confuso al inicio, debido a que hay
solo dos argumentos en paréntesis.  Sin embargo, el sujeto también
se considera como argumento, por lo cual todos juntos son tres.

Por cierto, un {\bf argumento posicional} es un argumento que
no tiene un nombre de parámetro, es decir, no es un argumento de
palabra clave.  En esta llamada a función:
\index{argumento posicional}
\index{posicional, argumento}

\begin{verbatim}
dibujar(loro, jaula, muerto=True)
\end{verbatim}

{\tt loro} y {\tt jaula} son posicionales, y {\tt muerto} es
un argumento de palabra clave.


\section{Un ejemplo más complicado}

Reescribir \verb"esta_despues" (de la Sección~\ref{isafter}) es un poco
más complicado porque toma dos objetos Tiempo como parámetros.  En
este caso, lo convencional es ponerle el nombre {\tt self} al primer parámetro
y {\tt other} al segundo parámetro: \index{other (nombre de parámetro)}
\index{parámetro!other}

\begin{verbatim}
# dentro de class Tiempo:

    def esta_despues(self, other):
        return self.tiempo_a_int() > other.tiempo_a_int()
\end{verbatim}
%
Para utilizar este método, tienes que invocarlo en un objeto y pasar
el otro como argumento:

\begin{verbatim}
>>> termino.esta_despues(comienzo)
    True
\end{verbatim}
%
Lo bonito de esta sintaxis es que se lee
casi textual: ``¿término está después de comienzo?''


\section{El método init}
\index{metodo@método!init}
\index{init, método}

El método init (abreviatura de ``initialization'') es
un método especial que se invoca cuando se instancia un objeto.
Su nombre completo es \verb"__init__" (dos caracteres de guión bajo,
seguido de {\tt init}, y luego dos guiones bajos más).  Un
método init para la clase {\tt Tiempo} se vería así:

\begin{verbatim}
# dentro de class Tiempo:

    def __init__(self, hora=0, minuto=0, segundo=0):
        self.hora = hora
        self.minuto = minuto
        self.segundo = segundo
\end{verbatim}
%
Es común que los parámetros de \verb"__init__"
tengan los mismos nombres que los atributos.  La sentencia

\begin{verbatim}
        self.hora = hora
\end{verbatim}
%
almacena el valor del parámetro {\tt hora} como un atributo
de {\tt self}.
\index{parámetro!opcional}
\index{opcional!parámetro}
\index{valor por defecto}
\index{anular}

Los parámetros son opcionales, por lo cual si llamas a {\tt Tiempo} sin
argumentos, obtienes los valores por defecto.

\begin{verbatim}
>>> tiempo = Tiempo()
>>> tiempo.imprimir_tiempo()
    00:00:00
\end{verbatim}
%
Si entregas un argumento, anula a {\tt hora}:

\begin{verbatim}
>>> tiempo = Tiempo(9)
>>> tiempo.imprimir_tiempo()
    09:00:00
\end{verbatim}
%
Si entregas dos argumentos, anulan a {\tt hora} y
{\tt minuto}.

\begin{verbatim}
>>> tiempo = Tiempo(9, 45)
>>> tiempo.imprimir_tiempo()
    09:45:00
\end{verbatim}
%
Y si entregas tres argumentos, anulan a los tres
valores por defecto.

Como ejercicio, escribe un método init para la clase {\tt Punto} que tome a
{\tt x} e {\tt y} como argumentos opcionales y los asigne
a los atributos correspondientes.
\index{clase!Punto}
\index{Punto!clase}


\section{El método {\tt \_\_str\_\_}}
\index{metodo@método!str@método \_\_str\_\_}
\index{\_\_str\_\_, método}

\verb"__str__" es un método especial, al igual que \verb"__init__",
que se supone que devuelve una representación de un objeto en forma de cadena.
\index{representación de cadena}

Por ejemplo, aquí hay un método {\tt str} para los objetos Tiempo:

\begin{verbatim}
# dentro de class Tiempo:

    def __str__(self):
        return '%.2d:%.2d:%.2d' % (self.hora, self.minuto, self.segundo)
\end{verbatim}
%
Cuando utilizas {\tt print} en un objeto, Python invoca al método {\tt str}:
\index{sentencia!print}
\index{print, sentencia}

\begin{verbatim}
>>> tiempo = Tiempo(9, 45)
>>> print(tiempo)
    09:45:00
\end{verbatim}
%
Cuando escribo una clase nueva, casi siempre comienzo escribiendo a
\verb"__init__", que hace más fácil instanciar objetos, y
\verb"__str__", que es útil para depurar.

Como ejercicio, escribe un método {\tt str} para la clase {\tt Punto}.
Crea un objeto Punto e imprímelo.


\section{Sobrecarga de operador}
\label{operator.overloading}

Definiendo otros métodos especiales, puedes especificar el comportamiento
de operadores en tipos definidos por el programador.  Por ejemplo, si defines
un método con nombre \verb"__add__" para la clase {\tt Tiempo}, puedes utilizar el
operador {\tt +} en objetos Tiempo.
\index{tipo!definido por el programador}
\index{definido por el programador!tipo}

Así es como podría verse la definición:
\index{metodo@método!add}
\index{add, método}

\begin{verbatim}
# dentro de class Tiempo:

    def __add__(self, other):
        segundos = self.tiempo_a_int() + other.tiempo_a_int()
        return int_a_tiempo(segundos)
\end{verbatim}
%
Y así es como podrías utilizarlo:

\begin{verbatim}
>>> comienzo = Tiempo(9, 45)
>>> duracion = Tiempo(1, 35)
>>> print(comienzo + duracion)
    11:20:00
\end{verbatim}
%
Cuando aplicas el operador {\tt +} a objetos Tiempo, Python invoca a
\verb"__add__".  Cuando imprimes el resultado, Python invoca a
\verb"__str__".  ¡Entonces están ocurriendo muchas cosas entre bastidores!
\index{sobrecarga de operador}

Cambiar el comportamiento de un operador para que funcione con
tipos definidos por el programador se llama {\bf sobrecarga de operador}.  Para cada
operador en Python hay un método especial, como
\verb"__add__", que le corresponde.  Para más detalles, ver
\url{http://docs.python.org/3/reference/datamodel.html#specialnames}.

Como ejercicio, escribe un método {\tt add} para la clase Punto.


\section{Despacho basado en tipo}

En la sección anterior sumamos dos objetos Tiempo, pero quizás
quieras también sumar un entero a un objeto Tiempo.  Lo
siguiente es una versión de \verb"__add__"
que verifica el tipo de {\tt other} e invoca a
\verb"sumar_tiempo" o {\tt aumentar}:

\begin{verbatim}
# dentro de class Tiempo:

    def __add__(self, other):
        if isinstance(other, Tiempo):
            return self.sumar_tiempo(other)
        else:
            return self.aumentar(other)

    def sumar_tiempo(self, other):
        segundos = self.tiempo_a_int() + other.tiempo_a_int()
        return int_a_tiempo(segundos)

    def aumentar(self, segundos):
        segundos += self.tiempo_a_int()
        return int_a_tiempo(segundos)
\end{verbatim}
%
La función incorporada {\tt isinstance} toma un valor y un
objeto de clase, y devuelve {\tt True} si el valor es una instancia
de la clase.
\index{función!isinstance}
\index{isinstance, función}

Si {\tt other} es un objeto Tiempo, \verb"__add__" invoca a
\verb"sumar_tiempo".  En caso contrario, supone que el parámetro
es un número e invoca a {\tt aumentar}.  Esta operación se
llama {\bf despacho basado en tipo} porque despacha el
cálculo a diferentes métodos basándose en el tipo de los
argumentos.
\index{despacho basado en tipo}
\index{tipo!despacho basado en}

Aquí hay ejemplos que utilizan el operador {\tt +} con tipos
diferentes:

\begin{verbatim}
>>> comienzo = Tiempo(9, 45)
>>> duracion = Tiempo(1, 35)
>>> print(comienzo + duracion)
    11:20:00
>>> print(comienzo + 1337)
    10:07:17
\end{verbatim}
%
Desafortunadamente, esta implementación de la suma no es conmutativa.
Si el entero es el primer operando, obtienes
\index{conmutatividad}

\begin{verbatim}
>>> print(1337 + comienzo)
    TypeError: unsupported operand type(s) for +: 'int' and 'instance'
\end{verbatim}
%
El problema es que, en lugar de pedirle al objeto Tiempo que sume un entero,
Python le está pidiendo a un entero que sume un objeto Tiempo, y no sabe
cómo hacerlo.  Sin embargo, hay una solución ingeniosa para este problema: el
método especial \verb"__radd__", que significa ``right-side add''.
Este método se invoca cuando un objeto Tiempo aparece en el lado derecho del
operador {\tt +}.  Esta es la definición:
\index{metodo@método!radd}
\index{radd, método}

\begin{verbatim}
# dentro de class Tiempo:

    def __radd__(self, other):
        return self.__add__(other)
\end{verbatim}
%
Y así es como se utiliza:

\begin{verbatim}
>>> print(1337 + comienzo)
    10:07:17
\end{verbatim}
%

Como ejercicio, escribe un método {\tt add} para Puntos que funcione tanto con
un objeto Punto como con una tupla:

\begin{itemize}

\item Si el segundo operando es un Punto, el método debería devolver un
Punto nuevo cuya coordenada $x$ es la suma de las coordenadas $x$ de los
operandos, y lo mismo para las coordenadas $y$.

\item Si el segundo operando es una tupla, el método debería sumar el
primer elemento de la tupla a la coordenada $x$ y el segundo
elemento a la coordenada $y$, y devolver un Punto nuevo con el resultado.

\end{itemize}




\section{Polimorfismo}
\label{polymorphism}

El despacho basado en tipo es útil si es necesario, pero (afortunadamente)
no siempre es necesario.  A menudo puedes evitarlo escribiendo funciones
que trabajen de manera correcta con argumentos de tipo diferente.
\index{despacho basado en tipo}
\index{tipo!despacho basado en}

Muchas de las funciones que escribimos para cadenas también
funcionan para otros tipos de secuencia.
Por ejemplo, en la Sección~\ref{histogram}
utilizamos {\tt histograma} para contar el número de veces que aparece cada letra
en una palabra.

\begin{verbatim}
def histograma(s):
    d = dict()
    for c in s:
        if c not in d:
            d[c] = 1
        else:
            d[c] = d[c] + 1
    return d
\end{verbatim}
%
Esta función también se puede utilizar con listas, tuplas e incluso diccionarios,
siempre que los elementos de {\tt s} sean hashables, por lo cual se pueden utilizar
como claves en {\tt d}.

\begin{verbatim}
>>> t = ['spam', 'egg', 'spam', 'spam', 'bacon', 'spam']
>>> histograma(t)
    {'bacon': 1, 'egg': 1, 'spam': 4}
\end{verbatim}
%
Las funciones que se pueden utilizar con varios tipos se llaman {\bf polimórficas}.
Los polimorfismos pueden facilitar la reutilización de código.  Por ejemplo, la función
incorporada {\tt sum}, que suma los elementos de una secuencia, funciona
siempre que los elementos de la secuencia admitan la suma.
\index{polimorfismo}

Dado que los objetos Tiempo proporcionan un método {\tt add}, funcionan
con {\tt sum}:

\begin{verbatim}
>>> t1 = Tiempo(7, 43)
>>> t2 = Tiempo(7, 41)
>>> t3 = Tiempo(7, 37)
>>> total = sum([t1, t2, t3])
>>> print(total)
    23:01:00
\end{verbatim}
%
En general, si todas las operaciones dentro de una función
se pueden utilizar con un tipo dado, la función se puede utilizar con ese tipo.

Los mejores polimorfismos son los involuntarios, donde
descubres que una función que ya escribiste se puede
aplicar a un tipo para el cual nunca planeaste.


\section{Depuración}
\index{depuración}

Es legal añadir atributos a los objetos en cualquier punto de la ejecución
de un programa, pero si tienes objetos con el mismo tipo que no
tienen los mismos atributos, es fácil cometer errores.
Se considera una buena idea
inicializar todo lo de un atributo de un objeto en el método init.
\index{metodo@método!init}
\index{inicializar atributo}

Si no sabes bien si un objeto tiene un atributo en particular,
puedes utilizar la función incorporada {\tt hasattr} (ver Sección~\ref{hasattr}).
\index{función!hasattr}
\index{hasattr, función}
\index{atributo dict@atributo \_\_dict\_\_}
\index{\_\_dict\_\_, atributo}

Otra manera de acceder a los atributos es la función incorporada {\tt vars},
que toma un objeto y devuelve un diccionario que mapea de los nombres
de atributos (como cadenas) a sus valores:

\begin{verbatim}
>>> p = Punto(3, 4)
>>> vars(p)
    {'y': 4, 'x': 3}
\end{verbatim}
%
Para fines de depuración, quizás encuentres útil tener esta
función a mano:

\begin{verbatim}
def imprimir_atributos(obj):
    for attr in vars(obj):
        print(attr, getattr(obj, attr))
\end{verbatim}
%
\verb"imprimir_atributos" recorre el diccionario
e imprime cada nombre de atributo y su valor correspondiente.
\index{recorrer!diccionario}
\index{diccionario!recorrer}

La función incorporada {\tt getattr} toma un objeto y un nombre
de atributo (como cadena) y devuelve el valor del atributo.
\index{función!getattr}
\index{getattr, función}


\section{Interfaz e implementación}

Uno de los objetivos del diseño orientado a objetos es hacer que el software sea más
mantenible, lo cual significa que puedes mantener el programa funcionando cuando
otras partes del sistema cambian, y modificar el programa para cumplir con nuevos
requisitos.
\index{interfaz}
\index{implementación}
\index{mantenible}
\index{diseño orientado a objetos}

Un principio de diseño que ayuda a alcanzar ese objetivo es mantener
las interfaces separadas de las implementaciones.  Para objetos, eso significa
que los métodos que proporciona una clase no deberían depender de la manera en que los
atributos estén representados.
\index{atributo}

Por ejemplo, en este capítulo desarrollamos una clase que representa
una hora del día.  Los métodos proporcionados por esta clase incluyen a
\verb"tiempo_a_int", \verb"esta_despues" y \verb"sumar_tiempo".

Podríamos implementar esos métodos de muchas maneras.  Los detalles de la
implementación dependen de cómo representemos al tiempo.  En este capítulo, los
atributos de un objeto {\tt Tiempo} son {\tt hora}, {\tt minuto} y
{\tt segundo}.

Como alternativa, podíamos reemplazar estos atributos con
un único entero que represente al número de segundos
desde la medianoche.  Esta implementación haría que algunos métodos,
como \verb"esta_despues", sean más fáciles de escribir, pero haría que otros métodos sean
más difíciles.

Después de que implementes una clase nueva, quizás descubras una mejor
implementación.  Si otras partes del programa están utilizando tu
clase, cambiar la interfaz podría consumir tiempo y ser propenso a
errores.

Sin embargo, si diseñaste la interfaz con cuidado, puedes
cambiar la implementación sin cambiar la interfaz, lo cual
significa que otras partes del programa no tienen que cambiar.


\section{Glosario}

\begin{description}

\item[lenguaje orientado a objetos:] Un lenguaje que proporciona características,
  tales como métodos y tipos definidos por el programador, que facilitan
  la programación orientada a objetos.
\index{lenguaje!orientado a objetos}

\item[programación orientada a objetos:] Un estilo de programación en el cual
los datos y las operaciones que los manipulan se organizan en clases
y métodos.
\index{programación orientada a objetos}

\item[método:] Una función que se define dentro de una definición de clase y
se invoca en instancias de esa clase.
\index{metodo@método}

\item[sujeto:] El objeto en el cual se invoca a un método.
\index{sujeto}

\item[argumento posicional:]  Un argumento que no incluye
un nombre de parámetro, por tanto no es un argumento de palabra clave.
\index{argumento posicional}
\index{posicional, argumento}

\item[sobrecarga de operador:] Cambiar el comportamiento de un operador como
{\tt +} para que funcione con un tipo definido por el programador.
\index{sobrecarga de operador}
\index{operador!sobrecarga de}

\item[despacho basado en tipo:] Un patrón de programación que verifica el tipo
de un operando e invoca a funciones diferentes para tipos diferentes.
\index{despacho basado en tipo} \index{tipo!despacho basado en}

\item[polimórfico:] Dicho de una función que puede trabajar con más de
  un tipo.
\index{polimorfismo}

\end{description}


\section{Ejercicios}

\begin{exercise}

Descarga el código de este capítulo en
\url{http://thinkpython.com/code/Time2.py}.  Cambia los atributos de
    {\tt Time} para que sea un único entero que represente los segundos desde
    la media noche.  Luego, modifica los métodos (y la función
    \verb"int_to_time") para que funcionen con la nueva implementación.  No
    deberías tener que modificar el código de prueba en {\tt main}.  Cuando
    termines, la salida debería ser la misma que antes.  Solución:
    \url{http://thinkpython.com/code/Time2_soln.py}.

\end{exercise}


\begin{exercise}
\label{kangaroo}
\index{valor por defecto!evitar mutable}
\index{objeto mutable como valor por defecto}
\index{el peor error de programación}
\index{error!de programación, el peor}
\index{clase!Kangaroo}
\index{Kangaroo, clase}

Este ejercicio es un cuento con moraleja acerca de uno de los errores
en Python más comunes y más difíciles de encontrar.
Escribe una definición de una clase con nombre {\tt Kangaroo} con los siguientes
métodos:

\begin{enumerate}

\item Un método \verb"__init__" que inicialice un atributo con nombre
\verb"pouch_contents" ({\em contenido de la bolsa}) a una lista vacía.

\item Un método con nombre \verb"put_in_pouch" ({\em poner en la bolsa}) que tome un objeto
de cualquier tipo y lo añada a \verb"pouch_contents".

\item Un método \verb"__str__" que devuelva una representación de cadena
del objeto Kangaroo y los contenidos de la bolsa.

\end{enumerate}
%
Prueba tu código
creando dos objetos {\tt Kangaroo}, asignándolos a variables
con nombres {\tt kanga} y {\tt roo}, y luego añadiendo {\tt roo} al
contenido de la bolsa de {\tt kanga}.

Descarga \url{http://thinkpython.com/code/BadKangaroo.py}.  Contiene
una solución al problema anterior con un gran y horrible error.
Encuentra y corrije el error.

Si te estancas, puedes descargar
\url{http://thinkpython.com/code/GoodKangaroo.py}, que explica el
problema y demuestra una solución.
\index{alias}
\index{objeto!incrustado}
\index{incrustado, objeto}

\end{exercise}



\chapter{Herencia}

La característica de un lenguaje que se asocia con más frecuencia a la programación
orientada a objetos es la {\bf herencia}.  La herencia es la posibilidad de
definir una clase nueva que sea una versión modificada de una clase existente.
En este capítulo demuestro la herencia utilizando clases que representan
cartas de juego, barajas de cartas y manos de póker.
\index{baraja}
\index{carta de juego}
\index{poker@póker}

Si no juegas
póker, puedes leer sobre este en
\url{http://en.wikipedia.org/wiki/Poker}, pero no tienes que hacerlo; 
te diré lo que necesitas saber para los ejercicios.

Los ejemplos de código de
este capítulo están disponibles en
\url{http://thinkpython.com/code/Card.py}.


\section{Objetos Carta}

Hay cincuenta y dos cartas en una baraja, cada una perteneciente a uno de
los cuatro palos y uno de los trece rangos.  Los palos son Picas, Corazones,
Diamantes y Tréboles (en orden descendiente en el bridge).  Los rangos son
As, 2, 3, 4, 5, 6, 7, 8, 9, 10, Jota, Reina y Rey.  Dependiendo del
juego que estés jugando, el valor de un As puede ser mayor al de un Rey
o menor que 2.
\index{rango}
\index{palo}

Si queremos definir un objeto nuevo que represente una carta, los
atributos que debería tener son obvios: {\tt rango} y
{\tt palo}.  Lo que no es obvio es qué tipo de atributos
deberían ser.  Una posibilidad es utilizar cadenas que contengan palabras como
\verb"'Picas'" para palos y \verb"'Reina'" para rangos.  Un problema que
tiene esta implementación es que no sería fácil comparar cartas para
ver cuál tiene un rango o palo con más valor.
\index{codificar}
\index{cifrar}
\index{mapear}
\index{representación}

Una alternativa es utilizar enteros para {\bf codificar} los rangos y palos.
En este contexto, ``codificar'' significa que vamos a definir un mapeo
entre números y palos, o entre números y rangos.  Este
tipo de codificación no está destinado a ser secreto (eso
sería ``cifrado'').

\newcommand{\mymapsto}{$\mapsto$}

Por ejemplo, esta tabla muestra los palos y sus correspondientes
códigos enteros:

\begin{tabular}{l c l}
Picas & \mymapsto & 3 \\
Corazones & \mymapsto & 2 \\
Diamantes & \mymapsto & 1 \\
Tréboles & \mymapsto & 0
\end{tabular}

Este código facilita la comparación entre cartas: dado que los palos con mayor valor mapean a
números mayores, podemos comparar palos comparando sus códigos.

El mapeo para los rangos es bastante obvio: cada uno de los rangos de número
mapean al entero correspondiente, y para cartas de figura:

\begin{tabular}{l c l}
Jota & \mymapsto & 11 \\
Reina & \mymapsto & 12 \\
Rey & \mymapsto & 13 \\
\end{tabular}

Utilizo el símbolo \mymapsto para que quede claro que estos mapeos
no son parte del programa en Python.  Son parte del diseño del
programa, pero no aparecen explícitamente en el código.
\index{clase!Carta}
\index{Carta, clase}

La definición de clase para {\tt Carta} se ve así:

\begin{verbatim}
class Carta:
    """Representa una carta de juego estándar."""

    def __init__(self, palo=0, rango=2):
        self.palo = palo
        self.rango = rango
\end{verbatim}
%
Como siempre, el método init toma un parámetro
opcional para cada atributo.  La carta por defecto es
el 2 de Tréboles.
\index{init, método}
\index{metodo@método!init}

Para crear una Carta, llamas a {\tt Carta} con el
palo y el rango de la carta que quieres.

\begin{verbatim}
reina_de_diamantes = Carta(1, 12)
\end{verbatim}
%


\section{Atributos de clase}
\label{class.attribute}
\index{atributo de clase}
\index{clase!atributo de}

Para imprimir objetos Carta de una manera que las personas puedan leer
fácilmente, necesitamos un mapeo de códigos enteros a los rangos y palos
correspondientes.  Una manera natural de
hacer eso es con listas de cadenas.  Podemos asignar estas listas a {\bf atributos
de clase}:

\begin{verbatim}
# dentro de class Carta:

    nombres_de_palo = ['Tréboles', 'Diamantes', 'Corazones', 'Picas']
    nombres_de_rango = [None, 'As', '2', '3', '4', '5', '6', '7',
              '8', '9', '10', 'Jota', 'Reina', 'Rey']

    def __str__(self):
        return '%s de %s' % (Carta.nombres_de_rango[self.rango],
                             Carta.nombres_de_palo[self.palo])
\end{verbatim}
%
Variables como \verb"nombres_de_palo" y \verb"nombres_de_rango", que se
definen dentro de una clase pero fuera de cualquier método, se llaman
atributos de clase porque están asociados al objeto de clase
{\tt Carta}.
\index{atributo de instancia}
\index{instancia!atributo de}

Este término los distingue de variables como {\tt palo} y {\tt
  rango}, que se llaman {\bf atributos de instancia} porque están
asociados a una instancia particular.
\index{notación de punto}

Para ambos tipos de atributo, se accede utilizando notación de punto.  Por
ejemplo, en \verb"__str__", {\tt self} es un objeto Carta
y {\tt self.rango} es su rango.  De manera similar, {\tt Carta}
es un objeto de clase y \verb"Carta.nombres_de_rango" es una
lista de cadenas asociadas a la clase.

Cada carta tiene su propio {\tt palo} y {\tt rango}, pero hay
solo una copia de \verb"nombres_de_palo" y \verb"nombres_de_rango".

Poniendo todos los elementos juntos, la expresión
\verb"Carta.nombres_de_rango[self.rango]" significa ``utiliza el atributo {\tt rango}
del objeto {\tt self} como índice dentro de la lista \verb"nombres_de_rango"
de la clase {\tt Carta}, y selecciona la cadena correspondiente.''

El primer elemento de \verb"nombres_de_rango" es {\tt None} porque no
hay carta con rango cero.  Incluyendo a {\tt None} como guardián de lugar,
obtenemos un mapeo con la genial propiedad de que el índice 2 mapea a la
cadena \verb"'2'", y así sucesivamente.  Para evitar este ajuste, podríamos haber
utilizado un diccionario en lugar de una lista.

Con los métodos que hemos visto hasta ahora, podemos crear e imprimir cartas:

\begin{verbatim}
>>> carta1 = Carta(2, 11)
>>> print(carta1)
    Jota de Corazones
\end{verbatim}

\begin{figure}
\centerline
{\includegraphics[scale=0.8]{figs/card1.pdf}}
\caption{Diagrama de objeto.}
\label{fig.card1}
\end{figure}

La Figura~\ref{fig.card1} es un diagrama del objeto de clase {\tt Carta} y
una instancia de Carta.  {\tt Carta} es un objeto de clase; su tipo es {\tt
  type}.  {\tt carta1} es una instancia de {\tt Carta}, por lo cual su tipo es
{\tt Carta}.  Para ahorrar espacio, no dibujé los contenidos de
\verb"nombres_de_palo" y \verb"nombres_de_rango".  \index{diagrama!de estado}
\index{estado, diagrama de} \index{diagrama!de objeto} \index{objeto!diagrama de}


\section{Comparar cartas}
\label{comparecard}
\index{relacional, operador}
\index{operador!relacional}

Para tipos incorporados, hay operadores relacionales
({\tt <}, {\tt >}, {\tt ==}, etc.)
que comparan
valores y determinan si uno es mayor, menor o igual al
otro.  Para tipos definidos por el programador, podemos anular el comportamiento de
los operadores incorporados con un método cuyo nombre es
\verb"__lt__", que significa ``menor que'' ({\em less than}).
\index{tipo!definido por el programador}
\index{definido por el programador!tipo}

\verb"__lt__" toma dos parámetros, {\tt self} y {\tt other},
y devuelve {\tt True} si {\tt self} es estrictamente menor que {\tt other}.
\index{anular}
\index{sobrecarga de operador}

El orden correcto para las cartas no es obvio.
Por ejemplo, ¿cuál
es mejor, el 3 de Tréboles o el 2 de Diamantes?  Uno tiene un rango
mayor, pero el otro tiene un palo con más valor.  Para comparar
cartas, tienes que decidir si el rango o el palo es más importante.

La respuesta podría depender del juego que estás jugando, pero para mantener
las cosas simples, haremos la elección arbitraria de que el palo es más
importante, así que todas las Picas superan a todos los Diamantes,
y así sucesivamente.
\index{metodo@método!\_\_cmp\_\_}
\index{\_\_cmp\_\_, método}

Con eso decidido, podemos escribir \verb"__lt__":

\begin{verbatim}
# dentro de class Carta:

    def __lt__(self, other):
        # verifica los palos
        if self.palo < other.palo: return True
        if self.palo > other.palo: return False

        # los palos son iguales... verifica los rangos
        return self.rango < other.rango
\end{verbatim}
%
Puedes escribir esto de manera más concisa utilizando comparación de tuplas:
\index{tupla!comparación de}
\index{comparación de tupla}

\begin{verbatim}
# dentro de class Carta:

    def __lt__(self, other):
        t1 = self.palo, self.rango
        t2 = other.palo, other.rango
        return t1 < t2
\end{verbatim}
%
Como ejercicio, escribe un método \verb"__lt__" para objetos Tiempo.  Puedes
utilizar comparación de tuplas, pero quizás también consideres
comparar enteros.


\section{Barajas}
\index{lista!de objetos}
\index{baraja de cartas}

Ahora que tenemos Cartas, el siguiente paso es definir Barajas.  Dado que una
baraja está compuesta de cartas, es natural que cada Baraja contenga una
lista de cartas como atributo.

\index{metodo@método!init}
\index{init, método}

Lo siguiente es una definición de clase para {\tt Baraja}.  El
método init crea el atributo {\tt cartas} y genera
el conjunto estándar de cincuenta y dos cartas:
\index{composición}
\index{bucle!anidado}
\index{clase!Baraja}
\index{Baraja, clase}

\begin{verbatim}
class Baraja:

    def __init__(self):
        self.cartas = []
        for palo in range(4):
            for rango in range(1, 14):
                carta = Carta(palo, rango)
                self.cartas.append(carta)
\end{verbatim}
%
La manera más fácil de llenar la baraja es con un bucle anidado.  El bucle externo
enumera los palos de 0 a 3.  El bucle interior enumera los
rangos de 1 a 13.  Cada iteración
crea una Carta nueva con el palo y el rango actuales,
y la anexa a {\tt self.cartas}.
\index{metodo@método!append}
\index{append, método}


\section{Imprimir la baraja}
\label{printdeck}
\index{metodo@método!str@método \_\_str\_\_}
\index{\_\_str\_\_, método}

Aquí hay un método \verb"__str__" para {\tt Baraja}:

\begin{verbatim}
# dentro de class Baraja:

    def __str__(self):
        res = []
        for carta in self.cartas:
            res.append(str(carta))
        return '\n'.join(res)
\end{verbatim}
%
Este método demuestra una manera eficiente de acumular una cadena
grande: construir una lista de cadenas y luego utilizar el método de cadena
{\tt join}.  La función incorporada {\tt str} invoca el método
\verb"__str__" en cada carta y devuelve la representación
de cadena.  \index{acumulador!cadena} \index{cadena!acumulador}
\index{metodo@método!join} \index{join, método} \index{nueva línea}

Dado que invocamos a {\tt join} en un carácter de nueva línea, las cartas
están separadas por caracteres de nueva línea.  Así es como se ve el resultado:

\begin{verbatim}
>>> baraja = Baraja()
>>> print(baraja)
    As de Tréboles
    2 de Tréboles
    3 de Tréboles
    ...
    10 de Picas
    Jota de Picas
    Reina de Picas
    Rey de Picas
\end{verbatim}
%
Si bien el resultado aparece en 52 líneas, es una única
cadena larga que contiene caracteres de nueva línea.


\section{Agregar, quitar, barajar y ordenar}

Para repartir cartas, nos gustaría un método que
quite una carta de la baraja y la entregue como valor de retorno.
El método de lista {\tt pop} proporciona una manera conveniente de hacer eso:
\index{metodo@método!pop}
\index{pop, método}

\begin{verbatim}
# dentro de class Baraja:

    def quitar_carta(self):
        return self.cartas.pop()
\end{verbatim}
%
Dado que {\tt pop} quita la {\em última} carta en la lista, estamos
repartiendo desde el fondo de la baraja.
\index{metodo@método!append}
\index{append, método}

Para agregar una carta, podemos utilizar el método de lista {\tt append}:

\begin{verbatim}
# dentro de class Baraja:

    def agregar_carta(self, carta):
        self.cartas.append(carta)
\end{verbatim}
%
Un método como este que utiliza otro método sin hacer
mucho trabajo a veces se llama {\bf enchapado} (en inglés, {\em veneer}).  La metáfora
viene de la carpintería, donde un enchapado es una capa
delgada de madera de buena calidad pegada a la superficie de un pedazo de madera
más barato, para mejorar la apariencia.
\index{enchapado}

En este caso \verb"agregar_carta" es un método ``delgado'' que expresa
una operación de lista en términos apropiados para barajas.  Este
mejora la apariencia, o interfaz, de la
implementación.

Como otro ejemplo, podemos escribir un método de Baraja con nombre {\tt barajar} 
utilizando la función {\tt shuffle} del módulo {\tt random}:
\index{modulo@módulo!random}
\index{random, módulo}
\index{función!shuffle}
\index{shuffle, función}

\begin{verbatim}
# dentro de class Baraja:

    def barajar(self):
        random.shuffle(self.cartas)
\end{verbatim}
%
No olvides importar a {\tt random}.

Como ejercicio, escribe un método de Baraja con nombre {\tt ordenar} que utilice el
método de lista {\tt sort} para ordenar las cartas en una {\tt Baraja}.  {\tt sort}
utiliza el método \verb"__lt__" que definimos para determinar el orden.
\index{metodo@método!sort} \index{sort, método}



\section{Herencia}
\index{herencia}
\index{programación orientada a objetos}

La herencia es la posibilidad de definir una clase nueva que sea una versión
modificada de una clase existente.  Como ejemplo, digamos que queremos una
clase que represente una ``mano'', es decir, las cartas sostenidas por un jugador.
Una mano es similar a una baraja: ambas están compuestas de una colección de
cartas, y ambas requieren operaciones como agregar y quitar cartas.

Una mano es también diferente de una baraja: hay operaciones para
manos que no tienen sentido para una baraja.  Por ejemplo, en el póker
podríamos comparar dos manos para ver cuál gana.  En el bridge, podríamos
calcular el puntaje de una mano para hacer un canto.

Estas relaciones entre clases ---similares, pero diferentes--- dan lugar
a la herencia.
Para definir una clase nueva que herede de una clase existente,
pones el nombre de la clase existente en paréntesis:
\index{paréntesis!clase padre en}
\index{clase!padre}
\index{padre, clase}
\index{clase!Mano}
\index{Mano, clase}

\begin{verbatim}
class Mano(Baraja):
    """Representa una mano de cartas de juego."""
\end{verbatim}
%
Esta definición indica que {\tt Mano} hereda de {\tt Baraja};
eso significa que podemos utilizar métodos como \verb"quitar_carta" y \verb"agregar_carta"
en Manos de igual manera que en Barajas.

Cuando una clase nueva hereda de una existente, la existente
se llama {\bf padre} y la clase nueva se
llama {\bf hija}.
\index{clase!padre}
\index{clase!hija}
\index{hija, clase}

En este ejemplo, {\tt Mano} hereda a \verb"__init__" de {\tt Baraja},
pero en realidad no hace lo que queremos: en lugar de llenar la mano
con 52 cartas nuevas, el método init para Manos debería inicializar {\tt
  cartas} con una lista vacía.  \index{anular} \index{metodo@método!init}
\index{init, método}

Si proporcionamos un método init en la clase {\tt Mano}, este anula el
de la clase {\tt Baraja}:

\begin{verbatim}
# dentro de class Mano:

    def __init__(self, etiqueta=''):
        self.cartas = []
        self.etiqueta = etiqueta
\end{verbatim}
%
Cuando creas una Mano, Python invoca este método init, no el
de {\tt Baraja}.

\begin{verbatim}
>>> mano = Mano('nueva mano')
>>> mano.cartas
    []
>>> mano.etiqueta
    'nueva mano'
\end{verbatim}
%
Los otros métodos se heredan de {\tt Baraja}, así que podemos utilizar a
\verb"quitar_carta" y \verb"agregar_carta" para repartir una carta:

\begin{verbatim}
>>> baraja = Baraja()
>>> carta = baraja.quitar_carta()
>>> mano.agregar_carta(carta)
>>> print(mano)
    Rey de Picas
\end{verbatim}
%
Un siguiente paso natural es encapsular este código en un método
llamado \verb"mover_cartas":

\index{encapsulamiento}

\begin{verbatim}
# dentro de class Baraja:

    def mover_cartas(self, mano, num):
        for i in range(num):
            mano.agregar_carta(self.quitar_carta())
\end{verbatim}
%
\verb"mover_cartas" toma dos argumentos, un objeto Mano y el número de
cartas a repartir.  Modifica tanto a {\tt self} como a {\tt mano}, y
devuelve {\tt None}.

En algunos juegos, las cartas se mueven de una mano a otra,
o de una mano de vuelta a la baraja.  Puedes utilizar a \verb"mover_cartas"
para cualquiera de estas operaciones: {\tt self} puede ser una Baraja
o una Mano, y {\tt mano}, a pesar del nombre, puede ser también una {\tt Baraja}.

La herencia es una característica útil.  Algunos programas que serían
repetitivos sin la herencia se pueden reescribir de manera más elegante
con esta.  La herencia puede facilitar la reutilización de código, dado que puedes
personalizar el comportamiento de las clases padres sin tener que
modificarlas.  En algunos casos, la estructura de la herencia refleja la estructura
natural del problema, lo cual hace al diseño más fácil
de entender.

Por otra parte, la herencia puede hacer a los programas difíciles de leer.
Cuando se invoca a un método, a veces no está claro dónde encontrar su
definición.  El código relevante puede estar distribuido en varios módulos.
Además, muchas de las cosas que se pueden hacer utilizando herencia se pueden
hacer igual de bien o mejor sin esta.


\section{Diagramas de clase}
\label{class.diagram}

Hasta ahora hemos visto diagramas de pila, que muestran el estado de
un programa, y diagramas de objeto, que muestran los atributos
de un objeto y sus valores.  Estos diagramas representan un instante
en la ejecución de un programa, por tanto cambian a medida que el programa
avanza.

Además, son altamente detallados; para algunos propósitos, muy
detallados.  Un diagrama de clase es una representación más abstracta
de la estructura de un programa.  En lugar de mostrar objetos
individuales, muestra clases y las relaciones entre estas.

Hay varios tipos de relaciones entre clases:

\begin{itemize}

\item Los objetos en una clase podrían contener referencias a objetos
en otra clase.  Por ejemplo, cada Rectángulo contiene una referencia
a un Punto, y cada Baraja contiene referencias a muchas Cartas.
Este tipo de relaciones se llama {\bf HAS-A} ({\em tiene un}), como en ``un Rectángulo
tiene un Punto.''

\item Una clase podría heredar de otra.  Esta relación
se llama {\bf IS-A} ({\em es un}), como en ``una Mano es un tipo de Baraja.''

\item Una clase podría depender de otra en el sentido de que objetos
de una clase tomen objetos de la segunda clase como parámetros, o
utilizar objetos de la segunda clase como parte de una computación.  Este
tipo de relación se llama {\bf dependencia}.

\end{itemize}
\index{relacion IS-A@relación IS-A}
\index{relacion HAS-A@relación HAS-A}
\index{diagrama!de clase}
\index{clase!diagrama de}

Un {\bf diagrama de clase} es una representación gráfica de estas
relaciones.  Por ejemplo, la Figura~\ref{fig.class1} muestra las
relaciones entre {\tt Carta}, {\tt Baraja} y {\tt Mano}.

\begin{figure}
\centerline
{\includegraphics[scale=0.8]{figs/class1.pdf}}
\caption{Diagrama de clase.}
\label{fig.class1}
\end{figure}

La flecha con una cabeza triangular hueca representa una relación IS-A;
en este caso indica que Mano hereda
de Baraja.

La fecha con una cabeza estándar representa una relación HAS-A;
en este caso una Baraja tiene referencias a objetos
Carta.
\index{multiplicidad (en diagrama de clase)}

El asterisco ({\tt *}) cerca de la cabeza de la flecha tiene una
{\bf multiplicidad}; esto indica cuántas Cartas tiene una Baraja.
Una multiplicidad puede ser un número simple, como {\tt 52}, un rango,
como {\tt 5..7} o un asterisco, que indica que una Baraja puede
tener cualquier número de Cartas.

No hay dependencias en este diagrama.  Normalmente se
mostrarían con una flecha punteada.  O bien si hay muchas
dependencias, a veces se omiten.

Un diagrama más detallado podría mostrar que una Baraja contiene
en realidad una {\em lista} de Cartas, pero los tipos incorporados
como list y dict generalmente no se incluyen en los diagramas de clase.


\section{Depuración}
\index{depuración}

La herencia puede dificultar la depuración porque cuando invocas a un
método en un objeto, podría ser difícil averiguar qué método
será invocado.
\index{herencia}

Supongamos que estás escribiendo una función que trabaja con objetos Mano.
Te gustaría trabajar con todos los tipos de Mano, como
ManoPóker, ManoBridge, etc.  Si invocas a un método como
{\tt barajar}, quizás obtengas el que se definió en {\tt Baraja},
pero si alguna de las subclases anula este método,
obtendrás esa versión en su lugar.  Este comportamiento es generalmente algo
bueno, pero puede ser confuso.

Cada vez que tengas insegurdad acerca del flujo de ejecución a través de tu
programa, la solución más simple es agregar sentencias print al
principio de métodos relevantes.  Si {\tt Baraja.barajar} imprime un
mensaje que diga algo como {\tt Ejecutando Baraja.barajar}, entonces
se sigue el flujo de ejecución a medida que el programa avanza.
\index{flujo de ejecución}

Como alternativa, podrías utilizar esta función, que toma un
objeto y un nombre de método (como cadena) y devuelve la clase que
proporciona la definición del método:

\begin{verbatim}
def encontrar_clase_definitoria(obj, nombre_metodo):
    for tipo in type(obj).mro():
        if nombre_metodo in tipo.__dict__:
            return tipo
\end{verbatim}
%
Aquí hay un ejemplo:

\begin{verbatim}
>>> mano = Mano()
>>> encontrar_clase_definitoria(mano, 'barajar')
    <class '__main__.Baraja'>
\end{verbatim}
%
Entonces el método {\tt barajar} para esta Mano es la que está en {\tt Baraja}.
\index{metodo@método!mro}
\index{mro, método}
\index{orden de resolución del método}

\verb"encontrar_clase_definitoria" utiliza el método {\tt mro} para obtener la lista de
objetos de clase (tipos) en los cuales serán buscados los métodos.  ``MRO''
significa ``orden de resolución de método'' (en inglés, {\em method resolution order}), que es la secuencia de
clases que Python busca para ``resolver'' un nombre de método.

Aquí hay una sugerencia de diseño: cuando anulas un método,
la interfaz del nuevo método debería ser la misma que el antiguo.  Debería
tomar los mismos parámetros, devolver el mismo tipo y cumplir las
mismas precondiciones y postcondiciones.  Si sigues esta regla,
encontrarás que cualquier función diseñada para trabajar con una instancia de una
clase padre, como Baraja, funcionará también con instancias de
clases hijas, como Mano y ManoPóker.
\index{anular}
\index{interfaz}
\index{precondición}
\index{postcondición}

Si violas esta regla, que se llama ``principio de sustitución
de Liskov'', tu código colapsará como (lo siento) un castillo de cartas.
\index{Liskov, principio de sustitución}


\section{Encapsulamiento de datos}

Los capítulos anteriores demuestran un plan de desarrollo que podríamos llamar
``diseño orientado a objetos''.  Identificamos objetos que necesitábamos ---como
{\tt Punto}, {\tt Rectángulo} y {\tt Tiempo}--- y clases definidas para
representarlos.  En cada caso, hay una correspondencia obvia
entre el objeto y alguna entidad del mundo real (o al menos de un
mundo matemático).
\index{plan de desarrollo!encapsulamiento de datos}

Sin embargo, a veces es menos obvio qué objetos necesitas
y cómo deberían interactuar.  En ese caso, necesitas un plan de desarrollo
diferente.  De la misma manera en que descubrimos interfaces
de función encapsulando y generalizando, podemos descubrir
interfaces de clase a través del {\bf encapsulamiento de datos}.
\index{encapsulamiento!de datos}

El análisis de Markov, de la Sección~\ref{markov}, proporciona un buen ejemplo.
Si descargas mi código en \url{http://thinkpython.com/code/markov.py},
verás que utiliza dos variables globales ---\verb"suffix_map" y
\verb"prefix"--- que se leen y escriben desde varias funciones.

\begin{verbatim}
suffix_map = {}
prefix = ()
\end{verbatim}

Dado que estas variables son globales, solo podemos ejecutar un análisis a la
vez.  Si leemos dos textos, sus prefijos y sufijos se
agregarían a la misma estructura de datos (lo que lo convierte en un interesante
texto generado).

Para ejecutar múltiples análisis, y mantenerlos separados, podemos encapsular
el estado de cada análisis en un objeto.
Así es como se ve eso:

\begin{verbatim}
class Markov:

    def __init__(self):
        self.suffix_map = {}
        self.prefix = ()
\end{verbatim}

Luego, transforamos las funciones en métodos.  Por ejemplo,
aquí está \verb"process_word":

\begin{verbatim}
    def process_word(self, word, order=2):
        if len(self.prefix) < order:
            self.prefix += (word,)
            return

        try:
            self.suffix_map[self.prefix].append(word)
        except KeyError:
            # si no hay entrada para este prefijo, crear una
            self.suffix_map[self.prefix] = [word]

        self.prefix = shift(self.prefix, word)
\end{verbatim}

Transformar un programa como este ---cambiar el diseño sin
cambiar el comportamiento--- es otro ejemplo de refactorización
(ver Sección~\ref{refactoring}).
\index{refactorización}

Este ejemplo sugiere un plan de desarrollo para diseñar objetos y
métodos:

\begin{enumerate}

\item Comienza escribiendo funciones que lean y escriban variables
globales (cuando sea necesario).

\item Una vez que hagas funcionar el programa, busca asociaciones
entre variables globales y las funciones que las utilizan.

\item Encapsula variables relacionadas como atributos de un objeto.

\item Transforma las funciones asociadas en métodos de una clase
nueva.

\end{enumerate}

Como ejercicio, descarga mi código de Markov en
\url{http://thinkpython.com/code/markov.py}, y sigue los pasos
descritos anteriormente para encapsular las variables globales como atributos de una clase
nueva llamada {\tt Markov}.  Solución:
\url{http://thinkpython.com/code/markov2.py}.


\section{Glosario}

\begin{description}

\item[codificar:]  Representar un conjunto de valores utilizando otro
conjunto de valores, construyendo un mapeo entre ambos.
\index{codificar}

\item[atributo de clase:] Un atributo asociado a un objeto
de clase.  Los atributos de clase se definen dentro de una
definición de clase pero afuera de cualquier método.
\index{atributo de clase}
\index{clase!atributo de}

\item[atributo de instancia:] Un atributo asociado a
una instancia de una clase.
\index{atributo de instancia}
\index{instancia!atributo de}

\item[enchapado:] Un método o función que proporciona una interfaz
diferente a otra función sin hacer muchas computaciones.
\index{enchapado}

\item[herencia:] La posibilidad de definir una clase nueva que sea una
versión modificada de una clase definida anteriormente.
\index{herencia}

\item[clase padre:] La clase desde la cual una clase hija hereda.
\index{clase!padre}

\item[clase hija:] Una clase nueva creada a través de herencia de una
clase existente; también llamada ``subclase''.
\index{clase!hija}
\index{hija, clase}

\item[relación IS-A:] Una relación entre una clase hija
y su clase padre.
\index{relacion IS-A@relación IS-A}

\item[relación HAS-A:] Una relación entre dos clases
donde instancias de una clase contienen referencias a instancias de
la otra.
\index{relacion HAS-A@relación HAS-A}

\item[dependencia:] Una relación entre dos clases
donde instancias de una clase utilizan instancias de otra clase,
pero no las almacenan como atributos.
\index{dependencia}

\item[diagrama de clase:] Un diagrama que muestra las clases en un programa
y las relaciones entre estas.
\index{diagrama!de clase}
\index{clase!diagrama}

\item[multiplicidad:] Una notación en un diagrama de clase que muestra, para
una relación HAS-A, cuántas referencias a instancias
de otra clase hay.
\index{multiplicidad (en diagrama de clase)}

\item[encapsulamiento de datos:]  Un plan de desarrollo de programa que
involucra a un prototipo que utiliza variables globales y una versión final
que convierte las variables globales en atributos de instancia.
\index{encapsulamiento!de datos}
\index{plan de desarrollo!encapsulamiento de datos}

\end{description}


\section{Ejercicios}

\begin{exercise}
Para el siguiente programa, dibuja un diagrama de clase UML que muestre
estas clases y las relaciones entre estas.

\begin{verbatim}
class PingPongPadre:
    pass

class Ping(PingPongPadre):
    def __init__(self, pong):
        self.pong = pong


class Pong(PingPongPadre):
    def __init__(self, pings=None):
        if pings is None:
            self.pings = []
        else:
            self.pings = pings

    def agregar_ping(self, ping):
        self.pings.append(ping)

pong = Pong()
ping = Ping(pong)
pong.agregar_ping(ping)
\end{verbatim}


\end{exercise}



\begin{exercise}
Escribe un método de Baraja llamado \verb"repartir_manos" que
tome dos parámetros, el número de manos y el número de cartas por
cada mano.  Debería crear el número apropiado de objetos Mano, repartir
el número apropiado de cartas por cada Mano y devolver una lista de Manos.
\end{exercise}


\begin{exercise}
\label{poker}

Lo siguiente son las posibles manos en el póker, en orden creciente
de valor y decreciente de probabilidad:
\index{poker@póker}

\begin{description}

\item[pareja:] dos cartas con el mismo rango
\vspace{-0.05in}

\item[doble pareja:] dos pares de cartas con el mismo rango
\vspace{-0.05in}

\item[trío:] tres cartas con el mismo rango
\vspace{-0.05in}

\item[escalera:] cinco cartas con rangos en secuencia (los ases pueden
ser altos o bajos, por lo cual {\tt As-2-3-4-5} es una escalera y {\tt
10-Jota-Reina-Rey-As} también lo es, pero {\tt Reina-Rey-As-2-3} no lo es.)
\vspace{-0.05in}

\item[color:] cinco cartas con el mismo palo
\vspace{-0.05in}

\item[full:] tres cartas con un rango, dos cartas con otro
\vspace{-0.05in}

\item[póker:] cuatro cartas con el mismo rango
\vspace{-0.05in}

\item[escalera de color:] cinco cartas en secuencia (como ya se definió) y
con el mismo palo
\vspace{-0.05in}

\end{description}
%
El objetivo de estos ejercicios es estimar
la probabilidad de extraer estas diferentes manos.

\begin{enumerate}

\item Descarga los siguientes archivos en \url{http://thinkpython.com/code}:

\begin{description}

\item[{\tt Card.py}]: Una versión completa de las clases {\tt Carta} (Card),
{\tt Baraja} (Deck) y {\tt Mano} (Hand) de este capítulo.

\item[{\tt PokerHand.py}]: Una implementación incompleta de una clase
que representa una mano de póker, y algo de código que la prueba.

\end{description}
%
\item Si ejecutas {\tt PokerHand.py}, reparte siete manos de póker de 7 cartas
y verifica si alguna de estas contiene una jugada ``color''.  Lee este
código con cuidado antes de continuar.

\item Agrega métodos a {\tt PokerHand.py} con nombre \verb"tiene_pareja",
\verb"tiene_doblepareja", etc. que devuelva True o False según si
la mano cumple o no los criterios relevantes.  Tu código debería
funcionar de manera correcta para ``manos'' que contengan cualquier número de cartas
(aunque 5 y 7 son los tamaños más comunes).

\item Escribe un método con nombre {\tt clasificar} que averigüe
la clasificación de mayor valor para una mano y ponga el atributo
{\tt etiqueta} según corresponda.  Por ejemplo, una mano de 7 cartas
podría contener un ``color'' y una ``pareja''; debería etiquetarse ``color''.

\item Cuando te convenzas de que tus métodos de clasificación
funcionan, el siguiente paso es estimar las probabilidades de varias
cartas.  Escribe una función en {\tt PokerHand.py} que baraje una baraja de
cartas, la divida en manos, clasifique las manos y cuente el
número de veces que aparecen varias clasificaciones.

\item Imprime una tabla con las clasificaciones y sus probabilidades.
Ejecuta tu programa con números más y más grandes de manos hasta que los
valores de salida converjan a un grado de exactitud razonable.  Compara
tus resultado con los valores en \url{http://en.wikipedia.org/wiki/Hand_rankings}.

\end{enumerate}

Solución: \url{http://thinkpython.com/code/PokerHandSoln.py}.
\end{exercise}


\chapter{Trucos extra}

Uno de mis objetivos para este libro ha sido enseñarte la menor cantidad de Python
posible.  Cuando había dos maneras de hacer algo, escogí
una y evité mencionar la otra.  O a veces puse la segunda
dentro de un ejercicio.

Ahora quiero volver a ver algunas de estas cosas buenas que dejé atrás.
Python proporciona un número de características que no son realmente necesarias ---puedes
escribir buen código sin estas--- pero con estas a veces puedes
escribir código más conciso, legible o eficiente, y a veces
las tres cosas a la vez.

% TODO: add the with statement

\section{Expresiones condicionales}

Vimos las sentencias condicionales en la Sección~\ref{conditional.execution}.
Las sentencias condicionales se utilizan a menudo para escoger uno de dos valores,
por ejemplo:
\index{expresión!condicional}
\index{condicional!expresión}

\begin{verbatim}
if x > 0:
    y = math.log(x)
else:
    y = float('nan')
\end{verbatim}

Esta sentencia verifica si {\tt x} es positivo.  Si lo es, calcula
{\tt math.log}.  Si no, {\tt math.log} levantaría un ValueError.  Para
evitar detener el programa, generamos un ``NaN'', que es un valor
de coma flotante especial que representa ``Not a Number''.
\index{NaN}
\index{coma flotante}

Podemos escribir esta sentencia de manera más concisa utilizando una {\bf expresión
condicional}:

\begin{verbatim}
y = math.log(x) if x > 0 else float('nan')
\end{verbatim}

Puedes leer esta línea casi como si estuviera en inglés: ``{\tt y} es log-{\tt x}
si {\tt x} es mayor que 0; de lo contrario es NaN''.

Las funciones recursivas a veces se pueden reescribir utilizando expresiones
condicionales.  Por ejemplo, aquí hay una versión recursiva de {\tt factorial}:
\index{factorial}
\index{función!factorial}

\begin{verbatim}
def factorial(n):
    if n == 0:
        return 1
    else:
        return n * factorial(n-1)
\end{verbatim}

Podemos reescribirla así:

\begin{verbatim}
def factorial(n):
    return 1 if n == 0 else n * factorial(n-1)
\end{verbatim}

Otro uso de las expresiones condicionales es al manejar argumentos
opcionales.  Por ejemplo, aquí está el método init de
{\tt GoodKangaroo} (ver Ejercicio~\ref{kangaroo}):
\index{argumento opcional}
\index{opcional!argumento}

\begin{verbatim}
    def __init__(self, name, contents=None):
        self.name = name
        if contents == None:
            contents = []
        self.pouch_contents = contents
\end{verbatim}

Podemos reescribirla así:

\begin{verbatim}
    def __init__(self, name, contents=None):
        self.name = name
        self.pouch_contents = [] if contents == None else contents
\end{verbatim}

En general, puedes reemplazar una sentencia condicional con una expresión
condicional si ambas ramas contienen expresiones simples que se
devuelven o asignan a la misma variable.
\index{sentencia!condicional}
\index{condicional!sentencia}



\section{Comprensiones de lista}

En la Sección~\ref{filter} vimos los patrones de mapa y filtro.  Por
ejemplo, esta función toma una lista de cadenas, mapea el método de cadena
{\tt capitalize} a los elementos y devuelve una nueva lista de cadenas:

\begin{verbatim}
def todas_con_mayuscula(t):
    res = []
    for s in t:
        res.append(s.capitalize())
    return res
\end{verbatim}

Podemos escribir esto de manera más concisa utilizando una {\bf comprensión de lista}:
\index{comprensión de lista}

\begin{verbatim}
def todas_con_mayuscula(t):
    return [s.capitalize() for s in t]
\end{verbatim}

Los operadores de corchetes indican que estamos construyendo una lista
nueva.  La expresión dentro de los corchetes especifica los elementos
de la lista, y la cláusula {\tt for} indica qué secuencia
estamos recorriendo.
\index{lista}
\index{for, bucle}

La sintaxis de una comprensión de lista es un poco incómoda porque
la variable del bucle, {\tt s} en este ejemplo, aparece en la expresión
antes de llegar a la definición.
\index{varlable del bucle}

Las comprensiones de lista también se pueden utilizar para filtrar.  Por ejemplo,
esta función selecciona solo los elementos de {\tt t} que están
en mayúsculas y devuelve una lista nueva:

\index{patrón!de filtro}
\index{filtro, patrón de}

\begin{verbatim}
def solo_mayusculas(t):
    res = []
    for s in t:
        if s.isupper():
            res.append(s)
    return res
\end{verbatim}

Podemos reescribirla utilizando una comprensión de lista

\begin{verbatim}
def solo_mayusculas(t):
    return [s for s in t if s.isupper()]
\end{verbatim}

Las comprensiones de lista son concisas y fáciles de leer, al menos para expresiones
simples.  Y generalmente son más rápidas que el equivalente para
bucles, a veces mucho más rápidas.  Así que si sientes molestia conmigo por no
mencionarlas antes, lo entiendo.

Sin embargo, en mi defensa, las comprensiones de lista son difíciles de depurar porque
no puedes poner una sentencia print dentro del bucle.  Sugiero que los
utilices solo si el cálculo es lo suficientemente simple para que sea probable
que lo hagas bien al primer intento.  Y para principiantes eso significa nunca.
\index{depuración}



\section{Expresiones generadoras}

Las {\bf expresiones generadoras} son similares a las comprensiones de lista, pero
con paréntesis en lugar de corchetes:
\index{expresión!generadora}
\index{generadora, expresión}

\begin{verbatim}
>>> g = (x**2 for x in range(5))
>>> g
    <generator object <genexpr> at 0x7f4c45a786c0>
\end{verbatim}
%
El resultado es un objeto generador que sabe cómo iterar a través de
una secuencia de valores.  Sin embargo, a diferencia de una comprensión de lista, este no
calcula todos los valores de una vez: espera hasta ser consultado.  La función
incorporada {\tt next} obtiene el siguiente valor del generador:
\index{objeto!generador}
\index{generador, objeto}

\begin{verbatim}
>>> next(g)
    0
>>> next(g)
    1
\end{verbatim}
%
Cuando llegas al final de la secuencia, {\tt next} levanta una
excepción StopIteration.  Puedes también utilizar un bucle {\tt for} para iterar
a través de los valores:
\index{StopIteration}
\index{excepción!StopIteration}

\begin{verbatim}
>>> for val in g:
...     print(val)
    4
    9
    16
\end{verbatim}
%
El objeto generador hace un seguimiento de dónde está en la secuencia,
por lo cual el bucle {\tt for} retoma donde {\tt next} lo dejó.  Una vez que el
generador se acaba, continúa para levantar {\tt StopIteration}:

\begin{verbatim}
>>> next(g)
    StopIteration
\end{verbatim}

Las expresiones generadoras son a menudo utilizadas con funciones como {\tt sum},
{\tt max} y {\tt min}:
\index{sum}
\index{función!sum}

\begin{verbatim}
>>> sum(x**2 for x in range(5))
    30
\end{verbatim}


\section{{\tt any} y {\tt all}}

Python proporciona una función incorporada, {\tt any}, que toma una secuencia
de valores booleanos y devuelve {\tt True} si alguno de los valores es {\tt
  True}.  Funciona en las listas:
\index{any, función incorporada}
\index{función!any}

\begin{verbatim}
>>> any([False, False, True])
    True
\end{verbatim}
%
Pero se utiliza a menudo con expresiones generadoras:
\index{expresión!generadora}
\index{generadora, expresión}

\begin{verbatim}
>>> any(letter == 't' for letter in 'monty')
    True
\end{verbatim}
%
Ese ejemplo no es muy útil porque hace lo mismo que
el operador {\tt in}.  Sin embargo, podríamos utilizar {\tt any} para reescribir
alguna de las funciones de búsqueda que escribimos en la Sección~\ref{search}.  Por
ejemplo, podríamos escribir {\tt excluye} así:
\index{patrón!de búsqueda}
\index{busqueda@búsqueda!patrón de}

\begin{verbatim}
def excluye(palabra, prohibidas):
    return not any(letra in prohibidas for letra in palabra)
\end{verbatim}
%
La función se lee casi como en inglés, ``{\tt palabra} excluye
{\tt prohibidas} si no hay alguna letra prohibida en {\tt palabra}.''

Utilizar {\tt any} con una expresión generadora es eficiente porque
se detiene inmediatamente si encuentra un valor {\tt True},
por lo cual no tiene que evaluar la secuencia completa.

Python proporciona otra función incorporada, {\tt all}, que devuelve
{\tt True} si cada elemento de la secuencia es {\tt True}.  Como
ejercicio, utiliza {\tt all} para reescribir \verb"usa_todas" de la
Sección~\ref{search}.
\index{all, función incorporada}
\index{any, función incorporada}


\section{Conjuntos (set)}
\label{sets}

En la Sección~\ref{dictsub} utilizo diccionarios para encontrar las palabras
que aparecen en un documento pero no en una lista de palabras.  La función
que escribí toma a {\tt d1}, que contiene las palabras del documento
como claves, y a {\tt d2}, que contiene la lista de palabras.  Este
devuelve un diccionario que contiene las claves de {\tt d1} que
no están en {\tt d2}.

\begin{verbatim}
def diferencia(d1, d2):
    res = dict()
    for clave in d1:
        if clave not in d2:
            res[clave] = None
    return res
\end{verbatim}
%
En todos estos diccionarios, los valores son {\tt None} porque
nunca los utilizamos.  Como resultado, desperdiciamos algo de espacio de almacenamiento.
\index{diferencia de diccionarios}

Python proporciona otro tipo incorporado, llamado {\tt set} (conjunto), que
se comporta como una colección de claves de diccionario sin valores.  Añadir
elementos a un conjunto es rápido; en consecuencia, lo es la verificación de pertenencia.  Además, los conjuntos
proporcionan métodos y operadores para calcular operaciones de conjunto comunes.
\index{conjunto}
\index{objeto!de conjunto}

Por ejemplo, la resta de conjuntos está disponible como método llamado
{\tt difference} o como operador, {\tt -}.  Entonces, podemos reescribir
la función {\tt diferencia} así:
\index{diferencia de conjuntos}

\begin{verbatim}
def diferencia(d1, d2):
    return set(d1) - set(d2)
\end{verbatim}
%
El resultado es un conjunto en lugar de un diccionario, pero para operaciones como
la iteración, el comportamiento es el mismo.

Algunos de los ejercicios de este libro se pueden hacer de manera concisa y
eficiente con conjuntos.  Por ejemplo, aquí hay una solución a
\verb"tiene_duplicados", del
Ejercicio~\ref{duplicate}, que utiliza un diccionario:

\begin{verbatim}
def tiene_duplicados(t):
    d = {}
    for x in t:
        if x in d:
            return True
        d[x] = True
    return False
\end{verbatim}

Cuando un elemento aparece por primera vez, se agrega al
diccionario.  Si el mismo elemento aparece otra vez, la función devuelve
{\tt True}.

Utilizando conjuntos, podemos escribir la misma función así:

\begin{verbatim}
def tiene_duplicados(t):
    return len(set(t)) < len(t)
\end{verbatim}
%
Un elemento puede aparecer en un conjunto una sola vez, por lo cual si un elemento en {\tt t}
aparece más de una vez, el conjunto será más pequeño que {\tt t}.  Si no
hay duplicados, el conjunto será del mismo tamaño que {\tt t}.
\index{duplicados}

Podemos también utilizar conjuntos para hacer algunos de los ejercicios del
Capítulo~\ref{wordplay}.  Por ejemplo, aquí hay una versión de
\verb"usa_solo" con un bucle:

\begin{verbatim}
def usa_solo(palabra, disponibles):
    for letra in palabra:
        if letra not in disponibles:
            return False
    return True
\end{verbatim}
%
\verb"usa_solo" verifica si todas las letras en {\tt palabra} están
en {\tt disponibles}.  Podemos reescribirla así:

\begin{verbatim}
def usa_solo(palabra, disponibles):
    return set(palabra) <= set(disponibles)
\end{verbatim}
%
El opereador \verb"<=" verifica si un conjunto es subconjunto de otro,
incluyendo la posibilidad de que sean iguales, lo cual es verdadero si todas
las letras en {\tt palabra} aparecen en {\tt disponibles}.
\index{subconjunto}

Como ejercicio, reescribe \verb"excluye" utilizando conjuntos.


\section{Contadores (Counter)}

Un contador ({\tt Counter}) es como un conjunto, excepto que si un elemento aparece más
de una vez, el contador hace un seguimiento de cuántas veces aparece.
Si la idea matemática de {\bf multiconjunto} se te hace familiar,
un contador es una manera natural de representar un multiconjunto.
\index{Counter}
\index{objeto!Counter}
\index{multiconjunto}

{\tt Counter} está definido en un módulo estándar llamado {\tt collections},
por lo cual tienes que importarlo.  Puedes inicializar un contador con una cadena,
lista, o cualquier otra cosa que soporte iteraciones:
\index{collections}
\index{modulo@módulo!collections}

\begin{verbatim}
>>> from collections import Counter
>>> contar = Counter('parrot')
>>> contar
    Counter({'r': 2, 't': 1, 'o': 1, 'p': 1, 'a': 1})
\end{verbatim}

Los contadores se comportan como diccionarios en muchas maneras: mapean de cada
clave al número de veces que esta aparece.  Al igual que en los diccionarios,
las claves tienen que ser hashables.

A diferencia de los diccionarios, los contadores no elevan una excepción si accedes a
un elemento que no aparece.  En su lugar, devuelven 0:

\begin{verbatim}
>>> contar['d']
    0
\end{verbatim}

Podemos utilizar contadores para reescribir \verb"es_anagrama" del
Ejercicio~\ref{anagram}:

\begin{verbatim}
def es_anagrama(palabra1, palabra2):
    return Counter(palabra1) == Counter(palabra2)
\end{verbatim}

Si dos palabras son anagramas, contienen la misma cantidad de letras con la mismas
cuentas, por tanto son equivalentes.

Los contadores proporcionan métodos y operadores para realizar operaciones como las de los conjuntos,
incluyendo la suma, resta, unión e intersección.  Además,
proporcionan un método que a menudo es útil, \verb"most_common", que
devuelve una lista de pares valor-frecuencia, ordenados desde el más común al
menos común:

\begin{verbatim}
>>> contar = Counter('parrot')
>>> for valor, frecuencia in contar.most_common(3):
...     print(valor, frecuencia)
    r 2
    p 1
    a 1
\end{verbatim}


\section{defaultdict}

El módulo {\tt collections} también proporciona a {\tt defaultdict}, que es
como un diccionario excepto que si accedes a una clave que no existe,
este puede generar un valor nuevo sobre la marcha.
\index{defaultdict}
\index{objeto!defaultdict}
\index{collections}
\index{modulo@módulo!collections}

Cuando creas un defaultdict, proporcionas una función que se utiliza para
crear valores nuevos.  Una función utilizada para crear objetos a veces es
llamada una {\bf fábrica}.  Las funciones incorporadas que crean listas, conjuntos
y otros tipos se pueden utilizar como fábricas:
\index{fabrica, funcion@fábrica, función}

\begin{verbatim}
>>> from collections import defaultdict
>>> d = defaultdict(list)
\end{verbatim}

Notar que el argumento es {\tt list}, que es un objeto de clase,
no {\tt list()}, que es una lista nueva.  La función que proporcionas
no es llamada a menos que accedas a una clave que no existe.

\begin{verbatim}
>>> t = d['clave nueva']
>>> t
    []
\end{verbatim}

La lista nueva, la cual llamamos {\tt t}, también es añadida al
diccionario.  Por lo tanto, si modificamos {\tt t}, el cambio aparece en {\tt d}:

\begin{verbatim}
>>> t.append('valor nuevo')
>>> d
    defaultdict(<class 'list'>, {'clave nueva': ['valor nuevo']})
\end{verbatim}

Si estás creando un diccionario de listas, a menudo puedes escribir código
más simple utilizando {\tt defaultdict}.  En mi solución al
Ejercicio~\ref{anagrams}, que puedes obtener en
\url{http://thinkpython.com/code/anagram_sets.py}, creo un
diccioinario que mapea de una cadena de letras ordenada a la lista de
palabras que se pueden escribir con esas letras.  Por ejemplo, {\tt
  'opst'} mapea a la lista {\tt ['opts', 'post', 'pots', 'spot',
    'stop', 'tops']}.

Este es el código original:

\begin{verbatim}
def all_anagrams(filename):
    d = {}
    for line in open(filename):
        word = line.strip().lower()
        t = signature(word)
        if t not in d:
            d[t] = [word]
        else:
            d[t].append(word)
    return d
\end{verbatim}

Esto se puede simplificar utilizando {\tt setdefault}, que tal vez
utilizaste en el Ejercicio~\ref{setdefault}:
\index{setdefault}

\begin{verbatim}
def all_anagrams(filename):
    d = {}
    for line in open(filename):
        word = line.strip().lower()
        t = signature(word)
        d.setdefault(t, []).append(word)
    return d
\end{verbatim}

Esta solución tiene el inconveniente de que crea una nueva lista
cada vez, independiente de si es necesaria.  Para las listas,
eso no es un gran problema, pero si la función
fábrica es complicada, podría serlo.
\index{fabrica, funcion@fábrica, función}

Podemos evitar este problema y
simplificar el código utilizando un {\tt defaultdict}:

\begin{verbatim}
def all_anagrams(filename):
    d = defaultdict(list)
    for line in open(filename):
        word = line.strip().lower()
        t = signature(word)
        d[t].append(word)
    return d
\end{verbatim}

Mi solución al Ejercicio~\ref{poker}, que puedes descargar en
\url{http://thinkpython.com/code/PokerHandSoln.py},
utiliza {\tt setdefault} en la función
\verb"has_straightflush".  Esta solución tiene el inconveniente de
crear un objeto {\tt Hand} en cada paso del bucle, cuando
es necesaria y cuando no.  Como ejercicio, reescríbela utilizando
un defaultdict.


\section{Tuplas con nombre (namedtuple)}

Muchos objetos simples son básicamente colecciones de valores relacionados.
Por ejemplo, el objeto Punto definido en el Capítulo~\ref{clobjects} contiene
dos números, {\tt x} e {\tt y}.  Cuando defines una clase como
esta, generalmente comienzas con un método init y un método str:

\begin{verbatim}
class Punto:

    def __init__(self, x=0, y=0):
        self.x = x
        self.y = y

    def __str__(self):
        return '(%g, %g)' % (self.x, self.y)
\end{verbatim}

Esto es mucho código para transmitir una cantidad pequeña de información.
Python proporciona una manera más concisa de decir lo mismo:

\begin{verbatim}
from collections import namedtuple
Punto = namedtuple('Punto', ['x', 'y'])
\end{verbatim}

El primer argumento es el nombre de la clase que quieres crear.
El segundo es una lista de atributos que los objetos Punto debieran tener,
en forma de cadenas.  El valor de retorno de {\tt namedtuple} es un objeto de clase:
\index{namedtuple}
\index{objeto!namedtuple}
\index{collections}
\index{modulo@módulo!collections}

\begin{verbatim}
>>> Punto
    <class '__main__.Punto'>
\end{verbatim}

{\tt Punto} automáticamente proporciona métodos como \verb"__init__" y
\verb"__str__", así que no tienes que escribirlos.
\index{objeto!de clase}
\index{clase!objeto}

Para crear un objeto Punto, utilizas la clase Punto como una función:

\begin{verbatim}
>>> p = Punto(1, 2)
>>> p
    Punto(x=1, y=2)
\end{verbatim}

El método init asigna argumentos a atributos utilizando nombres
que proporcionas.  El método str imprime una representación del objeto Punto
y sus atributos.

Puedes acceder a los elementos de la tupla con nombre, escribiendo sus nombres:

\begin{verbatim}
>>> p.x, p.y
   (1, 2)
\end{verbatim}

Sin embargo, puedes también tratar a una tupla con nombre como una tupla:

\begin{verbatim}
>>> p[0], p[1]
    (1, 2)

>>> x, y = p
>>> x, y
    (1, 2)
\end{verbatim}

Las tuplas con nombre proporcionan una manera rápida de definir clases simples.
El inconveniente es que las clases simples no siempre permanecen simples.
Tal vez decidas más tarde que quieres agregar métodos a una tupla con nombre.
En ese caso, podrías definir una clase nueva que tome herencia de
la tupla con nombre:
\index{herencia}

\begin{verbatim}
class SuperPunto(Punto):
    # agregar más métodos aquí
\end{verbatim}

O bien podrías cambiar a una definición de clase convencional.


\section{Reunir argumentos de palabra clave}

En la Sección~\ref{gather}, vimos cómo escribir una función que
reúne sus argumentos en una tupla:
\index{reunir}

\begin{verbatim}
def imprimir_todo(*args):
    print(args)
\end{verbatim}
%
Puedes llamar a esta función con cualquier número de argumentos posicionales
(es decir, argumentos que no tienen palabra clave):
\index{argumento posicional}
\index{posicional, argumento}

\begin{verbatim}
>>> imprimir_todo(1, 2.0, '3')
    (1, 2.0, '3')
\end{verbatim}
%
Sin embargo, el operador {\tt *} no reúne a los argumentos de palabra clave:
\index{argumento de palabra clave}
\index{palabra clave!argumento de}

\begin{verbatim}
>>> imprimir_todo(1, 2.0, tercero='3')
    TypeError: imprimir_todo() got an unexpected keyword argument 'tercero'
\end{verbatim}
%
Para reunir argumentos de palabra clave, puedes utilizar el operador {\tt **}:

\begin{verbatim}
def imprimir_todo(*args, **kwargs):
    print(args, kwargs)
\end{verbatim}
%
Puedes llamar al parámetro de reunión de palabras clave como quieras, pero
{\tt kwargs} es una elección común (del inglés, {\em keyword arguments}).  El resultado es un diccionario que mapea
palabras clave a valores:

\begin{verbatim}
>>> imprimir_todo(1, 2.0, tercero='3')
    (1, 2.0) {'tercero': '3'}
\end{verbatim}
%
Si tienes un diccionario de palabras clave y valores, puedes utilizar el
operador de dispersión {\tt **} para llamar a una función:
\index{dispersión}

\begin{verbatim}
>>> d = dict(x=1, y=2)
>>> Punto(**d)
    Punto(x=1, y=2)
\end{verbatim}
%
Sin el operador de dispersión, la función trataría a {\tt d} como
un argumento posicional único, por lo cual asignaría {\tt d} a
{\tt x} y reclamaría porque no hay nada para asignar a {\tt y}:

\begin{verbatim}
>>> d = dict(x=1, y=2)
>>> Punto(d)
    Traceback (most recent call last):
      File "<stdin>", line 1, in <module>
    TypeError: __new__() missing 1 required positional argument: 'y'
\end{verbatim}
%
Cuando estás trabajando con funciones que tienen un número grande de
parámetros, a menudo es útil crear y pasar diccionarios
que especifiquen las opciones utilizadas de manera frecuente.


\section{Glosario}

\begin{description}

\item[expresión condicional:] Una expresión que tiene uno de dos
valores, dependiendo de una condición.
\index{expresión!condicional}
\index{condicional!expresión}

\item[comprensión de lista:] Una expresión con un bucle {\tt for} en
corchetes que produce una lista nueva.
\index{comprensión de lista}

\item[expresión generadora:] Una expresión con un bucle {\tt for} en paréntesis
que produce un objeto generador.
\index{expresión!generadora}
\index{generadora, expresión}

\item[multiconjunto:] Una entidad matemática que representa un mapeo
entre los elementos de un conjunto y el número de veces que estos aparecen.

\item[fábrica:] Una función, generalmente pasada como parámetro, utilizada para
crear objetos.
\index{fabrica@fábrica}

\end{description}




\section{Ejercicios}

\begin{exercise}

Lo siguiente es una función que calcula el
coeficiente binomial de manera recursiva.

\begin{verbatim}
def coeficiente_binomial(n, k):
    """Calcula el coeficiente binomial "n sobre k".

    n: número de ensayos
    k: número de éxitos

    devuelve: int
    """
    if k == 0:
        return 1
    if n == 0:
        return 0

    res = coeficiente_binomial(n-1, k) + coeficiente_binomial(n-1, k-1)
    return res
\end{verbatim}

Reescribe el cuerpo de la función utilizando expresiones condicionales
anidadas.

Nota: esta función no es muy eficiente porque termina calculando
los mismos valores una y otra vez.  Podrías hacerla más eficiente
memoizando (ver Sección~\ref{memoize}).  Sin embargo, encontrarás que es más difícil
memoizar si la escribes utilizando expresiones condicionales.

\end{exercise}



\appendix

\chapter{Depuración}
\index{depuración}

Cuando estés depurando, deberías distinguir entre los diferentes
tipos de errores con el fin de rastrearlos de manera más rápida:

\begin{itemize}

\item Los errores de sintaxis son descubiertos por el intérprete cuando está
  traduciendo el código fuente a código byte.  Indican
  que hay algo mal en la estructura del programa.
  Ejemplo: omitir el signo de dos puntos al final de una sentencia {\tt def}
  genera el mensaje algo redundante {\tt SyntaxError: invalid
    syntax}.
\index{error!de sintaxis}
\index{sintaxis!error de}

\item Los errores de tiempo de ejecución son producidos por el intérprete si algo va
  mal mientras el programa se está ejecutando.  La mayoría de los mensajes de error de tiempo de ejecución
  incluyen información acerca de dónde ocurrió el error y qué
  funciones se estaban ejecutando.  Ejemplo: una recursividad infinita eventualmente
  causa el error de tiempo de ejecución ``maximum recursion depth exceeded''.
\index{error!de tiempo de ejecución}
\index{tiempo de ejecución, error de}
\index{excepción}

\item Los errores semánticos son problemas que tiene un programa que se ejecuta sin
  producir mensajes de error pero sin hacer lo correcto.  Ejemplo:
  una expresión puede que no sea evaluada en el orden que esperas, entregando
  un resultado incorrecto.
\index{error!semántico}
\index{semántico, error}

\end{itemize}

El primer paso en la depuración es averiguar con qué tipo de
error estás lidiando.  Aunque las siguientes secciones están
organizadas por tipo de error, algunas técnicas son
aplicables en más de una situación.


\section{Errores de sintaxis}
\index{mensaje de error}

Los errores de sintaxis son generalmente fáciles de arreglar una vez que averiguas cuáles
son.  Desafortunadamente, los mensajes de error a menudo no son útiles.
Los mensajes más comunes son {\tt SyntaxError: invalid syntax} y
{\tt SyntaxError: invalid token}, de los cuales ninguno es muy informativo.

Por otra parte, el mensaje sí te dice el lugar del programa donde
ocurrió el problema.  En realidad, te dice dónde Python
notó un problema, que no necesariamente es donde está
el error.  A veces el error está antes de la ubicación del mensaje
de error, a menudo en la línea precedente.
\index{desarrollo incremental}
\index{plan de desarrollo!incremental}

Si estás construyendo el programa de manera incremental, deberías tener
una buena idea acerca de dónde está el error.  Estará en la última
línea que agregaste.

Si estás copiando código de un libro, comienza comparando
tu código con el código del libro de manera muy cuidadosa.  Revisa cada carácter.
Al mismo tiempo, recuerda que el libro podría estar mal, por tanto
si ves algo que parece un error de sintaxis, puede serlo.

Aquí hay algunas maneras de evitar los errores de sintaxis más comunes:
\index{sintaxis}

\begin{enumerate}

\item Asegúrate de que no estás utilizando una palabra clave de Python para un nombre de variable.
\index{palabra clave}

\item Verifica que tienes un signo de dos puntos al final del encabezado de cada
sentencia compuesta, incluyendo las sentencias {\tt for}, {\tt while},
{\tt if} y {\tt def}.
\index{encabezado}
\index{signo de dos puntos}

\item Asegúrate de que todas las cadenas en el código tengan comillas
coincidentes.  Asegúrate de que todas las comillas son
``comillas rectas'', no ``comillas tipográficas''.\index{comillas}

\item Si tienes cadenas multilínea con comillas triples (simples o dobles),
asegúrate de que has terminado la cadena de manera apropiada.  Una cadena sin terminar
puede causar un error {\tt invalid token} al final de tu programa,
o puede tratar la siguiente parte del programa como una cadena hasta que
llega a la siguiente cadena.  En el segundo caso, ¡podría no producir ningún mensaje
de error!
\index{cadena!multilínea}
\index{multilínea, cadena}

\item Un operador de apertura no cerrado ---\verb+(+, \verb+{+ o
  \verb+[+--- hace que Python continúe con la línea siguiente como parte de la
  sentencia actual.  Generalmente, ocurre un error casi inmediatamente en
  la línea siguiente.

\item Revisa el clásico {\tt =} en lugar de {\tt ==} dentro
de un condicional.
\index{condicional}

\item Revisa la sangría para asegurarte de que esté alineada como se
supone que debe estar.  Python puede manejar el espacio y la tabulación, pero si los mezclas
puede causar problemas.  La mejor manera de evitar este problema
es utilizar un editor de texto que sepa sobre Python y genere
sangría consistente.
\index{sangría}
\index{espacio en blanco}

\item Si tienes caracteres no ASCII en el código (incluyendo cadenas
y comentarios), podría causar un problema, aunque Python 3 generalmente
maneja caracteres no ASCII.  Ten cuidado si pegas texto de
una página web u otra fuente.

\end{enumerate}

Si nada funciona, pasa a la siguiente sección...


\subsection{Sigo haciendo cambios y no hay diferencia.}

Si el intérprete dice que hay un error y tú no lo ves,
podría ser porque tú y el intérprete no están mirando el mismo
código.  Revisa tu entorno de programación para asegurarte de que el
programa que estás editando es el que Python está intentando ejecutar.

Si no sabes bien, intenta poniendo un error de sintaxis obvio y deliberado
al principio del programa.  Ahora ejecútalo de nuevo.  Si el
intérprete no encuentra el nuevo error, no estás ejecutando el
código nuevo.

Hay algunos posibles culpables:

\begin{itemize}

\item Editaste el archivo y olvidaste guardar los cambios antes de
ejecutarlo de nuevo.  Algunos entornos de programación hacen esto
por ti, pero otros no.

\item Cambiaste el nombre del archivo, pero todavía estás ejecutando
el nombre antiguo.

\item Algo en tu entorno de desarrollo está configurado
de manera incorrecta.

\item Si estás escribiendo un módulo y utilizando {\tt import},
asegúrate de que no le das a tu módulo el mismo nombre que uno
de los módulos estándar de Python.

\item Si estás utilizando {\tt import} para leer un módulo, recuerda
que tienes que reiniciar el intérprete o utilizar {\tt reload}
para leer un archivo modificado.  Si importas el módulo de nuevo,
no hace nada.
\index{modulo@módulo!reload}
\index{función!reload}
\index{reload, función}

\end{itemize}

Si te atascas y no puedes averiguar qué está pasando, una
manera de abordarlo es comenzar de nuevo con un nuevo programa como ``Hola, mundo''
y asegurarte de que puedes obtener un programa conocido para ejecutar.  Luego agrega gradualmente
los pedazos del programa original al nuevo programa.


\section{Errores de tiempo de ejecución}

Una vez que tu programa está sintácticamente correcto,
Python puede leerlo y al menos comenzar a ejecutarlo.  ¿Qué podría
salir mal?


\subsection{Mi programa no hace absolutamente nada.}

Este problema es más común cuando tu archivo se compone de funciones y
clases pero en realidad no invoca una función para comenzar la ejecución.
Esto puede ser intencional si solo planeas importar este módulo para
proporcionar clases y funciones.

Si no es intencional, asegúrate de que hay una llamada a función
en el programa, y asegúrate de que el flujo de ejecución
lo alcanza (ver ``Flujo de ejecución'' más adelante).


\subsection{Mi programa se congela.}
\index{bucle!infinito}
\index{recursividad!infinita}
\index{congelamiento}

Si un programa se detiene y parece que no hace nada, está ``congelado''.
A menudo eso significa que está atrapado en un bucle infinito o una
recursividad infinita.

\begin{itemize}

\item Si hay un bucle en particular del cual sospechas que es el
problema, agrega una sentencia {\tt print} inmediatamente antes del bucle que
diga ``entrando al bucle'' y otro inmediatamente después que diga
``saliendo del bucle''.

Ejecuta el programa.  Si obtienes el primer mensaje y no el segundo,
tienes un bucle infinito.  Ve a la sección ``Bucle infinito''
de más adelante.

\item La mayor parte del tiempo, una recursividad infinita causará que el
programa se ejecute por un momento y luego produzca un error ``RuntimeError: Maximum
recursion depth exceeded''.  Si eso ocurre, ve a la sección
``Recursividad infinita'' de más adelante.

Si no obtienes este error pero sospechas que hay un problema
con una función recursiva o método recursivo, todavía puedes utilizar las técnicas
de la sección ``Recursividad infinita''.

\item Si ninguno de esos pasos funciona, comienza a probar otros
bucles y otras funciones y métodos recursivos.

\item Si eso no funciona, entonces es posible que
no entiendas el flujo de ejecución de tu programa.
Ve a la sección ``Flujo de ejecución'' de más adelante.

\end{itemize}


\subsubsection{Bucle infinito}
\index{bucle!infinito}
\index{infinito, bucle}
\index{condición}
\index{bucle!condición}

Si crees que tienes un bucle infinito y crees que sabes
qué bucle está causando el problema, agrega una sentencia {\tt print}
al final del bucle que imprima los valores de las variables en
la condición y el valor de la condición.

Por ejemplo:

\begin{verbatim}
while x > 0 and y < 0 :
    # hacer algo a x
    # hacer algo a y 

    print('x: ', x)
    print('y: ', y)
    print("condición: ", (x > 0 and y < 0))
\end{verbatim}
%
Ahora cuando ejecutes el programa, verás tres líneas de salida
cada vez que se pase por el bucle.  En el último paso por el
bucle, la condición debería ser {\tt False}.  Si el bucle
continúa, podrás ver los valores de {\tt x} e {\tt y},
y podrías averiguar por qué no se están actualizando correctamente.


\subsubsection{Recursividad infinita}
\index{recursividad!infinita}
\index{infinita, recursividad}

La mayoría de las veces, la recursividad infinita causa que el programa se
ejecute por un momento y luego produzca un error
{\tt Maximum recursion depth exceeded}.

Si sospechas que una función está causando una recursividad
infinita, asegúrate de que haya un caso base.
Debería haber alguna condición que cause que la
función devuelva algo sin hacer una invocación recursiva.
Si no, necesitas volver a pensar el algoritmo e identificar un caso
base.

Si hay un caso base pero el programa no parece estar alcanzándolo,
agrega una sentencia {\tt print} al principio de la función
que imprima los parámetros.  Ahora cuando ejecutes el programa, verás
algunas líneas de salida cada vez que se invoca a la función,
y verás los valores de los parámetros.  Si los parámetros no se están moviendo
hacia el caso base, obtendrás algunas ideas sobre por qué no ocurre.


\subsubsection{Flujo de ejecución}
\index{flujo de ejecución}

Si no sabes bien cómo se está moviendo el flujo de ejecución a través
de tu programa, agrega sentencias {\tt print} al principio de cada
función con un mensaje como ``entrando a la función {\tt foo}'', donde
{\tt foo} es el nombre de la función.

Ahora cuando ejecutes el programa, imprimirá una señal de cada
función que se invoque.


\subsection{Cuando ejecuto el programa obtengo una excepción.}
\index{excepción}
\index{error!de tiempo de ejecución}

Si algo va mal durante el tiempo de ejecución, Python
imprime un mensaje que incluye el nombre de la
excepción, la línea del programa donde ocurrió el problema
y un rastreo.
\index{rastreo}

El rastreo identifica la función que se está ejecutando actualmente y
luego la función que la llamó, y luego la función que llamo a
{\em aquella}, y así sucesivamente.  En otras palabras, rastrea la secuencia de
llamadas a función que te llevaron a donde estás, incluyendo el número
de línea en tu archivo donde ocurrió cada llamada.

El primer paso es examinar el lugar del programa donde
ocurrió el error y ver si puedes averiguar lo que sucedió.
Estos son algunos de los errores de tiempo de ejecución más comunes:

\begin{description}

\item[NameError:]  Estás intentando utilizar una variable que no
existe en el entorno actual.  Revisa si el nombre
está bien escrito, o al menos de manera consistente.
Y recuerda que las variables locales son locales: 
no puedes referirte a estas desde afuera de la función donde se definieron.
\index{NameError}
\index{excepción!NameError}

\item[TypeError:] Hay varias causas posibles:
\index{TypeError}
\index{excepción!TypeError}

\begin{itemize}

\item  Estás intentando utilizar un valor de manera inapropiada.  Ejemplo: indexar
una cadena, lista o tupla con algo que no es un entero.
\index{indice@índice}

\item Hay una discordancia entre los ítems en una cadena de formato y
los ítems pasados para una conversión.  Eso puede ocurrir si el número
de ítems no coincide o si se pidió una conversión no válida.
\index{operador!de formato}
\index{formato, operador de}

\item Estás pasando el número equivocado de argumentos a una función.
Para los métodos, mira la definición del método y
verifica que el primer parámetro es {\tt self}.  Luego, mira la
invocación al método; asegúrate de que estás invocando al método en un
objeto con el tipo correcto y proporcionando los otros argumentos
de manera correcta.

\end{itemize}

\item[KeyError:]  Estás intentando acceder a un elemento de un diccionario
utilizando una clave que el diccionario no contiene.  Si las claves
son cadenas, recuerda que las mayúsculas importan.
\index{KeyError}
\index{excepción!KeyError}
\index{diccionario}

\item[AttributeError:] Estás intentando acceder a un atributo o método
  que no existe.  ¡Revisa la ortografía!  Puedes utilizar la función
  incorporada {\tt vars} para hacer una lista de los atributos que sí existen.
\index{dir function}
\index{función!dir}

Si un AttributeError indica que un objeto tiene {\tt NoneType},
eso significa que es {\tt None}.  Entonces el problema no es el
nombre de atributo, sino el objeto.

La razón por la cual el objeto es {\tt None} podría ser que olvidaste
devolver un valor desde una función; si llegas al final de
una función poniendo una sentencia {\tt return}, devuelve
{\tt None}.  Otra causa común es utilizar el resultado de
un método de lista, como {\tt sort}, que devuelve {\tt None}.
\index{AttributeError}
\index{excepción!AttributeError}

\item[IndexError:] El índice que estás utilizando
para acceder a una lista, cadena o tupla es mayor que
su longitud menos uno.  Inmediatamente antes del lugar del error,
agrega una sentencia {\tt print} para mostrar en pantalla
el valor del índice y la longitud de la secuencia.
¿Tiene la secuencia el tamaño correcto?  ¿Tiene el índice el valor correcto?
\index{IndexError}
\index{excepción!IndexError}

\end{description}

El depurador de Python ({\tt pdb}, {\em Python debugger}) es útil para rastrear
excepciones porque te permite examinar el estado del
programa inmediatamente antes del error.  Puedes leer
sobre {\tt pdb} en \url{https://docs.python.org/3/library/pdb.html}.
\index{depurador (pdb)}
\index{pdb (Python debugger)}


\subsection{Agregué tantas sentencias {\tt print} que me inundé con
la salida.}
\index{sentencia!print}
\index{print, sentencia}

Uno de los problemas al utilizar sentencias {\tt print} para depurar
es que puedes terminar enterrándote en la salida.  Hay dos maneras
de proceder: simplificar la salida o simplificar el programa.

Para simplificar la salida, puedes eliminar o poner como comentarios
las sentencias {\tt print} que no están ayudando, o combinarlas, o dar formato
a la salida para que sea más fácil de entender.

Para simplificar el programa, hay varias cosas que puedes hacer.  Primero,
reduce la escala del problema en el cual está trabajando el programa.  Por
ejemplo, si estás buscando una lista, busca una lista {\em pequeña}.  Si el
programa toma entrada del usuario, dale la entrada más simple que cause el
problema.
\index{codigo muerto@código muerto}

Segundo, limpia el programa.  Elimina el código muerto y reorganiza el
programa para hacerlo tan fácil de leer como sea posible.  Por ejemplo, si
sospechas que el problema está en una parte profundamente anidada del programa,
intenta reescribir esa parte con una estructura más simple.  Si sospechas de
una función grande, intenta separarla en funciones más pequeñas y probarlas
de manera separada.
\index{prueba!caso de prueba mínimo}
\index{caso de prueba mínimo}

A menudo el proceso de encontrar el caso de prueba mínimo te guía al
error.  Si encuentras que un programa funciona en una situación pero no en
otra, eso te da una pista sobre qué está pasando.

Del mismo modo, reescribir un pedazo de código puede ayudarte a encontrar
errores sutiles.  Si haces un cambio que crees que no debería afectar al
programa, y sí afecta, eso te puede dar una pista.


\section{Errores semánticos}

De alguna manera, los errores semánticos son los más difíciles de depurar,
porque el intérprete no proporciona información
sobre qué está mal.  Solo tú sabes lo que se supone que debe hacer el
programa.
\index{error!semántico}
\index{semántico, error}

El primer paso es hacer una conexión entre el texto del programa
y el comportamiento que ves.  Necesitas una hipótesis
sobre qué está haciendo realmente el programa.  Una de las cosas
que hace que eso sea difícil es que los computadores funcionan muy rápido.

A menudo desearás poder ralentizar el programa a velocidad humana,
y con algunos depuradores puedes hacerlo.  Pero el tiempo que toma insertar
unas pocas sentencias {\tt print} bien ubicadas es a menudo corto comparado 
con el de configurar el depurador, insertar y eliminar puntos de interrupción,
y avanzar ``paso a paso'' en el programa hasta donde ocurre el error.


\subsection{Mi programa no funciona.}

Deberías hacerte estas preguntas:

\begin{itemize}

\item ¿Hay algo que se supone que el programa debe hacer pero
que no parece estar ocurriendo?  Encuentra la sección del código
que realiza esa función y asegúrate de que se está ejecutando cuando
crees que debería.

\item ¿Ocurre algo que no debería?  Encuentra código en
tu programa que realiza esa función y ve si se está
ejecutando cuando no debería.

\item ¿Hay una sección de código produciendo un efecto que no es
lo que esperabas?  Asegúrate de que entiendes el código en
cuestión, especialmente si involucra funciones o métodos de
otros módulos de Python.  Lee la documentación para las funciones que llamas.
Pruébalas escribiendo casos de prueba simples y verificando los resultados.

\end{itemize}

Para programar, necesitas un modelo mental de cómo
funcionan los programas.  Si escribes un programa que no hace lo que esperas,
muchas veces el problema no está en el programa: está en tu modelo
mental.
\index{modelo mental}
\index{mental, modelo}

La mejor manera de corregir tu modelo mental es separar el programa
en sus componentes (generalmente las funciones y métodos) y probar
cada componente de manera independiente.  Una vez que encuentres la discrepancia
entre tu modelo y la realidad, puedes resolver el problema.

Por supuesto, deberías estar construyendo y probando componentes a medida que
desarrollas el programa.  Si encuentras un problema,
debería haber solo una pequeña cantidad de código nuevo
que no se sabe si es correcto.


\subsection{Tengo una expresión grande y fea, y no
hace lo que yo espero.}
\index{expresión!grande y fea}
\index{grande y fea, expresión}

Escribir expresiones complejas está bien mientras sean legibles,
pero pueden ser difíciles de depurar.  Muchas veces es una buena idea
separar una expresión compleja en una serie de asignaciones a
variables temporales.

Por ejemplo:

\begin{verbatim}
self.hands[i].addCard(self.hands[self.findNeighbor(i)].popCard())
\end{verbatim}
%
Esto se puede reescribir como:

\begin{verbatim}
neighbor = self.findNeighbor(i)
pickedCard = self.hands[neighbor].popCard()
self.hands[i].addCard(pickedCard)
\end{verbatim}
%
La versión explícita es más fácil de leer porque los nombres de variable
proporcionan documentación adicional, y es más fácil depurar
porque puedes verificar los tipos de las variables intermedias
y mostrar sus valores en pantalla.
\index{variable temporal}
\index{temporal, variable}

Otro problema que puede ocurrir con las expresiones grandes es que
el orden de evaluación puede que no sea lo que esperas.
Por ejemplo, si estás traduciendo la expresión
$\frac{x}{2 \pi}$ a Python, podrías escribir:

\begin{verbatim}
y = x / 2 * math.pi
\end{verbatim}
%
Eso no es correcto porque la multiplicación y la división tienen
la misma prioridad y se evalúan de izquierda a derecha.
Entonces esta expresión calcula $x \pi / 2$.
\index{orden de operaciones}
\index{prioridad}

Una buena manera de depurar expresiones es agregar paréntesis para hacer
que el orden de evaluación sea explícito:

\begin{verbatim}
 y = x / (2 * math.pi)
\end{verbatim}
%
Siempre que no sepas bien del orden de evaluación, utiliza
paréntesis.  No solo estará correcto el programa (en el sentido
de hacer lo que pretendías), también será más legible para
otras personas que no han memorizado el orden de las operaciones.


\subsection{Tengo una función que no devuelve lo que
yo espero.}
\index{sentencia!return}
\index{return, sentencia}

Si tienes una sentencia {\tt return} con una expresión compleja,
no tienes la posibilidad de imprimir el resultado antes de
devolverlo.  De nuevo, puedes utilizar una variable temporal.  Por
ejemplo, en lugar de:

\begin{verbatim}
return self.hands[i].removeMatches()
\end{verbatim}
%
podrías escribir:

\begin{verbatim}
count = self.hands[i].removeMatches()
return count
\end{verbatim}
%
Ahora tienes la oportunidad de mostrar en pantalla el valor de
{\tt count} antes de devolverlo.


\subsection{De verdad me atasqué y necesito ayuda.}

Primero, intenta alejarte del computador por algunos minutos.
Los computadores emiten ondas que afectan al cerebro, causando estos
síntomas:

\begin{itemize}

\item Frustración e ira.
\index{frustración}
\index{ira}
\index{depuración!respuesta emocional}
\index{depuración emocional}

\item Creencias supersticiosas (``el computador me odia'') y
pensamiento mágico (``el programa solo funciona cuando uso mi
gorra hacia atrás'').
\index{depuración!superstición}
\index{supersticiosa, depuración}

\item Programación de camino aleatorio (el intento de programar escribiendo
cada programa posible y escoger el que hace lo
correcto).
\index{programación de camino aleatorio}
\index{plan de desarrollo!programación de camino aleatorio}

\end{itemize}

Si te encuentras sufriendo alguno de estos síntomas, levántate
y ve a dar un paseo.  Cuando te hayas tranquilizado, piensa en el programa.
¿Qué está haciendo?  ¿Cuáles son algunas posibles causas de aquel
comportamiento?  ¿Cuándo fue la última vez que tuviste un programa eficaz
y qué hiciste después?

A veces solo toma tiempo encontrar un error de programación.  A menudo
encuentro errores cuando estoy lejos del computador y dejo vagar a mi mente.
Algunos de los mejores lugares para encontrar errores son los trenes, las
duchas y en la cama, justo antes de dormirte.


\subsection{No, realmente necesito ayuda.}

Sucede.  Incluso los mejores programadores se atascan ocasionalmente.
A veces trabajas en un programa tan largo que no puedes ver el
error.  Necesitas otro punto de vista.

Antes de traer a alguien más, asegúrate de tener todo preparado.
Tu programa debería ser tan simple
como sea posible, y deberías estar trabajando en la entrada más pequeña
que causa el error.  Deberías tener sentencias {\tt print} en los
lugares apropiados (y la salida que producen debería ser
comprensible).  Deberías entender el problema lo suficientemente bien
como para describirlo de manera concisa.

Cuando traigas a alguien para que te ayude, asegúrate de darle
la información que necesita:

\begin{itemize}

\item Si hay un mensaje de error, ¿cuál es
y qué parte del programa indica?

\item ¿Qué fue lo último que hiciste antes de que ocurriera este error?
¿Cuáles fueron las últimas líneas de código que escribiste o cuál es
el nuevo caso de prueba que falla?

\item ¿Qué has intentado hasta ahora y qué has aprendido?

\end{itemize}

Cuando encuentres el error, tómate un segundo para pensar sobre qué
podrías haber hecho para encontrarlo de manera más rápida.  La próxima vez
que veas algo similar, serás capaz de entontrar el error con más rapidez.

Recuerda, la meta no solo es hacer que el programa
funcione.  La meta es aprender cómo hacer que el programa funcione.


\chapter{Análisis de algoritmos}
\label{algorithms}

\begin{quote}
Este apéndice es un extracto de {\it Think Complexity}, por
Allen B. Downey, también publicado por O'Reilly Media (2012).  Cuando hayas
terminado con este libro, quizás quieras continuar con aquel.
\end{quote}

El {\bf análisis de algoritmos} es una rama de las ciencias de la computación que
estudia el rendimiento de los algoritmos, especialmente su tiempo de ejecución y
requerimientos de espacio.  Ver
\url{http://en.wikipedia.org/wiki/Analysis_of_algorithms}.
\index{algoritmo} \index{analisis de algoritmos@análisis de algoritmos}

El objetivo práctico del análisis de algoritmos es predecir el rendimiento
de diferentes algoritmos con el fin de guiar decisiones de diseño.

Durante la campaña presidencial de los Estados Unidos de 2008, al candidato
Barack Obama le pidieron que realizara un análisis improvisado cuando
visitó Google.  El director ejecutivo Eric Schmidt le preguntó en broma
por ``la manera más eficiente para ordenar un millón de enteros de 32 bit.''
Obama aparentemente se había dado un tropezón, porque rápidamente
respondió: ``Pienso que el ordenamiendo de burbuja sería la manera incorrecta de proceder.''
Ver \url{http://www.youtube.com/watch?v=k4RRi_ntQc8}.
\index{Obama, Barack}
\index{Schmidt, Eric}
\index{bubble sort}\index{ordenamiento!de burbuja}

Esto es verdad: el ordenamiento de burbuja (en inglés, {\em bubble sort}) es conceptualmente simple pero lento
para bases de datos grandes.  La respuesta que Schmidt probablemente estaba buscando es
``ordenamiento radix'' (\url{http://en.wikipedia.org/wiki/Radix_sort})\footnote{
Pero si tienes una pregunta como esta en una entrevista, pienso que
una mejor respuesta es, ``La manera más rápida de ordenar un millón de enteros
es utilizar cualquier función de ordenamiento que sea proporcionada por el lenguaje
que estoy utilizando.  Su rendimiento es suficientemente bueno para la gran mayoría
de las aplicaciones, pero si resulta que mi aplicación fue muy
lenta, utilizaría un analizador de rendimiento para ver dónde fue utilizado
el tiempo.  Si pareciera que un algoritmo de ordenamiento más rápido tendría
un efecto importante en el rendimiento, entonces buscaría
una buena implementación del ordenamiento radix.''}.

\index{radix sort}\index{ordenamiento!radix}

El objetivo del análisis de algoritmos es hacer comparaciones
significativas entre algoritmos, pero hay algunos problemas:
\index{comparar algoritmos}

\begin{itemize}

\item El rendimiento relativo de los algoritmos podría
depender de características del hardware, por lo cual un algoritmo
podría ser más rápido en la Máquina A y otro en la Máquina B.
La solución general a este problema es especificar un
{\bf modelo de máquina} y analizar el número de pasos, u
operaciones, que un algoritmo requiere según un modelo de máquina dado.
\index{modelo de máquina}

\item El rendimiento relativo podría depender de los detalles del
conjunto de datos.  Por ejemplo, algunos algoritmos
de ordenamiento se ejecutan de manera más rápida si los datos ya están parcialmente ordenados;
otros algoritmos se ejecutan de manera más lenta en este caso.
Una manera común de evitar este problema es analizar el
{\bf peor de los casos}.  A veces es útil
analizar el rendimiento del caso promedio, pero eso es generalmente más difícil,
y el conjunto de casos sobre el cual se establece el promedio podría no ser obvio.
\index{peor de los casos}
\index{caso promedio}

\item El rendimiento relativo depende también del tamaño del
problema.  Un algoritmo de ordenamiento que es rápido para listas pequeñas
podría ser lento para listas largas.
La solución usual a este problema es expresar el tiempo de ejecución
(o número de operaciones) como una función del tamaño del problema,
y agrupar funciones en categorías dependiendo de qué tan rápido
crecen a medida que el tamaño del problema aumenta.

\end{itemize}

Lo bueno de este tipo de comparaciones es que asegura
una clasificación simple de los algoritmos.  Por ejemplo,
si sé que el tiempo de ejecución del Algorigmo A tiende a ser
proporcional al tamaño de la entrada, $n$, y el algoritmo B
tiende a ser proporcional a $n^2$, entonces
espero que A sea más rápido que B, al menos para valores grandes de $n$.

Este tipo de análisis viene con algunas advertencias, pero llegaremos
a eso más adelante.


\section{Orden de crecimiento}

Supongamos que has analizado dos algoritmos y has expresado
sus tiempos de ejecución en términos del tamaño de la entrada:
al algoritmo A le toma $100n+1$ pasos resolver un problema con
tamaño $n$; al algoritmo B le toma $n^2 + n + 1$ pasos.
\index{orden de crecimiento}

La siguiente tabla muestra el tiempo de ejecución de estos algoritmos
para diferentes tamaños de problema:

\begin{tabular}{|r|r|r|}
\hline
Tamaño de & Tiempo de ejecución & Tiempo de ejecución \\
entrada   & del algoritmo A     & del algoritmo B \\
\hline
10        &   1 001             & 111         \\
100       &   10 001            & 10 101         \\
1 000     &   100 001           & 1 001 001         \\
10 000    &   1 000 001         & 100 010 001         \\
\hline
\end{tabular}

Cuando $n=10$, el algoritmo A se ve bastante mal; toma casi 10 veces
más que el algoritmo B.  Pero para $n=100$ son casi lo mismo, y
para valores más grandes A es mucho mejor.

La razón fundamental es que, para valores grandes de $n$, cualquier función
que contiene el término $n^2$ crecerá más rápido que una función cuyo
término principal es $n$.  El {\bf término principal} es el término con el
exponente más alto.
\index{termino principal@término principal}
\index{exponente}

Para el algoritmo A, el término principal tiene un coeficiente grande, 100, que
es el motivo por el cual B es mejor que A para $n$ pequeño.  Pero a pesar de los
coeficientes, siempre habrá algún valor de $n$ donde
$a n^2 > b n$, para cualquier valor de $a$ y $b$.
\index{coeficiente principal}

El mismo argumento se aplica a los términos no principales.  Incluso si el tiempo
de ejecución del algoritmo A fuera $n+1000000$, seguiría siendo mejor que el
algoritmo B para $n$ suficientemente grande.

En general, esperamos que un algoritmo con un término principal más pequeño sea un
mejor algoritmo para problemas grandes, pero para problemas más pequeños puede
haber un {\bf punto de cruce} donde otro algoritmo es mejor.  La
ubicación del punto de cruce depende de los detalles de los
algoritmos, las entradas y el hardware, por lo cual usualmente se ignora para
propósitos de análisis algorítmico.  Pero eso no significa que puedes
olvidarlo.
\index{punto de cruce}

Si dos algoritmos tienen el mismo término de orden principal, es difícil decir
cuál es mejor; nuevamente, la respuesta depende de los detalles.  Entonces para
análisis algorítmico, las funciones con el mismo término principal
se consideran equivalentes, incluso si tienen coeficientes diferentes.

Un {\bf orden de crecimiento} es un conjunto de funciones cuyo comportamiento
de crecimiento se considera equivalente.  Por ejemplo, $2n$, $100n$ y $n+1$
pertenecen al mismo orden de crecimiento, el cual se escribe $O(n)$ en
{\bf notación O grande} (en inglés, {\em Big-Oh}) y a menudo llamada {\bf lineal} porque cada función
de dicho conjunto crece de manera lineal con $n$.
\index{big-oh} \index{notación O grande}
\index{crecimiento!lineal}

Todas las funciones con término principal $n^2$ pertenecen a $O(n^2)$: se
llaman {\bf cuadráticas}.
\index{crecimiento!cuadrático}

La siguiente tabla muestra algunos de los órdenes de crecimiento que
aparecen más comúnmente en el análisis algorítmico,
en orcen creciente de ineficiencia.
\index{ineficiencia}

\begin{tabular}{|r|r|r|}
\hline
Orden de     &   Nombre      \\
crecimiento  &               \\
\hline
$O(1)$             & constante \\
$O(\log_b n)$      & logarítmica (para cualquier $b$) \\
$O(n)$             & lineal \\
$O(n \log_b n)$    & linearítmica \\
$O(n^2)$           & cuadrática    \\
$O(n^3)$           & cúbica    \\
$O(c^n)$           & exponencial (para cualquier $c$)    \\
\hline
\end{tabular}

Para los términos logarítmicos, la base del logarítmo no importa:
cambiar las bases es equivalente a multiplicar por una constante, la cual
no cambia el orden de crecimiento.  De igual manera, todas las
funciones exponenciales pertenecen al mismo orden de crecimiento, independiente de
la base del exponente.
Las funciones exponenciales crecen de manera muy rápida, por lo cual los algoritmos exponenciales
solo son útiles para problemas pequeños.
\index{crecimiento!logarítmico}
\index{crecimiento!exponencial}


\begin{exercise}

Lee la página de Wikipedia sobre la notación O grande en
\url{http://en.wikipedia.org/wiki/Big_O_notation} y
responde las siguientes preguntas:

\begin{enumerate}
\item ¿Cuál es el orden de crecimiento de $n^3 + n^2$?
¿Qué pasa con $1000000 n^3 + n^2$?
¿Qué pasa con $n^3 + 1000000 n^2$?

\item ¿Cuál es el orden de crecimiento de $(n^2 + n) \cdot (n + 1)$?  Antes
  de que empieces a multiplicar, recuerda que solo necesitas el término principal.

\item Si $f$ está en $O(g)$, para una función $g$ no especificada, ¿qué podemos
  decir de $af+b$, donde $a$ y $b$ son constantes?

\item Si $f_1$ y $f_2$ están en $O(g)$, ¿qué podemos decir de $f_1 + f_2$?

\item Si $f_1$ está en $O(g)$
y $f_2$ está en $O(h)$,
¿qué podemos decir de $f_1 + f_2$?

\item Si $f_1$ está en $O(g)$ y $f_2$ es $O(h)$,
¿qué podemos decir de $f_1 \cdot f_2$?
\end{enumerate}

\end{exercise}

Los programadores que se preocupan del rendimiento a menudo encuentran este tipo de
análisis difícil de tragar.  Ellos tienen un punto: a veces los
coeficientes y los términos no principales hacen una diferencia real.
A veces los detalles del hardware, el lenguaje de programación y
las características de la entrada hacen una gran diferencia.  Y para problemas
pequeños, el orden de crecimiento es irrelevante.

Pero si tienes en cuenta esas consideraciones, el análisis algorítmico es una
herramienta útil.  Al menos para problemas grandes, el ``mejor'' algoritmo
es generalmente mejor, y a veces es {\em mucho} mejor.  La
diferencia entre dos algoritmos con el mismo orden de crecimiento es
generalmente un factor constante, ¡pero la diferencia entre un buen algoritmo
y un mal algoritmo no tiene límites!


\section{Análisis de operaciones básicas de Python}

En Python, la mayoría de las operaciones aritméticas son de tiempo constante;
la multiplicación generalmente toma más tiempo que la adición y sustracción, y
la división toma aún más, pero estos tiempos de ejecución no dependen de la
magnitud de los operandos.  Los enteros muy grandes son una excepción: en
ese caso el tiempo de ejecución aumenta con el número de dígitos.
\index{analisis de primitivas@análisis de primitivas}

Las operaciones de indexado ---leer o escribir elementos en una secuencia
o diccionario--- también son de tiempo constante, independiente del tamaño
de la estructura de datos.
\index{indexar}

Un bucle {\tt for} que recorre una secuencia o diccionario es
generalmente lineal, siempre y cuando todas las operaciones en el cuerpo
del bucle sean de tiempo constante.  Por ejemplo, sumar los
elementos de una lista es lineal:

\begin{verbatim}
    total = 0
    for x in t:
        total += x
\end{verbatim}

La función incorporada {\tt sum} también es lineal porque hace
lo mismo, pero tiende a ser más rápida porque es una implementación
más eficiente; en el lenguaje del análisis algorítmico,
tiene un coeficiente principal más pequeño.

Como regla general, si el cuerpo de un bucle está en $O(n^a)$ entonces
el bucle completo está en $O(n^{a+1})$.  La excepción es cuando puedes
mostrar que el bucle se interrumpe después de un número constante de iteraciones.
Si un bucle se ejecuta $k$ veces independiente de $n$, entonces
el bucle está en $O(n^a)$, incluso para $k$ grande.

Multiplicar por $k$ no cambia el orden de crecimiento, ni tampoco
dividir.  Entonces, si el cuerpo de un bucle está en $O(n^a)$ y se ejecuta
$n/k$ veces, el bucle está en $O(n^{a+1})$, incluso para $k$ grande.

La mayoría de las operaciones de cadena y de tupla son lineales, excepto indexar y {\tt
  len}, que es de tiempo constante.  Las funciones incorporadas {\tt min} y
{\tt max} son lineales.  El tiempo de ejecución de una operación de trozo es
proporcional a la longitud de la salida, pero independiente del tamaño
de la entrada.
\index{metodo@método!de cadena}
\index{metodo@método!de tupla}

La concatenación de cadenas es lineal: el tiempo de ejecución depende de la suma
de las longitudes de los operandos.
\index{concatenación de cadenas}

Todos los métodos de cadena son lineales, pero si las longitudes de las
cadenas están acotadas por una constante ---por ejemplo, operaciones en caracteres
individuales--- se consideran de tiempo constante.
El método de cadena {\tt join} es lineal: el tiempo de ejecución depende de
la longitud total de las cadenas.
\index{join, método}

La mayoría de los métodos de lista son lineales, pero hay algunas excepciones:
\index{metodo@método!de lista}

\begin{itemize}

\item Agregar un elemento al final de una lista es de tiempo constante en
promedio; cuando se queda sin espacio, ocasionalmente es copiada
a una ubicación más grande, pero el tiempo total para $n$ operaciones
es $O(n)$, por lo cual el tiempo promedio para cada
operación es $O(1)$.

\item Eliminar un elemento del final de una lista es de tiempo constante.

\item El ordenamiento es $O(n \log n)$.
\index{ordenamiento}

\end{itemize}

La mayoría de las operaciones y métodos de diccionario son de tiempo constante, pero
hay algunas excepciones:
\index{metodo@método!de diccionario}

\begin{itemize}

\item El tiempo de ejecución de {\tt update} es
  proporcional al tamaño del diccionario que se pasa como parámetro,
  no del diccionario que está siendo actualizado.

\item {\tt keys}, {\tt values} e {\tt items} son de tiempo constante porque
  devuelven iteradores.  Pero
  si recorres un bucle con los iteradores, el bucle será lineal.
\index{iterador}

\end{itemize}

El rendimiento de los diccionarios es uno de los milagros menores de
las ciencias de la computación.  Veremos cómo funcionan en la
Sección~\ref{hashtable}.


\begin{exercise}

Lee la página de Wikipedia sobre algoritmos de ordenamiento en
\url{http://en.wikipedia.org/wiki/Sorting_algorithm} y responde
las siguientes preguntas:
\index{ordenamiento}

\begin{enumerate}

\item ¿Qué es un ``ordenamiento de comparación''? ¿Cuál es el mejor orden de crecimiento
  del peor de los casos para un ordenamiento de comparación?  ¿Cuál es el orden de crecimiento
  del peor de los casos para cualquier algoritmo de ordenamiento?
\index{ordenamiento!de comparación}

\item ¿Cuál es el orden de crecimiento del ordenamiento de burbuja, y por qué Barack
  Obama piensa que es ``la manera incorrecta de proceder''?

\item ¿Cuál es el orden de crecimiento del ordenamiento radix?  ¿Qué precondiciones
  necesitamos para utilizarlo?

\item ¿Qué es un ordenamiento estable y por qué podría importar en la práctica?
\index{ordenamiento!estable}

\item ¿Cuál es el peor algoritmo de ordenamiento (y cuál es su nombre)?

\item ¿Qué algoritmo de ordenamiento utiliza la librería de C?  ¿Qué algoritmo de ordenamiento
  utiliza Python?  ¿Son estos algorimos estables?  Quizás tengas que
  utilizar Google para encontrar estas respuestas.

\item Muchos de los algoritmos de ordenamiento diferentes al de comparación son lineales, entonces ¿por qué
  Python utiliza un algoritmo de comparación $O(n \log n)$?

\end{enumerate}

\end{exercise}


\section{Análisis de algoritmos de búsqueda}

Una {\bf búsqueda} es un algoritmo que toma una colección y un ítem objetivo
y determina si el objetivo está en la colección, a menudo
devolviendo el índice del objetivo.
\index{busqueda@búsqueda}

El algoritmo de búsqueda más simple es la ``búsqueda lineal'', que recorre
los ítems de la colección en orden, deteniéndose si encuentra al objetivo.
En el peor de los casos tiene que recorrer toda la colección, por lo cual el tiempo
de ejecución es lineal.
\index{busqueda@búsqueda!lineal}

El operador {\tt in} para secuencias utiliza una búsqueda lineal; los métodos
de cadena como {\tt find} y {\tt count} también.
\index{operador!in}

Si los elementos de la secuencia están en orden, puedes utilizar una {\bf
  búsqueda de bisección}, que es $O(\log n)$.  La búsqueda de bisección es
similar al algoritmo que podrías utilizar para buscar una palabra en un
diccionario (un diccionario de papel, no la estructura de datos).  En lugar de
comenzar del principio y revisar cada ítem en orden, comienzas
con el ítem en el medio y verificas si la palabra que buscas
viene antes o después.  Si viene antes, entonces buscas en la
primera mitad de la secuencia.  De lo contrario, buscas la segunda mitad.
De cualquier manera, disminuyes el número de ítems restantes a la mitad.
\index{busqueda@búsqueda!de bisección}

Si la secuencia tiene 1.000.000 de ítems, tomará cerca de 20 pasos
encontrar la palabra o concluir que no está.  Entonces, eso es cerca de 50.000
veces más rápido que una búsqueda lineal.

La búsqueda de bisección puede ser mucho más rápida que una búsqueda lineal, pero
requiere que la secuencia esté en orden, lo cual podría requerir
trabajo extra.

Hay otra estructura de datos, llamada {\bf tabla hash} (en inglés, {\em hashtable}) que
es incluso más rápida ---puede hacer una búsqueda en tiempo constante--- y
no requiere que los ítems estén ordenados.  Los diccionarios de Python
se implementan utilizando tablas hash, por lo cual la mayoría de las operaciones
de diccionario, incluyendo al operador {\tt in}, son de tiempo constante.


\section{Tablas hash}
\label{hashtable}

Para explicar cómo funcionan las tablas hash y por qué su rendimiento es tan
bueno, comienzo con una implementación simple de un mapa y
gradualmente lo mejoro hasta que sea una tabla hash.
\index{tabla hash}\index{hashtable}

Yo utilizo Python para demostrar estas implementaciones, pero en la vida
real no escribirías código como este en Python: ¡solo utilizarías un
diccionario!  Para el resto de este capítulo, tienes que imaginar que
los diccionarios no existen y quieres implementar una estructura de datos
que mapee de claves a valores.  Las operaciones que tienes que
implementar son:

\begin{description}

\item[{\tt add(k, v)}:] Agrega un nuevo ítem que mapea de una clave {\tt k}
a un valor {\tt v}.  Con un diccionario de Python, {\tt d}, esta operación
se escribe {\tt d[k] = v}.

\item[{\tt get(k)}:] Busca y devuelve el valor que corresponde
a la clave {\tt k}.  Con un diccionario de Python, {\tt d}, esta operación
se escribe {\tt d[k]} o {\tt d.get(k)}.

\end{description}

Por ahora, supongo que cada clave aparece una sola vez.
La implementación más simple de esta interfaz utiliza una lista de
tuplas, donde cada tupla es un par clave-valor.
\index{LinearMap@{\tt LinearMap}}

\begin{verbatim}
class LinearMap:

    def __init__(self):
        self.items = []

    def add(self, k, v):
        self.items.append((k, v))

    def get(self, k):
        for key, val in self.items:
            if key == k:
                return val
        raise KeyError
\end{verbatim}

{\tt add} anexa una tupla clave-valor a la lista de ítems, lo cual
toma tiempo constante.

{\tt get} utiliza un bucle {\tt for} para buscar la lista:
encuentra la clave objetivo y devuelve el valor correspondiente;
de lo contrario, ocurre un {\tt KeyError}.
Entonces, {\tt get} es lineal.
\index{KeyError@{\tt KeyError}}

Una alternativa es mantener la lista ordenada por clave.  Entonces {\tt get}
podría utilizar una búsqueda de bisección, que es $O(\log n)$.  Pero insertar un
nuevo ítem en el medio de una lista es lineal, por lo cual podría no ser la
mejor opción.  Hay otras estructuras de datos que pueden implementar {\tt
  add} y {\tt get} en tiempo logarítmico, pero eso aún no es tan bueno como
el tiempo constante, así que vamos a lo siguiente.
\index{red-black tree}

Una manera de mejorar {\tt LinearMap} es separar la lista de pares clave-valor
en listas más pequeñas.  Aquí hay una implementación llamada
{\tt BetterMap}, que es una lista de 100 LinearMaps.  Tal como veremos
en un segundo, el orden de crecimiento para {\tt get} aún es lineal,
pero {\tt BetterMap} es un paso en el camino hacia las tablas hash:
\index{BetterMap@{\tt BetterMap}}

\begin{verbatim}
class BetterMap:

    def __init__(self, n=100):
        self.maps = []
        for i in range(n):
            self.maps.append(LinearMap())

    def find_map(self, k):
        index = hash(k) % len(self.maps)
        return self.maps[index]

    def add(self, k, v):
        m = self.find_map(k)
        m.add(k, v)

    def get(self, k):
        m = self.find_map(k)
        return m.get(k)
\end{verbatim}

\verb"__init__" crea una lista de {\tt n} {\tt LinearMap}s.

\verb"find_map" es utilizado por
{\tt add} y {\tt get}
para averiguar en cuál mapa poner el
nuevo ítem, o en cuál mapa buscar.

\verb"find_map" utiliza la función incorporada {\tt hash}, que toma
casi cualquier objeto de Python y devuelve un entero.  Una limitación de esta
implementación es que solo funciona con claves hashables.  Los tipos
mutables como las listas y los diccionarios no son hashables.
\index{función!hash}

Los objetos hashables que se consideran equivalentes devuelven el mismo valor
hash, pero lo inverso no es necesariamente verdadero: dos objetos con
valores diferentes pueden devolver el mismo valor hash.

\verb"find_map" utiliza el operador de módulo para ajustar los valores hash
en el rango entre 0 y {\tt len(self.maps)}, por lo cual el resultado es un índice
legal en la lista.  Por supuesto, esto significa que muchos valores
hash diferentes se ajustarán al mismo índice.  Pero si la función hash
distribuye las cosas de manera muy uniforme (que es para lo que están diseñadas
las funciones hash), entonces esperamos $n/100$ ítems por cada LinearMap.

Dado que el tiempo de ejecución de {\tt LinearMap.get} es proporcional al
número de ítems, esperamos que BetterMap sea cerca de 100 veces más rápido
que LinearMap.  El orden de crecimiento es aún lineal, pero el
coeficiente principal es más pequeño.  Eso es bueno, pero aún no es
tan bueno como una tabla hash.

Aquí (finalmente) está la idea clave que hace que las tablas hash sean rápidas: si
puedes mantener acotada la longitud máxima de los LinearMaps, {\tt
  LinearMap.get} es de tiempo constante.  Todo lo que tienes que hacer es un seguimiento
del número de ítems y cuando el número de
ítems por cada LinearMap exceda un límite, cambiar el tamaño de la tabla hash
añadiendo más LinearMaps.
\index{acotado}

Aquí hay una implementación de una tabla hash:
\index{HashMap}

\begin{verbatim}
class HashMap:

    def __init__(self):
        self.maps = BetterMap(2)
        self.num = 0

    def get(self, k):
        return self.maps.get(k)

    def add(self, k, v):
        if self.num == len(self.maps.maps):
            self.resize()

        self.maps.add(k, v)
        self.num += 1

    def resize(self):
        new_maps = BetterMap(self.num * 2)

        for m in self.maps.maps:
            for k, v in m.items:
                new_maps.add(k, v)

        self.maps = new_maps
\end{verbatim}

\verb"__init__" crea un {\tt BetterMap} e inicializa a {\tt num}, que hace un seguimiento del número de ítems.

{\tt get} solo despacha a {\tt BetterMap}.  El trabajo real ocurre
en {\tt add}, que verifica el número de ítems y el tamaño del
{\tt BetterMap}: si son iguales, el número promedio de ítems por cada
LinearMap es 1, entonces llama a {\tt resize}.

{\tt resize} crea un nuevo {\tt BetterMap}, dos veces más grande que el
anterior, y luego ``rehashea'' los ítems desde el mapa antiguo hacia el nuevo.

Rehashear es necesario porque al cambiar el número de LinearMaps
cambia el denominador del operador de módulo en
\verb"find_map".  Eso significa que algunos objetos que solían
hacer hash en el mismo LinearMap se dividirán (que es
lo que queríamos, ¿verdad?).
\index{rehasheo}

El rehasheo es lineal, entonces
{\tt resize} es lineal, que podría parecer mal, dado que prometí
que {\tt add} sería de tiempo constante.  Pero recuerda que
no tenemos que cambiar de tamaño cada vez, entonces {\tt add} es generalmente
de tiempo constante y solo ocasionalmente lineal.  La cantidad total
de trabajo al ejecutar {\tt add} $n$ veces es proporcional a $n$,
¡entonces el tiempo promedio de cada {\tt add} es de tiempo constante!
\index{tiempo constante}

Para ver cómo funciona esto, piensa en comenzar con un
HashTable vacío y agregar una secuencia de ítems.  Comenzamos con 2 LinearMaps,
entonces los 2 primeros añadidos son rápidos (no se requiere cambiar tamaño).  Digamos
que estos toman una unidad de trabajo cada uno.  El siguiente añadido
requiere un cambio de tamaño, por lo cual tenemos que rehashear los primeros dos
ítems (digamos que 2 unidades más de trabajo) y luego
añadir el tercer ítem (una unidad más).  Añadir el siguiente ítem
cuesta 1 unidad, entonces el total hasta ahora es
de 6 unidades de trabajo para 4 ítems.

El siguiente {\tt add} cuesta 5 unidades, pero los siguientes tres
son solo una unidad cada uno, entonces el total es de 14 unidades para los
primeros 8 añadidos.

El siguiente {\tt add} cuesta 9 unidades, pero luego podemos añadir 7 más
antes del siguiente cambio de tamaño, entonces el total es de 30 unidades para los
primeros 16 añadidos.

Después de 32 añadidos, el costo total es de 62 unidades, y espero que comiences
a ver el patrón.  Después de $n$ añadidos, donde $n$ es potencia de dos, el
costo total es de $2n-2$ unidades, entonces el trabajo promedio por cada añadido es
un poco menos que dos unidades.  Cuando $n$ es una potencia de dos, ese es
el mejor caso; para los otros valores de $n$  el trabajo promedio es un poco
más alto, pero eso no es importante.  Lo importante es que
es $O(1)$.
\index{costo promedio}

La Figura~\ref{fig.hash} muestra cómo funciona esto de manera gráfica.  Cada
bloque representa una unidad de trabajo.  Las columnas muestran el trabajo
total para cada añadido en orden de izquierda a derecha: los primeros dos
{\tt adds} cuestan 1 unidad cada uno, el tercero cuesta 3 unidades, etc.

\begin{figure}
\centerline{\includegraphics[width=5.5in]{figs/towers.pdf}}
\caption{El costo de añadir a una tabla hash.\label{fig.hash}}
\end{figure}

El trabajo extra de rehashear aparece como una secuencia de torres
cada vez más altas cuyo espacio entre ellas es cada vez mayor.  Ahora si derribas
las torres, repartiendo el costo de cambiar de tamaño sobre todos los
añadidos, puedes ver gráficamente que el costo total después de $n$
añadidos es $2n - 2$.

Una característica importante de este algoritmo es que, cuando cambiamos el tamaño del
HashTable, este crece de manera geométrica, es decir, multiplicamos el tamaño por una
constante.  Si aumentas el tamaño de manera
aritmética ---añadiendo un número fijo cada vez--- el tiempo promedio
por cada {\tt add} es lineal.
\index{cambio de tamaño geométrico}

Puedes descargar mi implementación de HashMap en
\url{http://thinkpython.com/code/Map.py}, pero recuerda que no hay
razón para utilizarla: si quieres un mapa, solo utiliza un diccionario de Python.

\section{Glosario}

\begin{description}

\item[análisis de algoritmos:] Una manera de comparar algoritmos en términos de
su tiempo de ejecución y/o requerimientos de espacio.
\index{analisis de algoritmos@análisis de algoritmos}

\item[modelo de máquina:] Una representación simplificada de un computador utilizada
para describir algoritmos.
\index{modelo de máquina}

\item[peor de los casos:] La entrada que hace que un algoritmo dado se ejecute de la manera más lenta (o
requiera el mayor espacio).
\index{peor de los casos}

\item[término principal:] En un polinomio, el término con el exponente más alto.
\index{termino principal@término principal}

\item[punto de cruce:] El tamaño de un problema donde dos algoritmos requieren
el mismo tiempo de ejecución o espacio.
\index{punto de cruce}

\item[orden de crecimiento:] Un conjunto de funciones que crecen todas de una manera
considerada equivalente para propósitos del análisis de algoritmos.
Por ejemplo, todas las funciones que crecen linealmente pertenecen al mismo
orden de crecimiento.
\index{orden de crecimiento}

\item[notación O grande:] Notación para representar un orden de crecimiento;
por ejemplo, $O(n)$ representa el conjunto de funciones que crecen
linealmente.
\index{notación O grande}

\item[lineal:] Un algoritmo cuyo tiempo de ejecución es proporcional al
tamaño del problema, al menos para tamaños de problema grandes.
\index{lineal}

\item[cuadrático:] Un algoritmo cuyo tiempo de ejecución es proporcional a
$n^2$, donde $n$ es la medida del tamaño del problema.
\index{cuadrático}

\item[búsqueda:] El problema de localizar un elemento de una colección
(como una lista o diccionario) o determinar que este no está presente.
\index{busqueda@búsqueda}

\item[tabla hash:] Una estructura de datos que representa una colección de
pares clave-valor y realiza una búsqueda en tiempo constante.
\index{tabla hash}

\end{description}


\printindex

\clearemptydoublepage
%\blankpage
%\blankpage
%\blankpage


\end{document}
